\chapter{Perfectoid spaces}
    \begin{abstract}
        
    \end{abstract}
    
    \section{"The perfect(oid) space doesn't exi-"} \label{section: perfectoid_spaces}
        \subsection{Perfectoid spaces}
            \subsubsection{Some almost-ring theory}
                \paragraph{Almost modules}
                    To be able to understand the consequences of perfectoid fields being deeply ramified, we will be needing the language of almost-ring theory. Our main reference shall be \cite[Chapter 14]{gabber_ramero_almost_ring_theory}.
                    
                    \begin{convention}[\say{Basic setups}] \label{conv: basic_setups}
                        A \textbf{basic setup} for us shall always be a pair $(\scrV, \m)$ consisting of a commutative ring $\scrV$ along with an idempotent ideal $\m$ (i.e. one such that $\m^2 = \m$; note that this ideal need not be maximal, despite what our notation might suggest). One often speaks also of \textbf{flat basic setups}, which are basic setups $(\scrV, \m)$ such that $\m$ is flat over $\scrV$; such a specification can be made to ensure that the functor $- \tensor_{\scrV} \m$ is exact.
                    \end{convention}
                    
                    \begin{definition}[Almost-zero modules] \label{def: almost_zero_modules}
                        Let $(\scrV, \m)$ be a basic setup (cf. convention \ref{conv: basic_setups}) and fix a $\scrV$-algebra. Then, a $\scrV$-module $M$ (or should the situation calls for speification, a $(\scrV, \m)$-module $M$) is said to be \textbf{almost-zero} if and only if it is annihilated by $\m$, i.e.:
                            $$\m M = 0$$
                    \end{definition}
                    \begin{lemma} \label{lemma: tensor_powers_of_flat_modules}
                        Let $R$ be a commutative ring and let $N$ be a flat $R$-module. Then, $N \tensor_R N$ is also flat. Also, for all $R$-ideals $\m$, one has:
                            $$\m \tensor_R M \cong \m M$$
                    \end{lemma}
                        \begin{proof}
                            If $N$ is flat over $R$ then the functor $- \tensor_R N$ will be left-exact. The composition of two left-exact is again left-exact for trivial reasons. Thus $N \tensor_R N$ is flat over $R$.
                            
                            As for the assertion that $\m \tensor_R M \cong \m M$ whenever $\m$ is flat, we can rely on homological algebra. Specifically, because $\m$ is flat, we have:
                                $$\m M \cong \Tor_R^1(\m, M) \cong 0$$
                        \end{proof}
                    \begin{proposition}[Another definition of almost-zero modules] \label{prop: almost_zero_module_alt_def}
                        Let $(\scrV, \m)$ be a \textit{flat} basic setup (cf. convention \ref{conv: basic_setups}) and fix a $\scrV$-algebra. Then, a $\scrV$-module $M$ is said to be \textbf{almost-zero} if and only if:
                            $$\m \tensor_{\scrV} M  \cong0$$
                    \end{proposition}
                        \begin{proof}
                            \noindent
                            \begin{enumerate}
                                \item Suppose first of all that $M$ is \textit{non-zero} and almost-zero, i.e. that $\m M \cong 0$. Then, a direct application of lemma \ref{lemma: tensor_powers_of_flat_modules} (which is appropriate because $\m$ is flat over $\scrV$) tells us that:
                                    $$0 \cong \m M \cong \m \tensor_{\scrV} M$$
                                \item Conversely, suppose that $\m \tensor_{\scrV} M \cong 0$. Because $\m$ is flat over $\scrV$, this tells us that there exists the following short exact sequence:
                                    $$0 \to \m \tensor_{\scrV} \m M \to 0 \to M/\m M \tensor_{\scrV} \m \to 0$$
                                Using some abstract nonsense, we can then see that:
                                    $$\m \tensor_{\scrV} \m M \cong 0$$
                                We can apply the flatness assumption on $\m$ again, which gives:
                                    $$0 \cong \m \tensor_{\scrV} \m M \cong (\m \tensor_{\scrV} \m) M \cong \m^2 M$$
                                Lastly, because $\m$ is idempotent, the above implies that:
                                    $$0 \cong \m^2 M = \m M$$
                                i.e. $M$ is almost-zero as a $\scrV$-module.
                            \end{enumerate}
                        \end{proof}
                        
                    \begin{proposition}[Thick subcategories of almost-zero modules] \label{prop: thick_subcategories_of_almost_zero_modules}
                        Let $(\scrV, \m)$ be a flat basic setup. Then, almost-zero $\scrV$-modules form a thick subcategory of $\scrV\mod$, which we shall denote by $(\scrV, \m)\mod^0$.
                    \end{proposition}
                        \begin{proof}
                            \noindent
                            \begin{enumerate}
                                \item \textbf{(Full subcategories of almost-zero modules):} Let $M, N$ be two almost-zero $(\scrV, \m)$-modules. Then, because:
                                    $$\m M \cong \m N \cong 0$$
                                we have:
                                    $$M \cong M/\m M, N \cong N/\m N$$
                                and so any diagram of the following form, wherein the rows are short exact sequences, would commute:
                                    $$
                                        \begin{tikzcd}
                                        	0 & {\m M} & M & {M/\m M} & 0 \\
                                        	0 & {\m N} & N & {N/ \m N} & 0
                                        	\arrow[from=1-1, to=1-2]
                                        	\arrow[from=1-2, to=1-3]
                                        	\arrow[from=1-3, to=1-4]
                                        	\arrow[from=1-4, to=1-5]
                                        	\arrow[from=2-1, to=2-2]
                                        	\arrow[from=2-2, to=2-3]
                                        	\arrow[from=2-3, to=2-4]
                                        	\arrow[from=2-4, to=2-5]
                                        	\arrow[from=1-2, to=2-2]
                                        	\arrow[from=1-3, to=2-3]
                                        	\arrow[from=1-4, to=2-4]
                                        \end{tikzcd}
                                    $$
                                From this, we can infer that morphisms of almost-zero module are just module homomorphisms, which means that there exists a full subcategory $(\scrV, \m)\mod^0$ of $\scrV\mod$ spanned by almost-zero $(\scrV, \m)$-modules.
                                \item \textbf{(Thickness):} A \textit{full} subcategory $\S$ of a triangulated category $\T$ (cf. definition \ref{def: triangulated_infinity_categories}; our triangulated category is $\scrV\mod$) is said to be \textbf{thick} whenever it is closed under extensions, which is to say, for all short exact sequences:
                                    $$0 \to M' \to M \to M'' \to 0$$
                                should $M$ be an object of $\S$, then so must $M'$ and $M''$ too, and vice versa (one might also think of a thick subcategory as the homotopy category of a stable full $\infty$-subcategory of a triangulated $\infty$-category; cf. remark \ref{remark: elementary_properties_of_triangulated_categories}). We already have fullness, so it remains to show that $(\scrV, \m)\mod^0$ is closed under extensions. To this end, consider a short exact sequence of $\scrV$-modules, such as the following:
                                    $$0 \to M' \to M \to M'' \to 0$$
                                wherein $M$ is almost-zero. Then, because binary direct sums of modules are biproducts and because $\m$ is flat over $\scrV$, one can construct the following diagram with short exact rows out of the short exact sequence above:
                                    $$
                                        \begin{tikzcd}
                                        	0 & {\m \tensor_{\scrV} M'} & {\m \tensor_{\scrV} (M' \oplus M'')} & {\m \tensor_{\scrV} M''} & 0 \\
                                        	0 & {\m \tensor_{\scrV} M'} & {\m \tensor_{\scrV} M} & {\m \tensor_{\scrV} M''} & 0 \\
                                        	0 & {\m \tensor_{\scrV} M'} & {\m \tensor_{\scrV} (M' \oplus M'')} & {\m \tensor_{\scrV} M''} & 0
                                        	\arrow[from=1-1, to=1-2]
                                        	\arrow[from=1-2, to=1-3]
                                        	\arrow[from=1-3, to=1-4]
                                        	\arrow[from=1-4, to=1-5]
                                        	\arrow[from=2-1, to=2-2]
                                        	\arrow[from=2-2, to=2-3]
                                        	\arrow[from=2-3, to=2-4]
                                        	\arrow[from=2-4, to=2-5]
                                        	\arrow[from=3-1, to=3-2]
                                        	\arrow[from=3-2, to=3-3]
                                        	\arrow[from=3-3, to=3-4]
                                        	\arrow[from=3-4, to=3-5]
                                        	\arrow["{=}"{description}, from=2-2, to=3-2]
                                        	\arrow[dashed, tail, from=1-3, to=2-3]
                                        	\arrow[dashed, two heads, from=2-3, to=3-3]
                                        	\arrow["{=}"{description}, from=1-4, to=2-4]
                                        	\arrow["{=}"{description}, from=2-4, to=3-4]
                                        	\arrow["{=}"{description}, from=1-2, to=2-2]
                                        \end{tikzcd}
                                    $$
                                We can reuse the flatness assumption on $\m$ again (specifically, lemma \ref{lemma: tensor_powers_of_flat_modules}, which implies that $\m \tensor_{\scrV} M' \cong \m M'$ and $\m \tensor_{\scrV} M'' \cong \m M''$) as well as the assumption that $M'$ and $M''$ are almost-zero, and the fact that tensor products commute with finite direct sums, to condense the above diagram into:
                                    $$
                                        \begin{tikzcd}
                                        	0 & 0 & 0 & 0 & 0 \\
                                        	0 & 0 & {\m M} & 0 & 0 \\
                                        	0 & 0 & 0 & 0 & 0
                                        	\arrow[from=1-1, to=1-2]
                                        	\arrow[from=1-2, to=1-3]
                                        	\arrow[from=2-1, to=2-2]
                                        	\arrow[from=2-2, to=2-3]
                                        	\arrow[dashed, tail, from=1-3, to=2-3]
                                        	\arrow[dashed, two heads, from=2-3, to=3-3]
                                        	\arrow[from=1-2, to=2-2]
                                        	\arrow[from=2-2, to=3-2]
                                        	\arrow[from=3-1, to=3-2]
                                        	\arrow[from=3-2, to=3-3]
                                        	\arrow[from=1-3, to=1-4]
                                        	\arrow[from=1-4, to=1-5]
                                        	\arrow[from=2-3, to=2-4]
                                        	\arrow[from=2-4, to=2-5]
                                        	\arrow[from=1-4, to=2-4]
                                        	\arrow[from=2-4, to=3-4]
                                        	\arrow[from=3-3, to=3-4]
                                        	\arrow[from=3-4, to=3-5]
                                        \end{tikzcd}
                                    $$
                                We can now easily deduce that $\m M \cong 0$, i.e. that $M$ is almost-zero. This tells us that the full subcategory $(\scrV, \m)\mod^0$ is closed under extensions, and thus thick by definition.
                            \end{enumerate}
                        \end{proof}
                    \begin{remark}[Almost-zero modules over non-flat basic setups] \label{remark: almost_zero_modules_over_non_flat_basic_setups}
                        \noindent
                        \begin{itemize}
                            \item \textbf{(Removing the flatness assumption):} Proposition \ref{prop: thick_subcategories_of_almost_zero_modules} allows us to pull off quite a stunt, and for that matter, not even with much difficulty: the flatness assumption on $\m$ can be completely removed! Of course, the trade-off is that now, only the $\infty$-categorical version of proposition \ref{prop: thick_subcategories_of_almost_zero_modules} would hold: via the Dold-Kan Correspondence, one has that for $(\scrV, \m)$ a basic setup, $\scrV$-modules annihilated by $\m$ span a stable full $\infty$-subcategory $(\scrV, \m)\mod^0$ (or perhaps ${}^{\leq 0}_{(\scrV, \m)}\mod^0$) of the triangulated $\infty$-category ${}^{\leq 0}_{\scrV}\mod$ of projective resolutions of $\scrV$-modules. This descends naturally down to the level of derived categories. One can thus generalise proposition \ref{prop: almost_zero_module_alt_def} too: a $\scrV$-module is almost-zero if and only if:
                                $$\m \tensor_{\scrV}^{\L} M \cong_{\qis} 0$$
                            (i.e. the chain complex $\Tor_{\scrV}^*(\m, M)$ is homotopic to $0$).
                            \item \textbf{(Abelian categories of almost-zero modules):} If we extract the hearts of the t-structures out of ${}^{\leq 0}_{(\scrV, \m)}\mod^0$ and ${}^{\leq 0}_{\scrV}\mod$, we will be able to establish ${}^{\leq 0}_{(\scrV, \m)}\mod^{\almost, \heart}$ as a Serre subcategory of ${}^{\leq 0}_{\scrV}\mod^{\heart}$, and since hearts of t-structure are spanned by objects concentrated in degree $0$, one can furthurmore recognise ${}^{\leq 0}_{(\scrV, \m)}\mod^0$ as a Serre subcategory of $\scrV\mod$ whenever the basic setup $(\scrV, \m)$ is \textit{flat}. In particular, we have that the thick subcategories of almost-zero modules are \textit{abelian}.
                        \end{itemize}
                    \end{remark}
                    
                    \begin{definition}[Almost modules] \label{def: almost_modules}
                        Let $(\scrV, \m)$ be a basic setup. Then, the essential image of the functor:
                            $$\m \tensor_{\scrV}^{\L} -: {\scrV}^{\leq 0}\mod \to {\scrV}^{\leq 0}\mod$$
                        (i.e. the category spanned by objects of the form $\m \tensor_{\scrV}^{\L} M$, where $M$ is a $\scrV$-module) shall be called the category of \textbf{almost modules}. We shall denote this category by ${(\scrV, \m)}^{\leq 0}\mod^{\almost}$.
                    \end{definition}
                    \begin{remark}[Categories of almost-zero modules are kernels] \label{remark: categories_of_almost_zero_modules_are_kernels}
                        It is not hard to see that for $(\scrV, \m)$ a basic setup, the category ${(\scrV, \m)}^{\leq 0}\mod^0$ of almost-zero $(\scrV, \m)$-modules is the kernel or the essentially surjective functor:
                            $$\m \tensor_{\scrV}^{\L} -: {\scrV}^{\leq 0}\mod \to {(\scrV, \m)}^{\leq 0}\mod^{\almost}$$
                        Note that this kernel is well-defined because $\m \tensor_{\scrV}^{\L} -$ is an idempotent functor (cf. remark \ref{remark: almost_zero_modules_over_non_flat_basic_setups}).
                    \end{remark}
                    
                    \begin{proposition}[Localising at almost modules] \label{prop: localising_at_almost_modules}
                        Let $(\scrV, \m)$ be a basic setup (which need not be flat). Then, the tensor-hom adjunction establishes ${(\scrV, \m)}^{\leq 0}\mod^{\almost}$ as a localisation. This is to say, there exists the following $\infty$-adjunction:
                            $$
                                \begin{tikzcd}
                                	{{(\scrV, \m)}^{\leq 0}\mod^{\almost}} & {{\scrV}^{\leq 0}\mod}
                                	\arrow[""{name=0, anchor=center, inner sep=0}, "{j_*}"', shift right=2, hook, from=1-1, to=1-2]
                                	\arrow[""{name=1, anchor=center, inner sep=0}, "{j^*}"', shift right=2, from=1-2, to=1-1]
                                	\arrow["\dashv"{anchor=center, rotate=-90}, draw=none, from=1, to=0]
                                \end{tikzcd}
                            $$
                        wherein $j^* \cong \m \tensor_{\scrV}^{\L} -$ and $j_*(-) \cong \R\Hom_{\scrV}(\m, -)$.
                    \end{proposition}
                        \begin{proof}
                            The only thing to prove here is that $j_*$ ought to be fully faithful, which we can do via showing that the counit of the adjunction is naturally isomorphic to the identity; explicitly, this means showing that for all almost-zero $(\scrV, \m)$-modules $M$, one has the following isomorphism:
                                $$j^* j_* M \cong M$$
                            However, this is an automatic consequence of the fact that ${(\scrV, \m)}^{\leq 0}\mod^{\almost}$ is the essential image of $j^*: {\scrV}^{\leq}\mod \to {\scrV}^{\leq}\mod$.
                        \end{proof}
                    
                    \begin{theorem}[Four functors for almost modules] \label{theorem: four_functors_for_almost_modules}
                        Let $(\scrV, \m)$ be a \textit{flat} basic setup. Then, the usual four-functor pull-push yoga is applicable to almost modules over \textit{flat} basic setups, in the sense that one has the following adjunction quadruple:
                            $$
                                \begin{tikzcd}
                                	{{\scrV}^{\leq 0}\mod} & {{(\scrV, \m)}^{\leq 0}\mod^{\almost}}
                                	\arrow[""{name=0, anchor=center, inner sep=0}, "{j_! \ladjoint j^* \ladjoint j_* \ladjoint j^!}"', shift right=5, from=1-2, to=1-1]
                                	\arrow[""{name=1, anchor=center, inner sep=0}, shift right=5, from=1-1, to=1-2]
                                	\arrow[""{name=2, anchor=center, inner sep=0}, shift left=2, from=1-2, to=1-1]
                                	\arrow[""{name=3, anchor=center, inner sep=0}, shift left=2, from=1-1, to=1-2]
                                	\arrow["\dashv"{anchor=center, rotate=-90}, draw=none, from=0, to=3]
                                	\arrow["\dashv"{anchor=center, rotate=-90}, draw=none, from=3, to=2]
                                	\arrow["\dashv"{anchor=center, rotate=-90}, draw=none, from=2, to=1]
                                \end{tikzcd}
                            $$
                        wherein $j^* \cong \m \tensor_{\scrV}^{\L} -$ and $j_*(-) \cong \R\Hom_{\scrV}(\m, -)$.
                    \end{theorem}
                        \begin{proof}
                            We have already been provided with the middle adjunction $j^* \ladjoint j_*$, so let us just prove that the $!$-pushforwards and $!$-pullbacks exist and fit into the following adjunctions:
                                $$j_! \ladjoint j^*$$
                                $$j_* \ladjoint j^!$$
                            Although, we should note that we can have $j^* \cong \m \tensor_{\scrV}^{\L} -$ and $j_*(-) \cong \R\Hom_{\scrV}(\m, -)$ as two of the basic \href{https://ncatlab.org/nlab/show/six+operations}{\underline{four operations}} because ${(\scrV, \m)}^{\leq 0}\mod^{\almost}$ is the essential image of a functor out of ${\scrV}^{\leq 0}\mod$ (cf. proposition \ref{prop: localising_at_almost_modules}). 
                                \begin{enumerate}
                                    \item \textbf{($!$-pushforward):} To show that the further left-adjoint $j^!$ exists, we will be using Lurie's Adjoint Functor Theorem \cite[Corollary 5.5.2.9]{HTT}, which states that should $\C$ and $\D$ be presentable $\infty$-categories and $R: \C \to \D$ be a functor between them, then there exists an adjunction $L \ladjoint R$ if and only if $R$ is accessible (i.e. $R$ preserves all filtered colimits) and left-exact. For this, we will first need to show that ${(\scrV, \m)}^{\leq 0}\mod^{\almost}$ is a presentable $\infty$-category (it is well-known that module categories like ${\scrV}^{\leq 0}\mod$ are presentable, so we will not be providing a proof).
                                        \begin{enumerate}
                                            \item \textbf{(${(\scrV, \m)}^{\leq 0}\mod^{\almost}$ is presentable):} Recall that a presentable stable $\infty$-category is the same as an $\infty$-category that is:
                                                \begin{itemize}
                                                    \item accessible (cf. proposition \ref{prop: accessible_stable_infinity_categories}; accessible $\infty$-category is simply one that is its own ind-completion) and
                                                    \item equivalent to a left-exact localisation (i.e a left-exact functor with a full faithful right-adjoint) of a stable $\infty$-category.
                                                \end{itemize}  
                                                
                                            The second condition is an automatic consequence of proposition \ref{prop: localising_at_almost_modules} and the assumption that $\m$ is flat (which means, by definition, that $j^*(-) \cong \m \tensor_{\scrV} -$ is a left-exact functor), and so it remains to prove the first      condition. 
                                            
                                            For this, consider a filtered colimit:
                                                $$\underset{i \in I}{\colim} (\m \tensor_{\scrV}^{\L} M_i)$$
                                            of almost $(\scrV, \m)$-modules $\m \tensor_{\scrV}^{\L} M_i$ (see proposition \ref{prop: localising_at_almost_modules} for why almost modules are of this form). We can then apply the fact that $\m \tensor_{\scrV}^{\L} -$ commutes with all colimits to get:
                                                $$\m \tensor_{\scrV}^{\L} \underset{i \in I}{\colim} (\m \tensor_{\scrV}^{\L} M_i) \cong \m \tensor_{\scrV}^{\L} \underset{i \in I}{\colim} M_i$$
                                            Because ${\scrV}^{\leq 0}\mod$ is a cocomplete category (which implies, in particular, that $\underset{i \in I}{\colim} M_i$ is a $\scrV$-module), this tells us that ${(\scrV, \m)}^{\leq 0}\mod^{\almost}$ is closed under filtered colimits. 
                                            
                                            The category ${(\scrV, \m)}^{\leq 0}\mod^{\almost}$ is thus accessible by definition, and we have therefore shown that ${(\scrV, \m)}^{\leq 0}\mod^{\almost}$ is presentable as an $\infty$-category.
                                            \item \textbf{($j^*$ is accessible):} Because tensor products commute with all colimits, $j^*$ is trivially accessible by virtue of being naturally isomorphic to $\m \tensor_{\scrV}^{\L} -$. 
                                            \item \textbf{($j^*$ is left-exact):} This is an automatic consequence of the flatness assumption on $\m$.
                                        \end{enumerate}
                                    \item \textbf{($!$-pullback):} We have already shown that ${(\scrV, \m)}^{\leq 0}\mod^{\almost}$ is a presentable $\infty$-category, so we can apply Lurie's Adjoint Functor Theorem again, which tells us that we will only need to show that $j_*$ is right-exact. However, because corepresentable functors preserve colimits \textit{a priori} (cf. \cite{nlab:hom-functor_preserves_limits}), the functor:
                                        $$j_* \cong \R\Hom_{\scrV}(\m, -)$$
                                    is trivially right-exact.
                                \end{enumerate}
                        \end{proof}
                        
                    \begin{proposition}[Tensor products of almost modules] \label{prop: tensor_products_of_almost_modules}
                        Let $(\scrV, \m)$ be a basic setup. Then, ${(\scrV, \m)}^{\leq 0}\mod^{\almost}$ is a rigid monoidal category; in particular, there is a compatibility of tensor products in the following manner for all $\scrV$-modules $M$ and $N$:
                            $$j^*M \tensor^{\L}_{\scrV} j^*N \cong j^*(M \tensor^{\L}_{\scrV} N)$$
                        where $j^*$ is as in proposition \ref{prop: localising_at_almost_modules} and theorem \ref{theorem: four_functors_for_almost_modules}; the monoidal unit is $\m$.
                    \end{proposition}
                        \begin{proof}
                            
                        \end{proof}
                    \begin{convention}[Almost tensor products] \label{conv: almost_tensor_products}
                        Let $(\scrV, \m)$ be a basic setup. Then for all $M, N \in {\scrV}^{\leq 0}\mod$, let us abbreviate the quasi-isomorphism:
                            $$j^*M \tensor^{\L}_{\scrV} j^*N \cong j^*(M \tensor^{\L}_{\scrV} N)$$
                        by:
                            $$M^{\almost} \tensor^{\L}_{\scrV} N^{\almost} \cong (M \tensor^{\L}_{\scrV} N)^{\almost}$$
                        whenever the underlying basic setup $(\scrV, \m)$ is understood.
                    \end{convention}
                    \begin{remark}[(In)compatibility of monoidal structures] \label{remark: incompatible_monoidal_structures_almost_modules}
                        Let $(\scrV, \m)$ be a basic setup, which is possible non-flat.
                        \begin{itemize}
                            \item \textbf{(Tensor products of almost modules):} ${\scrV}^{\leq 0}\mod$ does \textit{not} admit ${(\scrV, \m)}^{\leq 0}\mod^{\almost}$ as a \textit{monoidal} subcategory, as the monoidal structures thereon are formed differently; in particular, their monoidal units ($\scrV$ and $\m$ respectively) do not coincide.
                            \item \textbf{(Tensor products of almost-zero modules):} The story is not quite the same for almost-zero modules, which form a subcategory of ${\scrV}^{\leq 0}\mod$. Thanks to the assumption that $\m$ is idempotent (in the sense that $\m \tensor_{\scrV}^{\L} \m \cong \m$; cf. convention \ref{conv: basic_setups} and remark \ref{remark: almost_zero_modules_over_non_flat_basic_setups}), one has the following for all almost-zero modules $M, N$:
                                $$\m \tensor_{\scrV}^{\L} (M \tensor_{\scrV}^{\L} N) \cong (\m \tensor_{\scrV}^{\L} \m) \tensor_{\scrV}^{\L} (M \tensor_{\scrV}^{\L} N) \cong (\m \tensor_{\scrV}^{\L} M) \tensor_{\scrV}^{\L} (\m \tensor_{\scrV}^{\L} N) = 0 \tensor_{\scrV}^{\L} 0 \cong 0$$
                            which proves that ${(\scrV, \m)}^{\leq 0}\mod^0$ is a monoidal subcategory of ${\scrV}^{\leq 0}\mod$. 
                        \end{itemize}
                    \end{remark}
                    
                    \begin{definition}[Almost flatness] \label{def: almost_flatness}
                    
                    \end{definition}
                
                \paragraph{Deep ramifications}
        
            \subsubsection{Perfectoid fields and rings}
                We begin by introducing the notion of perfectoid rings and affinoid perfectoid (pre)adic spaces.
            
                \begin{definition}[Perfectoid fields and perfectoid algebras over them] \label{def: perfectoid_fields}
                    A \textit{complete} Tate ring $R$ is \textbf{perfectoid} if and only if:
                        \begin{itemize}
                            \item \textbf{(Uniformity):} The subring of topologically bounded elements $R^{\circ}$ is itself bounded (in the sense that for all open neighbourhoods $U$ of $0$, there exist another open neighbourhood $V$ of $0$ such that $VR^{\circ} \subseteq U$), and 
                            \item \textbf{(Surjectivity of Frobenius):} There exists a pseudo-uniformiser $\varphi \in R^{\circ \circ}$ such that $\varphi^p \mid p$ and that the Frobenius on $R^{\circ}/\varphi$ is \textit{surjective}.
                        \end{itemize}
                \end{definition}
                \begin{remark}[Perfectoid fields contain all $p$-power roots] \label{remark: perfectoid_fields_have_p_power_roots}
                    The requirement that residual Frobenii are surjective is an especially important one for perfectoid spaces. There is a myriad of facts that it implies, but they are all further consequences of one: that perfectoid fields contain lots of $p^{th}$ power roots. Interestingly, this is a straightforward consequence of the definition of surjectivity, which tells us that for all perfectoid fields $K$, one has:
                        $$\forall x \in K^{\circ}/p: \exists y \in K^{\circ}/p: y^p = x$$
                    which is the same as:
                        $$\forall x \in K^{\circ}/p: \exists y \in K^{\circ}/p: y = x^{\frac1p}$$
                    and from this one gets the existence of roots of higher powers $x^{\frac{1}{p^n}}$. 
                    
                    This detail will be very important when we try to construct examples of perfectoid fields, as well as in lemma \ref{lemma: perfectoid_tilts_are_perfect} where we establish the fact that tilts of perfectoid fields are perfect (thereby justifying the name \say{perfectoid}).
                \end{remark}
                \begin{example}[Perfectoid rings and fields] \label{example: perfectoid_rings_and_fields}
                    Standard examples of perfectoid fields are the topological completions of $\Q_p(p^{1/p^{\infty}}), \Q_p(\mu_{p^{\infty}}), \Q_p^{\alg}$, and $\F_p(\!(t^{1/p^{\infty}})\!)$.
                \end{example}
                
                \begin{definition}[Tilts] \label{def: perfectoid_tilts}
                    Let $R$ be a perfectoid ring and let $\varpi \in R^{\circ \circ}$ be some pseudo-uniformiser. Then, first of all, one defines its multiplicative tilt $R^{\flat}$, where one takes the limit in the category of commutative monoids. Next, one gives $R^{\flat}$ the structure of a ring via:
                        $$R^{\flat} \cong (R^{\circ}/\varpi)^{\flat}[1/\varpi^{\flat}]$$
                \end{definition}
                \begin{example}
                    Different perfectoid fields can have isomorphic tilts, e.g. the tilts of both $\Q_p(p^{1/p^{\infty}})$ and $\Q_p(\mu_{p^{\infty}})$ are both isomorphic to $\F_p(\!(t^{1/p^{\infty}})\!)$. Later on, we will see that this gives rise to the moduli space of untilts (i.e. the Fargues-Fontaine Curve).
                \end{example}
                
                \begin{lemma}[Tilts are perfectoid] \label{lemma: perfectoid_tilts_are_perfect}
                    \noindent
                    \begin{enumerate}
                        \item The tilt of any perfectoid ring of residue characteristic $p > 0$ is a perfectoid ring of characteristic $p$. 
                        \item A complete Tate ring of prime characteristic $p$ is perfectoid if and only if it is perfect. 
                    \end{enumerate}
                \end{lemma}
                    \begin{proof}
                        
                    \end{proof}
                \begin{corollary}[Tilts are perfect] \label{coro: perfectoid_tilts_are_perfect}
                    The tilt of a perfectoid ring is a perfectoid ring that is perfect.
                \end{corollary}
                
                \begin{lemma}[Algebraic extensions of perfectoid fields]
                    Finite extensions and algebraic closures of perfectoid fields are also perfectoid.  
                \end{lemma}
                    \begin{proof}
                        
                    \end{proof}
                \begin{theorem}[The Tilting Equivalence for perfectoid fields] \label{theorem: tilting_equivalence_for_perfectoid_fields}
                    For any perfectoid field $K$, there exists a canoncial adjoint-equivalence of \href{https://stacks.math.columbia.edu/tag/0BMQ}{\underline{Galois categories}}\footnote{Recall that the category of finite extensions of any field $k$ is precisely $\Sch_{/\Spec k}^{\fet}$ (which also happens to be $\Sch_{/\Spec k}^{\aff, \fet}$, since fields are of relative dimension $0$ over one another), the category of schemes finite \'etale over $\Spec k$ (see \cite[\href{https://stacks.math.columbia.edu/tag/0BL6}{Tag 0BL6}]{stacks} and \cite[\href{https://stacks.math.columbia.edu/tag/00U3}{Tag 00U3}]{stacks}).} as follows:
                        $$
                            \begin{tikzcd}
                            	{\{\text{Finite extensions of $K$}\}} & {\{\text{Finite extensions of $K^{\flat}$}\}}
                            	\arrow[""{name=0, anchor=center, inner sep=0}, "{(-)^{\flat}}"', shift right=2, from=1-1, to=1-2]
                            	\arrow[""{name=1, anchor=center, inner sep=0}, "\Witt"', shift right=2, from=1-2, to=1-1]
                            	\arrow["\dashv"{anchor=center, rotate=-90}, draw=none, from=1, to=0]
                            \end{tikzcd}
                        $$
                    wherein $\Witt$ denotes the Witt vector functor.
                \end{theorem}
                    \begin{proof}
                        
                    \end{proof}
                \begin{corollary}[Fontaine-Wintenberger for perfectoid fields]
                    The absolute Galois groups of any perfectoid fields is \textit{canonically} isomorphic to that of its tilt.
                \end{corollary}
                \begin{example}
                    The classical version of theorem \ref{theorem: tilting_equivalence_for_perfectoid_fields} (due to Fontaine and Wintenberger) asserts that there is a canonical isomorphism $\bfG_{ \Q_p(p^{1/p^{\infty}})^{\wedge} } \cong \bfG_{ \F_p(\!(t^{1/p^{\infty}})\!)^{\wedge} }$.
                \end{example}
                
            \subsubsection{Perfectoid spaces}
                \begin{definition}[Perfectoid Huber pairs] \label{def: perfectoid_huber_pairs}
                    A Huber pair $(R, R^+)$ is \textbf{perfectoid} if and only if $R$ is a perfectoid ring. 
                \end{definition}
                
                \begin{lemma}[Perfectoid Huber pairs are sheafy] \label{lemma: perfectoid_huber_pairs_are_sheafy}
                    Any perfectoid Huber pair $(R, R^+)$ is sheafy.
                \end{lemma}
                    \begin{proof}
                        
                    \end{proof}
                \begin{definition}[Perfectoid spaces] \label{def: perfectoid_spaces}
                    A \textbf{perfectoid space} is an adic space that is locally isomorphic to affinoid perfectoid spaces $\Spa(R, R^+)$. 
                    
                    The category of perfectoid spaces is denoted by $\Perfd$.
                \end{definition}
                \begin{proposition}[Pullbacks and products of perfectoid spaces] \label{prop: pullbacks_products_of_perfectoid_spaces}
                    Finite pullbacks and finite products exist in $\Perfd$, but there are no terminal objects (which means that products are not pullbacks). 
                \end{proposition}
                    \begin{proof}
                        
                    \end{proof}
                
                \begin{lemma}[Tilts of perfectoid Huber pairs] \label{lemma: tilts_of_perfectoid_huber_pairs}
                    Let $(R, R^+)$ be a perfectoid Huber pair and let $\varpi \in R^{\circ \circ}$ be a choice of pseudo-uniformiser. Then, $(R^+)^{\flat} \cong (R^+/\varpi)^{\flat}$. Furtheremore $R^{\circ \flat} \cong R^{\flat \circ}$.
                \end{lemma}
                    \begin{proof}
                        
                    \end{proof}
                \begin{convention}
                    Usually, one writes $R^{\flat +}$ in place of $(R^+)^{\flat}$.
                \end{convention}
                \begin{proposition}[Functoriality of tilting] \label{prop: tilting_functoriality}
                    \noindent
                    \begin{enumerate}
                        \item \textbf{(Relative affinoid tilting functors)} Fix an affinoid perfectoid space $S$. Then, there exists a functor $(-)^{\flat, \affd}_{/S}: \Perfd_{/S}^{\affd} \to \Perfd_{/S^{\flat}}^{\affd}$. This functor admits a left-adjoint, that being the relative Witt vector functor over $S$. 
                        \item \textbf{(The absolute affinoid tilting functor):} There also exists a functor $(-)^{\flat, \affd}: \Perfd^{\affd} \to \Perfd^{\affd}$.
                    \end{enumerate}
                \end{proposition}
                    \begin{proof}
                        
                    \end{proof}
                    
                \begin{lemma}[The Affinoid Tilting Equivalence] \label{lemma: the_affinoid_tilting_equivalence}
                    Let $S$ be an affinoid perfectoid space. Then, one has an equivalence:
                        $$\Perfd^{\affd, \fet}_{/S} \cong \Perfd^{\affd, \fet}_{/S^{\flat}}$$
                    (via the tilting functor of proposition \ref{prop: tilting_functoriality}) of \'etale sites fibred over the absolute \'etale site $\Perfd^{\affd, \fet}$ of affinoid perfectoid spaces and finite \'etale morphisms between them.
                \end{lemma}
                    \begin{proof}
                        
                    \end{proof}
                \begin{theorem}[The Tilting Equivalence] \label{theorem: the_tilting_equivalence}
                    Let $X$ be a perfectoid space. Then, one has an equivalence:
                        $$(\Perfd_{/X})_{\et} \cong (\Perfd_{/X^{\flat}})_{\et}$$
                    (via the tilting functor of proposition \ref{prop: tilting_functoriality}) of \textit{small} \'etale sites fibred over the absolute \'etale site $\Perfd_{\et}$.
                \end{theorem}
                    \begin{proof}
                        
                    \end{proof}
                \begin{corollary}[The Tilting Equivalence for fibred \'etale topoi]
                    Let $X$ be a perfectoid space. Then, one has an equivalence:
                        $$X_{\et} \cong X^{\flat}_{\et}$$
                    (via the tilting functor of proposition \ref{prop: tilting_functoriality}) of \textit{small} \'etale topoi fibred over the absolute \'etale site $\Perfd_{\et}$. 
                \end{corollary}
                    
        \subsection{Perfectoid spaces arising from arithemtic jet spaces}
                
            \subsubsection{Attaching perfectoid spaces to smooth schemes}
                
            \subsubsection{Attaching perfectoid spaces to formal schemes}
        
    \section{Diamonds} \label{section: diamonds}
        \subsection{Descent theory for perfectoid spaces}
            We introduce the $v$-topology (v for \say{valuation}) and the pro-\'etale topology on categories of perfectoid spaces in this subsection.
            \subsubsection{Pro-\'etale and \texorpdfstring{$v$}{}-sites of perfectoid spaces} \label{subsubsection: pro_etale_sites_and_v_sites_of_perfectoid_spaces}
                Having known how the pro-\'etale topology is supposed to behave (heuristically speaking, that is) through schemes, we now return to the world of perfectoid spaces. We shall also be introducing a new topology, called the $v$-topology; it is closely related to the pro-\'etale topology.
                
                The upshot is that both topologies are rather well-behaved, and play the role of the fppf and fpqc topologies - respectively - for perfectoid spaces. For instance, they are both subcanonical (cf. proposition \ref{prop: perfectoid_topologies_subcanonicity}), and sheaves in these topologies do in fact obey a (weaker) version of Serre's Criterion for Affineness (cf. proposition \ref{prop: serre_affineness_for_perfectoid_spaces}). As a matter of fact, one has the following hierachy of topologies on $\Perfd_{/X}$ in terms of increasing fineness:
                    $$\text{analytic topology} \subset \text{\'etale topology} \subset \text{pro-\'etale topology} \subset \text{$v$-topology}$$
                (cf. proposition \ref{prop: perfectoid_topology_hierachy}) which are all \say{nice} in the aforementioned sense, and thus serve as perfectoid analogues of the Zariski, \'etale, fppf, and fpqc topologies (in that order) on schemes. 
                
                \begin{convention}
                    Throughout, we shall be working with a base perfectoid space $X$ of characteristic $p > 0$. 
                \end{convention}
                
                \begin{definition}[Pro-\'etale and $v$-topologies] \label{def: pro_etale_topology_and_v_topology}
                    \noindent
                    \begin{enumerate}
                        \item \textbf{(Pro-\'etale covers):} A \textbf{pro-\'etale cover} on each object $Y \in \Perfd_{/X}$ is defined to be a family of pro-\'etale morphisms $\{Y_i \to Y\}_{i \in I}$ which is jointly surjective and jointly quasi-compact on each \textit{quasi-compact} open subset $V \subseteq Y$.
                        \item \textbf{($v$-covers):} A \textbf{$v$-cover} on each object $Y \in \Perfd_{/X}$ is defined to be a family of morphisms $\{Y_i \to Y\}_{i \in I}$ which is only required to be jointly surjective and jointly quasi-compact on each \textit{quasi-compact} open subset $V \subseteq Y$.
                    \end{enumerate}
                \end{definition}
                \begin{remark}[Big and small sites]
                    \noindent
                    \begin{enumerate}
                        \item \textbf{(Pro-\'etale sites):} The \textbf{big pro-\'etale site} is simply $\Perfd_{/X}$ endowed with the pro-\'etale coverage. By restricting down onto perfectoid spaces which are pro-\'etale over $X$, one obtains the \textbf{small pro-\'etale site}. 
                        \item \textbf{($v$-sites):} There is, however, no small $v$-site. This is because, by design, the $v$-coverage includes all perfectoid spaces over $X$ (unlike the pro-\'etale coverage, which includes only perfectoid spaces that are pro-\'etale over $X$).
                    \end{enumerate}
                \end{remark}
                
                Having defined the two topologies, let us now establish a few of their basic properties.
                
                The first is a basic relationship between the two new topologies and the previously defined topologies on perfectoid spaces, namely the canonical analytic/rational topology and the (finite) \'etale topology.
                \begin{proposition}[The hierachy of topologies] \label{prop: perfectoid_topology_hierachy}
                    One has the following hierachy of topologies on $\Perfd_{/X}$ in terms of increasing fineness:
                        $$\text{analytic topology} \subset \text{\'etale topology} \subset \text{pro-\'etale topology} \subset \text{$v$-topology}$$
                \end{proposition}
                    \begin{proof}
                        The $v$-topology being finer than the pro-\'etale topology is a trivial consequence of their constructions, as well as the pro-\'etale topology being finer than the \'etale topology. The \'etale topology is finer than the analytic topology because open immersions are \'etale.
                    \end{proof}
                We can now discuss subcanonicity, a very good property that all topologies on perfectoid spaces share.
                \begin{proposition}[Subcanonicity] \label{prop: perfectoid_topologies_subcanonicity}
                    The $v$-topology is subcanonical.
                \end{proposition}
                    \begin{proof}
                        
                    \end{proof}
                \begin{corollary} \label{coro: perfectoid_topologies_subcanonicity}
                    The analytic, \'etale, and pro-\'etale topologies are all subcanonical.
                \end{corollary}
                    \begin{proof}
                        Due to the $v$-topology being the finest and also its construction as the coverage on $\Perfd_{/X}$ which includes all morphisms $Y \to X$, it is the maximal subcanonical topology on $\Perfd_{/X}$. As a result, the analytic, \'etale, and pro-\'etale topologies are all subcanonical.
                    \end{proof}
                
                Let us now check that sheaf cohomology in the pro-\'etale and $v$-topologies indeed behave as expected. In particular, we shall be establishing a version of Serre's well-known criterion for affineness in terms of vanishing of higher cohomologies; note that this version is \say{weaker} than the original schematic version, since we do not have a good theory of coherent sheaves for adic spaces (or for that matter, perfectoid spaces) and hence can not make any kind of general assertion regarding \say{coherent $\calO_X$-modules}.
                \begin{lemma}[Structure sheaves are pro-\'etale and $v$-sheaves] \label{lemma: structure_sheaves_are_pro_etale_sheaves_and_v_sheaves}
                    The presheaves $\calO_X$ and $\calO_X^+$ are sheaves in the $v$-topology (and hence in the pro-\'etale topology \textit{a priori}).
                \end{lemma}
                    \begin{proof}
                        
                    \end{proof}
                \begin{proposition}[A weak Serre's Affineness Criterion for perfectoid spaces] \label{prop: serre_affineness_for_perfectoid_spaces}
                    \noindent
                    \begin{enumerate}
                        \item If $X$ is an affinoid perfectoid space (of characteristic $p > 0$) then:
                            $$\text{$H^i_{\proet}(X, \calO_X) \cong 0$ and $H^i_v(X, \calO_X) \cong 0$}$$
                        for all $i > 0$. Additionally, $H^i_{\proet}(X, \calO_X)$ and $H^i_v(X, \calO_X)$ are almost zero, also for all $i > 0$.
                        \item Conversely, if $X$ is a quasi-compact perfectoid space of characterisitic $p > 0$, if:
                            $$\text{$H^i_{\proet}(X, \calO_X) \cong 0$ and $H^i_v(X, \calO_X) \cong 0$}$$
                        for all $i > 0$, and if $H^i_{\proet}(X, \calO_X)$ and $H^i_v(X, \calO_X)$ are almost zero (also for all $i > 0$), then $X$ is affinoid. 
                    \end{enumerate}
                \end{proposition}
                    \begin{proof}
                        \noindent
                        \begin{enumerate}
                            \item 
                            \item
                        \end{enumerate}
                    \end{proof}
    
        \subsection{Perfectoid stacks}        
            \subsubsection{Diamonds}
                \begin{definition}[Diamonds] \label{def: diamonds} \index{Diamonds}
                    Let $p$ be a prime and let $X$ be a base perfectoid space of characteristic $p$. Also, let $\kappa$ be a regular cardinal. A ($\kappa$-small) \textbf{diamond} is thus a quotient stack $[Y/R]$ internal to $X_{\kappa\-\proet}$ (see definition \ref{def: quotient_stacks} for what this means), where $Y$ is representable.
                \end{definition}
                
                \begin{proposition}[The category of diamonds] \label{prop: the_category_of_diamonds}
                    Let $p$ be a prime and let $X$ be a base perfectoid space of characteristic $p$. Also, let $\kappa$ be a regular cardinal.
                        \begin{enumerate}
                            \item \textbf{(Morphisms of diamonds):} A morphism of diamonds on $X$ is simply a morphism of groupoids internal to $X_{\kappa\-\proet}$. A full subcategory of $X_{\kappa\-\proet}$ spanned by ($\kappa$-small) diamonds on $X$ thus naturally exists, and we shall denote it by $\Dia_{/X}^{< \kappa}$. 
                            \item \textbf{(Limits and colimits of diamonds):}
                                \begin{enumerate}
                                    \item \textbf{(Limits):} Finite products and finite pullbacks exist in $\Dia_{/X}^{< \kappa}$. These limts are taken as limits of pro-\'etale sheaves on $X$. 
                                    \item \textbf{(Colimits):} Surjections (i.e. regular epimorphisms) and finite corproducts exist in $\Dia_{/X}^{< \kappa}$.
                                \end{enumerate}
                        \end{enumerate}
                \end{proposition}
                    \begin{proof}
                        \noindent
                        \begin{enumerate}
                            \item \textbf{(Morphisms of diamonds):}
                            \item \textbf{(Limits and colimits of diamonds):}
                                \begin{enumerate}
                                    \item \textbf{(Limits):}
                                    \item \textbf{(Colimits):}
                                \end{enumerate}
                        \end{enumerate}
                    \end{proof}
                    
                Let us now concern ourselves with the intermediary notion of small $v$-stacks, which are essentially diamonds in the $v$-topology.
                \begin{definition}[Small $v$-stacks] \label{def: small_v_stacks} \index{$v$-stacks!small}
                    A \textbf{small $v$-sheaf} covered by a perfectoid space (viewed as a representable $v$-sheaf; cf. proposition \ref{prop: perfectoid_topologies_subcanonicity}).
                \end{definition}
                \begin{remark}[Diamonds are small $v$-sheaves] \label{remark: dimaonds_are_small_v_sheaves}
                    The $v$-topology is finer than the pro-\'etale topology (cf. proposition \ref{prop: perfectoid_topology_hierachy}) so $v$-sheaves are \textit{a priori} pro-\'etale sheaves. Additionally, diamonds are quotients of perfectoid spaces by pro-\'etale equivalence relations by definition, so small $v$-stacks are automatically diamonds.  
                    
                    Furthermore, because the category of diamonds admits finite coproducts, quasi-compact $v$-stacks are automatically small.
                \end{remark}
            
            \subsubsection{Cohomologically smooth morphisms and perfectoid Artin stacks}
                \begin{definition}[Perfectoid Artin stacks] \label{def: perfectoid_artin_stacks} \index{Perfectoid Artin stacks}
                    
                \end{definition}
                
            \subsubsection{\texorpdfstring{$\Bun_G$}{} and Beauville-Laszlo Uniformisation}
        
        \subsection{Smoothness of diamonds}
            \subsubsection{Formal smoothness}
            
            \subsubsection{A Jacobian criterion}