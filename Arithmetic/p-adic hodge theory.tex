\chapter{\texorpdfstring{$p$}{}-adic Hodge theory}
    \begin{abstract}
        
    \end{abstract}
    
    \minitoc
    
    \section{Motivic \texorpdfstring{$p$}{}-adic Hodge theory}
        \subsection{Introduction}
            The subject that is nowadays known as \say{Hodge theory} is - as least as far as its motivic aspect is concerned - essentially the study of cohomology theories (\'etale, de Rham, crystalline, etc.) as topological invariants of (algebraic and analytic) varieties along with the various relationships between said cohomology theories. As it stands currently, Hodge theory consists of two parallel subfields, namely the theories over the complex numbers $\bbC$ and that over non-archimedean fields such as $\Q_p$, bot of which could be seen as probational steps towards a more grander all-unifying incarnation of Hodge theory over number fields, which for various technical reasons might be the key technical toolbox for tackling some of the deepest networks of conjectures in modern algebraic number theory, like the Global Langlands Correspondence over $\Q$. For algebraic varieties over the complex numbers $\bbC$ (or more precisely, their associated complex manifolds), the story is relatively simple: due to there being only two interesting Grothendieck topologies, namely the \'etale topology on smooth varieties and the complex-analytic topology on their corresponding complex manifolds, there exists only one fundamental comparison theorem, that being the Riemann-Hilbert Correspondence, which more-or-less relates \'etale cohomology of local systems and de Rham cohomology of their associated complex manifolds. Over non-archimedean local fields such as $\Q_p$ or $\F_p(\!(t)\!)$, however, the story is much more complicated; here, there is a plethora of cohomology theories, and since we would like for there to exist pairwise comparison theorems amongst them, we will have to look beyond attempting to simply compare the various Grothendieck topologies. Moreoever, due to $p$-adic Hodge theory being composed of two different facades, namely the theories over mixed characteristic fields such as $\Q_p$ and equicharacteristic fields such as $\F_p(\!(t)\!)$, we will also have to worry about how certain cohomology theories behave poorly over positive characteristics (e.g. de Rham cohomology) and as such require replacements (e.g. crystalline cohomology), which might be full of their own stock of shortcomings and technical difficulties. Luckily, geometry over non-archimedean fields (and particularly those of positive characteristics) is a much richer theory than its complex counterpart and as such, there are many intermediary objects that one might try to make use of in order to \say{interpolate} the various cohomoloy theories and thereby comparing them indirectly; this, incidentally, is an approach to $p$-adic Hodge theory that has now gained mainstream popularity.
    
    \section{Robba rings and \texorpdfstring{$\varphi$}{}-modules}
        \subsection{Robba rings}
        
        \subsection{\texorpdfstring{$\varphi$}{}-modules}
    
        \subsection{\texorpdfstring{$(\varphi, \Gamma)$}{}-modules}
            \subsubsection{Pseudo-coherent sheaves}
            
            \subsubsection{The homological algebra of \texorpdfstring{$(\varphi, \Gamma)$}{}-modules}
            
    \subsection{The Ax-Tate-Sen Theorem}
    
        \subsection{The Hodge-Tate Decomposition and the \'etale-de Rham comparison}
            \subsubsection{The Hodge-Tate Decomposition via perfectoid spaces}
                \paragraph{Scholze's de Rham period sheaf}
                    \begin{definition}[de Rham period sheaves] \label{def: de_rham_period_sheaves}
                        Let $K$ be a perfectoid field of mixed characteristic $(0, p)$ and let $X$ be a perfectoid space over $\Spa K$. Additionally, recall that the integral subsheaf $\calO_{X^{\flat}}^+$ is \textit{a priori} perfect (over characteristic $p$). We are interested in the following commutative ring objects of the pro-\'etale topos $X_{\proet}$:
                            \begin{enumerate}
                                \item \textbf{(Fontaine's infinitesimal period rings):} There is first of all the so-called \textbf{infinitesimal period ring of Fontaine}, usually denoted by $\A_{\inf}$, and is defined to be the ring of Witt vectors $\Witt(\calO_{X^{\flat}}^+)$. One also would usually be interested in its rational localisation, namely $\B_{\inf} := \A_{\inf}[1/p]$. 
                                
                                We care also about the canonical map $\theta: \A_{\inf} \to \calO_X^+$, which extends naturally to a map $\Theta: \B_{\inf} \to \calO_X$.
                                \item \textbf{(Scholze's de Rham period rings):} Scholze, in \cite[Definition 6.1]{scholze2012padic} then defined his \textbf{de Rham period ring} as the formal completion $(\B_{\inf}, \ker \Theta)^{\wedge}$; we will denote this period ring by $\B_{\dR}^+$. There exists also a rational localisation of this period ring, though this is a less than trivial fact.
                            \end{enumerate}
                    \end{definition}
                    
                    \begin{proposition}[$\theta$ is surjective] \label{prop: theta_is_surjective}
                        Let $K$ be a perfectoid field of characteristic $0$ and let $X = \Spa(R, R^+)$ be an affinoid perfectoid space over $\Spa K$.
                        \begin{enumerate}
                            \item The canonical map $\theta: \A_{\inf} \to R^+$ is a surjective continuous ring homomorphism.
                            \item $\ker \Theta$ is a non-zero principal ideal of $\B_{\inf}$.
                        \end{enumerate}
                    \end{proposition}
                        \begin{proof}
                            \noindent
                            \begin{enumerate}
                                \item \textbf{(Surjectivity of $\theta$):} 
                                \item \textbf{(Principality of $\ker \Theta$):} First of all, because localisation is a colimit, the map $\Theta: \B_{\inf} \to \calO_X$ must also be surjective as a consequence of $\theta: \A_{\inf} \to \calO_X^+$ being surjective; this ensures that the ideal $\ker \Theta$ is not trivial.
                            \end{enumerate}
                        \end{proof}
                    \begin{corollary}[Rational de Rham period rings] \label{coro: rational_de_rham_period_rings}
                        \noindent
                        \begin{enumerate}
                            \item Proposition \ref{prop: theta_is_surjective} applies also to non-affinoid perfectoid spaces.
                            \item Additionally, not only is the completion $\B_{\dR}^+ := (\B_{\inf}, \ker \Theta)^{\wedge}$ adic, but also, it admits a rational localisation: if $t \in \ker \Theta$ is any generator, then we can define $\B_{\dR} := \B_{\dR}^+[1/t]$. 
                        \end{enumerate}
                    \end{corollary}
                    
                    \begin{remark}
                        
                    \end{remark}
                
                \paragraph{The Hodge-Tate Decomposition}
            
            \subsubsection{The \'etale-de Rham comparison}
        
        \subsection{The \'etale-crystalline comparison}
    
    \section{Period sheaves and \texorpdfstring{$\varphi$}{}-modules over them}
        \subsection{Perfect period sheaves}
        
        \subsection{Imperfect period sheaves}