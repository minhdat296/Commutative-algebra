\chapter{\'Etale cohomology for schemes} \label{chapter: etale_cohomology_1}
    \begin{abstract}
        Galois theory via algebraic topology ? Game on!
    \end{abstract}
    
    \minitoc
    
    \section{Pr\'elude: Primes in integral extensions}
        \subsection{Behaviours of prime ideals in integral extensions}
            \subsubsection{Finite and integral extensions}
                \begin{definition}[Integral extensions] \label{def: integral_extensions} \index{Integral! extensions} \index{Integral! elements}
                    \noindent
                    \begin{enumerate}
                        \item \textbf{(Integral elements):} Let $A$ be a subring of a commutative ring $B$ (i.e. let their exist monic ring homomorphisms from $A$ to $B$). An element $b \in B$ will be called \textbf{integral} over $A$ if and only if $A[b]$ is a finitely generated commutative $A$-algebra; otherwise, it is called \textbf{transcendental}. When $A$ is a field, what one recovers are the notions of algebraic and transcendental elements (for instance, $\sqrt{2}$ is integral over $\Z$, as it is algebraic over $\Q$, whereas a formal variable $x$ is neither). 
                        \item \textbf{(Integral extensions):} A homomorphism between commutative rings $A \to B$ will be called an integral extension if all elements of $B$ are integral over $B$. Alternatively (and perhaps less confusingly), one may view an integral extension of a commutative ring $A$ as a tensor product $\bigotimes_{i \in I} A[b_i]$ wherein $\{b_i\}_{i \in I}$ is a (possibly infinite) set of elements that are integral over $A$; note that one has the following canonical isomorphism of commutative $A$-algebras:
                            $$\bigotimes_{i \in I} A[b_i] \cong A\left[\{b_i\}_{i \in I}\right]$$
                        thanks to the fact that left-adjoints (the polynomial ring free construction) commutes with colimits (tensor products of commutative algebras). Additionally, integral extensions are trivially injective ring homomorphisms.
                        \item \textbf{(Integral closures):} Let $\varphi: A \to B$ be a homomorphism of commutative rings. Then, the integral closure of $A$ inside $B$ (denoted by $\overline{A}$, $\overline{A_B}$, or $\overline{A_{\varphi}}$) is the subset of $B$ consisting of \textit{all} elements that are integral over $A$. It is not hard to show that integral closures are actually subrings of the codomains, and with this in mind, one can see that integral closures may be viewed as maximal integral extensions inside given commutative rings; this description can be succinctly summed up by the following expression:
                            $$\overline{A_{\varphi}} \cong \bigotimes_{\underset{\text{$b$ integral over $\im \varphi$}}{b \in B}} A[b]$$
                    \end{enumerate}
                \end{definition}
                \begin{example}[The ring of integers of a number field] \label{example: ring_of_integers}
                    \noindent
                    \begin{enumerate}
                        \item \textbf{(The global case):} Before we try to give a description of rings of integers inside global fields, let us fix a definition. To us, a global field is either a finite (hence \textit{a priori} algebraic) extension of $\Q$ or of $\F_p(t)$, the field of Laurent series with coefficients coming from the finite field of prime order $p$. With this definition in mind, let us then define the ring of integers of a global field $F$ as the maximal commutative ring (in terms of cardinality) that is integrally closed inside $F$. For example, $\Z$ is the ring of integers of $\Q$, but $\Z$ is not of $\R$ (not that $\R$ is a global field, nor can we even define the ring of integers of $\R$ anyway). This definition, while conceptually intuitive, is not very practical. That is because it is not entirely clear how one might trickle from a given global field down onto its largest subring that is integrally closed. Thus, one can define the ring of integers $\scrO_F$ of a global field $F$ alternatively as the set of all elements of $F$ that are integral over $\Z$ (in the event that $\chara F = 0$) or over $\F_p[t]$ (if $\chara F = p$, for some prime $p$). Per this definition, rings of integers are automatically closed inside their corresponding global fields. Furthermore, all global fields are equal to the field of fractions of their rings of integers.
                        \item \textbf{(The local case):} As above, let us first try to agree upon a notion of local fields: an \textit{archimedean} local field is a finite extension of $\R$ (so actually, just $\R$ and $\bbC$), and a \textit{non-archimedean} local field is either a finite extension of $\Q_p$ (i.e. a $p$-adic number field) - which we note to be of mixed characteristic $(0,p)$ - or a finite extension of $\F_p(\!(t))\!$ - which we note to be of equicharacteristic $(p, p)$ - i.e. the field of formal Laurent series over the finite field of order $p$. One can do some work to see that given a local field $K$, one can define a suitable sort of \say{absolute value} $|-|$ on it, and with respect to such an absolute value, one obtains either an archimedean metric topology or a non-archimedean ultrametric topology (hence the names). Then, consider the \textit{closed} unit ball inside $K$, i.e. the set:
                            $$\scrO_K := \{x \in K \mid |x| \leq 1\}$$
                        (for instance, $\Z_p$ and $\F_p[\![t]\!]$ are the closed unit balls inside $\Q_p$ and $\F_p(\!(t)\!)$ respectively, and $[-1, 1]$ is the the unit ball inside $\R$). In the non-archimedean case, this turns out to be a subring of $K$, which we dub the ring of integers of $K$. Interestingly, the ring of integers of a non-archimedean local field is integrally closed and one can show this by first showing that the \textit{open} unit ball inside $K$, i.e. the set:
                            $$\m_K := \{x \in K \mid |x| < 1\}$$
                        is the (necessarily unique) maximal ideal of $\scrO_K$; then, 
                        \\
                        Of course, one could also define the ring of integers of a non-archimedean local field $K$ as the set of all elements in $K$ that are integral over either $\Z_p$ or $\F_p[\![t]\!]$ (corresponding to $\chara K = 0$ and $\chara K = p$ respectively) and then show that such elements would have their absolute values bounded above by $1$. 
                    \end{enumerate}
                    
                    One interesting object that can be built out of global fields, local fields, along with rings of integers thereof are rings of a\`eles of global fields: the ring of ad\`eles of a global field $F$ is defined to be the following so-called \textbf{restricted product}:
                        $$\A_F := \hat{\prod_{v \in \Spec \scrO_F}} F_v := \underset{V \in \calP^{\fin}_{\Spec \scrO_F}}{\colim} \left(\prod_{v \in V} F_v \x \prod_{v \in \Spec \scrO_F \setminus V} \scrO_{F, v}\right)$$
                    wherein $\calP^{\fin}_{\Spec \scrO_F}$ is the poset of \textit{finite} subsets of $\Spec \scrO_F$, and for each place $v \in \Spec \scrO_F$, one writes $F_v$ for the $v$-adic completion of $F$ and $\scrO_{F, v}$ for the ring of integers of $F_v$. For instance, the ring of ad\`eles of $\Q$ is:
                        $$\A_{\Q} := \hat{\prod_{p \in \Spec \Z}} \Q_p := \underset{V \in \calP^{\fin}_{\Spec \Z}}{\colim} \left(\prod_{p \in V} \Q_p \x \prod_{q \in \Spec \scrO_F \setminus V} \Z_q\right)$$
                \end{example}
                \begin{example}[More instances of integrality]
                    \noindent
                    \begin{enumerate}
                        \item \textbf{(Dedekind domains):} Any Dedekind domain is integrally closed in its field of fractions. However, this is not the case for general integral domains, i.e. there are integral domains which are not 
                        \item \textbf{(Algebraic closures):} Because algebraic extensions are special cases of integral extensions, algebraic closures are nothing but instances of integral closures.
                        \item \textbf{(The ring of integers of a number field):} We have seen in example \ref{example: ring_of_integers} that given any number field $E$, the corresponding ring of integers $\scrO_E$ is its own integral closure in $E$. Let us now examine a few concrete instances of this phenomenon:
                            \begin{enumerate}
                                \item \textbf{(The Gaussian integers):} The ring of integers 
                                \item \textbf{(Quadratic extensions):}
                                \item \textbf{(Cyclotomic extensions):}
                                \item \textbf{(Algebraic integers):} The integral closure of $\Z$ inside the field $\overline{\Q}$ of algebraic numbers is the ring of algebraic integers (or in order to avoid tautological statements, the ring of integers of $\overline{\Q}$); one may draw the following diagram to understand the relationship between this example and that of $\Z$ inside $\Q$:
                                    $$
                                        \begin{tikzcd}
                                        	\overline{\Q} & {\scrO_{\overline{\Q}}} \\
                                        	\Q & \Z
                                        	\arrow[no head, from=2-1, to=1-1]
                                        	\arrow[no head, from=2-2, to=1-2]
                                        	\arrow[no head, from=1-1, to=1-2]
                                        	\arrow[no head, from=2-1, to=2-2]
                                        \end{tikzcd}
                                    $$
                            \end{enumerate}
                    \end{enumerate}
                \end{example}
                \begin{remark}[Finiteness and integrality] \label{remark: finite_implies_integral}
                    As it is the case with fields, finite extensions of commutative rings are integral, but the converse is not necessarily true. For instance, the extension $\Z[\{\sqrt{p}\}_{(p) \in \Spec \Z}]$ is certainly integral inside $\Q$, but definitely not finite. 
                \end{remark}
                \begin{convention}
                    From now on, integral extensions will be denoted like how field extensions are, i.e. as \say{quotients}.
                \end{convention}
                
                \begin{proposition}[Equivalent definitions of integrality]
                    Let $A$ be a subring of a commutative ring $B$ and let $b$ be an element of $B$. Then:
                        \begin{enumerate}
                            \item $b$ is integral over $A$ if and only if it is a root of a polynomial in $A[x]$. 
                            \item There exists a faithful $A[b]$-module that is finitely generated over $A$. 
                        \end{enumerate}
                \end{proposition}
                    \begin{proof}
                                    
                    \end{proof}
                
                \begin{proposition}
                    Compositions of integral extensions are themselves integral extensions. 
                \end{proposition}
                    \begin{proof}
                                    
                    \end{proof}
                
                \begin{proposition}[Integrality and localisations]
                    Let $A$ be a subring of a commutative ring $B$, let $\overline{A}$ denote the integral closure of $A$ inside $B$, and let $S$ be a multiplicative subset of $A$. Then, the integral closure of $S^{-1}A$ inside $S^{-1}$ is just $S^{-1}\overline{A}$. 
                \end{proposition}
                    \begin{proof}
                                    
                    \end{proof}
                
            \subsubsection{Lying Over, Going Up, and Going Down}
                \begin{definition}[Primes lying over one another]
                    Let $\pi: \Spec B \to \Spec A$ be a morphism of affine schemes. Then, a prime $\q \in |\Spec B|$ is said to lie over a prime $\p \in |\Spec A|$ if:
                        $$\q \in |\pi|^{-1}(\p)$$
                \end{definition}
            
                \begin{definition}[Going Up and Going Down]
                    Let $\varphi: A \to B$ be a homomorphism between commutative rings.
                        \begin{enumerate}
                            \item \textbf{(Going Up):} $\varphi$ is said to satisfy \textbf{Going Up} if for every pair of prime ideals $\p \subset \p'$ of $A$ and for every prime $\q$ lying over $\p$, there exists a prime ideal $\q'$ above $\p'$ such that $\q' \supset \q$.  
                            \item \textbf{(Going Down):} $\varphi$ is said to satisfy \textbf{Going Down} if for every pair of prime ideals $\p \subset \p'$ of $A$ and for every prime $\q'$ lying over $\p'$, there exists a prime ideal $\q$ above $\p$ such that $\q \subset \q'$. 
                        \end{enumerate}
                \end{definition}
                
                \begin{proposition}[Going Up and Going Down criteria] \label{prop: going_up_and_down_criteria}
                    \noindent
                    \begin{enumerate}
                        \item Integral (and hence finite; see remark \ref{remark: finite_implies_integral} for details) extensions satisfy Going Up.
                        \item Quotient maps satisfy Going Up.
                        \item Flat ring maps (and hence localisations; see \cite{stacks}, \href{https://stacks.math.columbia.edu/tag/00HT}{\underline{lemma 10.39.18}}; actually, we can prove this easily using the fact that left-adjoints commute with colimits) satisfy Going Down. 
                    \end{enumerate}
                \end{proposition}
                    \begin{proof}
                         
                    \end{proof}
                
            \subsubsection{Integral schemes, schemes of finite type, and normal schemes}
        
        \subsection{Ramification theory}
            \subsubsection{Extension of Dedekind domains and the splitting of primes in Galois extensions}
                \begin{lemma}[A ring of integers is a Dedekind domain]
                    Let $E$ be a number field (either local or global). Then, its ring of integers is a Dedekind domain. 
                \end{lemma}
                    \begin{proof}
                         
                    \end{proof}
                    
                \begin{theorem}[Extensions of Dedekind domains]
                    Let $L/K$ be a finite extension of fields. Then, there is an induced integral extension $\scrO_L/\scrO_K$ of Dedekind domains. In other words, $\scrO_L$ is the integral closure of $\scrO_K$ in $L$. 
                \end{theorem}
                    \begin{proof}
                        
                    \end{proof}
                \begin{corollary}[Splitting of primes in finite extensions] \label{coro: prime_splitting_finite_extensions}
                    Let $L/K$ be a finite field extension and let $\p$ be a prime ideal of $\scrO_K$. Then:
                        \begin{enumerate}
                            \item \textbf{(Primes splitting):} The ideal $\p\scrO_L$ factors uniquely into a \say{products} of primes $\q_1, ...\q_n$ of $\scrO_L$:
                                $$\p\scrO_L = \q_1^{e_1}...\q_n^{e_n}$$
                            (with the natural numbers $e_i$ being multiplicities, usually known as \textbf{ramification indices}).
                            \item \textbf{(Lying Over):} Thanks to the above unique factorisation 
                        \end{enumerate}
                \end{corollary}
                    \begin{proof}
                        
                    \end{proof}
                    
                \begin{definition}[Ramification indices and inertial degrees] \label{def: ramification_indices}
                    Let $L/K$ be a field extension of finite degree, and let $\p$ be a prime ideal of $\scrO_K$. Also, if there are no risks of confusion, let us write $\p$ instead of $\p\scrO_L$ from now on for the prime of $\scrO_L$ generated by $\p$.
                        \begin{enumerate}
                            \item \textbf{(Ramification indices):} The exponents of the prime ideals in the factorisation of $\p$ are called the \textbf{ramification indices} of said prime factors. Primes with ramification index $1$ are put into two further subclasses:
                                \begin{enumerate}
                                    \item \textbf{(Splitting primes):} Let $\p = \q_1^{e_1}...\q_n^{e_n}$. If $e_i = 1$ and $n > 1$, then we will say that $\p$ splits in $\scrO_L$.
                                    \item \textbf{(Inert primes):} However, if $e_i = 1$ for all $1 \leq i \leq n$ and $n = 1$ also, then we will say that $\p$ remains \textbf{inert} in $\scrO_L$.
                                    \item \textbf{(Ramifying primes):} Otherwise (i.e. if $e_i > 1$ for all $1 \leq i \leq n$ and $n Geq 1$), we will say that $\p$ \textbf{ramifies}, or that $\p$ is a \textbf{place of ramification}.
                                    \item If $e_i > 1$ for all $1 \leq i \leq n$ and $n = 1$ then we will say that $\p$ is a non-splitting prime that ramifies with index $e = e_i = e_1$.
                                \end{enumerate}
                            \item \textbf{(Inertial degrees):} Because $\p$ factors uniquely in $\scrO_L$ - say as $\q_1^{e_1}...\q_n^{e_n}$) - all the primes $\q_i$ are divisors of $\p$. Thus, any prime ideal $\p$ in a base Dedekind domain ($\scrO_K$ in this case) along with an integral extension of Dedekind domains (which is the canonical map $\scrO_K \to \scrO_L$ here) has prime divisors $\q_i$. To such prime divisors, there are associated \textbf{inertial degrees} $f_i$ that we are going to define as the degree of the field extension $(\scrO_L/\q_i)/(\scrO_K/\p)$, i.e.:
                                $$f_i := [\scrO_L/\q_i : \scrO_K/\p]$$
                            Often, we will just say that $f_i$ is the inertial degree of $\q_i$ over $\p$.
                        \end{enumerate}
                \end{definition}
                
                \begin{lemma}[Prime divisors lie over]
                    Let $L/K$ be a finite extension and let $\p$ be a prime of $\scrO_K$. Then, the following are equivalent:
                        \begin{enumerate}
                            \item A prime ideal $\q$ of $\scrO_L$ divides $\p\scrO_L$.
                            \item A prime ideal $\q$ of $\scrO_L$ contains $\p\scrO_L$.
                            \item The intersection of a prime ideal $\q$ of $\scrO_L$ with the subring $\scrO_K$ of $\scrO_L$ is $\p\scrO_L$.
                            \item The intersection of a prime ideal $\q$ of $\scrO_L$ with the subfield $K$ of $L$ is $\p\scrO_L$.
                        \end{enumerate}
                \end{lemma}
                    \begin{proof}
                        
                    \end{proof}
                    
                \begin{theorem}[The fundamental identity of ramification theory]
                    Let $L/K$ be a finite and separable field extension of degree $n$ and let $\p$ be a prime ideal of $\scrO_K$ that factors into primes of $\scrO_L$ as follows:
                        $$\p = \q_1^{e_1}...\q_n^{e_n}$$
                    and for each index $i$, let $f_i$ denote the inertial degree 
                \end{theorem}
                    \begin{proof}
                        
                    \end{proof}
                \begin{corollary}[An application to quadratic fields (\cite{christian_511_project}, proposition 2.15)] \label{coro: ramification_quadratic_fields}
                    Let $p$ be a prime, let $d$ be a square-free integer, and let $\Delta$ denote the discriminant of the quadratic field $\Q(\sqrt{d})$, and recall that this quantity is given by:
                        $$
                            \Delta := 
                            \begin{cases}
                                \text{$d$ if $d \equiv 1 \pmod{4}$}
                                \\
                                \text{$4d$ otherwise}
                            \end{cases}
                        $$
                    (essentially, $\Delta$ is the discriminant of the quadratic polynomial $x^2 - d$; the point is that the definition above comes from an application of the all-too-familiar quadratic formula to this polynomial). Also, let $\scrO_{\Delta} := \Z[\sqrt{d}]$ denote the ring of integers of $\Q(\sqrt{d})$. Then:
                        \begin{enumerate}
                            \item If $p \mid \Delta$ (i.e. if $p \mid d$ or $p = 2$) then $(p) \in \Spec \Z$ will be a non-splitting place of ramification of $\Q$ with ramification index $2$, i.e. there is a prime $\q$ of $\scrO_{\Delta}$ such that:
                                $$(p)\scrO_{\Delta} = \q^2$$
                            Also:
                                $$\scrO_{\Delta}/(p)\scrO_{\Delta} \cong \F_p[x]/(x^2)$$
                            \item When $p \ndiv d$, it is necessarily true that $p 
                            \not = 2$. We then obtain two subcases:
                                \begin{enumerate}
                                    \item If $\left(\frac{\Delta}{p}\right) = 1$ (see \href{https://ncatlab.org/nlab/show/quadratic+reciprocity+law}{\underline{here}} if a reminder of the definition of Legendre symbols is called for; note that the symbol $\left(\frac{\Delta}{p}\right)$ actually makes sense because $p$ has already been established to be an odd prime) then $(p)$ splits into two distinct prime factors $\q_1, \q_2$ inside $\scrO_{\Delta}$. Furthermore:
                                        $$\scrO_{\Delta}/\q_1\q_2 \cong \F_p \x \F_p$$
                                    \item If $\left(\frac{\Delta}{p}\right) = -1$ then $(p)$ remains inert in $\scrO_{\Delta}$, and:
                                        $$\scrO_{\Delta}/(p)\scrO_{\Delta} \cong \F_{p^2}$$
                                \end{enumerate}
                        \end{enumerate}
                \end{corollary}
                    \begin{proof}
                        
                    \end{proof}
                    
                \begin{convention}[Primes and places] \label{conv: places_and_primes} \index{Primes as places}
                    We probably should have mentioned this earlier, but since we have already got to this point without touching on it very often, this might be as good a time as any to discuss the terminologies \say{prime}, \say{prime ideals}, and \say{places}. Historically speaking, given a local field $K$, a \say{place} of $K$ is a valuation, and thanks to Ostrowski's theorem (\cite{koblitz_p_adic}, theorem I.1, pp.3) - which asserts that every non-trivial non-archimedean valuation on a local number field (i.e a $p$-adic field) is equivalent to the canonical $p$-adic valuation - equivalence classes thereof in the case where $K$ is a local number field. Thus, in the context of global number fields, a \say{place} is nothing but a prime ideal in the ring of integers, and hence it makes senses to interchange the words there. Furthermore, the terminology \say{place} helps us make sense of number-theoretic facts geometrically, as prime ideals are precisely points of affine schemes (spectra of rings of integers in this situation).
                    
                    As an example, consider $\Q$, a global number field in which a place is just a prime ideal of $\Z$, i.e. either $(0)$ or $(p)$, for some prime $p$. Below is an illustration wherein, by completing $\Q$ along a \textit{non-zero} prime $p$ of $\Z$, one gets a local number field $\Q_p$ that is complete with respect to the attached $p$-adic valuation, whereas by performing formal completion along $(0)$, one recovers $\Q$, which can be thought of as the corresponding \say{generic} number field due to its global nature:
                        \begin{figure}[H]
                            \centering
                            \includegraphics[width=\linewidth,height=\textheight,keepaspectratio]{Figures/places of Spec Z.png}
                            \caption{Places of $\Q$ (note that $(0)$ should be viewed as the generic place-at-infinity).}
                            \label{fig: places_of_Q}
                        \end{figure}
                \end{convention}
                
                \begin{proposition}[Local-global compatibility]
                
                \end{proposition}
                    \begin{proof}
                        
                    \end{proof}
                
                \begin{example}[Places of ramification of more general arithmetic schemes]
                    Below are examples of arithmetic schemes, i.e. schemes over $\Spec \Z$, on which there are primes lying over those of $\Spec \Z$ where ramifications take place.
                        \begin{enumerate}
                            \item \textbf{(The prime spectrum of the ring of integers of a number field):} Consider the following setting:
                                $$
                                    \begin{tikzcd}
                                    	{\Q(\sqrt{d})} & {\Z[\sqrt{d}]} \\
                                    	\Q & \Z
                                    	\arrow[no head, from=2-1, to=1-1]
                                    	\arrow[no head, from=2-2, to=1-2]
                                    	\arrow[no head, from=1-1, to=1-2]
                                    	\arrow[no head, from=2-1, to=2-2]
                                    \end{tikzcd}
                                $$
                            wherein we are consider the quadratic extension $\Q(\sqrt{d})/\Q$ along with the induced integral extension of Dedekind domains $\Z[\sqrt{d}]/\Z$; particularly, let us pick a prime ideal $\p$ of $\Z$ (i.e. either the zero ideal or an ideal generated by a prime number $p$). Now, do the primes of $\Z[\sqrt{d}]$ indeed lie over those of $\Z$ ? First of all, we will need to see what the prime ideals of $\Z[\sqrt{d}]$ actually look like, and luckily, we can apply corollary \ref{coro: prime_splitting_finite_extensions} to do this:
                                \begin{enumerate}
                                    \item Of course, the ideal $(0)\Z[\sqrt{d}]$ is just the zero ideal, and thus admits the trivial factorisation $(0) = (0)$. Because of this, let us only consider non-zero prime ideals of $\Z$ from now on.
                                    \item Then, there are three cases, according to corollary \ref{coro: ramification_quadratic_fields}:
                                        \begin{enumerate}
                                            \item 
                                            \item
                                            \item
                                        \end{enumerate}
                                \end{enumerate}
                            \item \textbf{(A conic over $\Spec \Z$):}
                            \item \textbf{(The line with double origin over $\Spec \Z$):}
                        \end{enumerate}
                    \end{example}
                    
            \subsubsection{Unramfied morphisms}
                \begin{definition}[Unramified morphisms] \label{def: unramified_morphisms}
                    A ring map is said to be \textbf{unramified} if it is of finite type and if the corresponding module of K\"ahler differentials is zero. 
                \end{definition}
                \begin{example}
                    \noindent
                    \begin{itemize}
                        \item \textbf{(\'Etale morphisms):} \'Etale morphisms (cf. definition \ref{def: etale_morphisms}) are trivially unramified. The converse statement is not necessarily true, since there are morphisms of finite type that are not of finite presentation, which is a necessary condition for \'etale-ness (or even just smoothness for that matter).
                        
                        As a concrete example, take any unramified finite extension $E/F$ (cf. definition \ref{def: ramification_indices}), and in the event that these fields have rings of integers (cf. example \ref{example: ring_of_integers}), the induced map:
                            $$\Spec E^{\circ} \to \Spec F^{\circ}$$
                        will also be unramified (in fact, they will be \'etale, since finite extensions come from finite presentations). As a result, for some fixed base scheme $S$, one can think of unramified morphisms $X \to S$ as coming from $S$-schemes $X$ such that preimages of points of $S$ consists merely of a single point. 
                        \item \textbf{(A counter example: non-\'etale smooth morphisms):} A smooth morphism of non-zero relative dimension can not be unramified, since its associated module of K\"ahler differential is not zero. 
                    \end{itemize}
                \end{example}
                
                \begin{proposition}[Locality of ramification] \label{prop: locality_of_ramification}
                    We say that 
                \end{proposition}
                
            \subsubsection{Extensions of discrete valuation rings}
                
            
            \subsubsection{Galois extensions and ramification}
    
    \section{\'Etale cohomology for schemes: The Chosen One}
        \subsection{\'Etale morphisms} \label{subsection: etale_morphisms}
            \begin{definition}[\'Etale morphisms] \label{def: etale_morphisms} \index{\'Etale-ness}
                An \'etale ring map is a smooth ring map whose cotangent complex is quasi-isomorphic to the zero complex. Equivalently, a ring homomorphism is \'etale if and only if it is smooth and of relative dimension $0$ (see proposition \ref{prop: smoothness_implies_almost_finiteness_of_cotangent_complex} for an explanation). 
            \end{definition}
            
            \begin{proposition}[The separable extension criterion for \'etale-ness] \label{prop: separable_criterion_for_etaleness} \index{\'Etale-ness! Separability Criterion}
                Let $\pi: F \to B$ be a ring map of finite presentation. Then, the following statements are equivalent:
                    \begin{enumerate}
                        \item $\pi$ is \'etale.
                        \item The field extension $\kappa_{\pi^{-1}(\q)}/\kappa_{\q}$ of the residue field at $\pi^{-1}(\q) \in |\Spec k|$ over that at $\q \in |\Spec B|$ is separable (and necessarily finite, as \'etale maps are \textit{a priori} of finite presentation) for all $\q \in |\Spec B|$. Note that the extension $\kappa_{\pi^{-1}(\q)}/\kappa_{\q}$ exists thanks to the fact that the stalk map:
                            $$\calO_{\Spec F, \pi^{-1}(\q)} \to \calO_{\Spec B, \q}$$
                        which is actually just: 
                            $$F_{\pi^{-1}(\q)} \to B_{\q}$$
                        is required to be a local homomorphism between local rings.
                        \item If $F$ is a field, then $B$, when viewed as an $F$-vector space, can be written as a (necessarily finite) direct sum of (necessarily finite) separable extensions of $F$ (of course, this assertion is only equivalent to the other two in the event that they are considered with $F$ a field as well).
                    \end{enumerate}
            \end{proposition}
                \begin{proof}
                    This is trivial when $\chara \kappa_{\q} = 0$, as every extension in characteristic $0$ is \textit{a priori} separable (for a proof, please consult \cite[\href{https://stacks.math.columbia.edu/tag/030Q}{Tag 030Q}]{stacks} and \cite[\href{https://stacks.math.columbia.edu/tag/030N}{Tag 030N}]{stacks}). Thus, let us assume that:
                        $$\chara \kappa_{\q} = p$$
                    for some prime $p$. Also, note that the extension $\kappa_{\pi^{-1}(\q)}/\kappa_{\q}$ decomposes into a separable part $\kappa_{\q}^{\sep}/\kappa_{\q}$ (with $\kappa_{\q}^{\sep}$ is the separable closure of $\kappa_{\q}$ inside $\kappa_{\pi^{-1}(\q)}$) and a purely inseparable part $\kappa_{\pi^{-1}(\q)}/\kappa_{\q}^{\sep}$ in the following manner \cite[\href{https://stacks.math.columbia.edu/tag/030K}{Tag 030K}]{stacks}:
                        $$
                            \begin{tikzcd}
                            	{\kappa_{\pi^{-1}(\q)}} \\
                            	{\kappa_{\q}^{\sep}} \\
                            	{\kappa_{\q}}
                            	\arrow[no head, from=3-1, to=2-1]
                            	\arrow[no head, from=2-1, to=1-1]
                            \end{tikzcd}
                        $$
                    \begin{enumerate}
                        \item 
                            \begin{enumerate}
                                \item \textbf{(1 implies 2):} Suppose firstly that \textbf{1} holds, i.e. that $\pi: F \to B$ is \'etale. According to definition \ref{def: etale_morphisms}, this tells us that $B$ is a smooth $F$-algebra that is of relative dimension $0$; in other words, we can write $B$ as $\frac{F[x_1, ..., x_n]}{(f_1, ..., f_n)}$ for some natural number $n$. Now, fix an arbitrary prime ideal $\q \in \left|\Spec \frac{F[x_1, ..., x_n]}{(f_1, ..., f_n)}\right|$, which should be noted to be nothing but a prime of $F[x_1, ..., x_n]$ containing the ideal $(f_1, ..., f_n)$, and note that:
                                    $$\left(\frac{F[x_1, ..., x_n]}{(f_1, ..., f_n)}\right)_{\q} \cong \frac{F[x_1, ..., x_n]_{\q}}{(f_1, ..., f_n)}$$
                                thanks to the fact that colimits commute. Now, let us suppose for the sake of deriving a contradiction, that $\kappa_{\pi^{-1}(\q)}/\kappa_{\q}$ is not a (finite) separable extension for our chosen prime $\q$, and observe that according to the preliminary discussion, this is the same as supposing that the purely inseparable extension $\kappa_{\pi^{-1}(\q)}/\kappa_{\q}^{\sep}$ is non-trivial. 
                                \item \textbf{(2 implies 1):} On the other hand, let us use \textbf{2} as our starting point. 
                            \end{enumerate}
                        \item 
                            \begin{enumerate}
                                \item \textbf{(2 implies 3):} Now, suppose that \textbf{2} is true and that $F$ is a field. Immediately, one sees that:
                                    $$\kappa_{\pi^{-1}{\q}} \cong \calO_{\Spec F, \pi^{-1}(\q)} \cong F_{\pi^{-1}(\q)} \cong F_{(0)} \cong F$$ 
                                which means that $F/\kappa_{\q}$ is a (finite) separable extension for all primes $\q \in \left|\Spec B\right|$. Also, thanks to the hypothesis whereby $F$ is a field, one can write the finitely presented $F$-algebra $B$ as some finite direct some of copies of $F$ (since algebras are first and foremost modules, and modules over fields are vector spaces, which are \textit{a priori} all free). Thus, the \'etale $F$-algebra $B$, when viewed as a vector space over $F$, can written as a finite direct sum of the finite separable extension $F$ of $\kappa_{\q}$, for all $\q \in |\Spec B|$. In other words, \textbf{2} implies \textbf{3}. 
                                \item \textbf{(3 implies 2):} Conversely, suppose that \textbf{3} is true, specifically that:
                                    $$B \cong F^{\oplus d}$$
                                for some natural number $d$, and suppose that the $F$-algebra of finite presentation $B$ is of the form $\frac{F[x_1, ..., x_N]}{(f_1, ..., f_n)}$, for some pair of natural numbers $n, N$. 
                            \end{enumerate}
                        Thus, we have managed to show that \textbf{1} is equivalent to \textbf{2}, and that \textbf{2} is in turn equivalent to \textbf{3}, and thus the three are jointly equivalent. 
                    \end{enumerate}
                \end{proof}
            \begin{corollary}[Finite separable extensions are \'etale] \label{coro: finite_separable_extensions_are_etale}
                Let $k$ be a field and let $B$ be a $k$-algebra of finite presentation. According to proposition \ref{prop: separable_criterion_for_etaleness}, $B$ is \'etale over $k$ if and only if when it is viewed as a $k$-vector space, $B$ is a direct sum of copies of some finite separable extension $K$ over $k$. But $K$ itself is a $k$-algebra, and thus every finite and separable field extension is an \'etale ring map. In fact, it is even better: via the adjoint equivalence:
                    $$
                        \begin{tikzcd}
                        	{{}^{k/}\Comm\Alg^{\op}} & {\Sch^{\aff}_{/\Spec k}}
                        	\arrow[""{name=0, anchor=center, inner sep=0}, "\Spec"', shift right=2, from=1-1, to=1-2]
                        	\arrow[""{name=1, anchor=center, inner sep=0}, "Gamma"', shift right=2, from=1-2, to=1-1]
                        	\arrow["\dashv"{anchor=center, rotate=-90}, draw=none, from=1, to=0]
                        \end{tikzcd}
                    $$
                any non-empty collection of finite separable extension $\{k \to K_{\alpha}\}_{\alpha \in A}$ of some given ground field $k$ corresponds to an \'etale covering sieve $\{\Spec K_{\alpha} \to \Spec k\}_{\alpha}$ (because spectra of fields are singletons, the joint surjection of presheaves:
                    $$
                        \begin{tikzcd}
                        	{\left(\coprod_{\alpha \in A} h_{\Spec K_{\alpha}}\right) \x_{h_{\Spec k}} \left(\coprod_{\alpha \in A} h_{\Spec K_{\alpha}}\right)} & {\coprod_{\alpha \in A} h_{\Spec K_{\alpha}}} & {h_{\Spec k}}
                        	\arrow["{\pr_2}"', shift right=2, from=1-1, to=1-2]
                        	\arrow["{\pr_1}", shift left=2, from=1-1, to=1-2]
                        	\arrow[dashed, from=1-2, to=1-3]
                        \end{tikzcd}
                    $$
                exists for all non-empty indexing sets $A$). In turn, this implies something remarkable, which is that for all Galois extensions $K/k$ (which are necessarily separable by definition, and hence \'etale) and for all \textit{sheaves} $\calF$ on ${}^{k/}\Comm\Alg^{\op, \petit}_{\et}$ (we refer the reader to paragraph \ref{paragraph: etale_descent} for the descent theory along \'etale morphisms) we have via the fact that finite colimits (in this case, the orbit space ) commute with filtered limits that and the Fundamental Theorem of Galois Theory that:
                    $$
                        \begin{aligned}
                            F(\Spec K)/Gal(K/k) & \cong \calF(\Spec E)/\left(\underset{E}{\lim} \Aut(E/k)\right)
                            \\
                            & \cong \underset{E}{\lim} \left(\calF(\Spec E)/\Aut(E/k)\right)
                            \\
                            & \cong \underset{E}{\lim} \left(\calF(\Spec k)/\Aut(E/k)\right)
                        \end{aligned}
                    $$
                wherein the limit is the filtered limit taken over the poset of finite separable subextensions $E/k$ of $K/k$; also, note that:
                    $$\calF(\Spec E) \cong \calF(\Spec k)$$
                for all finite separable extensions $E/k$ because sheaves satisfy descent, by definition.
            \end{corollary}
            \begin{remark}[Fibres over rational points of \'etale morphisms and inertia of primes] \label{remark: fibres_of_etale_maps_over_points}
                Let $k$ be a field and let $B$ be an \'etale $k$-algebra. From proposition \ref{prop: separable_criterion_for_etaleness}, we know that there exists a natural number $d$ such that:
                    $$B \cong K^{\oplus d} \cong (k^{\oplus [K : k]})^{\oplus d} \cong k^{\oplus ([K : k] \cdot d)}$$
                as $k$-vector spaces, where $K/k$ is some finite separable extension. Then, thanks to the fact that direct sums of algebra objects in the category $k\Vect$ of $k$-vector spaces are merely products in the commutative algebra category ${}^{k/}\Comm\Alg$, and by employing the adjoint equivalence:
                    $$
                        \begin{tikzcd}
                        	{{}^{k/}\Comm\Alg^{\op}} & {\Sch^{\aff}_{/\Spec k}}
                        	\arrow[""{name=0, anchor=center, inner sep=0}, "\Spec"', shift right=2, from=1-1, to=1-2]
                        	\arrow[""{name=1, anchor=center, inner sep=0}, "Gamma"', shift right=2, from=1-2, to=1-1]
                        	\arrow["\dashv"{anchor=center, rotate=-90}, draw=none, from=1, to=0]
                        \end{tikzcd}
                    $$
                one gets:
                    $$\Spec B \cong \Spec \prod_{i = 1}^d K \cong \coprod_{i = 1}^d \Spec K$$
                wherein the coproduct is taken in the category of locally ringed spaces over $\Spec k$, but lands in the category of affine schemes (also over $\Spec k$). Lastly, because spectra of fields are singletons, the \'etale $k$-algebra $B$ is nothing but a collection of $d$ disjoint points lying over the single point of $\Spec k$. 
                
                One implication of this is that given any scheme $X$ over $\Spec k$, its fibres over $k$-rational points $x \in X(k)$ are nothing but finite collections of points (even if the fibre is not affine, its affine patches would still necessarily be finite collections of points, as shown above, and thus the whole fibre would be so as well).
            \end{remark}
    
        \subsection{The interesting point: Galois cohomology}
            \subsubsection{Cohomology of profinite groups}
                Cohomology of profinite groups is nothing but group cohomology but for pro-objects in the category of finite groups (for us, cohomology shall always mean cohomology with abelian coefficients). Because of that, we shall pay less attention to the definition and general properties, and more on the special properties that cohomologies of profinite groups enjoy. By the end of it, we shall apply what we know to Galois groups, which are profinite thanks to the Fundamental Theorem of Galois Theory. 
                
                \paragraph{Group cohomology}
                    \begin{convention}
                        \noindent
                        \begin{itemize}
                            \item Throughout, we shall work with a sheaf topos $\E$ in which the category of internal abelian groups $\Ab(\E)$ has enough projectives (for instance, one can consider $\E \cong \Sets$ or the \'etale topos over any qcqs scheme). This essentially just means that we are assuming that the internal logic of $\E$ supports the Axiom of Choice (or more minimally, the Axiom of Presentation). 
                            
                            In particular, fix an abelian group $A \in \Ab(\E)$.
                            \item Additionally, fix a group object $G \in Grp(\E)$. Its group ring shall be denoted by $\Z[G]$ (which is not an abuse of notation, since a natural number object exists in $\E$, allowing us to define an integer object $\Z$ as the group completion of the monoid $\N$; alternatively, one can think of $\Z$ as the monoidal unit of $\Ab(\E)$); recall that this ring object is defined via the following forgetful-free adjunction, whose existence we shall let the reader prove as an exercise\footnote{Recall that $\Ring(\E)$ is the category of monoids internal to the symmetric monoidal category $\Ab(\E)$.}:
                                $$
                                    \begin{tikzcd}
                                    	{\Ring(\E)} & {\Grp(\E)}
                                    	\arrow[""{name=0, anchor=center, inner sep=0}, "\oblv"', shift right=2, from=1-1, to=1-2]
                                    	\arrow[""{name=1, anchor=center, inner sep=0}, "{\Z[-]}"', shift right=2, from=1-2, to=1-1]
                                    	\arrow["\dashv"{anchor=center, rotate=-90}, draw=none, from=1, to=0]
                                    \end{tikzcd}
                                $$
                                
                            We shall also want a $G$-action on $A$, which would make $A$ a $\Z[G]$-module.
                        \end{itemize}
                    \end{convention}
                    
                    \begin{definition}[Group cohomologies] \label{def: group_cohomologies}
                        The $n^{th}$ cohomology group of $G$ with coefficient in $A$ is defined as:
                            $$H^n(G, A) \cong \Ext^n_{\Z[G]}(\Z, A)$$
                        where we view $\Z$ as trivial $G$-representation on the right-hand side. 
                    \end{definition}
                    \begin{remark}
                        It is not hard to see via induction on the cohomological dimension $n \in \N$ that $H^n(G, A)$ (as in definition \ref{def: group_cohomologies}) actually has a $k$-module structure.
                    \end{remark}
                    
                    Let us now attempt to compute cohomologies of groups in low dimensions (namely, $n = 0, 1, 2$) as well as give meaning to these spaces, which we shall do using free (or at worst, projective) resolutions of the trivial $\Z[G]$-module $\Z$.
                    \begin{proposition}[Low-dimensional cohomologies of groups] \label{prop: low_dimensional_cohomologies_of_groups}
                        One has the following interpretations of the low-dimensional cohomologies of $G$ with coeffcients in $A$:
                        \begin{enumerate}
                            \item \textbf{($n = 0$: Invariants):}
                                $$H^0(G, A) \cong A^G$$
                            with $A^G$ denoting the space of $G$-fixed point of $A$.
                            \item \textbf{($n = 1$: Torsors):} 
                            \item \textbf{($n = 2$: Extensions):} 
                                $$H^2(G, A) \cong \{\text{extensions of $G$ by $A$}\}$$
                        \end{enumerate}
                    \end{proposition}
                        \begin{proof}
                            \noindent
                            \begin{enumerate}
                                \item \textbf{($n = 0$: Invariants):}
                                \item \textbf{($n = 1$: Torsors):}
                                \item \textbf{($n = 2$: Extensions):}
                            \end{enumerate}
                        \end{proof}
                    \begin{example}
                                    
                    \end{example}
                
                \paragraph{Properties of cohomologies of profinite groups}
            
            \subsubsection{Iwasawa theory}
    
        \subsection{\texorpdfstring{$\ell$}{}-adic cohomology}
            \subsubsection{\texorpdfstring{$\ell$}{}-adic cohomology for schemes: "This is where the fun begins!"}
                Let $X$ be a smooth projective scheme over a base field that is possibly of some prime characteristic $p$. Let $\ell$ be a prime different from $p$. By \say{$\ell$-adic cohomology}, we shall mean the cohomology theory whose cohomology groups are given by:
                    $$H^i_{\Q_{\ell}}(X) \cong \underset{n \in \N}{\lim} H_{\et}^i(X, \Z/\ell^{n + 1}\Z) \tensor_{\Z_{\ell}} \Q_{\ell}$$
                This means, among other things, that $\ell$-adic cohomology is effectively a \say{singular cohomology theory} for (smooth projective) schemes. This, however, is not the only amazing property that $\ell$-adic cohomology enjoys: it is also a Weil cohomology theory (see definition \ref{def: weil_cohomology_theories} for a description of what this means). At surface level, this might seem odd, since only \'etale cohomology with torsion coefficients gives satisfactory results. However, this is precisely why we take the limit over the torsion rings $\Z/\ell^{n + 1}\Z$ and then base change to $\Q_{\ell}$: $\Q_{\ell}$ is a field of characteristic $0$, and hence can serve as the field of a coefficients of a Weil cohomology theory (cf. definition \ref{def: weil_cohomology_theories}). 
                
                To check that $\ell$-adic cohomology is indeed a Weil cohomology theory, let us \say{simply} go through the axioms laid out in definition \ref{def: weil_cohomology_theories}. Before we can do that, however, we will first need to actually set up the theory of \'etale $\ell$-adic cohomology.
                
                Now, we are finally ready to verify that \'etale $\ell$-adic cohomology is a Weil cohomology theory.
                \paragraph{Finiteness}
            
                \paragraph{The K\"unneth Formula}
                
                \paragraph{Poincar\'e Duality}
            
                \paragraph{The Lefschetz Conditions}
        
            \subsubsection{The Artin comparison theorem: an \'etale GAGA theorem}
            
            \subsubsection{\texorpdfstring{$\ell$}{}-adic cohomology for algebraic stacks: "Another happy landing!"}
                \paragraph{Finiteness}
            
                \paragraph{The K\"unneth Formula}
                
                \paragraph{Poincar\'e Duality}
            
                \paragraph{The Lefschetz Conditions}
                
    