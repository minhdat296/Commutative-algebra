\chapter{Motives}
    \begin{abstract}
        
    \end{abstract}
    
    \minitoc
    
    \section{Weil cohomology theories}
        \subsection{Fantastic cohomology theories and where to find them}
            \subsubsection{Some intersection theory}
                \begin{remark}[Categories of smooth projective schemes] \label{remark: categories_of_smooth_projective_schemes}
                    Fix a ground field $k$. 
                    
                    Inspired by Serre's GAGA Theorem - which relates smooth projective varieties over $\Spec \bbC$ to (compact) complex analytic manifolds - we shall try to set up the so-called Weil cohomology theories so that they would behave well over smooth projective (algebraic) schemes over fields such as $k$ (not that we would object to these cohomology theories working over more general schemes, but one should also be reasonable with one's expectations). For that, we shall need to first see if smooth projective algebraic schemes form a category (otherwise, what even is the point ?); also, notice how even with two additional adjectives, the class smooth projective algebraic schemes still includes a lot of important examples, notable among which are elliptic curves and higher dimensional abelian varieties.
                    
                    Thankfully, we do have a category of smooth projective algebraic schemes over any given base field $k$, which we denote by $\Sch_{/\Spec k}^{\smooth, \proj}$. To see why this is the case, recall firstly that thanks to the universal property of the $\Proj$-construction, a projective schemes $X$ is nothing but an $\N$-filtration of schemes (i.e. a diagram:
                        $$X: \N \to \Sch$$
                    of shape $\N$ whose transition maps are \href{https://stacks.math.columbia.edu/tag/01L1}{\underline{monomorphism of schemes}}; note in particular, that any immersion, closed or open, is a monomorphism \cite[\href{https://stacks.math.columbia.edu/tag/01L7}{Tag 01L7}]{stacks}); by abstract nonsense, $X$ is thus simply a functor that preserves monomorphisms, as every arrow in $\N$ is a monomorphism.  
                \end{remark}
                
                \begin{definition}[Algebraic cycles] \label{def: algebraic_cycles}
                    Let $S$ be a Noetherian base scheme and let $f: X \to S$ be an $S$-scheme of finite type. Recall also that Noetherian schemes form a full subcategory of $\Sch$; let us denote it by $\Sch^{\Noeth}$. 
                        \begin{enumerate}
                            \item \textbf{(Relative cycles):} 
                            \item \textbf{(The Chow functors):} A \textbf{presheaf of $d$-dimensional relative cycles} over $f: X \to S$ (or a \textbf{$d$-dimensional Chow functor} over $f: X \to S$) shall be a functor:
                                $$\Chow_{X/S}(d, -): \Sch_{/S}^{\Noeth} \to \Sets$$
                            that associates to Noetherian $S$-schemes $g: T \to S$ the set $\Chow_{X/S}(d, T)$ of $d$-dimensional relative cycles on the $T$-scheme $X \x_{f, S, g} T \to T$.
                        \end{enumerate}
                \end{definition}
                
            \subsubsection{Weil cohomology theories}    
                \begin{definition}[Weil cohomology theories] \label{def: weil_cohomology_theories}
                    Let $k$ be an arbitrary base field, and let $F$ be a field of characteristic $0$, which will serve as a so-called \say{field of coefficients}. A \textbf{Weil cohomology theory} over $k$ is thus a contravariant functor:
                        $$\H^*: \Sch_{/\Spec k}^{\smooth, \proj, \op} \to [\Z, {}_F\Vect]$$
                    from the category of smooth projective algebraic schemes over $\Spec k$ into the category of $\Z$-graded $F$-vector spaces (which are just diagrams of shape $\Z$ of $F$-vector spaces) that satisfies the following axioms:
                        \begin{enumerate}
                            \item \textbf{(Finiteness):} Given a smooth projective algebraic scheme $X$ over $\Spec k$, we shall want all the vector spaces in the diagram:
                                $$
                                    \H^*(X) =
                                    \left(
                                        \begin{tikzcd}
                                        	\cdots & {\H^{-1}(X)} & {\H^0(X)} & {\H^1(X)} & \cdots
                                        	\arrow[from=1-2, to=1-3]
                                        	\arrow[from=1-3, to=1-4]
                                        	\arrow[from=1-4, to=1-5]
                                        	\arrow[from=1-1, to=1-2]
                                        \end{tikzcd}
                                    \right)
                                $$
                            to be \textit{finite-dimensional}. Additionally, we would like to require that:
                                $$
                                    \dim_F \H^i(X) = 
                                    \begin{cases}
                                        \text{$n_i \not = 0$ if $0 \leq i \leq 2\dim X$}
                                        \\
                                        \text{$0$ otherwise}
                                    \end{cases}
                                $$
                            \item \textbf{(Poincar\'e Duality):} This is to say that there is an isomorphism, called the \textbf{trace map}:
                                $$\int_X(-): \H^{2 \dim X}(X) \cong F$$
                            and that for each $i \in \Z$, there exists a non-degenerate bilinear pairing:
                                $$\<\cdot \mid \cdot\>: \H^i(X) \x \H^{2\dim X - i}(X) \to \H^{2 \dim X}(X)$$
                            which establishes an isomorphism between $\H^i(X) \x \H^{2\dim X - i}(X)$ and $F$ via the trace map $\int_X$. 
                            \item \textbf{(K\"unneth Formula/Monoidality):} For all $i \in \Z$, we have:
                                $$\H^i(X \x_{\Spec k} Y) \cong \H^i(X) \tensor_F \H^i(Y)$$
                            \item \textbf{(Algebraic cycles):}
                        \end{enumerate}
                \end{definition}
                \begin{remark}[Why these axioms ?] \label{remark: motivation_for_motives}
                    For the most part, the axioms laid out in definition \ref{def: weil_cohomology_theories} are there because we want Weil cohomology theories to behave how we have come to expect reasonable \say{geometric} cohomology theories to. In particular, we want for (smooth and projective) schemes cohomology theories that act more or less like singular cohomology or the classical de Rham cohomology for manifolds. In fact, most famous examples of Weil cohomology theories were conceived in the images of singular and de Rham cohomologies: \'etale cohomology (cf. chapter \ref{chapter: etale_cohomology_1}) is supposed to be the topological cohomology theory that works for schemes - especially those in positive characteristics - and crystalline cohomology exists so that we might have a \say{differential geometry} of schemes (cf. chapter \ref{chapter: crystals}). 
                \end{remark}
                \begin{remark}[What about the Lefschetz Conditions ?] \label{remark: lefschetz_axioms}
                    
                \end{remark}
                \begin{example}
                    \noindent
                    \begin{enumerate}
                        \item \textbf{(Over characteristic $0$):} 
                            \begin{enumerate}
                                \item \textbf{(Betti cohomology):}
                                \item \textbf{(de Rham cohomology):}
                                \item \textbf{(A counter-example: Zariski cohomology):}
                            \end{enumerate}
                        \item \textbf{(Over positive characteristics):}
                            \begin{enumerate}
                                \item \textbf{($\ell$-adic cohomology):}
                                \item \textbf{(Crystalline cohomology):}
                            \end{enumerate}
                    \end{enumerate}
                \end{example}
        
        \subsection{Motives}
    
    \section{Complex Hodge theory}
        \subsection{Hodge structures}
            \subsubsection{Pure Hodge structures on compact complex analytic manifolds}
                \begin{theorem}[Grothendieck's algebraic de Rham cohomology] \label{theorem: de_rham_cohomology}
                    Let $X$ be a smooth scheme over $\Spec \bbC$ and let $X^{\an}$ denote the analytic manifold whose underlying set is $X(\bbC)$. Then, one has the following comparison quasi-isomorphism between algebraic and complex-analytic de Rham cohomologies:
                        $$H^*_{\dR}(X) \cong_{\qis} H^*_{\dR}(X^{\an})$$
                \end{theorem}
            
            \subsubsection{Mixed Hodge structures}
        
        \subsection{Periods}
        
        \subsection{Moduli of Hodge structures and complex Shimura varieties}
        
    \section{Motivic \texorpdfstring{$p$}{}-adic Hodge theory}
        \subsection{The Ax-Tate-Sen Theorem}
        
        \subsection{Period rings and period sheaves}
    
        \subsection{The Hodge-Tate Decomposition and the \'etale-de Rham comparison}
            \subsubsection{The Hodge-Tate Decomposition via perfectoid spaces}
                \paragraph{The case for abelian schemes with good reductions}
                    \begin{convention}
                        From this point on, fix a prime $p$ along with a $p$-adic number field $K/\Q_p$. Also, fix a completion $C$ of an algebraic closure of $K$. Lastly, let $A_{/K^{\circ}}$ be an abelian scheme over $\Spec K^{\circ}$ with generic fibre $A_{/K}$.
                    \end{convention}
                
                \paragraph{The general case}
            
            \subsubsection{The \'etale-de Rham comparison}
        
        \subsection{The \'etale-crystalline comparison}
        
        \subsection{Perfectoid Shimura varieties}
        
        \subsection{The Fargues-Fontaine Curve}
            The following description of the Fargues-Fontaine Curve, aside from detailing its applications to Local Class Field Theory, will also serve as a continuation of section \ref{section: perfectoid_spaces}, for The Curve can be viewed as the moduli space of untilts a given algebraically closed perfectoid field of positive characteristic. Specifically, what we want to know are perfectoid fields $E$ of mixed characteristics $(0, p)$ such that $E^{\flat} \cong F$ for some prescribed perfectoid field $F$ of characteristic $p$. Also, let us note that this is in no way an ill-posed question, as for instance, we know that the tilts of both $\Q_p(p^{\frac{1}{p^{\infty}}})^{\wedge}$ and $\Q_p(\mu_{p^{\infty}})^{\wedge}$ are isomorphic to $\F_p(\!(t^{\frac{1}{p^{\infty}}})\!)$. The Fargues-Fontaine Curve, which we shall denote by $\calY_{\FF}$, should thus be some sheaf of sets on $\Perfd_{/\Spec \F_p}$ whose fibres over adic spectra of algebraic closures of adic residue fields (understood as \textit{geometric points}) shall be sets of untilts of that very algebraically closed field.
            
            \subsubsection{Constructing The Curve}
                \begin{convention}[Relative Witt vectors] \label{conv: relative_witt_vectors}
                    Let $p$ be a prime number. If $E/\Q_p$ is a finite $p$-adic number field with residue field $\F_q$ (for $q$ some power of $p$) and if $F$ is a perfectoid field of characteristic $p$ (which is \textit{a priori} perfect and therefore can contain $\F_q$; consider fields such as $\F_q(\!(t^{\frac{1}{p^{\infty}}})\!)$ or $\widehat{\overline{\F_q(\!(t)\!)}}$ for example), then let us write:
                        $$\W_{E^{\circ}}(F) \cong \W(F) \tensor_{\W(\F_q)} E^{\circ}$$
                    for the base change along $E^{\circ} \to $ of the ring $\W(F)$ of (unramified) $p$-typical Witt vectors with coefficients in $F$ along the canonically induced arrow $\W(\F_q) \to E^{\circ}$. For instance, when $q = p$ we have:
                        $$\W_{E^{\circ}}(\F_p(\!(t^{\frac{1}{p^{\infty}}})\!)) \cong \Z_p[p^{\frac{1}{p^{\infty}}}] \tensor_{\Z_p} \Z_p \cong \Z_p[p^{\frac{1}{p^{\infty}}}]$$
                    Slightly more generally, one can speak of a relative $p$-typical Witt vector functor from the category of perfect commutative $\F_q$-algebras to the category ${}^{E^{\circ}/}\Comm\Alg^{\wedge}$ of $p$-adically complete $E^{\circ}$-algebras:
                        $$\W_{E^{\circ}}(-): {}^{\F_q/}\Comm\Alg^{\perf} \to {}^{E^{\circ}/}\Comm\Alg^{\wedge}$$
                    which in particular, gives us the following canonical arrow:
                        $$\W_{E^{\circ}}(F^{\circ}) \to \W_{E^{\circ}}(F)$$
                \end{convention}
                
                \begin{remark}[Witt vectors over perfect rings] \label{remark: witt_vectors_over_perfect_rings}
                    \noindent
                    \begin{enumerate}
                        \item One somewhat non-trivial fact to keep in mind while reading this paper is that if $B$ is a \textit{perfect} $\F_q$-domain (for some power $q$ of a prime $p$) with field of fractions $K$, then we have the following natural characterisation of the ring of $p$-typical Witt vectors over $K$ (which we note to be trivially perfect as an $\F_q$-algebra):
                        $$(\W(B)_{(p)})^{\wedge} \cong \W(K)$$
                        In particular, $\W(B)_{(p)}$ is an unramified extension of $\W(\F_q)$. We refer the reader to \cite[Proposition 5.2]{shimomoto2014witt} for a proof.
                        \item This result extends trivially to the relative setting, as relative Witt vectors are defined via a pushout (cf. convention \ref{conv: relative_witt_vectors}), which is in particular a finite limit, and since adic completions are filtered limits, the two procedures can be exchanged, which gives:
                            $$(\W_{E^{\circ}}(B)_{(p)})^{\wedge} \cong \W_{E^{\circ}}(K)$$
                        where now, $E$ is a finite $p$-adic number field with residue field $\F_q$ (note that its pseudo-uniformiser is actually just $p$, like $\Q_p$).
                        \item Another notable property of rings (relative) Witt vectors is that if $B$ is a perfect $\F_q$-algebra (not necessarily a domain), then $\W_{E^{\circ}}(B)$ is $p$-torsion-free, and:
                            $$\W_{E^{\circ}}(B)/p \cong B$$
                    \end{enumerate}
                \end{remark}
                \begin{example}
                    One might think of the following example:
                        $$(\W_{E^{\circ}}(\F_q)_{(p)})^{\wedge} \cong \W_{E^{\circ}}(\F_q) \cong E^{\circ}$$
                    (note that $E^{\circ}$ is \textit{a priori} complete), or the following slightly subtler one:
                        $$( \W_{E^{\circ}}( \F_q[\![t^{\frac{1}{p^{\infty}}}]\!] )_{(p)} )^{\wedge} \cong E^{\circ}[p^{\frac{1}{p^{\infty}}}]^{\wedge} \cong \W_{E^{\circ}}( \F_q(\!(t^{\frac{1}{p^{\infty}}})\!) ) \cong E(p^{\frac{1}{p^{\infty}}})^{\wedge, \circ}$$
                    (indeed, $E^{\circ}[p^{\frac{1}{p^{\infty}}}]^{\wedge} \cong E^{\circ}[\![t]\!]/(t^{p^{\infty}} - p)$ and so $E^{\circ}[p^{\frac{1}{p^{\infty}}}]^{\wedge}/p \cong \F_q[\![t^{\frac{1}{p^{\infty}}}]\!]$). In both cases, note that $\F_q$ and $\F_q[\![t^{\frac{1}{p^{\infty}}}]\!]$ are both perfect domains (the latter being the $p$-tilt of $\F_q[\![t]\!]$).
                \end{example}
                
                \begin{remark}[Our hopes and dreams] \label{remark: motivating_the_fargues_fontaine_curve}
                    First of all, we would like our to-be adic Fargues-Fontaine Curve to be a curve over a base field that is:
                        \begin{itemize}
                            \item projective, so it would have a chance of admitting a GAGA-esque functor,
                            \item smooth, so that its topology would behave nicely (read: so that we would be able to apply \'etale cohomology) and so important cases such as elliptic curves would be covered, 
                            \item proper (or according to certain authors such as D. Gaitsgory, \say{complete}, although we will avoid this terminology since we would frequently be thinking of topologically complete objects), so the Proper Base Change Theorem from the theory of \'etale cohomology would apply 
                        \end{itemize}
                    and most importantly, such that it would have a very straight forward connection to the Weil group of a $p$-adic number field because then, certain local systems on The Curve would correspond with the Weil-Deligne representations of the aforementioned Weil group, which is something that we would want for the establishment of the Galois/spectral side of the Local Langlands Correspondence. Also, due to this last point, one should imagine the Fargues-Fontaine Curve as a sort of $p$-adic Riemann surface; in fact, we will see (eventually) that it does behave similarly to Riemann surfaces in many ways, particularly via its connection to $p$-adic Shimura varieties.
                \end{remark}
                
                \begin{definition}[Prediamonds] \label{def: prediamonds}
                    \noindent
                    \begin{enumerate}
                        \item \textbf{($2$-preadic spaces):} To us, an \textbf{$2$-preadic space} shall be nothing more than a prestack in groupoids on $\Ad^{\affd}$, the category of affinoid adic spaces.
                        \item \textbf{((Pre)diamonds):} A prediamond is thus nothing more than an $2$-preadic space over a category of affinoid perfectoid spaces.
                        
                        Now, let $p$ be a prime number and let $E$ be a finite $p$-adic number field with residue field $\F_q$, for some power $q$ of $p$. Then, the \textbf{diamond $p$-tilting functor} shall be the left-Kan extension along the perfectoid $p$-tilting functor:
                            $$(-)^{\flat, \perfd}: \Perfd^{\affd}_{/\Spa E^{\circ}} \to \Perfd^{\affd}_{/\Spa \F_q}$$
                        i.e. it is the following canonically induced left-adjoint pushforward:
                            $$
                                \begin{tikzcd}
                                	{\Pre\Dia_{/\Spa E^{\circ}}} & {\Pre\Dia_{/\Spa \F_q}} \\
                                	{\Perfd^{\affd}_{/\Spa E^{\circ}}} & {\Perfd^{\affd}_{/\Spa \F_q}}
                                	\arrow["{(-)^{\flat, \perfd}}", from=2-1, to=2-2]
                                	\arrow[""{name=0, anchor=center, inner sep=0}, shift left=2, from=1-2, to=1-1]
                                	\arrow["y", from=2-1, to=1-1]
                                	\arrow["y"', from=2-2, to=1-2]
                                	\arrow[""{name=1, anchor=center, inner sep=0}, "{(-)^{\flat, \diamond}}", shift left=2, from=1-1, to=1-2]
                                	\arrow["\dashv"{anchor=center, rotate=-90}, draw=none, from=1, to=0]
                                \end{tikzcd}
                            $$
                        We shall leave the proof of existence for the left-Kan extension $(-)^{\flat, \diamond}$ up to the reader, as it is entirely categorical.
                    \end{enumerate}
                \end{definition}
                
                \begin{definition}[The Fargues-Fontaine Curve] \label{def: the_fargues_fontaine_curve}
                    Let $p$ be a prime number and let $E/\Q_p$ be a finite $p$-adic number field with residue field $\F_q$ (for $q$ some power of $p$). Then:
                        \begin{enumerate}
                            \item Consider firstly the prediamond on $\Perfd^{\affd}_{/\Spa \F_q}$ that is given pointwise by:
                                $$\calY^{\ad}_{E, \FF}(\Spa(R, R^+)) \cong \Spa \W_{E^{\circ}}(R^+) \setminus \Spec R^+$$
                            We shall call this $2$-preadic space the \textbf{Covering Curve}, for reasons that will become clear shortly. Notice that this is an object in characteristic $0$, and also, that one might imagine it as the punctuation of the disc $\Spa \W_{E^{\circ}}(R^+)$ by the Zariski-closed \say{point} $\Spec R^+$.
                            \item We subsequently define the \textbf{relative Fargues-Fontain Curve} $\calX^{\ad}_{E, \FF}$ over $E$ (or rather, over $\Spa E^{\circ}$) to be the quotient of the prediamond $\calY^{\ad}_{E, \FF}$ by the equivalence relation generated by the $p^{th}$ power Frobenius thereon:
                                $$\calX^{\ad}_{E, \FF} \cong \left[\calY^{\ad}_{E, \FF} \bigg/ \calY^{\ad}_{E, \FF} \x_{\Frob, \calY^{\ad}_{E, \FF}, \id} \calY^{\ad}_{E, \FF}\right]$$
                            (note that the equivalence relation is well-define, as $\calY^{\ad}_{E, \FF}$ is in characteristic $0$, and the associated Frobenius must therefore be injective; we can then use the fact that limits commute to see why the above pullback is a subobject of the product $\calY^{\ad}_{E, \FF} \x_{\Spa E^{\circ}} \calY^{\ad}_{E, \FF}$); we often simply write:
                                $$\calX^{\ad}_{E, \FF} \cong \calY^{\ad}_{E, \FF}/\Frob$$
                        \end{enumerate}
                \end{definition}
                \begin{remark}[Tilting the Fargues-Fontaine Curve] \label{remark: tilting_the_fargues_fontaine_curve}
                    Since the Fargues-Fontaine Curve $\calY^{\ad}_{E, \FF}$ over some finite extension $E/\Q_p$ is a prediamond of characteristic $0$, it admits a $p$-tilt $\calX^{\ad, \flat, \diamond}_{E, \FF}$. Even better, we have:
                        $$\calX^{\ad, \flat, \diamond}_{E, \FF} \cong (\calY^{\ad}_{E, \FF}/\Frob)^{\flat, \diamond} \cong \calY^{\ad, \flat, \diamond}_{E, \FF}/\Frob$$
                    as the left-Kan extension $(-)^{\flat, \diamond}$, by virtue of being a left-adjoint, preserves colimits \textit{a priori}. Actually, we even have:
                        $$\calY^{\ad, \flat, \diamond}_{E, \FF} \cong \calY^{\ad, \flat, \diamond}_{E, \FF}/\Frob$$
                    since perfectoid algebras of characteristic $p$ are perfect, and because prediamonds all admit covers by affinoid perfectoid spaces, the Frobenius on the prediamond $\calY^{\ad, \flat, \diamond}_{E, \FF}$ of characteristic $p$ must be an automorphism. Thus:
                        $$\calX^{\ad, \flat, \diamond}_{E, \FF} \cong \calY^{\ad, \flat, \diamond}_{E, \FF}$$
                \end{remark}
                
                \begin{proposition}[The Fargue-Fontaine Curve as the moduli curve of untilts] \label{prop: fargues_fontaine_curve_as_moduli_space_of_untilts}
                    
                \end{proposition}
                    \begin{proof}
                        
                    \end{proof}
                    
                \begin{proposition}[The Fargue-Fontaine Curve as a punctured disc] \label{prop: fargues_fontaine_curve_as_punctured_disc}
                    
                \end{proposition}
                
                \begin{lemma}[The Covering Curve is a diamond] \label{the_covering_curve_is_a_diamond}
                    Let $p$ be a prime number and let $E/\Q_p$ be a finite $p$-adic number field with residue field $\F_q$ (for $q$ some power of $p$). Then, the Covering Curve is a sheaf on $\Perfd^{\affd}_{/\Spa E^{\circ}, \proet}$. In other words, it is a diamond.
                \end{lemma}
                    \begin{proof}
                        
                    \end{proof}
                \begin{proposition}[Perfectoidification of the Covering Curve] \label{perfectoidifying_the_fargues_fontaine_curve}
                    Let $p$ be a prime number and let $E/\Q_p$ be a finite $p$-adic number field with residue field $\F_q$ (for $q$ some power of $p$). Then, the canonical base change:
                        $$\calY^{\ad}_{E, \FF} \x_{\Spa E^{\circ}} \Spa E(p^{\frac{1}{p^{\infty}}})^{\wedge, \circ}$$
                    of the Covering Curve along the canonical arrow:
                        $$\Spa E(p^{\frac{1}{p^{\infty}}})^{\wedge, \circ} \to \Spa E^{\circ}$$
                    is a perfectoid space of characteristic $0$ over $\Spa E(p^{\frac{1}{p^{\infty}}})^{\wedge, \circ}$. 
                \end{proposition}
                    \begin{proof}
                        
                    \end{proof}
                \begin{corollary}[Discontinuity of Frobenius] \label{coro: discontinuity_of_frobenius}
                    
                \end{corollary}
            
            \subsubsection{A GAGA theorem}
            
            \subsubsection{Vector bundles on The Curve}