\chapter{Hodge theory, Weil cohomology theories, and motives}
    \begin{abstract}
        
    \end{abstract}
    
    \minitoc
    
    \section{Complex Hodge theory}
        \subsection{Hodge structures}
            \subsubsection{Pure Hodge structures on compact complex analytic manifolds}
                \begin{theorem}[Grothendieck's algebraic de Rham cohomology] \label{theorem: de_rham_cohomology}
                    Let $X$ be a smooth scheme over $\Spec \bbC$ and let $X^{\an}$ denote the analytic manifold whose underlying set is $X(\bbC)$. Then, one has the following comparison quasi-isomorphism between algebraic and complex-analytic de Rham cohomologies:
                        $$H^*_{\dR}(X) \cong_{\qis} H^*_{\dR}(X^{\an})$$
                \end{theorem}
            
            \subsubsection{Mixed Hodge structures}
        
        \subsection{Periods}
        
        \subsection{Moduli of Hodge structures}
    
    \section{Weil Cohomology Theories}
        \subsection{Fantastic cohomology theories and where to find them}
            \subsubsection{Some intersection theory; Chow Motives}
                \begin{remark}[Categories of smooth projective schemes] \label{remark: categories_of_smooth_projective_schemes}
                    Fix a ground field $k$. 
                    
                    Inspired by Serre's GAGA Theorem - which relates smooth projective varieties over $\Spec \bbC$ to (compact) complex analytic manifolds - we shall try to set up the so-called Weil Cohomology Theories so that they would behave well over smooth projective (algebraic) schemes over fields such as $k$ (not that we would object to these cohomology theories working over more general schemes, but one should also be reasonable with one's expectations). For that, we shall need to first see if smooth projective algebraic schemes form a category (otherwise, what even is the point ?); also, notice how even with two additional adjectives, the class smooth projective algebraic schemes still includes a lot of important examples, notable among which are elliptic curves and higher dimensional abelian varieties.
                    
                    Thankfully, we do have a category of smooth projective algebraic schemes over any given base field $k$, which we denote by $\Sch_{/\Spec k}^{\smooth, \proj}$. To see why this is the case, recall firstly that thanks to the universal property of the $\Proj$-construction, a projective schemes $X$ is nothing but an $\N$-filtration of schemes (i.e. a diagram:
                        $$X: \N \to \Sch$$
                    of shape $\N$ whose transition maps are \href{https://stacks.math.columbia.edu/tag/01L1}{\underline{monomorphism of schemes}}; note in particular, that any immersion, closed or open, is a monomorphism \cite[\href{https://stacks.math.columbia.edu/tag/01L7}{Tag 01L7}]{stacks}); by abstract nonsense, $X$ is thus simply a functor that preserves monomorphisms, as every arrow in $\N$ is a monomorphism.  
                \end{remark}
                
                \begin{definition}[Algebraic cycles] \label{def: algebraic_cycles}
                    Let $S$ be a Noetherian base scheme and let $f: X \to S$ be an $S$-scheme of finite type. Recall also that Noetherian schemes form a full subcategory of $\Sch$; let us denote it by $\Sch^{\Noeth}$. 
                        \begin{enumerate}
                            \item \textbf{(Relative cycles):} 
                            \item \textbf{(The Chow functors):} A \textbf{presheaf of $d$-dimensional relative cycles} over $f: X \to S$ (or a \textbf{$d$-dimensional Chow functor} over $f: X \to S$) shall be a functor:
                                $$\Chow_{X/S}(d, -): \Sch_{/S}^{\Noeth} \to \Sets$$
                            that associates to Noetherian $S$-schemes $g: T \to S$ the set $\Chow_{X/S}(d, T)$ of $d$-dimensional relative cycles on the $T$-scheme $X \x_{f, S, g} T \to T$.
                        \end{enumerate}
                \end{definition}
                
            \subsubsection{Weil Cohomology Theories}    
                \begin{definition}[Weil Cohomology Theories] \label{def: weil_cohomology_theories}
                    Let $k$ be an arbitrary base field, and let $F$ be a field of characteristic $0$, which will serve as a so-called \say{field of coefficients}. A \textbf{Weil Cohomology Theory} over $k$ is thus a contravariant functor:
                        $$\H^*: (\Sch_{/\Spec k}^{\smooth, \proj})^{\op} \to \Func(\Z, \Vect(F))$$
                    from the category of smooth projective algebraic schemes over $\Spec k$ into the category of $\Z$-graded $F$-vector spaces (which are just diagrams of shape $\Z$ of $F$-vector spaces) that satisfies the following axioms:
                        \begin{enumerate}
                            \item \textbf{(Finiteness):} Given a smooth projective algebraic scheme $X$ over $\Spec k$, we shall want all the vector spaces in the diagram:
                                $$
                                    \H^*(X) =
                                    \left(
                                        \begin{tikzcd}
                                        	\cdots & {\H^{-1}(X)} & {\H^0(X)} & {\H^1(X)} & \cdots
                                        	\arrow[from=1-2, to=1-3]
                                        	\arrow[from=1-3, to=1-4]
                                        	\arrow[from=1-4, to=1-5]
                                        	\arrow[from=1-1, to=1-2]
                                        \end{tikzcd}
                                    \right)
                                $$
                            to be \textit{finite-dimensional}. Additionally, we would like to require that:
                                $$
                                    \dim_F \H^i(X) = 
                                    \begin{cases}
                                        \text{$n_i \not = 0$ if $0 \leq i \leq 2\dim X$}
                                        \\
                                        \text{$0$ otherwise}
                                    \end{cases}
                                $$
                            \item \textbf{(Poincar\'e Duality):} This is to say that there is an isomorphism, called the \textbf{trace map}:
                                $$\int_X(-): \H^{2 \dim X}(X) \cong F$$
                            and that for each $i \in \Z$, there exists a non-degenerate bilinear pairing:
                                $$\<\cdot \mid \cdot\>: \H^i(X) \x \H^{2\dim X - i}(X) \to \H^{2 \dim X}(X)$$
                            which establishes an isomorphism between $\H^i(X) \x \H^{2\dim X - i}(X)$ and $F$ via the trace map $\int_X$. 
                            \item \textbf{(K\"unneth Formula/Monoidality):} For all $i \in \Z$, we have:
                                $$\H^i(X \x_{\Spec k} Y) \cong \H^i(X) \tensor_F \H^i(Y)$$
                            \item \textbf{(Algebraic cycles):}
                        \end{enumerate}
                \end{definition}
                \begin{remark}[Why these axioms ?] \label{remark: motivation_for_motives}
                    For the most part, the axioms laid out in definition \ref{def: weil_cohomology_theories} are there because we want Weil Cohomology Theories to behave how we have come to expect reasonable \say{geometric} cohomology theories to. In particular, we want for (smooth and projective) schemes cohomology theories that act more or less like singular cohomology or the classical de Rham cohomology for manifolds. In fact, most famous examples of Weil Cohomology Theories were conceived in the images of singular and de Rham cohomologies: \'etale cohomology (cf. chapter \ref{chapter: etale_cohomology_1}) is supposed to be the topological cohomology theory that works for schemes - especially those in positive characteristics - and crystalline cohomology exists so that we might have a \say{differential geometry} of schemes (cf. chapter \ref{chapter: crystals}). 
                \end{remark}
                \begin{remark}[What about the Lefschetz Conditions ?] \label{remark: lefschetz_axioms}
                    
                \end{remark}
                \begin{example}
                    \noindent
                    \begin{enumerate}
                        \item \textbf{(Over characteristic $0$):} 
                            \begin{enumerate}
                                \item \textbf{(Betti cohomology):}
                                \item \textbf{(de Rham cohomology):}
                                \item \textbf{(A counter-example: Zariski cohomology):}
                            \end{enumerate}
                        \item \textbf{(Over positive characteristics):}
                            \begin{enumerate}
                                \item \textbf{($\ell$-adic cohomology):}
                                \item \textbf{(Crystalline cohomology):}
                            \end{enumerate}
                    \end{enumerate}
                \end{example}
                
        \subsection{Chern classes}
        
        \subsection{Decategorification via K-groups}
                
    \section{Mixed motives}
        \subsection{The motivic category}
            Let us first lay down some conventions.
            \begin{convention} \label{conv: motivic_category_scheme_convention}
                Following \cite[Subsection I.i.1.1, pp. 9]{levine_1998_mixed_motives}, by \say{scheme}, we shall always implicitly mean \say{Noetherian separated schemes} throughout this entire subsection. 
            \end{convention}
            
            We shall also be needing the notion of so-called \say{localisations} of schemes\footnote{Which, again, are Noetherian and separated from this point on.} of finite type over a given base scheme.  
            \begin{definition}[Morphisms essentially of finite type] \label{def: scheme_morphisms_essentially_of_finite_type}
                For a moment, let us suppose that our schemes are general.
                \begin{enumerate}
                    \item Following \cite[Definition 2.1]{nayak_essentially_of_finite_type}, a morphism of schemes:
                        $$f: U \to X$$
                    is a \textbf{localisation} if and only if it is affine-schematic (cf. definition \ref{def: affine_schematic}) and each pullback of the form:
                        $$U \x_X \Spec A$$
                    is not only affine (say, isomorphic to $\Spec B$), but also comes from a localisation ring map, i.e. there must exist a multiplicative submonoid $W \subseteq A$ such that:
                        $$B \cong W^{-1}A$$
                    \item If furthermore, $U$ is of finite type (respectively, of finite presentation) over $X$ in addition to being a localisation of $X$, then we shall say that $U$ is \textbf{essentially of finite type} (respectively, \textbf{essentially of finite presentation}).
                \end{enumerate}
            \end{definition}
            \begin{remark}
                Given a base scheme $S$ (as in convention \ref{conv: motivic_category_scheme_convention}), there exists natural subcategories of $\Sch_{/S}$, denoted by $\Sch_{/S}^{\ess.\ft}$ and $\Sch_{/S}^{\ess.\fp}$, whose objects are (respectively) schemes essentially of finite type and essentially of finite presentation over $S$. It is not hard to see that these categories admit finite products.
            \end{remark}
            
            \subsubsection{The motivic dg-category}
                \begin{claim}
                    For any base scheme $S$ (as in convention \ref{conv: motivic_category_scheme_convention}), quasi-projective smooth $S$-schemes form a category has a natural symmetric monoidal structure given by products (i.e. fibred products over $S$).
                \end{claim}
                    \begin{proof}
                        This amounts to proving that binary products (over $S$) of quasi-projective smooth $S$-schemes are once more quasi-projective and smooth over $S$, since $S$ is so as a scheme over itself. Preservation of smoothness is easy (cf. proposition \ref{prop: compositions_and_base_changes_of_smooth_morphisms}), so we shall focus on showing preservation of quasi-projectivity. For this, recall that an $S$-scheme $\pi: X \to S$ is quasi-projective if and only if it is of finite type and there exists a $\pi$-relatively ample line bundle. Being of finite type is obviously preserved by binary products, so it remains to check the existence of relatively ample line bundles over binary products. For this, we shall verify the conditions spelled out in definition \ref{def: relatively_ample_line_bundles}:
                            \begin{itemize}
                                \item Any structural morphism $\pi: X \to S$ has already been assumed to be separated, and by being smooth, it must also be of finite type \textit{a fortiori}. Both these properties are preserved under binary products. Thus, we need only to prove that being universally closed is preserved under binary products. But this is a straightforward consequence of the definition. Therefore, properness is preserved by binary products and hence one can define relatively ample line bundles on binary products.
                                \item Now, consider two quasi-projective smooth $S$-schemes $\pi_1: X_1 \to S$ and $\pi_2: X_2 \to S$; choose two relatively ample line bundles $\calL_1$ and $\calL_2$ thereon. We thus claim that:
                                    $$\pi_1^*\calL_1 \tensor_{\calO_{X_1 \x_S X_2}} \pi_2^*\calL_2$$
                                is $\pi_1 \x \pi_2$-relatively ample. 
                            \end{itemize}
                    \end{proof}
            
                \begin{convention}
                    From now on, we work over a \textit{reduced} base scheme $S$. Furthermore, let us fix a symmetric monoidal subcategory $\V \subseteq \Sch_{/S}^{\smooth, \qproj, \ess.\ft}$ that is closed under all coproducts that exist in $\Sch_{/S}$. 
                \end{convention}
                
                \begin{definition}[The category of smooth sections] \label{def: category_of_smooth_sections}
                    For us, $\Sm(\V)$ shall be used for denoting the category fibred over $\Sch_{/S}$, wherein:
                        \begin{itemize}
                            \item objects over each $X \in \Sch_{/S}$ are smooth sections $\sigma: X \to X'$, which we shall view as commutative diagrams of the form:
                                $$
                                    \begin{tikzcd}
                                    	& {Y} \\
                                    	X & X
                                    	\arrow["\pi", from=1-2, to=2-2]
                                    	\arrow["{\sigma}", from=2-1, to=1-2]
                                    	\arrow["{\id_X}", from=2-1, to=2-2]
                                    \end{tikzcd}
                                $$
                            \item and morphisms are commutative diagrams of the form:
                                $$
                                    \begin{tikzcd}
                                    	Y && {Y'} \\
                                    	& X \\
                                    	& X
                                    	\arrow["\pi"', from=1-1, to=3-2]
                                    	\arrow["\sigma"', from=2-2, to=1-1]
                                    	\arrow["{\id_X}"{description}, from=2-2, to=3-2]
                                    	\arrow[from=1-1, to=1-3]
                                    	\arrow["{\sigma'}", from=2-2, to=1-3]
                                    	\arrow["{\pi'}", from=1-3, to=3-2]
                                    \end{tikzcd}
                                $$
                        \end{itemize}
                    We call this category the category of \textbf{smooth liftings} over $S$.
                \end{definition}
        
        \subsection{Motivic cohomology}
        
        \subsection{Relations to K-theory}
        
        \subsection{Duality}
        
        \subsection{Realisation functors out of the motivic category}
        
        \subsection{Comparison theorems}
        
    