\chapter{Motivic Hodge theory}
    \begin{abstract}
        
    \end{abstract}
    
    \minitoc
    
    \section{Weil Cohomology Theories}
        \subsection{Fantastic cohomology theories and where to find them}
            \subsubsection{Some intersection theory; Chow Motives}
                \begin{remark}[Categories of smooth projective schemes] \label{remark: categories_of_smooth_projective_schemes}
                    Fix a ground field $k$. 
                    
                    Inspired by Serre's GAGA Theorem - which relates smooth projective varieties over $\Spec \bbC$ to (compact) complex analytic manifolds - we shall try to set up the so-called Weil Cohomology Theories so that they would behave well over smooth projective (algebraic) schemes over fields such as $k$ (not that we would object to these cohomology theories working over more general schemes, but one should also be reasonable with one's expectations). For that, we shall need to first see if smooth projective algebraic schemes form a category (otherwise, what even is the point ?); also, notice how even with two additional adjectives, the class smooth projective algebraic schemes still includes a lot of important examples, notable among which are elliptic curves and higher dimensional abelian varieties.
                    
                    Thankfully, we do have a category of smooth projective algebraic schemes over any given base field $k$, which we denote by $\Sch_{/\Spec k}^{\smooth, \proj}$. To see why this is the case, recall firstly that thanks to the universal property of the $\Proj$-construction, a projective schemes $X$ is nothing but an $\N$-filtration of schemes (i.e. a diagram:
                        $$X: \N \to \Sch$$
                    of shape $\N$ whose transition maps are \href{https://stacks.math.columbia.edu/tag/01L1}{\underline{monomorphism of schemes}}; note in particular, that any immersion, closed or open, is a monomorphism \cite[\href{https://stacks.math.columbia.edu/tag/01L7}{Tag 01L7}]{stacks}); by abstract nonsense, $X$ is thus simply a functor that preserves monomorphisms, as every arrow in $\N$ is a monomorphism.  
                \end{remark}
                
                \begin{definition}[Algebraic cycles] \label{def: algebraic_cycles}
                    Let $S$ be a Noetherian base scheme and let $f: X \to S$ be an $S$-scheme of finite type. Recall also that Noetherian schemes form a full subcategory of $\Sch$; let us denote it by $\Sch^{\Noeth}$. 
                        \begin{enumerate}
                            \item \textbf{(Relative cycles):} 
                            \item \textbf{(The Chow functors):} A \textbf{presheaf of $d$-dimensional relative cycles} over $f: X \to S$ (or a \textbf{$d$-dimensional Chow functor} over $f: X \to S$) shall be a functor:
                                $$\Chow_{X/S}(d, -): \Sch_{/S}^{\Noeth} \to \Sets$$
                            that associates to Noetherian $S$-schemes $g: T \to S$ the set $\Chow_{X/S}(d, T)$ of $d$-dimensional relative cycles on the $T$-scheme $X \x_{f, S, g} T \to T$.
                        \end{enumerate}
                \end{definition}
                
            \subsubsection{Weil Cohomology Theories}    
                \begin{definition}[Weil Cohomology Theories] \label{def: weil_cohomology_theories}
                    Let $k$ be an arbitrary base field, and let $F$ be a field of characteristic $0$, which will serve as a so-called \say{field of coefficients}. A \textbf{Weil Cohomology Theory} over $k$ is thus a contravariant functor:
                        $$\H^*: (\Sch_{/\Spec k}^{\smooth, \proj})^{\op} \to [\Z, {}_F\Vect]$$
                    from the category of smooth projective algebraic schemes over $\Spec k$ into the category of $\Z$-graded $F$-vector spaces (which are just diagrams of shape $\Z$ of $F$-vector spaces) that satisfies the following axioms:
                        \begin{enumerate}
                            \item \textbf{(Finiteness):} Given a smooth projective algebraic scheme $X$ over $\Spec k$, we shall want all the vector spaces in the diagram:
                                $$
                                    \H^*(X) =
                                    \left(
                                        \begin{tikzcd}
                                        	\cdots & {\H^{-1}(X)} & {\H^0(X)} & {\H^1(X)} & \cdots
                                        	\arrow[from=1-2, to=1-3]
                                        	\arrow[from=1-3, to=1-4]
                                        	\arrow[from=1-4, to=1-5]
                                        	\arrow[from=1-1, to=1-2]
                                        \end{tikzcd}
                                    \right)
                                $$
                            to be \textit{finite-dimensional}. Additionally, we would like to require that:
                                $$
                                    \dim_F \H^i(X) = 
                                    \begin{cases}
                                        \text{$n_i \not = 0$ if $0 \leq i \leq 2\dim X$}
                                        \\
                                        \text{$0$ otherwise}
                                    \end{cases}
                                $$
                            \item \textbf{(Poincar\'e Duality):} This is to say that there is an isomorphism, called the \textbf{trace map}:
                                $$\int_X(-): \H^{2 \dim X}(X) \cong F$$
                            and that for each $i \in \Z$, there exists a non-degenerate bilinear pairing:
                                $$\<\cdot \mid \cdot\>: \H^i(X) \x \H^{2\dim X - i}(X) \to \H^{2 \dim X}(X)$$
                            which establishes an isomorphism between $\H^i(X) \x \H^{2\dim X - i}(X)$ and $F$ via the trace map $\int_X$. 
                            \item \textbf{(K\"unneth Formula/Monoidality):} For all $i \in \Z$, we have:
                                $$\H^i(X \x_{\Spec k} Y) \cong \H^i(X) \tensor_F \H^i(Y)$$
                            \item \textbf{(Algebraic cycles):}
                        \end{enumerate}
                \end{definition}
                \begin{remark}[Why these axioms ?] \label{remark: motivation_for_motives}
                    For the most part, the axioms laid out in definition \ref{def: weil_cohomology_theories} are there because we want Weil Cohomology Theories to behave how we have come to expect reasonable \say{geometric} cohomology theories to. In particular, we want for (smooth and projective) schemes cohomology theories that act more or less like singular cohomology or the classical de Rham cohomology for manifolds. In fact, most famous examples of Weil Cohomology Theories were conceived in the images of singular and de Rham cohomologies: \'etale cohomology (cf. chapter \ref{chapter: etale_cohomology_1}) is supposed to be the topological cohomology theory that works for schemes - especially those in positive characteristics - and crystalline cohomology exists so that we might have a \say{differential geometry} of schemes (cf. chapter \ref{chapter: crystals}). 
                \end{remark}
                \begin{remark}[What about the Lefschetz Conditions ?] \label{remark: lefschetz_axioms}
                    
                \end{remark}
                \begin{example}
                    \noindent
                    \begin{enumerate}
                        \item \textbf{(Over characteristic $0$):} 
                            \begin{enumerate}
                                \item \textbf{(Betti cohomology):}
                                \item \textbf{(de Rham cohomology):}
                                \item \textbf{(A counter-example: Zariski cohomology):}
                            \end{enumerate}
                        \item \textbf{(Over positive characteristics):}
                            \begin{enumerate}
                                \item \textbf{($\ell$-adic cohomology):}
                                \item \textbf{(Crystalline cohomology):}
                            \end{enumerate}
                    \end{enumerate}
                \end{example}
        
        \subsection{Mixed motives}
            \subsubsection{Introduction}
                Grothendieck once dreamed of a categorical framework for a universal cohomology theory for algebraic varieties, something which was eventually called \say{\textbf{the category of motives}}. Even though the mere existence of such a category remains unknown, many desirable features has been described over the past few decades:
                    \begin{itemize}
                        \item The category of mixed motives, which we shall denote by $\MM$, is to be fibred over the category $\Sch$ of schemes.
                        \item 
                        \item 
                    \end{itemize}
            
            \subsubsection{The motivic category}
    
    \section{Complex Hodge theory}
        \subsection{Hodge structures}
            \subsubsection{Pure Hodge structures on compact complex analytic manifolds}
                \begin{theorem}[Grothendieck's algebraic de Rham cohomology] \label{theorem: de_rham_cohomology}
                    Let $X$ be a smooth scheme over $\Spec \bbC$ and let $X^{\an}$ denote the analytic manifold whose underlying set is $X(\bbC)$. Then, one has the following comparison quasi-isomorphism between algebraic and complex-analytic de Rham cohomologies:
                        $$H^*_{\dR}(X) \cong_{\qis} H^*_{\dR}(X^{\an})$$
                \end{theorem}
            
            \subsubsection{Mixed Hodge structures}
        
        \subsection{Periods}
        
        \subsection{Moduli of Hodge structures}
        
    \section{Motivic \texorpdfstring{$p$}{}-adic Hodge theory}
        \subsection{The Ax-Tate-Sen Theorem}
    
        \subsection{The Hodge-Tate Decomposition and the \'etale-de Rham comparison}
            First, let us recall some relevant facts about Witt vectors.
            \begin{convention}[Relative Witt vectors] \label{conv: relative_witt_vectors}
                Let $p$ be a prime number. If $E/\Q_p$ is a finite $p$-adic number field with residue field $\F_q$ (for $q$ some power of $p$) and if $F$ is a perfectoid field of characteristic $p$ (which is \textit{a priori} perfect and therefore can contain $\F_q$; consider fields such as $\F_q(\!(t^{\frac{1}{p^{\infty}}})\!)$ or $\widehat{\overline{\F_q(\!(t)\!)}}$ for example), then let us write:
                    $$\W_{E^{\circ}}(F) \cong \W(F) \tensor_{\W(\F_q)} E^{\circ}$$
                for the base change along $E^{\circ} \to $ of the ring $\W(F)$ of (unramified) $p$-typical Witt vectors with coefficients in $F$ along the canonically induced arrow $\W(\F_q) \to E^{\circ}$. For instance, when $q = p$ we have:
                    $$\W_{E^{\circ}}(\F_p(\!(t^{\frac{1}{p^{\infty}}})\!)) \cong \Z_p[p^{\frac{1}{p^{\infty}}}] \tensor_{\Z_p} \Z_p \cong \Z_p[p^{\frac{1}{p^{\infty}}}]$$
                Slightly more generally, one can speak of a relative $p$-typical Witt vector functor from the category of perfect commutative $\F_q$-algebras to the category ${}^{E^{\circ}/}\Comm\Alg^{\wedge}$ of $p$-adically complete $E^{\circ}$-algebras:
                    $$\W_{E^{\circ}}: {}^{\F_q/}\Comm\Alg^{\perf} \to {}^{E^{\circ}/}\Comm\Alg^{\wedge}$$
                which in particular, gives us the following canonical arrow:
                    $$\W_{E^{\circ}}(F^{\circ}) \to \W_{E^{\circ}}(F)$$
            \end{convention}
        
            \begin{remark}[Witt vectors over perfect rings] \label{remark: witt_vectors_over_perfect_rings}
                \noindent
                \begin{enumerate}
                    \item One somewhat non-trivial fact to keep in mind is that if $B$ is a \textit{perfect} $\F_q$-domain (for some power $q$ of a prime $p$) with field of fractions $K$, then we have the following natural characterisation of the ring of $p$-typical Witt vectors over $K$ (which we note to be trivially perfect as an $\F_q$-algebra):
                    $$(\W(B)_{(p)})^{\wedge} \cong \W(K)$$
                    In particular, $\W(B)_{(p)}$ is an unramified extension of $\W(\F_q)$. We refer the reader to \cite[Proposition 5.2]{shimomoto2014witt} for a proof.
                    \item This result extends trivially to the relative setting, as relative Witt vectors are defined via a pushout (cf. convention \ref{conv: relative_witt_vectors}), which is in particular a finite limit, and since adic completions are filtered limits, the two procedures can be exchanged, which gives:
                        $$(\W_{E^{\circ}}(B)_{(p)})^{\wedge} \cong \W_{E^{\circ}}(K)$$
                    where now, $E$ is a finite $p$-adic number field with residue field $\F_q$ (note that its pseudo-uniformiser is actually just $p$, like $\Q_p$).
                    \item Another notable property of rings (relative) Witt vectors is that if $B$ is a perfect $\F_q$-algebra (not necessarily a domain), then $\W_{E^{\circ}}(B)$ is $p$-torsion-free, and:
                        $$\W_{E^{\circ}}(B)/p \cong B$$
                \end{enumerate}
            \end{remark}
            \begin{example}
                One might think of the following example:
                    $$(\W_{E^{\circ}}(\F_q)_{(p)})^{\wedge} \cong \W_{E^{\circ}}(\F_q) \cong E^{\circ}$$
                (note that $E^{\circ}$ is \textit{a priori} complete), or the following slightly subtler one:
                    $$( \W_{E^{\circ}}( \F_q[\![t^{\frac{1}{p^{\infty}}}]\!] )_{(p)} )^{\wedge} \cong E^{\circ}[p^{\frac{1}{p^{\infty}}}]^{\wedge} \cong \W_{E^{\circ}}( \F_q(\!(t^{\frac{1}{p^{\infty}}})\!) ) \cong E(p^{\frac{1}{p^{\infty}}})^{\wedge, \circ}$$
                (indeed, $E^{\circ}[p^{\frac{1}{p^{\infty}}}]^{\wedge} \cong E^{\circ}[\![t]\!]/(t^{p^{\infty}} - p)$ and so $E^{\circ}[p^{\frac{1}{p^{\infty}}}]^{\wedge}/p \cong \F_q[\![t^{\frac{1}{p^{\infty}}}]\!]$). In both cases, note that $\F_q$ and $\F_q[\![t^{\frac{1}{p^{\infty}}}]\!]$ are both perfect domains (the latter being the $p$-tilt of $\F_q[\![t]\!]$).
            \end{example}
        
            \subsubsection{The Hodge-Tate Decomposition via perfectoid spaces}
                \paragraph{Scholze's de Rham period sheaf}
                    \begin{definition}[de Rham period sheaves] \label{def: de_rham_period_sheaves}
                        Let $K$ be a perfectoid field of mixed characteristic $(0, p)$ and let $X$ be a perfectoid space over $\Spa K$. Additionally, recall that the integral subsheaf $\calO_{X^{\flat}}^+$ is \textit{a priori} perfect (over characteristic $p$). We are interested in the following commutative ring objects of the pro-\'etale topos $X_{\proet}$:
                            \begin{enumerate}
                                \item \textbf{(Fontaine's infinitesimal period rings):} There is first of all the so-called \textbf{infinitesimal period ring of Fontaine}, usually denoted by $\A_{\inf}$, and is defined to be the ring of Witt vectors $\W(\calO_{X^{\flat}}^+)$. One also would usually be interested in its rational localisation, namely $\B_{\inf} := \A_{\inf}[1/p]$. 
                                
                                We care also about the canonical map $\theta: \A_{\inf} \to \calO_X^+$, which extends naturally to a map $\Theta: \B_{\inf} \to \calO_X$.
                                \item \textbf{(Scholze's de Rham period rings):} Scholze, in \cite[Definition 6.1]{scholze2012padic} then defined his \textbf{de Rham period ring} as the formal completion $(\B_{\inf}, \ker \Theta)^{\wedge}$; we will denote this period ring by $\B_{\dR}^+$. There exists also a rational localisation of this period ring, though this is a less than trivial fact.
                            \end{enumerate}
                    \end{definition}
                    
                    \begin{proposition}[$\theta$ is surjective] \label{prop: theta_is_surjective}
                        Let $K$ be a perfectoid field of characteristic $0$ and let $X = \Spa(R, R^+)$ be an affinoid perfectoid space over $\Spa K$.
                        \begin{enumerate}
                            \item The canonical map $\theta: \A_{\inf} \to R^+$ is a surjective continuous ring homomorphism.
                            \item $\ker \Theta$ is a non-zero principal ideal of $\B_{\inf}$.
                        \end{enumerate}
                    \end{proposition}
                        \begin{proof}
                            \noindent
                            \begin{enumerate}
                                \item \textbf{(Surjectivity of $\theta$):} 
                                \item \textbf{(Principality of $\ker \Theta$):} First of all, because localisation is a colimit, the map $\Theta: \B_{\inf} \to \calO_X$ must also be surjective as a consequence of $\theta: \A_{\inf} \to \calO_X^+$ being surjective; this ensures that the ideal $\ker \Theta$ is not trivial.
                            \end{enumerate}
                        \end{proof}
                    \begin{corollary}[Rational de Rham period rings] \label{coro: rational_de_rham_period_rings}
                        \noindent
                        \begin{enumerate}
                            \item Proposition \ref{prop: theta_is_surjective} applies also to non-affinoid perfectoid spaces.
                            \item Additionally, not only is the completion $\B_{\dR}^+ := (\B_{\inf}, \ker \Theta)^{\wedge}$ adic, but also, it admits a rational localisation: if $t \in \ker \Theta$ is any generator, then we can define $\B_{\dR} := \B_{\dR}^+[1/t]$. 
                        \end{enumerate}
                    \end{corollary}
                    
                    \begin{remark}
                        
                    \end{remark}
                
                \paragraph{The Hodge-Tate Decomposition}
            
            \subsubsection{The \'etale-de Rham comparison}
        
        \subsection{The \'etale-crystalline comparison}