\chapter{\texorpdfstring{$p$}{}-adic \'etale cohomology}
    \begin{abstract}
        
    \end{abstract}
    
    \minitoc
    
    \section{Pro-\'etale cohomology for schemes: "Another fine cohomology theory to add to my collection"}
        Despite its successes - particularly its involvement in the proofs of the Weil Conjectures - \'etale cohomology has its shortcomings. For one, \'etale cohomology modules are not actually computed over \'etale sites, but rather, as scalar extensions (to $\overline{\Q_{\ell}}$ in the case of $\ell$-adic cohomology for instance) of the actually \'etale cohomology modules over \'etale sites. Pro-\'etale cohomology theories exist so that such issues could be rectified. 
        
        \begin{remark}[What is the pro-\'etale topology ?]
            For now, we refer the reader to \cite[Definition 4.1.1 and Remark 4.1.3]{bhatt_scholze_2014_pro_etale}.
        \end{remark}
        
        \subsection{The pro-\'etale topology for schemes}
        
        \subsection{Infinite Galois categories and pro-\'etale fundamental groups}
            \subsubsection{The abstract framework}
                \begin{definition}[Infinite Galois categories] \label{def: infinite_galois_categories}
                    \noindent
                    \begin{itemize}
                        \item \textbf{(Infinite Galois categories):} An \textbf{infinite Galois category} is defined via the data contained in a pair $(\calG, F)$ consisting of:
                        \begin{itemize}
                            \item a \textit{cocomplete and \textit{finitely} complete} small category $\calG$, wherein objects can all be written as a coproduct of connected objects, and
                            \item a functor $F: \calG \to \Sets$, called the \textbf{fibre functor}, which we shall require to reflect and preserve all colimits and all finite limits\footnote{In \cite[Definition 7.2.1]{bhatt_scholze_2014_pro_etale}, it is furthermore required that the fibre functor would be faithful. However, because any functor that preserves and reflect monomorphisms must also be faithful, we need not make this requirement.}.
                        \end{itemize}
                        \item \textbf{(Galois objects):} An object $Y$ of an infinite Galois category $\calG$ is a \textbf{Galois object} if and only if $Y/\Aut(Y)$ is terminal as an object of $\calG$ (which exists as $\calG$ is finitely complete by definition).
                        \item \textbf{(Galois functors):} A \textbf{Galois functor} is a functor $\Phi: \calG \to \calG'$ between infinite Galois categories $(\calG, F), (\calG', F')$ which preserves connected objects, colimits, finite limits, and commute with the fibre functors in the following manner:
                            $$
                                \begin{tikzcd}
                                	\calG && {\calG'} \\
                                	& \Sets
                                	\arrow["F"', from=1-1, to=2-2]
                                	\arrow["{F'}", from=1-3, to=2-2]
                                	\arrow["\Phi", from=1-1, to=1-3]
                                \end{tikzcd}
                            $$
                    \end{itemize}
                \end{definition}    
                \begin{definition}[Fundamental groups of infinite Galois categories] \label{def: fundamental_groups_of_infinite_galois_categories}
                    The \textbf{fundamental group} of a given infinite Galois category $(\calG, F)$, denoted by $\pi_1(\calG, F)$, is defined to be the automorphism group $\Aut(F)$\footnote{Compare this to definition \ref{def: fundamental_groups_of_finite_galois_categories} and note the lack of profinite completion the current situation.}.
                \end{definition}
                Arbitrary infinite Galois categories turn out to be rather pathological, so we shall introduce a more refined notion, namely that of infinite Galois categories which are \textbf{tame}.  
                \begin{definition}[Tame infinite Galois categories] \label{def: tame_infinite_galois_categories}
                    \footnote{Note that this is slightly different from \cite[Definition 7.2.4]{bhatt_scholze_2014_pro_etale}, in that we require furthermore that the fibre functor preserves connectedness instead of merely that $\pi_1(\calG, F)$ acts transitively on $F(X)$ whenever $X$ is connected.} One says that an infinite Galois category $(\calG, F)$ is \textbf{tame}\footnote{The terminology comes from the fact given any Galois extension $L/K$, the corresponding Galois group $\Gal(L/K)$ acts transitively on $L$ (cf. \cite[Proposition I.9.1]{neukirch_2010_algebraic_number_theory}).} whenever the fibre functor $F: \calG \to \Sets$ preserves connectedness\footnote{Compare this to corollary \ref{coro: fundamental_groups_of_finite_galois_categories_act_transitively_on_connected_objects}.} and $\pi_1(\calG, F)$ acts transitively on $F(X)$ for all $X$ connected.
                \end{definition}
                
                Let us now introduce the notion of Noohi groups, which shall serve in this new theory of infinite Galois categories as replacements for the more restrictive notion of profinite groups. The idea is that, similar to how fundamental groups of finite Galois categories, the fundamental group of a \textit{tame} infinite Galois category is Noohi. We shall also see that the notion of Noohi groups subsumes the notion of profinite groups and as such, so too does the notion of tame infinite Galois categories subsume that of finite Galois categories.
                \begin{definition}[Noohi groups] \label{def: noohi_groups}
                    \cite[Defintion 7.1.1]{bhatt_scholze_2014_pro_etale} A \textbf{Noohi group} is a Hausdorff topological group $G$ such that $G \cong \Aut(\oblv_G)$ \textit{as topological groups} (here, $\Aut(\oblv_G)$ carries the compact-open topology), with $\oblv_G: G\-\Sets \to \Sets$ being the forgetful functor from the category of sets with continuous $G$-actions into the category of sets. In addition, an (in)finite Galois category that is Galois-equivalent to $G\-\Sets$ for some Noohi group $G$ shall be called a \textbf{Noohi category}.
                \end{definition}
                \begin{remark}[Completions of topological groups] \label{remark: completions_of_topological_groups}
                    If $X$ is a topological space then we shall denote its Cauchy completion by $X^+$ (i.e. $X^+$ is the minimal topological space containing $X$, in which all Cauchy filters converge). 
                        \begin{itemize}
                            \item By \cite[Theorem 3.6.10]{topological_groups_and_related_structures}, we know that such a completion exists and is unique for all topological groups; furthermore, every topological group is continuously isomorphic to a dense subgroup of its Cauchy completion. 
                            \item \cite[Theorem 3.6.22]{topological_groups_and_related_structures} Any small product of complete topological groups is also a complete topological group.
                            \item Any closed subgroup of a complete topological group is complete.
                        \end{itemize}
                \end{remark}
                \begin{proposition}[Noohi groups are complete] \label{prop: noohi_groups_are_complete}
                    \cite[Proposition 7.1.5]{bhatt_scholze_2014_pro_etale} Let $G$ be a Hausdorff topological group with a basis generated by open subsets. Then, there is an homeomorphic group isomorphism $\psi: \Aut(\oblv_G) \to G^+$. In fact, a Hausdorff topological group is a Noohi group if and only if it is complete and has a basis generated by open subgroups.
                \end{proposition}
                \begin{lemma}[Fundamental groups of infinite Galois categories are Noohi] \label{lemma: fundamental_groups_of_infinite_galois_categories_are_noohi}
                    A topological group is Noohi if and only if it is the fundamental group of an infinite Galois category.
                \end{lemma}
                    \begin{proof}
                        If $G$ is a Noohi group then by definition $\Aut(\oblv_G) \cong G$. Thus, it shall suffice to demonstrate that the pair $(G\-\Sets, \oblv_G)$ is an infinite Galois category; this is routine (cf. \cite[Section 3]{nlab:category_of_G_sets}), so we shall leave it up to our readers as an exercise.
                    
                        Conversely, if $(\calG, F)$ is an infinite Galois category then it shall suffice to show that $\pi_1(\calG, F)$ will be a closed subgroup of $\prod_{Y \in \Ob(\calG)} \Aut(F(Y))$ (which is a well-defined group since it is a small product of groups), as we can demonstrated this product group to be Noohi. For this, we use the fact that $\Aut(S)$ for any set $S$ is complete in the compact-open topology (cf. \cite[Lemma 7.1.4]{bhatt_scholze_2014_pro_etale}), and that $\Aut(S)$ admits a basis generated by open subgroups (cf. \cite[\href{https://stacks.math.columbia.edu/tag/0BMC}{Tag 0BMC}]{stacks}): by combining this with the fact that products of complete groups are complete (cf. remark \ref{remark: completions_of_topological_groups}), one infers that $\prod_{Y \in \Ob(\calG)} \Aut(F(Y))$ endowed with the product topology is complete and admits a basis generated by open subgroups, meaning that it is Noohi (cf. proposition \ref{prop: noohi_groups_are_complete}). Then, to show that $\pi_1(\calG, F)$ is a closed subgroup of $\prod_{Y \in \Ob(\calG)} \Aut(F(Y))$, we shall show that it is complete (see remark \ref{remark: completions_of_topological_groups}). For this, see \cite[\href{https://stacks.math.columbia.edu/tag/0BMR}{Tag 0BMR}]{stacks}.
                    \end{proof}
                \begin{corollary}[Noohi categories are tame] \label{coro: noohi_categories_are_tame}
                    Noohi categories, as in definition \ref{def: noohi_groups}, are tame infinite Galois categories. 
                \end{corollary}
                    
                \begin{remark}[Action of fundamental groups on fibres] \label{remark: action_of_fundamental_groups_on_fibres_infinite_galois_categories}
                    By arguing as in remark \ref{remark: action_of_fundamental_groups_on_fibres_finite_galois_categories}, one sees that the fibre functor $F: \calG \to \Sets$ of any infinite Galois category $(\calG, F)$ must factor through $\pi_1(\calG, F)\-\Sets$ in the following manner:
                        $$
                            \begin{tikzcd}
                            	\calG && {\pi_1(\calG, F)\-\Sets} \\
                            	& \Sets
                            	\arrow[dashed, from=1-1, to=1-3]
                            	\arrow["F"', from=1-1, to=2-2]
                            	\arrow["{\oblv_{\pi_1(\calG, F)}}", from=1-3, to=2-2]
                            \end{tikzcd}
                        $$
                    As a consequence, functions $F(X) \to F(Y)$ coming from morphisms $X \to Y$ in $\calG$ are automatically $\pi_1(\calG, F)$-equivariant. 
                \end{remark}
                \begin{theorem}[The Infinite Galois Correspondence] \label{theorem: infinite_categorical_galois_correspondence}
                    \cite[Theorem 7.2.5(3)]{bhatt_scholze_2014_pro_etale} Any tame infinite Galois categories $(\calG, F)$, the functor $F: \calG \to \pi_1(\calG, F)\-\Sets$ is a Galois equivalence\footnote{As such, every tame infinite Galois category is Noohi.} (we equip $\pi_1(\calG, F)\-\Sets$ with the forgetful functor to $\Sets$ to make it a Noohi category; cf. corollary \ref{coro: noohi_categories_are_tame}).
                \end{theorem}
                    \begin{proof}
                        Because $F$ reflects monomorphisms (or more generally, all finite limits, as it is an isomorphism-reflecting left-exact functor; cf. definition \ref{def: infinite_galois_categories}) and because any function $f: S \to T$ can be viewed as a unique monomorphism $\Gamma_f: S \to S \x T$ such that $\pr_1 \circ \Gamma_f = \id_S$, any morphism $p: X \to Y$ in $\calG$ can be viewed as a unique monomorphism $\Gamma_p: X \to X \x Y$ such that $\pr_1 \circ \Gamma_p = \id_X$, which comes from. By combining this with the fact that $F$ commutes with finite limits, and that functions $F(X) \to F(Y)$ coming from morphisms $X \to Y$ in $\calG$ are automatically $\pi_1(\calG, F)$-equivariant (cf. remark \ref{remark: action_of_fundamental_groups_on_fibres_infinite_galois_categories}), we obtain the following bijections, which proves that $F: \calG \to \pi_1(\calG, F)\-\Sets$ is fully faithful:
                            $$
                                \begin{aligned}
                                    \calG(X, Y) & \cong \{\text{Monomorphisms $\Gamma_p: X \to X \x Y$ such that $\pr_1 \circ \Gamma_p = \id_X$}\}
                                    \\
                                    & \cong \{\text{Monomorphisms $\Gamma_f: F(X) \to F(X) \x F(Y)$ such that $\pr_1 \circ \Gamma_f = \id_{F(X)}$}\} 
                                    \\
                                    & \cong \pi_1(\calG, F)\-\Sets(F(X), F(Y))
                                \end{aligned}
                            $$
                        As $\pi_1(\calG, F)\-\Sets$ is an infinite Galois category, every object $S \in \pi_1(\calG, F)\-\Sets$ admits a decomposition $S \cong \coprod_{i \in I_S} S_i$ into connected objects $S_i \in \pi_1(\calG, F)\-\Sets$. But the functor $F: \calG \to \pi_1(\calG, F)\-\Sets$ reflects colimits by definition, so such a coproduct gives rise to a corresponding coproduct $Y_S := \coprod_{i \in I_S} Y_i$ in $\calG$, and since $\calG$ has all colimits, $Y_S$ is an object thereof. As such, for any $S \in \pi_1(\calG, F)\-\Sets$, there exists a corresponding object $Y_S \in \calG$ such that $F(Y_S) \cong S$. This implies that the functor $F: \calG \to \pi_1(\calG, F)\-\Sets$ is essentially surjective, which when combined with the fact that it is fully faithful (as shown above), implies that this functor is an equivalence of categories. 
                        
                        It remains to show that this equivalence of categories between $\calG$ and $\pi_1(\calG, F)\-\Sets$ is a Galois equivalence between Noohi categories. For this, observe that because $\pi_1(\calG, F)$ is a Noohi group (cf. lemma \ref{lemma: fundamental_groups_of_infinite_galois_categories_are_noohi}), and because $G\-\Sets$ is a Noohi category for all Noohi groups $G$ (cf. corollary \ref{coro: noohi_categories_are_tame}), $F: \calG \to \pi_1(\calG, F)\-\Sets$ is a Galois functor, by virtue of being a functor between (tame) infinite Galois categories that preserves colimits, finite limits, and connected objects.
                    \end{proof}
                
            \subsubsection{Pro-\'etale fundamental groups}
        
    \section{Finiteness of \'etale cohomology for smooth proper rigid-analytic varieties}
    
    \section{Interlude: Condensed mathematics} \label{section: condensed_mathematics}
        \begin{remark}[Strong limit cardinals]
            We will be using the notion of strong limit cardinals often. For details on the notion, see definition \ref{def: limit_cardinal}.
        \end{remark}
    
        \subsection{Basics of condensed mathematics}
            \subsubsection{Condensed sets}
                \begin{definition}[Condensation] \label{def: condensation}
                    Let $\kappa$ be a fixed strong limit cardinal and let $\C$ be a hypercomplete $\infty$-category with enough $\kappa$-small limits and enough $\kappa$-small filtered colimits. We then define so-called \textbf{condensed objects} of $\C$ to be $\C$-valued sheaves over the $\kappa$-small pro-\'etale site of a point (i.e. the pro-\'etale site of the spectrum of a field). 
                    
                    Clearly condensed objects of a given $\infty$-category $\C$ satisfying the above conditions form a category. We shall denote it by $\C^{\cond}$.
                \end{definition}
                \begin{remark}
                    For now, we refer the reader to \cite[Definition 4.1.1 and Remark 4.1.3]{bhatt_scholze_2014_pro_etale} for the definition of pro-\'etale coverages. In particular, recall that the $\kappa$-small pro-\'etale site of a point is equivalent to the site $\Pro_{\kappa}(\Sets^{\fin})$ of $\kappa$-small profinite sets (\textit{viewed as totally disconnected $\kappa$-small compact Hausdorff spaces}), whose coverage is generated by jointly surjective families. 
                \end{remark}
                \begin{example}
                    \noindent
                    \begin{itemize}
                        \item \textbf{(\textit{Small} condensed sets):} The pro-\'etale topos $\Sh(*_{\kappa\-\proet})$ over a point is, by definition, the category of sheaves of sets on the pro-\'etale site of a point. Therefore, this topos is the category of \textbf{$\kappa$-small condensed sets}. Whenenver we wish to put emphasis on the fact that $\Sh(*_{\kappa\-\proet})$ is actually the category of $\kappa$-small condensed sets, we will write $(\Sets^{\cond})^{< \kappa}$ instead.
                        \item \textbf{(Condensed abelian groups and modules):} If $R$ is a condensed commutative ring, then the category $R\mod^{\cond}$ of condensed $R$-modules is a \href{https://ncatlab.org/nlab/show/Grothendieck+category}{\underline{Grothendieck category}}; on the other hand, topological abelian groups even fails to form an abelian category, and as we well know: no abelian categories means no homological algebra. This is a biggy, so we shall bestow upon it the dignity of theorem-hood (see theorem \ref{theorem: abelian_categories_of_condensed_modules}).
                    \end{itemize}
                \end{example}
                \begin{remark}[Condensation and profiniteness] \label{remark: condensation_and_profiniteness}
                    \noindent
                    \begin{enumerate}
                        \item Fix a strong limit cardinal $\kappa$. The $\kappa$-pro-\'etale site of a point is equivalent to the category $\Pro_{\kappa}(\Sets^{\fin})$ of $\kappa$-small profinite sets equipped with the coverage generated by jointly surjective finite families of surjective functions. One can show this using the fact that the pro-\'etale site of a point is the same as the pro-\'etale site of the spectrum of a field, and subsequently, the Fundamental Theorem of Galois Theory, namely the fact that the Galois group of any Galois extension $L/K$ is the filtered limit over the galois groups $\Gal(E/K)$ over all \textit{finite} Galois subextensions $E/K$.
                        \item One important fact to keep in mind is that profinite sets are compact and Hausdorff. This will be used in lemma \ref{lemma: sheaves_over_compact_hausdorff_spaces} to show that the sheaf tops over the category of small compact Hausdorff spaces is the same as the sheaf topos of $\kappa$-small condensed sets.
                    \end{enumerate}
                \end{remark}
                \begin{remark}[Set-theoretic technicalities] \label{remark: condensed_sets_set_theoretic_issues}
                    It might seem as though the fixture of a strong limit cardinal $\kappa$ is an unnecessary gimmick and that the issues that one might run into when removing this cardinal bound are purely philosophical. However, because the unbounded pro-\'etale site $*_{\proet}$ (or for that matter, the category of all profinite sets) is large and sheaves on large sites may not form topoi, and because we rely crucially on the premise that the category of condensed sets would be a sheaf topos, we really do need to take these set-theoretic issues seriously and pre-suppose that we are only working over the $\kappa$-small pro-\'etale site of a point.
                \end{remark}
                
                \begin{lemma}[Sheaves over compact Hausdorff spaces] \label{lemma: sheaves_over_compact_hausdorff_spaces}
                    Fix a strong limit cardinal $\kappa$ and denote by $\Sh(\Comp^{< \kappa})$ the sheaf topos over the site of $\kappa$-small compact Hausdorff topological spaces\footnote{This is sometimes referred to as the topos of $\kappa$-small pyknotic sets.} (with coverage given by jointly surjective families of continuous functions). Then, one has the following equivalence of topoi:
                        $$\Sh(\Comp^{< \kappa}) \cong (\Sets^{\cond})^{< \kappa}$$
                    between the aforementioned sheaf topos and the topos of $\kappa$-small condensed sets.
                \end{lemma}
                    \begin{proof}
                        We can make use of the fact that the $\kappa$-small pro-\'etale site of a point is equivalent to the category of $\kappa$-small profinite sets to see that there exist a functor:
                            $$\beta: *_{\kappa\-\proet} \to \Comp^{< \kappa}$$
                        that is naturally isomorphic to the Stone-\v{C}ech Compactification functor restricted from $\Top^{< \kappa}$ down to $\Pro_{\kappa}(\Sets^{\fin})$. 
                    \end{proof}
                
                \begin{definition}[Extremally disconnected sets] \label{def: extrememly_disconnected_sets}
                    Extremally disconnected compact Hausdorff spaces are projective objects (cf. definition \ref{def: projective_and_injective_objects} and proposition \ref{prop: projectives_and_injectives_lifting_property}) in the category of compact Hausdorff spaces.
                \end{definition}
                \begin{example}
                    \noindent
                    \begin{enumerate}
                        \item \textbf{(Stone-\v{C}ech compactifications):}
                        \item \textbf{(A counter-example: the $p$-adics):} For a fixed prime $p$, the $p$-adic rationals $\Q_p$ is only totally disconnected, not extremally disconnected. 
                    \end{enumerate}
                \end{example}
                
                \begin{lemma}[Small condensed sets and extremally disconnected sets] \label{lemma: small_condensed_sets_and_extremally_disconnected_sets}
                    Fix a strong limit cardinal $\kappa$ and denote the topos of sheaves of sets on the site of $\kappa$-small extremally disconnected sets with coverage given by \textit{finite} jointly surjective families by $\Sh(\sfExt^{< \kappa})$. Then, one has the following equivalence of topoi:
                        $$\Sh(\sfExt^{< \kappa}) \cong (\Sets^{\cond})^{< \kappa}$$
                    between the aforementioned sheaf topos and the topos of $\kappa$-small condensed sets.
                \end{lemma}
                    \begin{proof}
                        
                    \end{proof}
                    
                \begin{lemma}[Enlargements of condensed sets] \label{lemma: large_condensed_sets}
                    Fix strong limit cardinals $\kappa < \lambda$ (cf. definition \ref{def: limit_cardinal}). The natural embedding of $(\Sets^{\cond})^{< \kappa}$ into $(\Sets^{\cond})^{< \lambda}$ is compatible with pro-\'etale sheafification. This is to say, the following diagram of topoi commutes:
                        $$
                            \begin{tikzcd}
                            	{(\Sets^{\cond})^{< \kappa}} & {\Psh(*_{\kappa\-\proet})} \\
                            	{(\Sets^{\cond})^{< \lambda}} & {\Psh(*_{\lambda\-\proet})}
                            	\arrow[hook, from=1-1, to=2-1]
                            	\arrow[hook, from=1-2, to=2-2]
                            	\arrow["{{}^{\sh}(-)}"', from=1-2, to=1-1]
                            	\arrow["{{}^{\sh}(-)}"', from=2-2, to=2-1]
                            \end{tikzcd}
                        $$
                \end{lemma}
                    \begin{proof}
                        
                    \end{proof}
                
            \subsubsection{Abelian categories of condensed objects}
                \begin{theorem}[Abelian categories of condensed modules] \label{theorem: abelian_categories_of_condensed_modules}
                    Fix a regular cardinal $\kappa$. Categories of modules $R\mod^{\cond}$ over $\kappa$-condensed commutative rings $R$ satisfy the following Grothendieck homological axioms:
                        \begin{enumerate}
                            \item They are abelian categories.
                            \item \textbf{($AB3$ \& $AB3^*$):} They are $\kappa$-small complete and $\kappa$-small cocomplete.
                            \item \textbf{($AB3$ \& $AB4^*$):} Coproducts of monics remain monic, and dually, products of epics remain epic.
                            \item \textbf{($AB5$):} Filtered colimits of exact sequences remain exact. Furthermore, categories of condensed modules are compactly generated: this is to say, its generator, call it $\Lambda$ is a compact object (i.e. the copresheaf $R\mod^{\cond}(\Lambda, -)$ preserves filtered colimits).
                            \item \textbf{($AB6$):} $\kappa$-small products commute with $\kappa$-small filtered colimits.
                        \end{enumerate}
                \end{theorem}
                    \begin{proof}
                        \noindent
                        \begin{enumerate}
                            \item Categories of condensed modules are categories of internal modules - specifically to the pro-\'etale topos over a point - and so are trivially abelian. 
                            \item \textbf{($AB3$ \& $AB3^*$):} Again, condensed modules are internal modules, and thus the categories they form are \textit{a priori} $AB5$, and hence $AB3$.
                            \item \textbf{($AB4$ \& $AB4^*$):} 
                            \item \textbf{($AB5$):} We have already shown that categories of condensed modules are $AB5$, so it remains to show that the generator of $R\mod^{\cond}$ is compact.
                            \item \textbf{($AB6$):} 
                        \end{enumerate}
                    \end{proof}
                \begin{corollary}[Properties of categories of condensed modules] \label{coro: condensed_modules_properties}
                    By virtue of being a Grothendieck category (i.e. an $AB5$-category with a generator), any condensed module category $R\mod^{\cond}$ enjoy the following properties:
                        \begin{enumerate}
                            \item 
                                \begin{enumerate}
                                    \item If a presheaf $F: (R\mod^{\cond})^{\op} \to \Sets$ preserves $\kappa$-small limits, then it is representable (i.e. the Yoneda embedding on $R\mod^{\cond}$ preserves all $\kappa$-small limits, not just the finite ones).
                                    \item If a presheaf $F: (R\mod^{\cond})^{\op} \to \Sets$ commutes with all $\kappa$-small colimits, then it possesses a right-adjoint. 
                                \end{enumerate}
                            \item By the \href{https://ncatlab.org/nlab/show/Gabriel-Popescu+theorem}{\underline{Gabriel-Popescu Theorem}}, we can realise $R\mod^{\cond}$ as a reflective localisation of some category of modules over a commutative ring. 
                            \item $R\mod^{\cond}$ is presentable. 
                        \end{enumerate}
                \end{corollary}
                
            \subsubsection{Cohomology of condensed modules}
            
        \subsection{Symmetric monoidal structures; solidity}
            \begin{convention}[Condensed local systems]
                From now on, if $L \in \Sets$ is a set then the corresponding condensed local system (i.e. pro-\'etale local system over a point) shall be suggestively denoted by $L^{\cond}$.
            \end{convention}
        
            First of all, let us clarify that it is not that tensor products of condensed modules do not exist. However, such tensor products will usually end up being topologically pathological or just outright nonsensical. Take for instance, the local system $\Z_p^{\cond} \in \Sets^{\cond}$. Its tensor product with other non-archimedean local systems can be easily topologised via formal completion, but if we were to consider say, $\Z_p^{\cond} \tensor \R^{\cond}$ or $\Z_p^{\cond} \tensor \Z_{\ell}^{\cond}$ (where $\ell \not = p$ is another prime), then it is not very clear what the corresponding topological completion should be.  
            
        \subsection{Analyticity}
    
    \section{\'Etale cohomology of diamonds}
    
    \section{Cohomology of solid pro-\'etale sheaves}
        
    \section{\texorpdfstring{$p$}{}-adic monodromy}
        \subsection{The Weight-Monodromy Conjecture}
            \subsubsection{Toric varieties}
            
            \subsubsection{The Weight-Monodromy Conjecture}
            
        \subsection{Around \texorpdfstring{$p$}{}-adic differential equations}