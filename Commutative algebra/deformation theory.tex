\chapter{Deformation theory}
    \begin{abstract}
        
    \end{abstract}
    
    \minitoc

    \section{Schlessinger's classical deformation theory}
        \subsection{Thickenings and deformation functors}
            \subsubsection{An obesity epidemic among Artinian rings}
                Since the notion is somewhat less popular than that of being Noetherian, let us first recall what it means for a commutative ring to be Artinian, and what the topological consequences of this condition are.
                
                \begin{definition}[Artinian rings] \label{def: artinian_rings} \index{Artinian rings}
                    A commutative ring $\Lambda$ is said to be \textbf{Artinian} if and only if there exist no non-terminating descending chain of ideals (up to bijections, of course). Alternatively, since ideals of commutative rings corespond to Zariski-closed sets, one can define Artinian rings $\Lambda$ as those whose prime spectra $|\Spec \Lambda|$ are topological spaces with no non-terminating \textit{ascending} chains of closed subsets. 
                    \\
                    It is not hard to see that for every given base commutative ring $R$ there is a category whose objects are local Artinian $R$-algebras and whose morphisms are local homomorphisms between them. We shall denote this category by ${}^{R/}\Comm\Alg^{\loc, \Art}$. Furthermore, for each $R$-algebra $k$ that is a field, there is a corresponding full subcategory of ${}^{R/}\Comm\Alg^{\loc, \Art}$, which we shall denote by ${}^{R/}\Comm\Alg_{/k}^{\loc, \Art}$ spanned by local Artinian $R$-algebras whose residue field is $k$. 
                \end{definition}
                
                \begin{proposition}[Basic properties of local Artinian rings] \label{prop: artinian_rings_properties}
                    In deformation theory, we are greatly interested in local Artinian rings, for reasons that we shall divulge later (in remark \ref{remark: why_local_artinian_rings}), so let us try to establish some elementary and (hopefully) geometrically intuitive properties of theirs.
                        \begin{enumerate}
                            \item Quotients and localisations of Artinian rings are also Artinian.
                            \item \cite[\href{https://stacks.math.columbia.edu/tag/00J6}{Tag 00J6}]{stacks} Finitely generated algebras over fields are Artinian.
                            \item A local Artinian ring $(\Lambda, \m)$ with residue field $\kappa$ is a finitely generated $\kappa$-algebra and admits the splitting:
                                $$\Lambda \cong \kappa \oplus \m$$
                            \item \cite[\href{https://stacks.math.columbia.edu/tag/00J7}{Tag 00J7}]{stacks} Artinian rings only have finitely many maximal ideals.
                            \item \cite[\href{https://stacks.math.columbia.edu/tag/00J8}{Tag 00J8}]{stacks} Let $A$ be an Artinian ring. Then, its Jacobson radical is nilpotent. In fact, its Jacobson radical shall be the same as its nilradical.
                            \item \cite[\href{https://stacks.math.columbia.edu/tag/00JA}{Tag 00JA}]{stacks} Any commutative ring with finitely many maximal ideals and locally nilpotent Jacobson radical (such as Artinian rings) can be decomposed into the direct sum of its localisations at the maximal ideals. Furthermore, any prime ideal in such a ring is automatically maximal.
                            \item \cite[\href{https://stacks.math.columbia.edu/tag/00JB}{Tag 00JB}]{stacks} A commutative ring $A$ is simultaneously Artinian and Noetherian if and only if $A$ has finite length as a module over itself. 
                        \end{enumerate}
                \end{proposition}
                    \begin{proof}
                        \noindent
                        \begin{enumerate}
                            \item Let $A$ be an Artinian ring, which we shall view as the Artinian toplogical space $|\Spec A|$. If we would recall that localisations and quotients of $A$ correspond to Zariski-open and Zariski-closed subsets of $|\Spec A|$ respectively (cf. lemma \ref{lemma: localisations_are_open}, corollary \ref{coro: localisation_at_primes}, and corollary \ref{coro: quotients_are_closed}), then we would see how localisations and quotients of Artinian rings being Artinian themselves is entirely evident as a topological phenomenon. 
                            \item Let $k$ be a field and let $A$ be a finitely generated $k$-algebra. Such an algebra is, first and foremost, a finite-dimensional $k$-vector space, and ideals of which are (necessarily finite-dimensional) vector subspaces. This tells us that all descending chains of $A$-ideals are just certains chains of finite-dimensional vector subspaces of the finite-dimensional vector space $A$; they therefore must all be finite length, and they thus all terminate. This means that $A$ is Artinian by definition.
                            \item First of all, we need to show that $\Lambda$ is an algebra over its residue field $\kappa$, i.e. that there exists a homomorphism of commutative rings:
                                $$\kappa \to \Lambda$$
                            
                            \item Suppose to the contrary that there is an Artinian ring $A$ with infinitely many distinct maximal ideal, and without loss of generality, let us also assume that it has \textit{countably} many maximal ideals; let us organise them into a sequence $\{\m_n\}_{n \in \N}$. From such a sequence, we can construct the following descending chain of $A$-ideals:
                                $$\m_0 \supseteq \m_0 \cap \m_1 \supseteq ... \supseteq \bigcap_{n \in \N} \m_n$$
                            But this is manifestly an \textit{infinite} chain of ideals (note that intersections of ideals are still ideals), which means that its existence violates the Artinian assumption on $A$. Thus, $A$ can have only finitely many maximal ideals.
                            
                            One thing to note is that this proof does not imply that every Artinian rings only have finitely many proper ideals. 
                            \item Recall firstly, that the Jacobson radical of a commutative ring $A$ is defined to be the intersection of all its maximal ideals (which we note to still be an ideal, as intersections of ideals are ideals):
                                $$\J(A) = \bigcap_{\m \in |\Spm A|} \m$$
                            and also, that the nilradical of a commutative ring is the same as the intersection of all its prime ideals (cf. proposition \ref{prop: radical_properties}):
                                $$\Nil(A) = \bigcap_{\p \in |\Spec A|} \p$$
                            Then, consider the following (for which we shall assume that the Law of Excluded Middle holds):
                                $$
                                    \begin{aligned}
                                        & x \in \Nil(A)
                                        \\
                                        \iff & x \in \bigcap_{\p \in |\Spec A|} \p
                                        \\
                                        \iff & \bigwedge_{\p \in |\Spec A|} (\p \ni x)
                                        \\
                                        \iff & \neg \neg \bigwedge_{\p \in |\Spec A|} (\p \ni x)
                                        \\
                                        \iff & \neg \bigvee_{\p \in |\Spec A|} \neg (\p \ni x)
                                        \\
                                        \iff & \bigwedge_{\p \in |\Spec A|} \neg \left(\p \in D_A(x)\right)
                                        \\
                                        \iff & \bigwedge_{\p \in |\Spec A|} \left(\p \in V_A(x)\right)
                                    \end{aligned}
                                $$
                            wherein the last line holds thanks to remark \ref{remark: basic_opens_complements}. We can also obtain the following through reasoning similarly:
                                $$x \in \J(A) \iff \bigwedge_{\m \in |\Spm A|} \left(\m \in V_A(x)\right)$$
                            and because maximal ideals are prime (which in particular means that $|\Spm A|$ is a subset of $|\Spec A|$), these tell us that:
                                $$\left(x \in \Nil(A)\right) \implies \left(x \in \J(A)\right)$$
                            i.e.:
                                $$\Nil(A) \subseteq \J(A)$$
                            Now, suppose to the contrary that the complement $\J(A) \setminus \Nil(A)$ is non-empty, and from the above analysis, we know that this would imply the existence of non-maximal primes in $V_A(x)$ for all $x \in \Nil(A)$.  
                            \item
                            \item 
                        \end{enumerate}
                    \end{proof}
                \begin{corollary}
                    \cite[\href{https://stacks.math.columbia.edu/tag/00JB}{Tag 00JB}]{stacks} Let $A$ be a commutative ring that is simultaneously Artinian and Noetherian. Then, all primes of $A$ are maximal ideals and there are only finitely many of them. As a consequence, $|\Spec A|$ is a totally disconnected set in the Zariski topology consisting merely of finitely many closed points.
                \end{corollary}
                    \begin{proof}
                        
                    \end{proof}
                
                \begin{definition}[Deformation categories] \label{def: deformation_categories}
                     
                \end{definition}
                \begin{remark}[Why local Artinian rings] \label{remark: why_local_artinian_rings}
                    
                \end{remark}
        
            \subsubsection{Deformation functors}
        
        \subsection{Schlessinger's criterion}
    
    \section{Deformations in derived algebraic geometry}
        \begin{convention}[Everything is derived!] \label{conv: deformation_theory_everything_is_derived}
            \noindent
            \begin{itemize}
                \item From now on until the end of the chapter, everything will be assumed to be derived. 
                \item By $\Cat^1$, or simply $\Cat$, we shall actually mean ${}^{(\infty, 1)}\Cat^1$, i.e. the $(\infty, 1)$-category of $(\infty, 1)$-categories and functors between them, and by $\Cat^2$ we will be referring to the $(\infty, 2)$-category of $(\infty, 1)$-categories, functors between them, and natural transformations between these functors. 
                
                Similarly, by $\Grpd^1$, or simply $\Grpd$, we will actually mean the $(\infty, 1)$-category of $\infty$-groupoids and functors between them, and by $\Grpd^2$, we shall mean the $(\infty, 2)$-category of $\infty$-groupoids, functors between them, and natural transformations between these functors.
                \item A subcategory of $\Cat$ this is of particular interest is $\Cat^{\dg, \cont}$, the $(\infty, 2)$-category of stable linear (i.e. differential-graded) $(\infty, 1)$-categories (see section \ref{section: homological_algebra} for the notion of stable $(\infty, 1)$-categories). Of course, we can also view $\Cat^{\dg, \cont}$ as a mere $(\infty, 1)$-category.
            \end{itemize}
        \end{convention}
        
        \subsection{Admittance of deformations}
            \subsubsection{Differential cohesiveness}
                
        
            \subsubsection{Spaces admitting deformations}
                \begin{definition}[Prestacks admitting deformations] \label{def: prestacks_admitting_deformations}
                    Let $k$ be an arbitrary base commutative ring. A prestack $\calX$ on ${}^{k/}\Comm\Alg^{\op}$ is said to \textbf{admit deformations} if it satisfies the following conditions:
                        \begin{itemize}
                            \item \textbf{(Convergence):} $\calX$ is convergent, i.e. for all affine schemes $S$ over $\Spec k$, one has:
                                $$\calX(S) \cong \underset{n \in \N}{\lim} \calX({}^{\leq n}S)$$
                            \item \textbf{(Admittance of a cotangent complex):} $\calX$ must admit a pro-cotangent complex. Should said pro-cotangetn complex be an actual cotangent complex, then we will say that $\calX$ \textbf{admits corepresentable deformations}.
                            \item \textbf{(Cohesiveness):} $\calX$ has to be differentially cohesive.
                        \end{itemize}
                \end{definition}
            
            \subsubsection{Consequences of admitting deformations}
            
    \section{Formal schemes and inf-schemes} \label{section: formal_schemes_and_inf_schemes}
        \subsection{Formal schemes}
            \subsubsection{The geomety of formal schemes}
                This subsubsection will, for the most part, rather straightforward and formal, owing to the fact that ind-schemes are defined rather simply.
                
                We start first of all with that simple definition of ind-schemes.
                \begin{definition}[Ind-schemes] \label{def: ind-schemes}
                    Let $k$ be an arbitrary base commutative ring and let $\kappa$ be a regular cardinal\footnote{Which we will never mention again beyond this definition}. The category of ind-schemes over $\Spec k$ is thus the $\kappa$-ind-completion $\Ind_{\kappa}({}^{< \infty}\Sch_{/\Spec k})$ of the category ${}^{< \infty}\Sch_{/\Spec k}$ of convergent schemes over $\Spec k$. This category shall be denoted by $\Ind\Sch_{/\Spec k}$.
                \end{definition}
                \begin{remark}[Ind-schemes vs. formal schemes] \label{remark: ind_schemes_vs_formal_schemes}
                    It should be noted while formal schemes are trivially ind-schemes, the converse statement is not necessarily true. This is because formal schemes are (small) filtered colimits of \textit{quasi-compact} schemes taken along \textit{closed immersions}. This use of terminologies is slightly contradictory to that of \cite[Definition I.2.1.1.2]{GR2}, wherein the authors define ind-schemes as what we refer to here as formal schemes. We have chosen to make this modification both to keep to a more traditional and popular etymological convention, but also, to put emphasis on the fact that unlike general ind-schemes, formal schemes are not \textit{just} filtered colimits of schemes, but rather certain special filtered colimits.
                    
                    For topological reasons (cf. proposition \ref{prop: topologically_complete_adic_modules}), we usually would want to work with Noetherian formal schemes. Note that because Noetherian topological spaces are \textit{a priori} quasi-compact, but there exist quasi-compact spaces that are not Noetherian (e.g. finite disjoint unions of non-Noetherian spaces), the category $(\Sch^{\wedge})^{\Noeth}$ of \textit{Noetherian} formal schemes are not quite the same as the \textit{sub}category of the category $\Ind\Sch^{\qc, \closed}$ of filtered colimits of closed immersions of quasi-compact schemes. The category spanned by closed immersions of \textit{locally} Notherian quasi-compact ind-schemes, however, is precisely equivalent to that of formal schemes. In short, one has the following chain of containment of categories:
                        $$\Ind\Sch^{\qc, \closed, \loc\Noeth} \cong (\Sch^{\wedge})^{\Noeth} \subset \Sch^{\wedge} \cong \Ind\Sch^{\qc, \closed} \subset \Ind\Sch^{\qc} \subset \Ind\Sch$$
                \end{remark}
                \begin{remark}[Morphisms of formal schemes] \label{remark: morphisms_of_formal_schemes}
                    Due to the fact that filtered colimits commute with finite limits, the category $\Ind\Sch^{\qc, \closed}$ spanned by filtered colimits of closed immersions of quasi-compact schemes (which is the same as the category $\Sch^{\wedge}$ of formal schemes), as well as any subcategories thereof, has only monomorphisms as arrows (recall that closed immersions are monomorphic). One can then also rather easily show that these monomorphisms of ind-schemes are actually closed immersions themselves. This, first of all, justifies the isomorphism:
                        $$\Sch^{\wedge} \cong \Ind\Sch^{\qc, \closed}$$
                    and second of all, tells us that the \href{https://ncatlab.org/nlab/show/skeleton}{\underline{skeleton}} of the category $\Sch^{\wedge}$ is a partial order, wherein the ordering is given by the canonical closed immersions. 
                \end{remark}
                
                \begin{definition}[Descriptors for formal scheme] \label{def: formal_schemes_descriptors}
                    There are many descriptors that one can use to describe formal schemes. Notable examples are:
                        \begin{itemize}
                            \item $n$-coconnective (for some natural number $n$),
                            \item affine,
                            \item (locally) almost of finite type,
                            \item classical,
                            \item reduced.
                        \end{itemize}
                    and so on. Let $\scrP$ be any one of these properties, or properties which imply any one of the above (e.g. smoothness, as it implies being of finite type). Then, the subcategory $(\Sch^{\wedge})^{\scrP}$ of formal schemes with property $\scrP$ shall be nothing but the ind-completion of $\Sch^{\qc, \closed, \scrP}$, the category of quasi-compact schemes with the same property $\scrP$ and closed immersions between them, i.e.:
                        $$(\Sch^{\wedge})^{\scrP} \cong \Ind(\Sch^{\qc, \closed, \scrP})$$
                \end{definition}
                \begin{remark}
                    Let $\scrP$ be a property as elaborated on above. Then, one can also characterise the category $(\Sch^{\wedge})^{\scrP}$ via:
                        $$(\Sch^{\wedge})^{\scrP} \cong \Sch^{\wedge} \cap (\Spec \Z)^{\scrP}$$
                    wherein the \say{intersection} is taken at both the level of objects and of (higher) morphisms.
                \end{remark}
                
                \begin{remark}[Other geometric facts about formal schemes] \label{remark: geometric_facts_about_formal_schemes}
                    \noindent
                    \begin{itemize}
                        \item \textbf{(Formal schemes are sheaves):} As schemes satisfy Zariski, \'etale, fppf, and fpqc descent, so do formal schemes (or more generally, ind-schemes). This is due to the fact that sheaf topoi are cocomplete.
                        \item \textbf{(Formal schemes preserve coconnectivity of quasi-compact schemes):} Let $\calX$ be a formal scheme and let $S \in {}^{\leq n}\Sch^{\qc}$ be an $n$-coconnective quasi-compact scheme. Then, the space $\calX(S)$ of $S$-points of $\calX$ will be $n$-truncated. To see why this ought to be true, note first of all that the truncation level of $\calX(S)$ has to be finite, as $\calX$ is convergent by definition. Second of all, 
                        
                        As a corollary, one sees that should $\calX$ be isomorphic to say, $\underset{i \in I}{\colim} X_i$, where $\{X_i\}_{i \in I}$ is small filtered diagram of closed immersions of quasi-compact schemes, then:
                            $$\calX(S) \cong \underset{i \in I}{\colim} X_i(S)$$
                        wherein $S$ is as above.
                    \end{itemize}
                \end{remark}
                
            \subsubsection{Deformations of formal schemes}
        
        \subsection{(Ind)-inf-schemes}
        
        \subsection{Ind-coherent sheaves on ind-(inf)-schemes}
            \subsubsection{Ind-coherent sheaves on formal schemes}
            
            \subsubsection{Ind-coherent sheaves on ind-(inf)-schemes}
        
    \section{Formal moduli}
        \begin{convention}[Everything is derived!] \label{conv: moduli_everything_is_derived}
            \noindent
            \begin{itemize}
                \item From now on until the end of the chapter, everything will be assumed to be derived. 
                \item By $\Cat^1$, or simply $\Cat$, we shall actually mean ${}^{(\infty, 1)}\Cat^1$, i.e. the $(\infty, 1)$-category of $(\infty, 1)$-categories and functors between them, and by $\Cat^2$ we will be referring to the $(\infty, 2)$-category of $(\infty, 1)$-categories, functors between them, and natural transformations between these functors. 
                
                Similarly, by $\Grpd^1$, or simply $\Grpd$, we will actually mean the $(\infty, 1)$-category of $\infty$-groupoids and functors between them, and by $\Grpd^2$, we shall mean the $(\infty, 2)$-category of $\infty$-groupoids, functors between them, and natural transformations between these functors.
                \item A subcategory of $\Cat$ this is of particular interest is $(\Cat^{\dg, \cont})^2$ (or simply $\Cat^{\dg, \cont}$), the $(\infty, 2)$-category of stable linear (i.e. differential-graded) $(\infty, 1)$-categories (see section \ref{section: homological_algebra} for the notion of stable $(\infty, 1)$-categories). Of course, we can also view $\Cat^{\dg, \cont}$ as a mere $(\infty, 1)$-category; when necessary, we shall write $(\Cat^{\dg, \cont})^1$ to put emphasis on the disregard of $2$-morphisms.
            \end{itemize} 
        \end{convention}
    
        \subsection{Formal moduli problems}
            \begin{definition}[Formal moduli problems] \label{def: formal_moduli_problems}
                Let $k$ be a commutative ring and let $\calX$ be an object of $[\Spec k]^{\laft}$ (i.e. a prestack over $\Spec k$ that is locally almost of finite type) and define the category $\Moduli_{/\calX}$ of \textbf{formal moduli problems over $\calX$} to be the full subcategory $[\Spec k]^{\laft, \defm}_{/\calX}$ spanned by nil-isomorphisms:
                    $$\calY \to \calX$$
                which are \textit{inf-schematic} (note that inf-schematic prestacks necessarily admit deformations).
            \end{definition}
        
        \subsection{Formal groupoids}