\chapter{Derived schemes} \label{chapter: cohomology_and_derived_schemes}
    \begin{abstract}
        
    \end{abstract}
    
    \minitoc
    
    \section{Sheaf cohmology}
        \subsection{Vanishing theorems}
            \subsubsection{Quasi-coherent sheaves and Serre's affineness criterion}
            
            \subsubsection{Higher direct images}
        
        \subsection{Base change}
        
        \subsection{Cohomology of projective varieties}
        
        \subsection{Some useful theorems}
            \subsubsection{Theorems on duality}
            
            \subsubsection{The theorem on formal functions}
            
            \subsubsection{The Grothendieck Existence Theorem; Algebraisation Theorems}

    \section{Derived schemes}
        \subsection{(Un)necessary \texorpdfstring{$\infty$}{}-categorical technicalities}
            \begin{convention}[Some typographical conventions]
                For the sake of linguistic simplicity (and since typing \say{$\infty$} gets very tedious very quickly), we shall refrain from specfifying the homotopicality of many $\infty$-categorical operations. For instance, $(\infty, 1)$-limits shall almost always be referred to simply as \say{limits}, and so on. Also, we shall only refer to $(\infty, 1)$-categories by their \say{full name}, so to say, when $(\infty, 2)$-categories are not around; otherwise, they shall simply be known as $\infty$-categories.
            \end{convention}
            
            \subsubsection{What on earth is an \texorpdfstring{$\infty$}{}-category ?}
        
            \subsubsection{\texorpdfstring{$\infty$}{}-topoi and \texorpdfstring{$\infty$}{}-stacks}
                \begin{remark}[Regular cardinals]
                    From now on we will be using the notion of regular cardinals often. For details on the notion, see definition \ref{def: limit_cardinal}.
                \end{remark}
            
                \begin{definition}[$\infty$-topoi] \label{def: infinity_topoi}
                    \noindent
                    \begin{enumerate}
                        \item \textbf{(Localisations of $\infty$-categories):} 
                            \begin{enumerate}
                                \item \textbf{(Reflexivity):} A \textit{fully faithful} $\infty$-functor:
                                    $$R: \C \to \D$$
                                identifies an $\infty$-category $\C$ as a \textbf{reflexive full $\infty$-subcategory} of another $\infty$-category $\D$ if it admits a left-$(\infty, 1)$-adjoint. Should this left-adjoint functor preserve all finite $(\infty, 1)$-limits, then the full faithful reflexive embedding $R: \C \to \D$ shall be called \textbf{exact}.
                                \item \textbf{(Localisations):} An $\infty$-functor:
                                    $$L: \D \to \C$$
                                is called a \textbf{localisation} if it admits a fully faithful right-$(\infty, 1)$-adjoint, which we note to necessarily be a reflexive embedding, by definition; we shall thus dub the right-adjoint component the \textbf{reflector}. Alternatively, one may characterise the \textbf{localisation} of an $\infty$-category $\D$ at a full $\infty$-subcategory $\C$ as an $(\infty, 1)$-adjoint pair:
                                    $$
                                        \begin{tikzcd}
                                        	\C & \D
                                        	\arrow[""{name=0, anchor=center, inner sep=0}, "L"', shift right=2, from=1-2, to=1-1]
                                        	\arrow[""{name=1, anchor=center, inner sep=0}, "R"', shift right=2, hook, from=1-1, to=1-2]
                                        	\arrow["\dashv"{anchor=center, rotate=-90}, draw=none, from=0, to=1]
                                        \end{tikzcd}
                                    $$
                                whose right-adjoint component is fully faithful. 
                                
                                A localisation that is \textit{exact} shall be called a \textbf{topological localisation}. 
                            \end{enumerate}
                        \item \textbf{(Accessibility and geometric embeddings/localisations):}
                            \begin{enumerate}
                                \item \textbf{(Accessibility):} Let $\kappa$ be some \href{https://ncatlab.org/nlab/show/regular+cardinal}{\underline{regular cardinal}}. An $\infty$-category $\C'$ is said to be \textbf{$\kappa$-accessible} if and only if it is equivalent to the $(\infty, 1)$-ind-completion of some $\kappa$-small $\infty$-category $\C$, i.e. one should be able to identify $\C'$ as a full $\infty$-subcategory of $\Psh_{(\infty, 1)}(\C)$ which is closed under all $\kappa$-small $(\infty, 1)$-filtered colimits.
                                \item \textbf{($(\infty, 1)$-geometric embeddings and $\infty$-topoi):} An \textbf{$\infty$-topos} \textit{\`a la} Rezk-Lurie is the topological localisation of some $\infty$-category of $(\infty, 1)$-presheaves at some \textit{accessible} full $\infty$-subcategory; the adjoint pair defining this localisation may either be referred to as an \textbf{$\infty$-geometric embedding} or an \textbf{$\infty$-geometric localisation}, depending on whether we wish to put emphasis on the left-adjoint or right-adjoint component.
                                
                                One very important thing to note is that every $(\infty, 1)$-presheaf $\infty$-category is trivially an $\infty$-topos; we refer to them as $(\infty, 1)$-presheaf $\infty$-topoi. 
                            \end{enumerate}
                    \end{enumerate}
                \end{definition}
                
                In algebraic geometry, one cares as much about sheaves as the topoi they span. Therefore, it would be nice if $\infty$-topoi were to behave as homotopical analogues of Grothendieck $1$-topoi (and let us recall that there is a geometric embedding from every Grothendieck $1$-topos into a presheaf $1$-topos), in that its objects could be realised as higher sheaves on higher sites. Luckily, such a characterisation of $\infty$-topos is available. We shall, however, need to roll our sleeves up a bit to obtain it.
                
                \begin{definition}[$\infty$-sites and $\infty$-descent theory] \label{def: infinity_sites}
                    \noindent
                    \begin{enumerate}
                        \item \textbf{($\infty$-coverages):} 
                            \begin{enumerate}
                                \item \textbf{($\infty$-sieves):} An $\infty$-sieve on an object $X$ of some $\infty$-category $\C$ is nothing more than an $(\infty, 1)$-subpresheaf of the representable presheaf $h_X$; i.e. for every test object $X_0$ of $\C_{/X}$ and every $\infty$-sieve $\U$ on $X$, one can identify $\U(X_0)$ as a full $\infty$-subcategory ($\infty$-subgroupoid, actually) of the $\infty$-groupoid $\C(X_0, X)$.
                                
                                An $\infty$-sieve $\U$ is said to \textbf{cover} an object $X$ of an $\infty$-category $\C$ if and only if $X$ is the $(\infty, 1)$-colimit of the \v{C}ech nerve $\U^{\bullet}_{/X}$. 
                                \item \textbf{(Axioms for $\infty$-coverages):} The following are conditions for a class of sieves on objects of some base $\infty$-category to qualify as an \textbf{$\infty$-coverage}:
                                    \begin{itemize}
                                        \item Representable $(\infty, 1)$-presheaves - viewed as $\infty$-sieves - cover the objects they represent.
                                        \item $(\infty, 1)$-pullbacks of covering $\infty$-sieves must also be covering $\infty$-sieves themselves. 
                                        \item Should the $(\infty, 1)$-pullback of an $\infty$-sieve be one that covers, then the first $\infty$-sieve must also be a covering $\infty$-sieve.
                                    \end{itemize}
                                An $\infty$-category equipped with an $\infty$-coverage is an \textbf{$\infty$-site.}
                            \end{enumerate}
                        \item \textbf{($\infty$-descent theory):} An $\infty$-prestack $\calY$ (cf. convention \ref{conv: infinity_prestacks}) on some $\infty$-category $\C$ equipped with an $\infty$-coverage $J$ (i.e. an $\infty$-site $(\C, J)$) is said to \textbf{satisfy $J$-descent} if and only if for all objects $X$ of $\C$ and for all $\infty$-sieves $\U$ that covers $X$, one has the following equivalence of $\infty$-groupoids:
                            $$\calY(X) \cong \calY\left((\infty, 1)\-\colim \U^{\bullet}_{/X}\right)$$
                        In particular, $(\infty, 1)$-presheaves that satisfy descent are \textbf{$(\infty, 1)$-sheaves}.
                    \end{enumerate}
                \end{definition}
                
                \begin{definition}[Presentability] \label{def: presentable_infinity_categories}
                    Let $\kappa$ be regular cardinal. Then, a $\kappa$-accessible $\infty$-category is said to be \textbf{$\kappa$-presentable} if it is furthermore $\kappa$-small $(\infty, 1)$-cocomplete. In other words, a $\kappa$-accessible $\infty$-category is a \textit{full} $\infty$-subcategory of the $(\infty, 1)$-presheaf $\infty$-topos over some underlying $\infty$-category that is \textit{closed under all $\kappa$-small $(\infty, 1)$-colimits}; clearly, every object in a $\kappa$-presentable category \textit{can be built out of a $\kappa$-small set of objects using $(\infty, 1)$-colimits}.
                \end{definition}
                \begin{example}[Examples of presentable $\infty$-categories]
                    
                \end{example}
                
                \begin{definition}[Universal $(\infty, 1)$-colimits] \label{def: universal_colimits}
                    Given any finitely $(\infty, 1)$-complete $\infty$-category $\E$, one can build so-called pullback $\infty$-functors:
                        $$f^*: \E_{/y} \to \E_{/x}$$
                    via $(\infty, 1)$-pullbacks along arrows $f: x \to y$. If a $(\infty, 1)$-colimit in $\E$ is preserved by all these pullback $\infty$-functors, then we shall call it \textbf{universal}.
                \end{definition}
                
                \begin{definition}[Subobject classifiers in $\infty$-categories] \label{def: subobject_classifiers}
                    \noindent
                    \begin{enumerate}
                        \item \textbf{(Monomorphisms):} A morphism:
                            $$f: x \to y$$
                        in an $\infty$-category $\E$ is a \textbf{monomorphism} if and only if $\E(x, -)$ is an $(\infty, 1)$-subfunctor of $\E(y, -)$: that is to say, for all test objects $x_0$ of $\E$, $\E(x, x_0)$ is a full $\infty$-subcategory of the $\infty$-groupoid $\E(y, x_0)$. 
                        \item \textbf{(Subobject classifiers):} Let $\E$ be a $(\infty, 1)$-complete $\infty$-category and let us denote its terminal object by $1$ (note that such a terminal object exists as a consequence of the finite-completeness hypothesis). Then, a morphism:
                            $$t: 1 \to \Omega$$
                        of $\E$ will be called a \textbf{subobject classifier} if and only if it is terminal in the (non-full) $\infty$-subcategory of the $\infty$-category $\E^{[1]} \cong \infty\-1-\Cat([1], \E)$ of arrows and commutative squares in $\E$ spanned by monomorphisms.
                        
                        Thanks to the assumption that $\E$ is finitely $(\infty, 1)$-complete, if a subobject classifier $t: 1 \to \Omega$ exists, one will be able to write any monomorphism $u \to x$ in $\E$ as the pullback of the subobject classifier along some \textit{unique} morphism $\chi_x: x \to \Omega$.
                    \end{enumerate}
                \end{definition}
                
                \begin{definition}[(De)looping] \label{def: looping} \index{Loop spaces} \index{Classifying spaces} 
                    \noindent
                    \begin{enumerate}
                        \item \textbf{(Loop spaces):} Let $\S$ be a finitely $(\infty, 1)$-complete $\infty$-category of \say{spaces} and let $X$ be some object thereof; also, let $*$ be a terminal object of $\S$. Then, the \textbf{loop space} $\loopspace X$ fits into the following $(\infty, 1)$-pullback square:
                            $$
                                \begin{tikzcd}
                                	{\loopspace X} & {*} \\
                                	{*} & X
                                	\arrow[from=1-1, to=2-1]
                                	\arrow[from=2-1, to=2-2]
                                	\arrow[from=1-1, to=1-2]
                                	\arrow[from=1-2, to=2-2]
                                	\arrow["\lrcorner"{anchor=center, pos=0.125}, draw=none, from=1-1, to=2-2]
                                \end{tikzcd}
                            $$
                        \item \textbf{(Classifying spaces):} Within the same framework, one can also define the so-called \textbf{classifying space} or \textbf{delooping space} of an object $H$ as the $(\infty, 1)$-pushout of the unique morphism $H \to *$ along itself:
                            $$
                                \begin{tikzcd}
                                	H & {*} \\
                                	{*} & {\bfB H}
                                	\arrow[from=1-1, to=2-1]
                                	\arrow[from=2-1, to=2-2]
                                	\arrow[from=1-1, to=1-2]
                                	\arrow[from=1-2, to=2-2]
                                	\arrow["\lrcorner"{anchor=center, pos=0.125, rotate=180}, draw=none, from=2-2, to=1-1]
                                \end{tikzcd}
                            $$
                        Note that because our ambient $\infty$-category is only finitely $(\infty, 1)$-complete, deloopings of its objects may or may not exist; should they do, however, they would guarantee the existence of all $(\infty, 1)$-coequalisers in $\E$ (we will use this fact in the proof of lemma \ref{lemma: building_infinity_topoi_out_of_colimits}).
                    \end{enumerate}
                    It is not too hard to see that:
                        $$\loopspace \bfB H \cong H$$
                    and if $(\infty, 1)$-pushouts were to exist in $\E$:
                        $$\bfB \loopspace X \cong X$$
                \end{definition}
                \begin{example}
                    \noindent
                    \begin{enumerate}
                        \item \textbf{((De)looping of spheres):} The simplest setting in which one can perform (de)looping on objects of a finitely $(\infty, 1)$-complete $\infty$-category is when $\S$ is the homotopy category $\Ho\Top$ of the category $\Top$ of topological spaces, which happens to be $(\infty, 1)$-cocomplete. For instance, for all positive integers $n$, the loop space of the $(n + 1)$-sphere is nothing but the $n$-sphere:
                            $$
                                \begin{tikzcd}
                                	{\bbS^n} & {*} \\
                                	{*} & {\bbS^{n + 1}}
                                	\arrow[from=1-1, to=2-1]
                                	\arrow[from=2-1, to=2-2]
                                	\arrow[from=1-1, to=1-2]
                                	\arrow[from=1-2, to=2-2]
                                	\arrow["\lrcorner"{anchor=center, pos=0.125, rotate=180}, draw=none, from=2-2, to=1-1]
                                \end{tikzcd}
                            $$
                        and conversely, the delooping of the $n$-sphere is the $(n + 1)$-sphere.
                        \item \textbf{(\v{C}ech nerves):} The delooping of the \v{C}ech nerve $u^{\bullet}_{/x}$ on some object $x$ of a finitely $(\infty, 1)$-complete $\infty$-category is exactly $x$. 
                    \end{enumerate}
                \end{example}
                
                \begin{lemma}[Building $\infty$-topoi out of colimits] \label{lemma: building_infinity_topoi_out_of_colimits}
                    Let $\kappa$ be a regular cardinal. Then, every $\kappa$-small $\infty$-topos is necessarily $\kappa$-presentable, has subobject classifiers, and the $(\infty, 1)$-colimits therein (which are all $\kappa$-small) are all universal.
                \end{lemma}
                    \begin{proof}
                        Fix a base $\kappa$-small $\infty$-category $\C$ and the following topological localisation defining an $\infty$-topos $\E$, which is necessarily $\kappa$-small by virtue of being a full $\infty$-subcategory of the $\kappa$-small $\infty$-category $\Psh_{(\infty, 1)}(\C)$:
                            $$
                                \begin{tikzcd}
                                	\E & {\Psh_{(\infty, 1)}(\C)}
                                	\arrow[""{name=0, anchor=center, inner sep=0}, "L"', shift right=2, from=1-2, to=1-1]
                                	\arrow[""{name=1, anchor=center, inner sep=0}, "R"', shift right=2, hook, from=1-1, to=1-2]
                                	\arrow["\dashv"{anchor=center, rotate=-90}, draw=none, from=0, to=1]
                                \end{tikzcd}
                            $$
                        \begin{enumerate}
                            \item \textbf{(Presentability):} Because $\kappa$-small $\infty$-topos are already $\kappa$-accessible by default (cf. definition \ref{def: infinity_topoi}), it shall suffice to show that they are $\kappa$-small $(\infty, 1)$-cocomplete. For this, we shall attempt to show that $\E$ has all $(\infty, 1)$-coproducts and all $(\infty, 1)$-coequalisers. 
                                \begin{enumerate}
                                    \item \textbf{(Coproducts):} By viewing our accessible $\infty$-category $\E$ as a quasi-category (which in particular, have underlying topological spaces), we can simply apply the Seifert-van Kampen Theorem to see how $(\infty, 1)$-coproducts ought to exist in $\E$ via the existence of liftings inside pushout squares of the following form: 
                                        $$
                                            \begin{tikzcd}
                                            	\varnothing & \bullet \\
                                            	\bullet & \bullet
                                            	\arrow[from=1-1, to=2-1]
                                            	\arrow[from=2-1, to=2-2]
                                            	\arrow[from=1-1, to=1-2]
                                            	\arrow[from=1-2, to=2-2]
                                            	\arrow[dashed, from=2-1, to=1-2]
                                            	\arrow["\lrcorner"{anchor=center, pos=0.125, rotate=180}, draw=none, from=2-2, to=1-1]
                                            \end{tikzcd}
                                        $$
                                    These liftings exist thanks to the fact that topological spaces have underlying sets, and one can always order their cardinalities. One thing to note here is that the Seifert-van Kampen Theorem can be applied here because disjoint spaces intersect at the empty set (of course!) and subsequently because the empty set is path-connected (recall that the Seifert-van Kampen Theorem can only be applied when the intersection is path-connected).
                                    \item \textbf{(Coequalisers):} $(\infty, 1)$-coequalisers are nothing but internal equivalence relations, and since these are nothing more than certain kinds of internal groupoids, let us show the stronger assertion that all internal groupoids in $\E$ can be delooped (cf. definition \ref{def: looping}). To that end, note that because $\E$ is $\kappa$-accessible, we can write this as a $\kappa$-small filtered $(\infty, 1)$-colimit of other internal groupoids $\left( s^{(i)}, t^{(i)}: H_1^{(i)} \toto H_0^{(i)} \right)$ in $\E$, which we might as well take to be deloopable:
                                        $$(s, t: H_1 \to H_0) \cong (\infty, 1)\-\underset{i \in I}{\colim} \left( s^{(i)}, t^{(i)}: H_1^{(i)} \toto H_0^{(i)} \right)$$
                                    But we $(\infty, 1)$-colimits commute with one another, and so it is precisely because the groupoids $\left( s^{(i)}, t^{(i)}: H_1^{(i)} \toto H_0^{(i)} \right)$ are deloopable that we can also deloop the filtered $(\infty, 1)$-colimit $(s, t: H_1 \toto H_0)$. Thus, every internal groupoid in $\E$ is deloopable, which as stated above, implies that $\E$ has all $(\infty, 1)$-coequalisers.
                                \end{enumerate}
                            \item \textbf{(Universality of colimits):} Now that we have established the $(\infty, 1)$-cocompleteness of $\infty$-topoi, let us check if the $(\infty, 1)$-colimits therein are all universal. To that end, let $x$ be an object of $\E$ and let:
                                $$f: x \to y$$
                            be a morphism. Also, let:
                                $$F: \D \to \E_{/y}$$
                            be a diagram of shape $\D$ in $\E_{/y}$, which we note to necessarily be $\kappa$-small; a pullback of this diagram is just a lifting in $\infty\-1-\Cat^{\leq \kappa}$ of the following form:
                                $$
                                    \begin{tikzcd}
                                    	& {\E_{/x}} \\
                                    	\D & {\E_{/y}}
                                    	\arrow["{f^*}"', from=2-2, to=1-2]
                                    	\arrow["F"', from=2-1, to=2-2]
                                    	\arrow["{f^*F}", dashed, from=2-1, to=1-2]
                                    \end{tikzcd}
                                $$
                            If the diagram $F$ were to be filtered, then the assertion would be a trivial consequence of the fact that filtered colimits commute with finite limits (recall that $f^*$ is given by pulling back along $f: x \to y$), so let us assume that is is not filtered; actually, we can simply assume that our diagram $F$ is discrete, since every diagram can built out of filtered and discrete subdiagrams using coproducts of functors. With this assumption in place, we would only need to show that every $(\infty, 1)$-coproduct in $\E$ is universal. 
                            \item \textbf{(Subobject classifiers):}
                        \end{enumerate}
                    \end{proof}
                \begin{proposition}[$\infty$-categories of $(\infty, 1)$-sheaves] \label{prop: infinity_categories_of_higher_sheaves}
                    Let $(\C, J)$ be a $\kappa$-small $\infty$-site, for some regular cardinal $\kappa$. Then, the category of $(\infty, 1)$-sheaves on $(\C, J)$, denoted by $\Sh_{(\infty, 1)}(\C, J)$, is a full $\infty$-subcategory of the $(\infty, 1)$-presheaf $\infty$-topos $\Psh_{(\infty, 1)}(\C, J)$. Furthermore, $\infty$-categories of $(\infty, 1)$-sheaves:
                        \begin{enumerate}
                            \item they are always $\kappa$-presentable (assuming that the underlying $\infty$-site is $\kappa$-small, of course),
                            \item they are $\infty$-categories where all $\kappa$-small $(\infty, 1)$-colimits are universal, and
                            \item they have subobject classifiers.
                        \end{enumerate}
                \end{proposition}
                    \begin{proof}
                        \noindent
                        \begin{enumerate}
                            \item \textbf{(Presentability):} 
                            \item \textbf{(Universality of colimits):}
                            \item \textbf{(Subobject classifiers):}
                        \end{enumerate}
                    \end{proof}
                \begin{corollary}[$\infty$-topoi are $\infty$-categories of $(\infty, 1)$-sheaves] \label{coro: infinity_topoi_are_sheaf_topoi}
                    Let:
                        $$
                            \begin{tikzcd}
                            	\E & \Psh_{(\infty, 1)}(\C)
                            	\arrow[""{name=0, anchor=center, inner sep=0}, "L"', shift right=2, from=1-2, to=1-1]
                            	\arrow[""{name=1, anchor=center, inner sep=0}, "R"', shift right=2, hook, from=1-1, to=1-2]
                            	\arrow["\dashv"{anchor=center, rotate=-90}, draw=none, from=0, to=1]
                            \end{tikzcd}
                        $$
                    be a topological localisation that defines some $\infty$-topos $\E$. Then, there exists an $\infty$-site $(\C, J)$ such that $\E$ is equivalent to the $\infty$-category of $(\infty, 1)$-sheaves over $(\C, J)$:
                        $$\E \cong \Sh_{(\infty, 1)}(\C, J)$$
                    Because of this, the left-adjoint component of the topological localisation $L \ladjoint R$ is usually known as the $(\infty, 1)$-sheafification functor with respect to the $\infty$-coverage on $(\C, J)$, and we shall denote it by ${}^{(\infty, 1), \sh}(-)_J$.
                \end{corollary}
            
                \begin{remark}[The $(\infty, 2)$-category of $(\infty, 1)$-topoi] \label{remark: (infinity, 1)_topoi_categories} \index{$\infty$-topoi} \index{$\infty$-topoi! $\infty\-\Sh\Topos$}
                    We shall leave the verification of the following facts to the reader.
                    \\
                    $(\infty, 1)$-topoi naturally form a \textit{weak} $(\infty, 2)$-category, wherein:
                        \begin{itemize}
                            \item Objects are $\infty$-topoi, which we should note to be $(\infty, 1)$-categories.
                            \item $(\infty, 1)$-morphisms are $(\infty, 1)$-geometric morphisms, i.e. pairs of $(\infty, 1)$-adjoint $(\infty, 1)$-functors with the left-$(\infty, 1)$-adjoint component being left-$(\infty, 1)$-exact; also, compositions of these left-$(\infty, 1)$-adjoint components are merely associative up to invertible strict $(\infty, 1)$-natural transformations.
                            \item $(\infty, 1)$-morphisms are $(\infty, 1)$-natural transformations between the left-$(\infty, 1)$-adjoint components of weak $(\infty, 1)$-geometric morphisms. 
                        \end{itemize}
                    This weak $(\infty, 2)$-category shall be denoted by $\infty\-\Sh\Topos$. It is finitely weakly $(\infty, 2)$-complete:
                        \begin{itemize}
                            \item \textbf{(Products):} It admits weak $(\infty, 2)$-products and weak $(\infty, 2)$-pullbacks, along with a terminal object, that being the $(\infty, 1)$-category $\infty\-\Grpd$ of small $\infty$-groupoids, $(\infty, 1)$-(ana)functors between them, and $(\infty, 1)$-(ana)natural transformations between those functors. 
                            \item \textbf{(Monomorphisms):} Its \href{https://ncatlab.org/nlab/show/monomorphism+in+an+\%28infinity\%2C1\%29-category}{\underline{monomorphisms}} are topological localisations.
                        \end{itemize}
                \end{remark}
        
                \begin{convention}[\textcolor{red}{\underline{IMPORTANT}} $\infty$-prestacks and $\infty$-stacks] \label{conv: infinity_prestacks} \index{$\infty$-prestacks} \index{$\infty$-stacks}
                    \noindent
                    \begin{enumerate}
                        \item \textbf{(The descent-theoretic perspective):} \begin{enumerate}
                            \item \textbf{($\infty$-prestacks):} To us, a \textbf{prestack} on an $(\infty, 1)$-category $\C$ will always be a \textit{weak} $(\infty, 2)$-functor (i.e. a $(\infty, 2)$-functor that respects compositions only up to natural isomorphisms) from $\C^{\op}$ (viewed as a tautological bicategory) into the weak $(\infty, 2)$-category $(\infty, 1)\-1-\Cat$ of small $(\infty, 1)$-categories, $(\infty, 1)$-(ana)functors between them, and $(\infty, 1)$-(ana-)natural transformations between these $(\infty, 1)$-(ana-)functors. In other terminologies, an $\infty$-prestack is an $(\infty, 1)$-pseudo-functor with values in $(\infty, 1)\-1-\Cat$. Prestacks over a given $(\infty, 1)$-category $\C$ form a $(\infty, 1)$-category in an obvious manner; we shall denote it by $\infty\-\Pre\Stk(\C)$. Actually, for all base $(\infty, 1)$-categories $\C$, the $(\infty, 1)$-category $\infty\-\Pre\Stk(\C)$ can also be endowed with the structure of a weak $(\infty, 2)$-category, determined on the $(\infty, 2)$-categorical level by $(\infty, 2)$-morphisms which are \textit{strict} $(\infty, 2)$-natural transformations
                            
                            Additionally, we shall assume the Axiom of Choice (which incidentally, forces us to adopt definition \ref{def: internal_categories}). The advantage in this is that for all base $(\infty, 1)$-categories $\C$, we will automatically be given a \textit{weak} $2$-equivalence of \textit{weak} $(\infty, 2)$-categories:
                                $$\infty\-\Pre\Stk(\C) \cong \Fib_{(\infty, 1)}(\C)$$
                            between the weak $(\infty, 2)$-category of $\infty$-prestacks on $\C$ and that of fibred $(\infty, 1)$-categories on $\C$. Logicians might scoff at such a practice, but since choosing cleavages in algebraic geometry is mostly just asking for trouble, we shall try to bear the shame of having Choice.
                            \item \textbf{($\infty$-stacks):} Let us build upon the above notion of $\infty$-prestacks and declare that from this point on, the term \say{\textbf{$\infty$-stack}} shall mean \say{sheaf of $(\infty, 1)$-categories}, i.e. an $\infty$-stack is a fibred $(\infty, 1)$-category which satisfies $(\infty, 1)$-descent; concretely, an $\infty$-stack $\calX$ over a small \href{https://ncatlab.org/nlab/show/(infinity,1)-site}{\underline{$\infty$-site}} $(\C, J)$ is an $\infty$-prestack such that for all objects $X$ of $\C$ and all covering $J$-sieves $\U_{/X}$ thereon, one has the following equivalence of $(\infty, 1)$-categories:
                                $$\calX(X) \cong \calX\left((\infty, 1)\-\underset{U \in \U_{/X}}{\colim} h_U\right)$$
                            $\infty$-stacks on $(\C, J)$ form a full weak $(\infty, 2)$-subcategory of the weak $(\infty, 2)$-category $\infty\-\Pre\Stk(\C)$ of $\infty$-prestacks on $\C$, which shall be denoted by $\infty\-\Stk(\C, J)$. Furthermore, $\infty$-stacks of $\infty$-groupoids (i.e. $(\infty, 1)$-categories fibred in $\infty$-groupoids that satisfy $(\infty, 1)$-descent) on small $\infty$-sites $(\C, J)$ form an $\infty$-topos which is written $\Sh_{(\infty, 1)}(\C, J)$; occasionally, we might refer to these as sheaves of $\infty$-groupoids; naturally, $\Sh_{(\infty, 1)}(\C, J)$ comes equipped with a \href{https://ncatlab.org/nlab/show/(infinity,1)-topos#AsAGeometricEmbedding}{\underline{$(\infty, 1)$-geometric embedding}} into the weak $(\infty, 2)$-category $\Psh_{(\infty, 1)}(\C)$ of $\infty$-prestacks of $\infty$-groupoids on $\C$.
                        \end{enumerate}
                        \item \textbf{(The internal point of view):} From the internal point of view (which in the opinion of the author, is a lot more intuitive; this is, however, purely personal), $\infty$-(pre)stacks are nothing but internal $(\infty, 1)$-categories inside $\infty$-(pre)sheaf topoi. Note that $\infty$-(pre)stacks fibred in $\infty$-groupoids are precisely $(\infty, 1)$-(pre)sheaves, so we do not to treat them as a cases of (pre)stacks which are not (pre)sheaves, unlike how $(2, 1)$-(pre)sheaves having to be considered as phenomena more general than (pre)sheaves of sets; this is thanks to the fact that the notion of $\infty$-groupoids subsumes both those of $1$-groupoids and sets (which may be viewed as $0$-groupoids): in particular, \textit{sets are $0$-truncated $\infty$-groupoids, and $1$-groupoids are $1$-truncated $\infty$-groupoids}. Incidentally, this is also a first glimpse into the myriads of reasons why $\infty$-categories might help us simplify instead of further complicating constructions in algebraic geometry, especially when it comes to taking colimits (recall how in general, quotients stacks are not sheaves of sets but rather stacks in groupoids).
                    
                        Readers may also have heard of mysterious entities known as \textbf{$\infty$-gerbes}. There are many definitions floating around, but again, let us fix one meaning for the term. To us, an $\infty$-gerbe on a small $\infty$-site $(\C, J)$ will always be an group object internal to $\infty\-\Stk(\C, J)$ (i.e. an object of $\Sh_{\infty\-\Grpd}(\C, J)$ whose connected component is the connected component of the terminal object of $\Sh_{\infty\-\Grpd}(\C, J)$). Additionally, the notion of gerbes coincides with that of so-called principal $\infty$-bundles.
                    \end{enumerate}
                \end{convention}
                
                \begin{remark}[Derived affine schemes over general $\infty$-(pre)stacks] \label{remark: weak_(infinity, 2)_yoneda}
                    Instead of embedding a base $(\infty, 1)$-category $\C$ into the $(\infty, 1)$-category of $(\infty, 1)$-presheaves of $\infty$-groupoids $\Psh_{\infty\-\Grpd}(\C)$ thereon, we shall be viewing $\C$ as a tautological weak $(\infty, 2)$-category, and shall instead be embedding it into the weak $(\infty, 2)$-category $\infty\-\Pre\Stk(\C)$ of $\infty$-prestacks over $\C$ via a fully faithful $(\infty, 2)$-functor, known as the \textbf{$(\infty, 2)$-Yoneda embedding}. Via the $(\infty, 2)$-Yoneda embedding, one can view objects of $\C$ as $\infty$-prestacks fibred in $\infty$-groupoids (which we note to be $(\infty, 2)$-categories wherein all $(\infty, 1)$-cells and $(\infty, 2)$-cells are identities), and therefore, the notion of objects of $\C$ with mappings into an arbitrarily given base $\infty$-prestack $\calY \in \infty\-\Pre\Stk(\C)$ is well-defined; in particular, one can meaningfully talk about so-called derived affine schemes over $\infty$-prestacks on any (symmetric monoidal) dg-category of commutative algebras ${}^{\dg, k/}\Comm\Alg^{\op}$ (more on this later). 
                    \\
                    For a detailed discussion of the weak $(\infty, 2)$-Yoneda embedding, the reader may consult \cite{nlab:yoneda_lemma_for_bicategories}.
                \end{remark}
                
            \subsubsection{Truncation; connectivity and co-connectivity}
                \begin{definition}[Truncated objects] \label{def: truncated_objects} \index{Truncated! objects}
                    Let $n \geq -2$ be an integer.
                    \begin{enumerate}
                        \item \textbf{(Truncated spaces):} An $\infty$-groupoid $H$ said to be $n$-truncated if:
                            \begin{itemize}
                                \item  
                                \item
                            \end{itemize}
                        \item \textbf{(Truncated objects \cite[Definition 5.5.6.1]{HTT}):} Let $n \geq -2$. An object $X$ of an $\infty$-category $\S$ is said to be $n$-truncated if
                    \end{enumerate}
                \end{definition}
    
        \subsection{Derived geometric stacks}
            \subsubsection{Generalities}
                \begin{definition}[Geometric $\infty$-stacks] \label{def: derived_geometric_stacks} \index{$\infty$-stacks! geometric}
                    \noindent
                    \begin{enumerate}
                        \item \textbf{(Representable morphisms):} A morphism:
                            $$f: \calX \to \calY$$
                        of $\infty$-stacks on some subcanonical small $\infty$-site $(\C, J)$ is called \textbf{representable} if and only if for all morphism:
                            $$\pi: h_V \to \calY$$
                        from a representable sheaves $h_V$, the $(\infty, 1)$-pullback $\calX \x_{f, \calY, \pi} h_V$ is also representable. When the underlying site is a site of commutative dg-algebras, representable morphisms are might also be referred to as \textbf{affine-schematic}.
                        \item \textbf{(Geometric $\infty$-stacks):} A \textbf{geometric $\infty$-stack} over some small $\infty$-site $(\C, J)$ is a quotient $\infty$-stack $\calX$ (i.e. an equivalence relation internal to $\Sh_{(\infty, 1)}(\C, J)$) whose diagonal morphism:
                            $$\Delta_{\calX}: \calX \to \calX \x^{(\infty, 1)} \calX$$
                        is representable. Geometric $\infty$-stacks over small $\infty$-sites of commutative dg-algebras are commonly known as algebraic $\infty$-stacks; in particular, algebraic $\infty$-stacks over small \'etale sites are colloquially known as \textbf{derived Deligne-Mumford stacks}, and those over small smooth $\infty$-sites (in the algebraic sense of the word \say{smooth}) are known as \textbf{derived Artin stacks}.   
                    \end{enumerate}
                \end{definition}
                
                \begin{lemma}[$\infty$-categories of geometric $\infty$-stacks] \label{lemma: derived_geometric_stack_categories}
                    Let $(\C, J)$ be a small $\infty$-site. Then, we have the following tower of full weak $(\infty, 2)$-subcategories that are all \textit{closed under arbitrary weak $(\infty, 2)$-limits and under finite weak $(\infty, 2)$-colimits}:
                        $$
                            \begin{tikzcd}
                            	{\infty\-\Stk(\C, J)} \\
                            	{\Sh_{(\infty,1)}(\C, J)} \\
                            	{\infty\-\Stk^{\geom}(\C, J)}
                            	\arrow[no head, from=3-1, to=2-1]
                            	\arrow[no head, from=2-1, to=1-1]
                            \end{tikzcd}
                        $$
                    wherein $\infty\-\Stk^{\geom}(\C, J)$ is the weak $(\infty, 2)$-category of geometric $\infty$-stacks on $(\C, J)$.
                \end{lemma}
                    \begin{proof}
                        
                    \end{proof}
                
                \begin{theorem}[$(\infty, 1)$-sheafification of geometric $\infty$-prestacks] \label{theorem: (infinity,1)_sheafification_of_derived_geometric_prestacks}
                    Let $(\C, J)$ be a small $\infty$-site. Then, every weak $(\infty, 2)$-geometric embedding of the $\infty$-topos $\Sh_{(\infty, 1)}(\C, J)$ into the $(\infty, 1)$-presheaf $\infty$-topos $\Psh_{(\infty, 1)}(\C, J)$ can be restricted down to a weak $(\infty, 2)$-geometric embedding of the weak $(\infty, 2)$-category of geometric $\infty$-stacks $\Stk^{\geom}(\C, J)$ into the weak $(\infty, 2)$-category of geometric $\infty$-prestacks $\infty\-\Pre\Stk^{\geom}(\C)$ (i.e. geometric $\infty$-stacks on $\C$ equipped with the chaotic topology); pictorially, one might think of this statement as the following diagram being commutative in the weak $(\infty, 2)$-category ${}^{(\infty, 2)}1-\Cat$ of weak $2$-categories: 
                        $$
                            \begin{tikzcd}
                            	{\Sh_{(\infty, 1)}(\C, J)} & {\Psh_{(\infty, 1)}(\C)} \\
                            	{\infty\-\Stk^{\geom}(\C, J)} & {\infty\-\Pre\Stk^{\geom}(\C)}
                            	\arrow[""{name=0, anchor=center, inner sep=0}, "{{}^{\sh}(-)}"', shift right=2, shorten <=2pt, from=1-2, to=1-1]
                            	\arrow[""{name=1, anchor=center, inner sep=0}, "i"', shift right=2, hook, from=1-1, to=1-2]
                            	\arrow[shift left=2, hook', from=2-1, to=1-1]
                            	\arrow[shift left=2, hook', from=2-2, to=1-2]
                            	\arrow[""{name=2, anchor=center, inner sep=0}, "{i^{\geom}}"', shift right=2, hook, from=2-1, to=2-2]
                            	\arrow[""{name=3, anchor=center, inner sep=0}, "{{}^{\sh}(-)^{\geom}}"', shift right=2, from=2-2, to=2-1]
                            	\arrow["\dashv"{anchor=center, rotate=-90}, draw=none, from=0, to=1]
                            	\arrow["\dashv"{anchor=center, rotate=-90}, draw=none, from=3, to=2]
                            \end{tikzcd}
                        $$
                    Note that the restricted geometric embedding $({}^{\sh}(-)^{\geom} \ladjoint i^{\geom})$ is well-defined thanks to categories of geometric (pre)stacks embedding fully faithfully into $(\infty, 1)$-(pre)sheaf $\infty$-topoi.
                \end{theorem}
                    \begin{proof}
                        
                    \end{proof}
            
            \subsubsection{Connectivity and co-connectivity of derived geometric stacks}
        
        \subsection{Derived schemes and derived algebraic spaces}
            \subsubsection{As derived DM-stacks}
            
            \subsubsection{As structured spaces \textit{\`a la} Lurie}
        
    \section{Quasi-coherent sheaves} \label{section: qcoh}
        Let's be real: people - especially modern representation theorists disguised as geometers - care absolutely zilch about the \textit{actual} geometric objects in algebraic geometry. Instead, we sought after a notion of \say{local-global compatible} linear algebra, i.e. a way to probe geometric structures using the abundance of knowledge that has been accumulated through the centuries in the field of linear algebra. For instance, and to bring up representation theory again, by gaining and understanding of their linear representations, one effectively knows all there is to know about mysterious and exotic objects such as groups and Lie algebras, which (and especially the case of groups) are ultimately geometric in nature: groups are nothing but local versions of group schemes. Now that's all well and good, but how do we actually perform this linear-algebraic black magic ? The answer is via quasi-coherent modules.
        
        Quasi-coherent modules are quite literally everywhere in algebraic geometry. Classically (i.e. \`a la EGA or Stacks Project), one would define these objects as sheaves of modules on schemes/algebraic spaces/what-have-you or even just topological spaces and sites that satisfy certain local-to-global compatibility conditions (see \cite[\href{https://stacks.math.columbia.edu/tag/01BD}{Tag 01BD}]{stacks} for a reminder of this traditional formulation). This approach, while seemingly technically simple, has one flaw that is rather hard to ignore: the local-to-global transition is not very \say{categorical}, meaning that this seemingly perfectly usable definition of quasi-coherent modules will might bring home technical difficulties that at some point, might make us (read: the author) decide to throw in the towel and go downstairs for a biscuit instead. It is due to this reason that in this book, we are going to approach quasi-coherent modules from a $2$-categorical angle: namely, we shall be studying the symmetric monoidal (abelian) categories of quasi-coherent modules and continuous functors between them, instead of objects therein (in fact, we care little about the actual quasi-coherent modules themselves), how these categories are parametrised by underlying sites of schemes, and how everything is packaged together by so-called \say{stacks of quasi-coherent modules}, which form $2$-categories in a somewhat obvious manner. We should also note that such an approach is $2$-categorical due to the important fact that the associativity that compositions of base change functors between categories of quasi-coherent modules are \textit{supposed} to enjoy is only preserved up to natural equivalences; heuristically, one can think about how the functors:
            $$(- \tensor R) \tensor S$$
        and:
            $$- \tensor (R \tensor S)$$
        are only naturally isomorphic and not identical. 
        
        Admittedly, this is a very high-tech approach, and while the classical low-tech approach has been employed to produce astounding results such as Serre's Criterion for Affineness and the Grothendieck-Riemann-Roch Theorem, we wholeheartedly believe that it has merits. In particular:
            \begin{enumerate}
                \item The stack of quasi-coherent modules over site of schemes tells us intuively as well as \textit{formally} (i.e. \textit{sans} hand-wavy sheaf restriction) how local and global sections of quasi-coherent sheaves are related via change of scalar operations. 
                \item The fact that quasi-coherent module categories possess symmetric monoidal structures is given much emphasis in this formulation.
                \item The homotopy-isation of our setup shall be rather self-evident.
            \end{enumerate}
        With that being said, let us proceed by discussing how the category of quasi-coherent modules on a given scheme is defined via a \textit{stack} on whatever site one might associate to said scheme.
    
        \begin{convention}[Everything is derived!] \label{conv: schemes_2_everything_is_derived}
            \noindent
            \begin{itemize}
                \item From now on until the end of the chapter, everything will be assumed to be derived. 
                \item By $1-\Cat_1$, or simply $1-\Cat$, we shall actually mean $(\infty, 1)\-1-\Cat_1$, i.e. the $(\infty, 1)$-category of $(\infty, 1)$-categories and functors between them, and by $1-\Cat_2$ we will be referring to the $(\infty, 2)$-category of $(\infty, 1)$-categories, functors between them, and natural transformations between these functors. 
                
                Similarly, by $\Grpd_1$, or simply $\Grpd$, we will actually mean the $(\infty, 1)$-category of $\infty$-groupoids and functors between them, and by $\Grpd_2$, we shall mean the $(\infty, 2)$-category of $\infty$-groupoids, functors between them, and natural transformations between these functors.
                \item A subcategory of $1-\Cat$ this is of particular interest is $\dg\Cat^{\cont}_2$ (or simply $\dg\Cat^{\cont}$), the $(\infty, 2)$-category of stable linear (i.e. differential-graded) $(\infty, 1)$-categories (see section \ref{section: homological_algebra} for the notion of stable $(\infty, 1)$-categories). Of course, we can also view $\dg\Cat^{\cont}$ as a mere $(\infty, 1)$-category; when necessary, we shall write $\dg\Cat^{\cont}_1$ to put emphasis on the disregard of $2$-morphisms.
            \end{itemize} 
        \end{convention}
        
        \subsection{Categories of quasi-coherent sheaves}
            \begin{definition}[Quasi-coherent modules] \label{def: qcoh_def}
                \noindent
                \begin{enumerate}
                    \item \textbf{(Quasi-coherent modules):} Our goal is to construct categories of quasi-coherent sheaves on schemes and algebraic spaces/stacks in a manner that is as adaptable to the world of derived algebraic geometry as possible, because at the end of the day, one cares most about cohomologies of quasi-coherent modules, and these \say{things} naturally inhabit stable linear $(\infty,1)$-categories - whatever that means - and to that end, let us consider firstly a sketch of the theory. For every affine scheme $\Spec R$, let us \textit{declare} that:
                        $$\QCoh(\Spec R) \cong R\mod$$
                    and for prestacks of groupoids $\calY$ on $\Cring^{\op}$ (a class of objects which subsumes that of affine schemes; schemes and algebraic stacks are instances of prestacks on $\Cring^{\op}$), fitting into commutative diagrams of prestacks as follows:
                        $$
                            \begin{tikzcd}
                            	{\Spec R'} && {\Spec R} \\
                            	& \calY
                            	\arrow["{y'}"', from=1-1, to=2-2]
                            	\arrow["y", from=1-3, to=2-2]
                            	\arrow["f", from=1-1, to=1-3]
                            \end{tikzcd}
                        $$
                    suppose that there is a functor $f^*: \QCoh(\Spec R) \to \QCoh^*(\Spec R')$ such that for all objects $M_{\Spec R', y'} \in \QCoh(\Spec R')$, there exists an object $M_{\Spec R, y} \in \QCoh^*(\Spec R)$ so that:
                        $$M_{\Spec R', y'} \cong f^* M_{\Spec R, y}$$
                    We shall be calling the categories of the form $\QCoh^*(\calY)$ categories of \textbf{quasi-coherent modules} on $\calY$.
                    \item \textbf{(Categories of quasi-coherent modules):} More precisely, an object $M \in \QCoh^*(\calY)$ is a prestack (cf. convention \ref{conv: prestacks}):
                        $$M: (\Sch^{\aff}_{/\calY})^{\op} \to 1-\Cat$$
                    which:
                        \begin{enumerate}
                            \item sends the affine schemes $y: \Spec R \to \calY$ to the categories $\QCoh(\Spec R)$.
                            \item and sends commutative diagrams in $\Sch^{\aff}_{/\calY}$ as below:
                                $$
                                    \begin{tikzcd}
                                    	{\Spec R''} & {\Spec R'} & {\Spec R} \\
                                    	& \calY
                                    	\arrow["{y''}"', from=1-1, to=2-2]
                                    	\arrow["{y'}", from=1-2, to=2-2]
                                    	\arrow["{f'}", from=1-1, to=1-2]
                                    	\arrow["y", from=1-3, to=2-2]
                                    	\arrow["f", from=1-2, to=1-3]
                                    \end{tikzcd}
                                $$
                            to diagrams in $1-\Cat$ as below:
                                $$
                                    \begin{tikzcd}
                                    	{\QCoh(\Spec R'')} & {\QCoh(\Spec R')} & {\QCoh(\Spec R)}
                                    	\arrow["{M(f')}"', from=1-2, to=1-1]
                                    	\arrow["{M(f)}"', from=1-3, to=1-2]
                                    \end{tikzcd}
                                $$
                            such that one has natural isomorphisms between $M(f' \circ f)$ and $M(f') \circ M(f)$ that are not necessarily the identity.
                        \end{enumerate}
                    and a morphism in $\QCoh(\calY)$ is just a \href{https://ncatlab.org/nlab/show/pseudonatural+transformation}{\underline{pseudo-natural transformation}} (i.e. a morphism of pseudo-functors).
                    \item \textbf{(Prestacks of quasi-coherent modules):} Having defined categories of quasi-coherent modules on categories of affine schemes over prestacks of groupoids on $\Cring^{\op}$, let us try to define the \textbf{prestack of quasi-coherent modules} on the category $\Pre\Stk$ of prestacks on $\Cring^{\op}$, which we shall denote by:
                        $$\QCoh^*: \Pre\Stk^{\op} \to 1-\Cat$$
                    Such a prestack will associate to each commutative diagrams in $\Pre\Stk$ as below:
                        $$
                            \begin{tikzcd}
                            	{\calY''} & {\calY'} & \calY
                            	\arrow["{\varphi'}", from=1-1, to=1-2]
                            	\arrow["\varphi", from=1-2, to=1-3]
                            \end{tikzcd}
                        $$
                    to diagrams in $1-\Cat$ as below:
                        $$
                            \begin{tikzcd}
                            	{\QCoh^*(\calY'')} & {\QCoh^*(\calY')} & {\QCoh^*(\calY)}
                            	\arrow["{\QCoh^*(\varphi')}"', from=1-2, to=1-1]
                            	\arrow["{\QCoh^*(\varphi)}"', from=1-3, to=1-2]
                            \end{tikzcd}
                        $$
                    such that one has natural isomorphisms between $\QCoh^*(\varphi' \circ \varphi)$ and $\QCoh^*(\varphi') \circ \QCoh^*(\varphi)$ that are not necessarily the identity. In short, $\QCoh^*(-)$ is a prestack that sends objects $\calY \in \Pre\Stk^{\op}$ to $1$-categories $\QCoh^*(\calY)$ of prestacks $M$ that in turn assign to affine schemes over $\calY$ the appropriate module categories. 
                    
                    In the event that the prestacks $\calY'', \calY'$, and $\calY$ are affine schemes over some base prestack $\calY_0$, i.e. by replacing the category $\Pre\Stk^{\op}$ with $\Sch^{\aff}_{/\calY_0}$, one recovers the prestack $\QCoh^*(\calY_0)$. 
                \end{enumerate}
            \end{definition}
            \begin{remark}[The purpose of quasi-coherent modules]
                Ultimately, we are trying to build a theory of quasi-coherent modules wherein categories thereof over schemes can be obtained via gluing together those on the covering affine schemes. In other words, the theory of quasi-coherent modules ought to look like a globalisation of the theory of modules over commutative rings. This goal shall be realised fully via descent theory (cf. \cite{vistoli_descent}).
            \end{remark}
            \begin{remark}[Why prestacks ?]
                We should note that definition \ref{def: qcoh_def} is not standard. Most textbooks will define \textit{objects} of quasi-coherent module categories as modules over structure sheaves that satisfy certain cohomological conditions. The problem with this formulation is that even though abstract entities such as pseudo-functors are not needed for it, it relies very heavily on the idea of sheaf restriction, which is not very well-defined. Prestacks allow us to avoid that technical hiccup; they also highlight an important point: non-quasi-coherent modules are not very \say{algebraic}.
            \end{remark}
            
            Now that we have managed to write down the definition of what it means for a category to have quasi-coherent modules as objects, let us examine some basic properties of the prestack of quasi-coherent modules on $\Pre\Stk$. The first of these is its universal property.
            \begin{proposition}[Universal property of quasi-coherent modules] \label{prop: qcoh_universal_property}
                Fix a base commutative ring $k$. Then, the prestack:
                    $$\QCoh^*: \Pre\Stk_{/\Spec k}^{\op} \to 1-\Cat$$
                of quasi-coherent modules on the $1$-category $\Pre\Stk_{/\Spec k}$ of prestacks of groupoids on ${}^{k/}\Comm\Alg^{\op}$ is the \textit{right}-Kan extension of:
                    $$\QCoh^*|_{\Sch^{\aff}_{/\Spec k}}: \Sch^{\aff, \op}_{/\Spec k} \to 1-\Cat$$
                along the canonical embedding $\Sch^{\aff, \op} \hookrightarrow \Pre\Stk_{/\Spec k}^{\op}$. In particular, we have:
                    $$\QCoh^*(\calY) \cong \underset{S \in \Sch^{\aff}_{/\calY}}{\lim} \QCoh^*(S)$$
                for all $\calY \in \Pre\Stk_{/\Spec k}$.
            \end{proposition}
                \begin{proof}
                    
                \end{proof}
            \begin{remark}[The correct definition of quasi-coherent modules]
                To be quite honest, definition \ref{def: qcoh_def} should be thought of less as a proper definition of the prestack of quasi-coherent modules and more of a preliminary discussion. We have only granted it the title of \say{Definition} and the honour that goes along with it because proposition \ref{prop: qcoh_universal_property} is a rather non-trivial phenomenon. 
            \end{remark}
            
        \subsection{\texorpdfstring{$*$}{}-pushforwards of quasi-coherent modules} \label{subsubsection: qcoh_*_pushforwards}
                    
        \subsection{The rigid symmetric monoidal structure on \texorpdfstring{$\QCoh$}{}}
        
        \subsection{Actions of monoidal categories; 1-affineness}
            Informally, a sheaf of categories with coefficients in some category $\C$ over a prestack $\calY$ is an assignment:
                $$\Sch^{\aff}_{/\calY} \to \dg\Cat: (S \to \calY) \mapsto \bfGamma(S, \C)$$
            of affine schemes $S$ over $\calY$ to dg-categories $\bfGamma(S, \C)$ with a $\QCoh(S)$-action, i.e. to objects $\bfGamma(S, \C) \in \QCoh(S)\mod$. Furthermore, we shall require that this assingment is functorial (up to homotopies, of course), in the sense that given any arrow $f: S' \to S$ in $\Sch^{\aff}_{/\calY}$, we get a corresponding isomorphism of $\QCoh(S)$-modules:
                $$\bfGamma(S', \C) \cong \QCoh(S') \tensor_{\QCoh(S)} \bfGamma(S, \C)$$
            Lastly, the functor $\Gamma(-, \C): \Sch^{\aff}_{/\calY} \to \dg\Cat$ shall have to somehow satisfy descent in order to be a sheaf. 
            
            In this subsection, we shall attempt to set up such a theory.
            
            \subsubsection{Quasi-coherent sheaves of categories}
            
            \subsubsection{1-affineness}
        
    \section{Ind-coherent sheaves} \label{section: indcoh}
        \subsection{Categories of ind-coherent sheaves} \label{subsection: categories_of_ind_coherent_sheaves}
            \subsubsection{Ind-coherent sheaves over locally almost of finite type schemes}
                \begin{definition}[Ind-coherent sheaves] \label{def: ind_coherent_sheaves_on_laft_schemes}
                    Let $k$ be an arbitrary base commutative ring, $X$ be a scheme \textit{locally almost of finite type} over $\Spec k$. Then, the category $\Ind\Coh(X)$ of ind-coherent modules on $X$ is precisely the ind-completion of $\Coh(X)$. 
                \end{definition}
            
                Having stated the definition, let us now investigate the (desired) formal properties of ind-coherent sheaves over the simplest setup possible (at least from a homological point-of-view), namely that of locally almost of finite type schemes. 
                \begin{convention} \label{conv: indcoh_to_qcoh_functor}
                    $\Ind\Coh(\calY)$ is a cocomplete full subcategory of $\QCoh(\calY)$ for all prestacks $\calY$ locally almost of finite type. Typically, the evident fully faithful embedding is denoted by $\Psi_{\calY}: \Ind\Coh(\calY) \to \QCoh(\calY)$. 
                \end{convention}
                
                \paragraph{Homological characterisations of ind-coherent sheaves}
                    \begin{theorem}[Ind-coherent modules on classical schemes] \label{theorem: indcoh_on_classically__regular_and_smooth_schemes}
                        If a $0$-coconnective scheme $X$ is locally almost of finite type\footnote{Recall that locally almost of finite type $0$-coconnective schemes are just locally Noetherian in the usual sense.} and regular then $\Psi_X$ is an equivalence. This implication is an equivalence if and only if $X$ is smooth.
                    \end{theorem}
                        \begin{proof}
                            It is known that whenever $X$ is Noetherian as a $0$-coconnective scheme, $\QCoh(X)$ is compactly generated by its (small) subcategory of perfect complexes, i.e. $\QCoh(X) \cong \Ind(\QCoh(X)^{\perf})$. It is also known that over $0$-coconnective regular schemes $X$, one has $\Coh(X) \cong \QCoh(X)^{\perf}$. Thus, if $X$ is a $0$-coconnective regular scheme then one has an equivalence $\Psi_X: \Ind\Coh(X) \cong \QCoh(X)$.
                            
                            \todo[inline]{Finish this up}
                        \end{proof}
                        
                    Next, recall that for any scheme $X$, $\Coh(X)$ carries a natural t-structure (cf. definition \ref{def: t_structures}), which is a sub-t-structure of the t-structure of $\QCoh(X)$. As it happens, this t-structure gets passed along to $\Ind\Coh(X)$ whenever $X$ is locally almost of finite type.
                    \begin{lemma}[t-structures of ind-coherent sheaves] \label{lemma: t_structure_of_ind_coherent_sheaves}
                        Let $X$ be a locally almost of finite type scheme. Then:
                            \begin{enumerate}
                                \item $\Ind\Coh(X)$ inherits a $t$-structure from $\Coh(X)$ whose accompanying truncation functors commute with filtered colimits. 
                                \item the canonical fully faithful embedding $\Coh(X) \subset \Ind\Coh(X)$ is $t$-exact.
                            \end{enumerate}
                    \end{lemma}
                        \begin{proof}
                                        
                        \end{proof}
                    \begin{proposition}[Truncations of ind-coherent sheaves] \label{prop: truncations_of_ind_coherent_sheaves}
                        Let $X$ be a locally almost of finite type scheme. Then for all $n \in \Z$, the truncated canonical embeddings:
                            $$\Psi_X^{\geq n}: \Ind\Coh(X)^{\geq n} \hookrightarrow \QCoh(X)^{\geq n}$$
                        are actually equivalences of categories.
                    \end{proposition}
                        \begin{proof}
                                        
                        \end{proof}
                    \begin{corollary}[Connectivity of ind-coherent sheaves] \label{coro: connectivity_of_ind_coherent_sheaves}
                        \noindent
                        \begin{enumerate}
                            \item An ind-coherent module of over a locally almost of finite type scheme is connective if and only if it is so as a quasi-coherent module.
                            \item For every locally almost of finite type scheme, one has an equivalence $\Ind\Coh(X)^{\perf} \cong \Coh(X)$.
                        \end{enumerate}
                    \end{corollary}
                    
                    \begin{proposition}[$\QCoh$ is the left-completion of $\Ind\Coh$] \label{prop: qcoh_is_the_left_completion_of_indcoh}
                        Over a Notherian scheme $X$, one recognises $\QCoh(X)$ as the left-completion of $\Ind\Coh(X)$ in its t-structure.
                    \end{proposition}
                        \begin{proof}
                            
                        \end{proof}
                    
                \paragraph{An action of \texorpdfstring{$\QCoh(X)$}{} on \texorpdfstring{$\Ind\Coh(X)$}{}}
                    It is not hard to see that for any prestack locally almost of finite type $\calY$, the category $\Ind\Coh(\calY)$ of ind-coherent sheaves over it has a natural structure of a dg-category. What this means is that $\Ind\Coh(\calY)$ is an object of the $1$-category $\dg\Cat^{\cont}_1$ of dg-categories and continuous functors between them. On this category, there exists a natural monoidal structure inherited from the $1$-category of stable $\infty$-categories, which is the Lurie tensor product $\tensor$. 
                
                    Now, one thing to note is that for each locally almost of finite type scheme $X$, the evident embedding $\Psi_X: \Ind\Coh(X) \hookrightarrow \QCoh(X)$ is a continuous functor. In particular, this means one can construct continuous functors $\QCoh(X) \x \Ind\Coh(X) \to \Ind\Coh(X)$, and hence the Lurie tensor product $\QCoh(X) \tensor \Ind\Coh(X)$, which satisfies the following universal property:
                        $$
                            \begin{tikzcd}
                            	{\QCoh(X) \tensor \Ind\Coh(X)} & \calA \\
                            	{\QCoh(X) \x \Ind\Coh(X)}
                            	\arrow[from=2-1, to=1-2]
                            	\arrow[dashed, from=1-1, to=1-2]
                            	\arrow["\tensor", from=2-1, to=1-1]
                            \end{tikzcd}
                        $$
                    (wherein, of course, $\calA$ is an arbitrary dg-category and the arrows are continuous functors). As a consequence, we can construct so-called \textbf{actions} of $\QCoh(X)$ on $\Ind\Coh(X)$, which are just continuous functors:
                        $$\alpha: \QCoh(X) \tensor \Ind\Coh(X) \to \Ind\Coh(X)$$
                    via continuous functors $\underline{\alpha}: \QCoh(X) \x \Ind\Coh(X) \to \Ind\Coh(X)$
                        
                    \begin{proposition}[The canonical action of $\QCoh$ on $\Ind\Coh$] \label{prop: canonical_action_of_qcoh_on_indcoh}
                        Let $X$ be a locally almost of finite type scheme. Then, there exists a continuous functor $\bar{\alpha}_X: \QCoh(X) \x \Ind\Coh(X) \to \Ind\Coh(X)$ rendering the following diagram commutative:
                            $$
                                \begin{tikzcd}
                                	{\QCoh(X) \x \Ind\Coh(X)} & {\Ind\Coh(X)} \\
                                	{\QCoh(X) \x \QCoh(X)} & {\QCoh(X)}
                                	\arrow["{\id \x \Psi_X}"', from=1-1, to=2-1]
                                	\arrow["\tensor", from=2-1, to=2-2]
                                	\arrow["{\bar{\alpha}_X}", from=1-1, to=1-2]
                                	\arrow["{\Psi_X}", from=1-2, to=2-2]
                                \end{tikzcd}
                            $$
                    \end{proposition}
                        \begin{proof}
                            
                        \end{proof}
                    \begin{corollary} \label{coro: canonical_action_of_qcoh_on_indcoh}
                        For each Notherian scheme $X$, there exists a canonically defined $\QCoh(X)$-action on $\Ind\Coh(X)$ coming from $\bar{\alpha}_X$ as above; this means that this action is given by:
                            $$\alpha_X(\calF \tensor \E) \cong \calF \tensor \Psi_X(\E)$$
                    \end{corollary}
                
                \paragraph{Ind-coherent sheaves over eventually coconnective schemes}
                    \begin{definition}[Eventually coconnective schemes] \label{def: eventually_coconnective_schemes}
                        A scheme which is locally almost of finite type is said to be \textbf{eventually coconnective} if and only if it admits a Zariski atlas by affine schemes almost of finite type.
                        
                        Equivalently - and perhaps more succinctly - a scheme $X$ is eventually coconnective if and only if $\calO_X \in \Coh(X)$.
                    \end{definition}
                    \begin{remark}
                        We note that being eventually coconnective implies being locally almost of finite type, since the latter notion is only that the structure sheaf is Zariski-locally coherent. In particular, this means that one can still consider ind-coherent sheaves (which are only defined over schemes locally of finite type) over eventually coconnective schemes.
                    \end{remark}
                    
                    \begin{proposition}[A canonical adjunction] \label{prop: canonical_adjunction} 
                        Let $X$ be an eventually coconnective scheme. Then, there exists an adjunction as follows, wherein $\Xi_X$ is fully faithful:
                            $$
                                \begin{tikzcd}
                                	{\Ind\Coh(X)} & {\QCoh(X)}
                                	\arrow[""{name=0, anchor=center, inner sep=0}, "{\Psi_X}"', shift right=2, from=1-1, to=1-2]
                                	\arrow[""{name=1, anchor=center, inner sep=0}, "{\Xi_X}"', shift right=2, hook', from=1-2, to=1-1]
                                	\arrow["\dashv"{anchor=center, rotate=-90}, draw=none, from=1, to=0]
                                \end{tikzcd}
                            $$
                    \end{proposition}
                        \begin{proof}
                            
                        \end{proof}
                
            \subsubsection{Ind-coherent sheaves over schemes not locally of finite type}
            
        \subsection{Basic functoriality via \texorpdfstring{$*$}{}-pullbacks and \texorpdfstring{$*$}{}-pushforwards}
            The upshot of this subsection is that ind-coherent sheaves, via pullback and pushforward functors enjoy a certain flavour of functoriality. In subsection \ref{subsection: categories_of_ind_coherent_sheaves}, we have already studied the objects of these to-be categories, so now, let us figure out what the ($1$-)morphisms ought to be. In fact, the resulting category $\Ind\Coh$ will turn out to be fibred over $\Sch^{\aft}$, and moreover inherits many important properties from $\QCoh$.
        
            \begin{convention}
                We continue to work over schemes almost of finite type (read: Noetherian) throughout this subsection.
            \end{convention}
            
            \subsubsection{\texorpdfstring{$*$}{}-pushforwards}
                \begin{proposition}[Existence of $*$-pushforwards] \label{prop: indcoh_*_pushforwards}
                    Let $f: X \to Y$ be any morphism in $\Sch^{\aft}$. Then, there exists a corresponding uniquely defined t-exact functor:
                        $$f_*|_{\Ind\Coh}: \Ind\Coh(X) \to \Ind\Coh(Y)$$
                    fitting into the following commutative diagram in $\dg\Cat^{\cont}_1$:
                        $$
                            \begin{tikzcd}
                            	{\Ind\Coh(X)} & {\QCoh(X)} \\
                            	{\Ind\Coh(Y)} & {\QCoh(X)}
                            	\arrow["{\Psi_X}", from=1-1, to=1-2]
                            	\arrow["{f_*|_{\Ind\Coh}}"', from=1-1, to=2-1]
                            	\arrow["{f_*}", from=1-2, to=2-2]
                            	\arrow["{\Psi_Y}", from=2-1, to=2-2]
                            \end{tikzcd}
                        $$
                \end{proposition}
                    \begin{proof}
                        
                    \end{proof}
                \begin{corollary}[Compatibility of $*$-pushforwards and $\QCoh$-actions] \label{coro: *_pushforwards_and_qcoh_actions}
                    Let $f: X \to Y$ be any morphism in $\Sch^{\aft}$. Then, via the pushforward functor $f_*|_{\Ind\Coh}: \Ind\Coh(X) \to \Ind\Coh(Y)$, one obtains a natural $\QCoh(Y)$-action on $\Ind\Coh(X)$ that is compatible with the $\QCoh(Y)$-action on $\QCoh(X)$ induced by $f_*: \QCoh(X) \to \QCoh(Y)$.
                \end{corollary}
                
                Now, to establish a proper functoriality for ind-coherent sheaves (cf. theorem \ref{theorem: indcoh_functoriality}), we will need to go on a technical detour.
                \begin{convention}
                    \noindent
                    \begin{itemize}
                        \item Let $(\dg\Cat^{\cont, \co\compl}_{\tstructure^{\pm}, \access(\geq 0), \comp\gen(+)})_1$ be the $1$-full subcategory of $\dg\Cat^{\cont}_1$ wherein:
                            \begin{itemize}
                                \item objects are \textit{cocomplete} dg-categories $\calA$ endowed with t-structures such that $\calA$ is compactly generated by $\calA^+$ and that $\calA^{\geq 0}$ is accessible, and
                                \item $1$-morphisms are \textit{continuous} functors $F: \calA \to \calB$ between such non-cocomplete dg-categories which are left-t-exact up a finite shift and such that the restriction $F|_{\calA^{\geq 0}}$ is accessible. 
                            \end{itemize}
                        \item Let $(\dg\Cat^{\non\co\compl}_{\tstructure^+, \access(\geq 0)})_1$ be the $1$-full subcategory of $\dg\Cat_1$ wherein:
                            \begin{itemize}
                                \item objects are \textit{non-cocomplete} dg-categories $\calA$ endowed with t-structures such that:
                                    $$\calA \cong \calA^+$$
                                (i.e. the objects of $\calA$ are all eventually coconnective) and that $\calA^{\geq 0}$ is accessible, and
                                \item $1$-morphisms are functors $F: \calA \to \calB$ between such non-cocomplete dg-categories which are left-t-exact up a finite shift and such that the restriction $F|_{\calA^{\geq 0}}$ is accessible. 
                            \end{itemize}
                    \end{itemize}
                \end{convention}
                \begin{lemma} \label{lemma: selecting_a_compact_generator}
                    There exists a natural $1$-fully faithful embedding:
                        $$(\dg\Cat^{\cont, \co\compl}_{\tstructure^{\pm}, \access(\geq 0), \comp\gen(+)})_1 \to (\dg\Cat^{\non\co\compl}_{\tstructure^+, \access(\geq 0)})_1$$
                    defined via the assignment:
                        $$\calA \mapsto \calA^+$$
                \end{lemma}
                    \begin{proof}
                        Denote the assignment in question by $(-)^+$. Then, observe that for every object $\calA \in (\dg\Cat^{\cont, \co\compl}_{\tstructure^{\pm}, \access(\geq 0), \comp\gen(+)})_1$, the corresponding object $\calA^+ \in (\dg\Cat^{\non\co\compl}_{\tstructure^+, \access(\geq 0)})_1$ compactly generates $\calA$ by the very construction of $(\dg\Cat^{\cont, \co\compl}_{\tstructure^{\pm}, \access(\geq 0), \comp\gen(+)})_1$. Also, note that any $\calA' \in (\dg\Cat^{\non\co\compl}_{\tstructure^+, \access(\geq 0)})_1$ satisfies $\calA' \cong (\calA')^+$ by definition. By putting the two observations together one sees that there is a natural equivalence:
                            $$\Maps(\calA, \calB) \cong \Maps(\calA^+, \calB^+)$$
                        which implies that indeed, there exists a natural $1$-fully faithful embedding:
                            $$(\dg\Cat^{\cont, \co\compl}_{\tstructure^{\pm}, \access(\geq 0), \comp\gen(+)})_1 \to (\dg\Cat^{\non\co\compl}_{\tstructure^+, \access(\geq 0)})_1$$
                        defined via the assignment:
                            $$\calA \mapsto \calA^+$$
                    \end{proof}
                \begin{lemma}
                    
                \end{lemma}
                    \begin{proof}
                        
                    \end{proof}
                \begin{theorem}[Functoriality of $\Ind\Coh$] \label{theorem: indcoh_functoriality}
                    There exists a functor:
                        $$\Ind\Coh_*|_{\Sch^{\aft}}: \Sch^{\aft} \to \dg\Cat^{\cont}_1$$
                    along with a natural tranformation:
                        $$\Psi|_{\Sch^{\aft}}: \Ind\Coh_*|_{\Sch^{\aft}} \to \QCoh_*|_{\Sch^{\aft}}$$
                    which at the level of components, associates to each morphism $f: X \to Y$ in $\Sch^{\aft}$ a commutative diagram as follows:
                        $$
                            \begin{tikzcd}
                            	{\Ind\Coh(X)} & {\QCoh(X)} \\
                            	{\Ind\Coh(Y)} & {\QCoh(X)}
                            	\arrow["{\Psi_X}", from=1-1, to=1-2]
                            	\arrow["{f_*|_{\Ind\Coh}}"', from=1-1, to=2-1]
                            	\arrow["{f_*}", from=1-2, to=2-2]
                            	\arrow["{\Psi_Y}", from=2-1, to=2-2]
                            \end{tikzcd}
                        $$
                \end{theorem}
                    \begin{proof}
                        
                    \end{proof}
    
            \subsubsection{\texorpdfstring{$*$}{}-pullbacks}
                \paragraph{\texorpdfstring{$*$}{}-pullbacks along eventually coconnective morphisms}
                
                \paragraph{Base change along eventually coconnective morphisms}
                
            \subsubsection{Restricting down to subschemes}
            
            \subsubsection{Descent for \texorpdfstring{$\Ind\Coh$}{}}
        
        \subsection{\texorpdfstring{$\Ind\Coh$}{} as a functor out of the category of correspondences}
            \begin{convention}
                Since everything is derived (cf. convention \ref{conv: schemes_2_everything_is_derived}), by \say{$n$-category} we will actually mean \say{$(\infty, n)$-category}.
            \end{convention}
                
            \subsubsection{Ind-coherent sheaves via correspondences}
        
            \subsubsection{!-pullbacks and Grothendieck duality}