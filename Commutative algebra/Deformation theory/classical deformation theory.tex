\section{Classical deformation theory}
    \subsection{Formal deformation theory}
        \subsubsection{A high-level overview}
            Deformation theory is a large topic, so let us preface our treatment of it with an overview of its features, along with the notable results in the subject. Our main references are \cite[\href{https://stacks.math.columbia.edu/tag/06G7}{Tag 06G7}]{stacks} and \cite[\href{https://stacks.math.columbia.edu/tag/08KW}{Tag 08KW}]{stacks}.
            
            For now, assume that $(\Lambda, \m, k)$ is the data of a Noetherian local ring with maximal ideal $\m$ and residue field $k$ and let $\C_{\Lambda, k}$ be the category of local Artinian $\Lambda$-algebras (always assumed to be commutative and unital)\footnote{The $\C$ stands for \say{deformation \textbf{c}ontext} whose residue fields are isomorphic to $k$, referred to in \cite[\href{https://stacks.math.columbia.edu/tag/06G7}{Tag 06G7}]{stacks} simply as the \say{base category}.}. At the same time, consider a left-exact functor (or should the reader prefer, a so-called \say{prestack} on $\C_{\Lambda, k}^{\op}$):
                $$F: \C_{\Lambda, k} \to \Grpd$$
            which we shall refer as a \textbf{predeformation functor}. Now, let $\hat{\C}_{\Lambda, k}$ be the category of strict pro-objects associated to $\C_{\Lambda, k}$, which is the full subcategory of the pro-completino $\Pro(\C_{\Lambda, k})$ spanned by cofiltered diagrams whose vertices are objects of $\C_{\Lambda, k}$ and whose (directed) edges are epimorphisms. In deformation theory, one seeks criteria under which predeformation functors are \textbf{strictly pro-representable}, which is to say that they are represented by objects of $\hat{\C}_{\Lambda, k}$ and thus can be thought of as objects belonging to formal geometry. As a consequence, many foundationally important results such as Grothendieck's Criteria for Pro-representability or Schlessinger's Criterion are of this nature.
    
        \subsubsection{Local Artinian algebras and deformation contexts}
            \paragraph{The deformation context of local Artinian algebras}
                We begin by investigating the category $\C_{\Lambda, k}$ of local Artinian $\Lambda$-algebras (for $\Lambda$ a Noetherian local ring with residue field $k$) whose residue fields are isomorphic to $k$. This is a category that is inherently and concretely geometric in nature and as such shall serve as a template according to which we shall be able to axiomatically define more general so-called \say{\textbf{deformation contexts}}.
                \begin{definition}[Artinian rings] \label{def: artinian_rings} \index{Artinian rings}
                    A commutative ring $\Lambda$ is said to be \textbf{Artinian} if and only if there exist no non-terminating descending chain of ideals (up to bijections, of course). Alternatively, since ideals of commutative rings corespond to Zariski-closed sets, one can define Artinian rings $\Lambda$ as those such that the underlying topological spaces of their corresponding affine schemes $\Spec \Lambda$ have no non-terminating \textit{ascending} chains of closed subsets. 
                    
                    It is not hard to see that for every given base commutative ring $\Lambda$ there is a category whose objects are local Artinian $\Lambda$-algebras and whose morphisms are local homomorphisms between them. We shall denote this category by $\C_{\Lambda}$. Furthermore, for each $R$-algebra $k$ that is a field, there is a corresponding full subcategory of $\C_{\Lambda}$, which we shall denote by $\C_{\Lambda, k}$ spanned by local Artinian $\Lambda$-algebras whose residue field is $k$. 
                \end{definition}
                
                \begin{lemma}[Basic properties of local Artinian rings] \label{lemma: artinian_rings_properties}
                    \noindent
                    \begin{enumerate}
                        \item Quotients and localisations of Artinian rings are also Artinian.
                        \item \cite[\href{https://stacks.math.columbia.edu/tag/00J6}{Tag 00J6}]{stacks} Finitely generated algebras over fields are Artinian.
                        \item A local Artinian ring $(\Lambda, \m)$ with residue field $\kappa$ is a finitely generated $\kappa$-algebra and admits a splitting $\Lambda \cong \kappa \oplus \m$.
                        \item \cite[\href{https://stacks.math.columbia.edu/tag/00J7}{Tag 00J7}]{stacks} Artinian rings only have finitely many maximal ideals.
                        \item \cite[\href{https://stacks.math.columbia.edu/tag/00J8}{Tag 00J8}]{stacks} Let $\Lambda$ be an Artinian ring. Then, its Jacobson radical is nilpotent. In fact, its Jacobson radical shall be the same as its nilradical.
                        \item \cite[\href{https://stacks.math.columbia.edu/tag/00JA}{Tag 00JA}]{stacks} Any commutative ring with finitely many maximal ideals and locally nilpotent Jacobson radical (such as Artinian rings) can be decomposed into the direct sum of its localisations at the maximal ideals. Furthermore, any prime ideal in such a ring is automatically maximal.
                        \item \cite[\href{https://stacks.math.columbia.edu/tag/00JB}{Tag 00JB}]{stacks} A commutative ring $A$ is simultaneously Artinian and Noetherian if and only if $A$ has finite length as a module over itself. 
                    \end{enumerate}
                \end{lemma}
                
                \begin{definition}[Categories of local Artinian algebras] \label{def: categories_of_local_artinian_algebras}
                     Let $(\Lambda, \m, k)$ is the data of a Noetherian local ring with maximal ideal $\m$ and residue field $k$. We denote by $\C_{\Lambda, k}$ the category whose objects are local Artinian $\Lambda$-algebras whose residue fields are isomorphic to $k$ and whose morphisms are local $\Lambda$-algebra homomorphisms between them.
                \end{definition}
                \begin{proposition}[Finite completeness of categories of local Artinian algebras] \label{prop: finite_completeness_of_categories_of_local_artinian_algebras}
                    Let $(\Lambda, \m, k)$ is the data of a Noetherian local ring with maximal ideal $\m$ and residue field $k$. Then, $\C_{\Lambda, k}$ is a finitely complete Artinian category.
                \end{proposition}
                    \begin{proof}
                        That the category $\C_{\Lambda, k}$ is Artinian is obvious, and that it is finitely complete is a trivial consequence of lemma \ref{lemma: artinian_rings_properties}(3).
                    \end{proof}
                    
                \begin{definition}[Strict pro-objects] \label{def: strict_pro_objects}
                    Let $\C$ be a small category and let $\Pro(\C)$ denote its pro-completion. Then, there exists a full subcategory $\hat{\C} \subset \Pro(\C)$ whose objects are cofiltered diagrams whose vertices are objects of $\C$ and whose (directed) edges are epimorphisms; objects of $\hat{\C}$ are referred to as \textbf{strict pro-objects} of $\C$. 
                \end{definition}
                \begin{example}[Completed Artinian local algebras] \label{example: completed_artinian_local_algebras}
                    Let $(\Lambda, \m, k)$ is the data of a Noetherian local ring with maximal ideal $\m$ and residue field $k$. Then, the strict pro-completion $\hat{\C}_{\Lambda, k}$ shall be spanned by pro-objects of $\C_{\Lambda, k}$ which are of the form $\{\cdots \to A/\m_A^n \to \cdots \to A/\m_A^2 \to A/\m_A\}$ for some $(A, \m_A, k) \in \C_{\Lambda, k}$. 
                    
                    Because limits commute, and because $\C_{\Lambda, k}$ is finitely complete and Artinian (cf. proposition \ref{prop: finite_completeness_of_categories_of_local_artinian_algebras}), its strict pro-completion $\hat{\C}_{\Lambda, k}$ is must also be finitely complete and Artinian. It is also not hard to see that via taking limits of the cofiltered diagrams that are objects of $\hat{\C}_{\Lambda, k}$ is equivalent to the category of complete Noetherian local $\hat{\Lambda}$-algebras with residue field isomorphic to $k$.
                \end{example}
                \begin{theorem}[Grothendieck's pro-representability criterion] \label{theorem: grothendieck_pro_representability_criterion}
                    \cite[Proposition 3.1]{grothendieck_fga_2} Let $\C$ be a finitely complete Artinian small category. Then, a functor $F: \C \to \Fin\Sets$ is strictly pro-representable if and only if it is left-exact.
                \end{theorem}
                
                \begin{proposition}[Pushouts and coproducts of completed Artinian local algebras] \label{prop: pushouts_and_coproducts_of_completed_artinian_local_algebras}
                    Let $(\Lambda, \m, k)$ is the data of a Noetherian local ring with maximal ideal $\m$ and residue field $k$. Then, the category $\hat{\C}_{\Lambda, k}$ admits pushouts and initial objects\footnote{.. and as a result, all coproducts}.
                \end{proposition}
                    \begin{proof}
                        
                    \end{proof}
                \begin{corollary}
                    Let $(\Lambda, \m, k)$ is the data of a Noetherian local ring with maximal ideal $\m$ and residue field $k$. Then, the category $\hat{\C}_{\Lambda, k}$ is cocomplete.
                \end{corollary}
                    \begin{proof}
                        This is a direct consequence of the fact that a category is cocomplete if and only if it has all coproducts and coequalisers (cf. \cite[Theorem V.2.1]{maclane}).
                    \end{proof}
                    
            \paragraph{Abstract deformation contexts}
                \begin{definition}[Deformation contexts] \label{def: deformation_context}
                    A \textbf{deformation context} is a finitely complete category whose strict pro-completion is cocomplete. 
                \end{definition}
                \begin{definition}[Predeformation functors] \label{def: predeformation_functors}
                    For $\C$ a deformation context, a \textbf{predeformation functor} on $\C$ is a left-exact functor $F: \C \to \Grpd$. One says that such a funcotr is \textbf{discrete} should its domain be the full subcategory $\Sets$ of $\Grpd$.
                \end{definition}
                \begin{example}[Completed local Artinian algebras and predeformation functors on them]
                    Let $(\Lambda, \m, k)$ is the data of a Noetherian local ring with maximal ideal $\m$ and residue field $k$. Then, $\C_{\Lambda, k}$ will have a natural structure of a deformation context, as shown via propositions \ref{prop: finite_completeness_of_categories_of_local_artinian_algebras} and \ref{prop: pushouts_and_coproducts_of_completed_artinian_local_algebras}.
                \end{example}
                
                \begin{definition}[Predeformation fibrations] \label{def: predeformation_fibrations}
                    Let $p: \calF \to \C$ be a category cofibred in 
                \end{definition}
    
        \subsubsection{Thickenings and deformation functors}
        
    \subsection{Deformations of ringed topoi}
    
    \subsection{Examples of deformation problems}
        \subsubsection{Deformations of singularities}
        
        \subsubsection{Deformations of Galois representations}