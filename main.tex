\input{preambles}

\renewcommand{\cong}{\simeq}
\newcommand{\ladjoint}{\dashv}
\newcommand{\radjoint}{\vdash}
\newcommand{\<}{\langle}
\renewcommand{\>}{\rangle}
\newcommand{\ndiv}{\hspace{-2pt}\not|\hspace{5pt}}
\newcommand{\cond}{\blacksquare}

\newcommand{\N}{\mathbb{N}}
\newcommand{\Z}{\mathbb{Z}}
\newcommand{\Q}{\mathbb{Q}}
\newcommand{\R}{\mathbb{R}}
\newcommand{\bbC}{\mathbb{C}}
\NewDocumentCommand{\x}{e{_^}}{%
  \mathbin{\mathop{\times}\displaylimits
    \IfValueT{#1}{_{#1}}
    \IfValueT{#2}{^{#2}}
  }%
}
\NewDocumentCommand{\pushout}{e{_^}}{%
  \mathbin{\mathop{\sqcup}\displaylimits
    \IfValueT{#1}{_{#1}}
    \IfValueT{#2}{^{#2}}
  }%
}
\newcommand{\im}{\operatorname{im}}
\newcommand{\coker}{\operatorname{coker}}
\newcommand{\id}{\mathrm{id}}
\newcommand{\chara}{\operatorname{char}}
\newcommand{\trdeg}{\operatorname{trdeg}}
\newcommand{\rank}{\operatorname{rank}}
\newcommand{\length}{\operatorname{length}}
\newcommand{\height}{\operatorname{height}}
\renewcommand{\span}{\operatorname{span}}
\newcommand{\e}{\epsilon}
\newcommand{\p}{\mathfrak{p}}
\newcommand{\q}{\mathfrak{q}}
\newcommand{\m}{\mathfrak{m}}
\newcommand{\n}{\mathfrak{n}}
\newcommand{\calF}{\mathcal{F}}
\newcommand{\calG}{\mathcal{G}}
\newcommand{\calO}{\mathcal{O}}
\newcommand{\F}{\mathbb{F}}
\DeclareMathOperator{\lcm}{lcm}
\newcommand{\gr}{\mathbf{gr}}

\newcommand{\GL}{\mathbf{GL}}
\newcommand{\frakgl}{\mathfrak{gl}}
\newcommand{\SL}{\mathbf{SL}}
\newcommand{\bfO}{\mathbf{O}}
\newcommand{\SO}{\mathbf{SO}}
\newcommand{\SU}{\mathbf{SU}}
\newcommand{\bfU}{\mathbf{U}}
\newcommand{\Spec}{\mathbf{Spec}}
\newcommand{\Spf}{\mathbf{Spf}}
\newcommand{\Spm}{\mathbf{Spm}}
\newcommand{\Spv}{\mathbf{Spv}}
\newcommand{\Spa}{\mathbf{Spa}}
\newcommand{\Spd}{\mathbf{Spd}}
\newcommand{\Proj}{\mathbf{Proj}}
\newcommand{\Gr}{\mathbf{Gr}}
\newcommand{\Sht}{\mathbf{Sht}}
\newcommand{\Quot}{\mathbf{Quot}}
\newcommand{\Hilb}{\mathbf{Hilb}}
\newcommand{\Pic}{\mathbf{Pic}}
\newcommand{\Jac}{\mathbf{Jac}}
\newcommand{\Bun}{\mathbf{Bun}}
\newcommand{\loopspace}{\mathbf{\Omega}}
\newcommand{\suspension}{\mathbf{\Sigma}}
\newcommand{\bfT}{\mathbf{T}} %Tate module

\newcommand{\Ring}{\mathsf{Ring}}
\newcommand{\Cring}{\mathsf{CRing}}
\newcommand{\Alg}{\mathsf{Alg}}
\newcommand{\Leib}{\mathsf{Leib}} %leibniz algebras
\newcommand{\Fld}{\mathsf{Fld}}
\newcommand{\Sets}{\mathsf{Sets}}
\newcommand{\Cat}{\mathsf{Cat}}
\newcommand{\Grp}{\mathsf{Grp}}
\newcommand{\Ab}{\mathsf{Ab}}
\newcommand{\Sch}{\mathsf{Sch}}
\newcommand{\Coh}{\mathsf{Coh}}
\newcommand{\QCoh}{\mathsf{QCoh}}
\newcommand{\Desc}{\mathsf{Desc}}
\newcommand{\Sh}{\mathsf{Sh}}
\newcommand{\Psh}{\mathsf{PSh}}
\newcommand{\Fib}{\mathsf{Fib}}
\newcommand{\Mod}{\mathsf{Mod}}
\newcommand{\Vect}{\mathsf{Vect}}
\newcommand{\Rep}{\mathsf{Rep}}
\newcommand{\Grpd}{\mathsf{Grpd}}
\newcommand{\Arr}{\mathsf{Arr}}
\newcommand{\Esp}{\mathsf{Esp}}
\newcommand{\Ob}{\mathsf{Ob}}
\newcommand{\Mor}{\mathsf{Mor}}
\newcommand{\Mfd}{\mathsf{Mfd}}
%\newcommand{\LR}{\mathsf{LR}}
%\newcommand{\RSpc}{\mathsf{RSpc}}
\newcommand{\Spc}{\mathsf{Spc}}
\newcommand{\Top}{\mathsf{Top}}
\newcommand{\Topos}{\mathsf{Topos}}
\newcommand{\Nil}{\mathfrak{Nil}}
\newcommand{\J}{\mathfrak{J}}
\newcommand{\Stk}{\mathsf{Stk}}
\newcommand{\Pre}{\mathsf{Pre}}
\newcommand{\simp}{\mathsf{\Delta}}
\newcommand{\Ind}{\mathsf{Ind}}
\newcommand{\Pro}{\mathsf{Pro}}
\newcommand{\Mon}{\mathsf{Mon}}
\newcommand{\Comm}{\mathsf{Comm}}
\newcommand{\Fin}{\mathsf{Fin}}
\newcommand{\Assoc}{\mathsf{Assoc}}
\newcommand{\Co}{\mathsf{Co}}
\newcommand{\Comp}{\mathsf{Comp}} %compact hausdorff spaces
\newcommand{\Stone}{\mathsf{Stone}} %stone spaces
\newcommand{\sfExt}{\mathsf{Ext}}
\newcommand{\Ouv}{\mathsf{Ouv}}
\newcommand{\Str}{\mathsf{Str}}
\newcommand{\Func}{\mathsf{Func}}
\newcommand{\Crys}{\mathsf{Crys}}
\newcommand{\LocSys}{\mathsf{LocSys}}
\newcommand{\Sieves}{\mathsf{Sieves}}
\newcommand{\pt}{\mathsf{pt}}
\newcommand{\Graphs}{\mathsf{Graphs}}
\newcommand{\Lie}{\mathsf{Lie}}
\newcommand{\Env}{\mathsf{Env}}
\newcommand{\Ho}{\mathsf{Ho}}
\newcommand{\Cov}{\mathsf{Cov}}
\newcommand{\Frames}{\mathsf{Frames}}
\newcommand{\Locales}{\mathsf{Locales}}
\newcommand{\Span}{\mathsf{Span}}
\newcommand{\Corr}{\mathsf{Corr}}
\newcommand{\Monad}{\mathsf{Monad}}
\newcommand{\Var}{\mathsf{Var}}
\newcommand{\sfN}{\mathsf{N}} %nerve
\newcommand{\Dia}{\mathsf{Dia}}
\newcommand{\co}{\mathsf{co}}
\newcommand{\bi}{\mathsf{bi}}
\newcommand{\Nat}{\mathsf{Nat}}
\newcommand{\Hopf}{\mathsf{Hopf}}
\newcommand{\DMod}{\mathsf{DMod}}
\newcommand{\Perv}{\mathsf{Perv}}
\newcommand{\Sph}{\mathsf{Sph}}
\newcommand{\Moduli}{\mathsf{Moduli}}
\newcommand{\Pseudo}{\mathsf{Pseudo}}
\newcommand{\Lax}{\mathsf{Lax}}
\newcommand{\Strict}{\mathsf{Strict}}
\newcommand{\Opd}{\mathsf{Opd}} %operads
\newcommand{\Shv}{\mathsf{Shv}}
\newcommand{\Huber}{\mathsf{Huber}}
\newcommand{\Tate}{\mathsf{Tate}}
\newcommand{\Ad}{\mathsf{Ad}} %adic spaces
\newcommand{\Perfd}{\mathsf{Perfd}} %perfectoid spaces
\newcommand{\Sub}{\mathsf{Sub}} %subobjects
\newcommand{\Ideals}{\mathsf{Ideals}}
\newcommand{\Isoc}{\mathsf{Isoc}}

\newcommand{\Aut}{\mathbf{Aut}}
\newcommand{\Stab}{\mathbf{Stab}}
\newcommand{\Cent}{\mathbf{Cent}}
\newcommand{\Conj}{\mathbf{Conj}}
\newcommand{\Gal}{\mathbf{Gal}}
\newcommand{\bfG}{\mathbf{G}} %absolute galois group
\newcommand{\Frac}{\mathbf{Frac}}
\newcommand{\Ann}{\mathbf{Ann}}
\newcommand{\Val}{\mathbf{Val}}
\newcommand{\Chow}{\mathbf{Chow}}
\newcommand{\Sym}{\mathbf{Sym}}
\newcommand{\End}{\mathbf{End}}

\newcommand{\colim}{\operatorname{colim}}
\renewcommand{\lim}{\operatorname{lim}}
\newcommand{\toto}{\rightrightarrows}
\NewDocumentCommand{\tensor}{e{_^}}{%
  \mathbin{\mathop{\otimes}\displaylimits
    \IfValueT{#1}{_{#1}}
    \IfValueT{#2}{^{#2}}
  }%
}
\newcommand{\eq}{\operatorname{eq}}
\newcommand{\coeq}{\operatorname{coeq}}
\newcommand{\Hom}{\mathrm{Hom}}
\newcommand{\Tor}{\mathrm{Tor}}
\newcommand{\Ext}{\mathrm{Ext}}
\newcommand{\Isom}{\mathbf{Isom}}
\newcommand{\lex}{\mathbf{lex}}
\newcommand{\stalk}{\mathbf{stalk}}
\newcommand{\RKE}{\operatorname{RKE}}
\newcommand{\LKE}{\operatorname{LKE}}
\newcommand{\oblv}{\mathbf{oblv}}
\newcommand{\const}{\mathbf{const}}
%\newcommand{\forget}{\mathbf{forget}}
\newcommand{\adrep}{\mathbf{ad}} %adjoint representation
\newcommand{\NL}{\mathbf{NL}} %naive cotangent complex
\newcommand{\bfL}{\mathbf{L}} %cotangent complex
\newcommand{\pr}{\operatorname{pr}}
\newcommand{\Der}{\mathbf{Der}}
\newcommand{\Frob}{\bm{\varphi}} %Frobenius
\newcommand{\bfpt}{\mathbf{pt}}
\newcommand{\bfloc}{\mathbf{loc}}
\newcommand{\1}{\mathbbm{1}}
\newcommand{\Jet}{\mathbf{Jet}}
\newcommand{\Split}{\mathbf{Split}}
\newcommand{\Sq}{\mathbf{Sq}}
\newcommand{\Zero}{\mathbf{Z}}
\newcommand{\SqZ}{\Sq\Zero}
\newcommand{\frakLie}{\mathfrak{Lie}}
\newcommand{\Pol}{\mathbf{Pol}}

\newcommand{\bbU}{\mathbb{U}}
\newcommand{\V}{\mathbb{V}}
\newcommand{\U}{\mathscr{U}}
\newcommand{\bfV}{\mathbf{V}}
\newcommand{\C}{\mathcal{C}}
\newcommand{\D}{\mathcal{D}}
\newcommand{\T}{\mathscr{T}}
\newcommand{\calM}{\mathcal{M}}
\newcommand{\calN}{\mathcal{N}}
\newcommand{\calP}{\mathcal{P}}
\newcommand{\calQ}{\mathcal{Q}}
\newcommand{\A}{\mathbb{A}}
\renewcommand{\P}{\mathbb{P}}
\newcommand{\calL}{\mathcal{L}}
\newcommand{\E}{\mathcal{E}}
\renewcommand{\H}{\mathbf{H}}
\newcommand{\calX}{\mathcal{X}}
\newcommand{\calY}{\mathcal{Y}}
\newcommand{\calZ}{\mathcal{Z}}
\newcommand{\calA}{\mathcal{A}}
\newcommand{\calB}{\mathcal{B}}
\newcommand{\sfT}{\mathsf{T}}
\renewcommand{\S}{\mathcal{S}}
\newcommand{\B}{\mathbb{B}}
\newcommand{\bbD}{\mathbb{D}}
\newcommand{\G}{\mathbb{G}}
\newcommand{\horn}{\mathbf{\Lambda}}
\renewcommand{\L}{\mathbb{L}}
\renewcommand{\a}{\mathfrak{a}}
\renewcommand{\b}{\mathfrak{b}}
\renewcommand{\r}{\mathfrak{r}}
\newcommand{\bbX}{\mathbb{X}}
\newcommand{\g}{\mathfrak{g}}
\newcommand{\h}{\mathfrak{h}}
\newcommand{\del}{\partial}
\newcommand{\bbE}{\mathbb{E}}
\newcommand{\scrO}{\mathscr{O}}
\newcommand{\scrA}{\mathscr{A}}
\newcommand{\scrB}{\mathscr{B}}
\newcommand{\scrF}{\mathscr{F}}
\newcommand{\scrG}{\mathscr{G}}
\newcommand{\scrM}{\mathscr{M}}
\newcommand{\scrN}{\mathscr{N}}
\newcommand{\scrP}{\mathscr{P}}
\newcommand{\frakS}{\mathfrak{S}}
\newcommand{\calI}{\mathcal{I}}
\newcommand{\calJ}{\mathcal{J}}
\newcommand{\scrV}{\mathscr{V}}
\newcommand{\bbS}{\mathbb{S}}
\newcommand{\scrH}{\mathscr{H}}
\newcommand{\bfB}{\mathbf{B}}
\newcommand{\W}{\mathbb{W}}
%\newcommand{\bfA}{\mathbf{A}}
\renewcommand{\O}{\mathbb{O}}

\newcommand{\aff}{\mathrm{aff}}
\newcommand{\ft}{\mathrm{ft}}
\newcommand{\fp}{\mathrm{fp}}
\newcommand{\aft}{\mathrm{aft}}
\newcommand{\lft}{\mathrm{lft}}
\newcommand{\laft}{\mathrm{laft}}
\newcommand{\cmpt}{\mathrm{cmpt}}
\newcommand{\qc}{\mathrm{qc}}
\newcommand{\qs}{\mathrm{qs}}
\newcommand{\lcmpt}{\mathrm{lcmpt}}
%\newcommand{\conv}{\mathrm{conv}}
\newcommand{\red}{\mathrm{red}}
\newcommand{\fin}{\mathrm{fin}}
\newcommand{\petit}{\mathrm{petit}}
\newcommand{\gros}{\mathrm{gros}}
\newcommand{\loc}{\mathrm{loc}}
\newcommand{\ringed}{\mathrm{ringed}}
\newcommand{\qcoh}{\mathrm{qcoh}}
\newcommand{\cl}{\mathrm{cl}}
\newcommand{\et}{\mathrm{\acute{e}t}}
\newcommand{\fet}{\mathrm{f\acute{e}t}}
\newcommand{\proet}{\mathrm{pro\acute{e}t}}
\newcommand{\Zar}{\mathrm{Zar}}
\newcommand{\fppf}{\mathrm{fppf}}
\newcommand{\fpqc}{\mathrm{fpqc}}
\newcommand{\smooth}{\mathrm{smooth}}
\newcommand{\sh}{\mathrm{sh}}
\newcommand{\op}{\mathrm{op}}
\newcommand{\open}{\mathrm{open}}
\newcommand{\geom}{\mathrm{geom}}
\newcommand{\alg}{\mathrm{alg}}
\newcommand{\sober}{\mathrm{sober}}
\newcommand{\dR}{\mathrm{dR}}
\newcommand{\rad}{\mathrm{rad}}
\newcommand{\discrete}{\mathrm{discrete}}
%\newcommand{\add}{\mathrm{add}}
%\newcommand{\lin}{\mathrm{lin}}
\newcommand{\Krull}{\mathrm{Krull}}
\newcommand{\qis}{\mathrm{qis}}
\newcommand{\sep}{\mathrm{sep}}
\newcommand{\nil}{\mathrm{nil}}
\newcommand{\defm}{\mathrm{defm}}
\newcommand{\Art}{\mathrm{Art}}
\newcommand{\Noeth}{\mathrm{Noeth}}
\newcommand{\affd}{\mathrm{affd}}
%\newcommand{\adic}{\mathrm{adic}}
\newcommand{\pre}{\mathrm{pre}}
\newcommand{\perf}{\mathrm{perf}}
\newcommand{\perfd}{\mathrm{perfd}}
\newcommand{\rat}{\mathrm{rat}}
\newcommand{\cont}{\mathrm{cont}}
\newcommand{\dg}{\mathrm{dg}}
\newcommand{\almost}{\mathrm{a}}
\newcommand{\stab}{\mathrm{stab}}
\newcommand{\heart}{\heartsuit}
\newcommand{\proj}{\mathrm{proj}}
\newcommand{\pd}{\mathrm{pd}}
\newcommand{\crys}{\mathrm{crys}}
\newcommand{\prisma}{\mathrm{prisma}}
\newcommand{\FF}{\mathrm{FF}}
\newcommand{\sph}{\mathrm{sph}}
\newcommand{\lax}{\mathrm{lax}}
\newcommand{\weak}{\mathrm{weak}}
\newcommand{\strict}{\mathrm{strict}}
\newcommand{\mon}{\mathrm{mon}}
\newcommand{\sym}{\mathrm{sym}}
\newcommand{\lisse}{\mathrm{lisse}}
\newcommand{\an}{\mathrm{an}}
\newcommand{\ad}{\mathrm{ad}}
\newcommand{\sch}{\mathrm{sch}}
\newcommand{\rig}{\mathrm{rig}}

%prism custom command
\usepackage{relsize}
\usepackage[bbgreekl]{mathbbol}
\usepackage{amsfonts}
\DeclareSymbolFontAlphabet{\mathbb}{AMSb} %to ensure that the meaning of \mathbb does not change
\DeclareSymbolFontAlphabet{\mathbbl}{bbold}
\newcommand{\prism}{{\mathlarger{\mathbbl{\Delta}}}}

\begin{document}
    \frontmatter

	\title{Foundations of geometric representation theory}
	
	\author{Dat Minh Ha}
	\maketitle
	
	{
      \hypersetup{} 
      \dominitoc
      \tableofcontents %sort sections alphabetically
    }
	
	\newpage
	
	{
      \hypersetup{hidelinks} 
      \listoftodos
    }
    
    \chapter*{Introduction}
    \begin{abstract}
        
    \end{abstract}
    
    \minitoc
    
    \section{The Global Correspondence}
        Let $X$ be a curve that is smooth, proper, and geometrically connected algebraic curve (for instance, we can take $X$ be an elliptic curve or $\P^1$) and suppose that $G$ is a reductive group (think $\GL_n$ or $\SL_n$, or more concretely, $\GL_1$, or groups of diagonal matrices); both shall be over a field $k$ of characteristic $0$. Additionally, denote the function field of our curve $X$ by $K_X$, the completions of said field at (closed) points $x \in |X|$ by $K_{X, x}$, and we shall write $\scrO_{X, x}$ for the associated rings of integers (note how they coincide with the adic completions $\calO_{X, x}^{\wedge}$).
            
        The end goal for us, shall be to construct some semblance of an equivalence of derived/abelian/stable $\infty$-categories:
            $$\Dmod\left(\Bun_G(X)\right) \cong \Ind\Coh\left(\LocSys_F(X)^{\check{G}}\right)$$
        between:
            \begin{itemize}
                \item the category $\Dmod\left(\Bun_G(X)\right)$ of D-modules on the moduli stack $\Bun_G(X)$ of $G$-bundles on $X$, and
                \item the category $\Ind\Coh\left(\LocSys_F(X)^{\check{G}}\right)$ of ind-coherent sheaves (cf. section \ref{section: indcoh}) on the moduli stack of $\check{G}$-equivariant local systems on $X$ with coefficients in some implicitly understood suitable field $F$. 
            \end{itemize}
        When $G$ is a torus - i.e. when it is abelian - the above correspondence is a bit simpler:
            $$\Dmod\left(\Bun_G(X)\right) \cong \QCoh\left(\LocSys_F(X)^{\check{G}}\right)$$
        (notice how now, we can work with the entire category of quasi-coherent sheaves instead of having to restrict ourselves to ind-coherent sheaves). One thing that needs to be made clear right away, however, is that aside from a few very special cases such as $G = \GL_1$ and $G = \SL_2$, this equivalence is \textit{entirely conjectural}. Nevertheless, we do have a rough idea of how to eventually obtain a proper theorem from this vision:
            \begin{enumerate}
                \item The very first thing to do is to understand the construction of D-modules on (pre)stacks locally of finite type, and we can do this by learning about crystals (in the sense of Grothendieck) and their infinitesimal/crystalline cohomology over base fields of characteristic $0$ (crystalline cohomology over base fields of positive characteristics and the accompanying theory of arithmetic D-modules is significantly more complicated than their characteristic $0$ counterparts, which incidentally is why we have required that $\chara k = 0$).
                \item Then, we must know what $\check{G}$ actually is, i.e. we must understand Langlands duals. There is a tool for this, which is the Geometric Satake Equivalence. However, we are going to have to go through two substeps:
                    \begin{enumerate}
                        \item To begin, we shall need to understand what the affine Grassmannian is and its roles in the representation theory of algebraic groups.
                        \item We shall also have to know what it means to have a group act upon a (nice enough) category so as to be able to define the category of so-called \textbf{spherical D-modules}, which are certain kinds of equivariant D-modules.
                        \item We shall then establish the Geometric Satake Correspondence to be a Tannakian equivalence:
                            $$\Rep^{\heart}_F(\check{G}_{K_{X, x}}) \cong \Sph^{\heart}_{G, X, x}$$
                        between the hearts of the t-structures of the rigid monoidal derived categories of $F$-linear representations of the $K_{X, x}$-points of the Langlands dual group $\check{G}$ and of $G(\scrO_{X, x})$-equivariant/spherical D-modules over the local affine Grassmannian $\Gr_{G, X, x}$.
                    \end{enumerate}
                \item Lastly, we shall seek to understand the subtle technical differences between quasi-coherent sheaves and ind-coherent sheaves, and why restricting ourselves to the case of tori allows us to forego the ind-coherent sheaf machinery. 
            \end{enumerate}
        Of course, before embarking on this journey, we might also want to learn some (derived) algebraic geometry, which will help us understand $\Bun_G(X)$ and $\LocSys_F(X)^{\check{G}}$, what these categories have to do with the theory of Galois representations (because at the end of the day, the Langlands Programme is all about understanding higher reciprocity laws), or even simply why we have required that our curve $X$ is smooth (spoiler: smoothness helps us identify $\QCoh(X)$ with the category $\QCoh(X)^{\perf}$ of perfect complexes on $X$), proper, and geometrically connected, beyond wanting our machineries to be applicable to important classes of examples such as elliptic curves and abelian varieties. For details, see chapters \ref{chapter: schemes} and \ref{chapter: cohomology_and_derived_schemes}.
        
        We should also make some remarks about the above equivalence of categories as well. Thanks to Grothendieck's Galois theory, the left-hand side can be thought of as the \say{\textbf{Automorphic Side}} of the Langlands Correspondence, which holds information about Galois representations. Drawing inspiration from another one of Grothendieck's major contributions, $\ell$-adic \'etale cohomology, the right-hand side in turn can be thought of as the \say{\textbf{Spectral Side}}, which tells interesting stories\footnote{Fairy tales, really...} through harmonic analysis.
    
        \subsection{The Categorical-Geometric Langlands Correspondence for algebraic tori}
            This section, as the title suggests, shall be dedicated to outlining our hopes and dreams (or the lack thereof) for a Categorical-Geometric Langlands Correspondence for algebraic tori; specifically, we would like to present of a list of key results known to be involved in a proof of the Correspondence. We will also give run-down of the various technical tools used for establishing said key results. 
            
            \subsubsection{Equivariant local systems}
        
            \subsubsection{The Fourier-Muka\"i-Laumon Transform}
            
            \subsubsection{Factor-wise Langlands duality}
            
        \subsection{The Conjecture for non-abelian groups}
        
        \subsection{Outline of the proof for the case of \texorpdfstring{$G = \GL_2$}{}}
        
    \section{The Local Correspondence for complex loop groups}
        \subsection{The appearance of Langlands parameters}
            Consider the formal loop group $G(\!(t)\!)$ associated to some chosen connected complex reductive group $G$. 
            
            Let us start by describing the absolute Galois group of the field $\bbC(\!(t)\!)$. First of all, notice that:
                $$\Gal(\bbC(\!(t^{\frac1n})\!)/\bbC(\!(t)\!)) \cong \Z/n\Z$$
            and so:
                $$\Gal(\overline{\bbC(\!(t)\!)}/\bbC(\!(t)\!)) \cong \hat{\Z}$$
            which is a canonical homeomorphism of topological groups obtained via the Fundamental Theorem of Galois Theory. Now, one thing to note is that for some fixed power $q$ of a prime $p$, one also has:
                $$\Gal(\overline{\F_q}/\F_q) \cong \hat{\Z}$$
            but unlike the complex case, the group $\Gal(\overline{\F_q(\!(t)\!)}/\F_q(\!(t)\!))$ surjects (continuously) onto the non-trivial group $\Gal(\overline{\F_q}/\F_q)$ ($\bbC$ is algebraically closed so $\Gal(\bar{\bbC}/\bbC)$ is trivial), a fact known through local class field theory. As a consequence, describing the Weil group (and by extension, Weil-Deligne representations thereof) attached to $\bbC(\!(t)\!)$ will - hopefully - be somewhat simpler than that of $\F_q(\!(t)\!)$ and might therefore help us gain insight into the nature of the Langlands Correspondence. Better yet, we have via Grothendieck's Galois Theory, that:
                $$\Gal(\overline{\bbC(\!(t)\!)}/\bbC(\!(t)\!)) \cong \hat{\Z} \cong \pi_1^{\et}(\bbD^{\x}_{\bbC})$$
            wherein $\bbD^{\x}_{\bbC} \cong \Spec \bbC(\!(t)\!)$; through the discussion above, one sees that this is not the case for $\bbD^{\x}_{\F_q}$, i.e.:
                $$\Gal(\overline{\F_q(\!(t)\!)}/\F_q(\!(t)\!)) \not \cong \pi_1^{\et}(\bbD^{\x}_{\F_q})$$
            Since representations of the (\'etale) fundamental group correspond to certain D-modules, we essentially have access to the theory of D-modules in studying the Langlands Correspondence for the case of $G\!(t)\!)$, which roughly postulates a bijective relationship between homomorphisms $W_{\bbC(\!(t)\!)} \to \check{G}$ and certain representations of $G(\!(t)\!)$.
        
        \subsection{Representations of loop groups; Kac-Moody algebras}
    
    \section{Deformation quantisation of the Local Correspondence}
	
	\input{conventions}
	
	\mainmatter
	
	\part{Commutative algebra}  
        \chapter{Introduction}
    \begin{abstract}
        
    \end{abstract}
	
	    \input{Commutative algebra/schemes 1}
	    
	    \chapter{Derived schemes} \label{chapter: cohomology_and_derived_schemes}
    \begin{abstract}
        
    \end{abstract}
    
    \minitoc
    
    \section{Sheaf cohmology}
        \subsection{Vanishing theorems}
            \subsubsection{Quasi-coherent sheaves and Serre's affineness criterion}
            
            \subsubsection{Higher direct images}
        
        \subsection{Base change}
        
        \subsection{Cohomology of projective varieties}
        
        \subsection{Some useful theorems}
            \subsubsection{Theorems on duality}
            
            \subsubsection{The theorem on formal functions}
            
            \subsubsection{The Grothendieck Existence Theorem; Algebraisation Theorems}

    \section{Derived schemes}
        \subsection{(Un)necessary \texorpdfstring{$\infty$}{}-categorical technicalities}
            \begin{convention}[Some typographical conventions]
                For the sake of linguistic simplicity (and since typing \say{$\infty$} gets very tedious very quickly), we shall refrain from specfifying the homotopicality of many $\infty$-categorical operations. For instance, $(\infty, 1)$-limits shall almost always be referred to simply as \say{limits}, and so on. Also, we shall only refer to $(\infty, 1)$-categories by their \say{full name}, so to say, when $(\infty, 2)$-categories are not around; otherwise, they shall simply be known as $\infty$-categories.
            \end{convention}
            
            \subsubsection{What on earth is an \texorpdfstring{$\infty$}{}-category ?}
        
            \subsubsection{\texorpdfstring{$\infty$}{}-topoi and \texorpdfstring{$\infty$}{}-stacks}
                \begin{remark}[Regular cardinals]
                    From now on we will be using the notion of regular cardinals often. For details on the notion, see definition \ref{def: limit_cardinal}.
                \end{remark}
            
                \begin{definition}[$\infty$-topoi] \label{def: infinity_topoi}
                    \noindent
                    \begin{enumerate}
                        \item \textbf{(Localisations of $\infty$-categories):} 
                            \begin{enumerate}
                                \item \textbf{(Reflexivity):} A \textit{fully faithful} $\infty$-functor:
                                    $$R: \C \to \D$$
                                identifies an $\infty$-category $\C$ as a \textbf{reflexive full $\infty$-subcategory} of another $\infty$-category $\D$ if it admits a left-$(\infty, 1)$-adjoint. Should this left-adjoint functor preserve all finite $(\infty, 1)$-limits, then the full faithful reflexive embedding $R: \C \to \D$ shall be called \textbf{exact}.
                                \item \textbf{(Localisations):} An $\infty$-functor:
                                    $$L: \D \to \C$$
                                is called a \textbf{localisation} if it admits a fully faithful right-$(\infty, 1)$-adjoint, which we note to necessarily be a reflexive embedding, by definition; we shall thus dub the right-adjoint component the \textbf{reflector}. Alternatively, one may characterise the \textbf{localisation} of an $\infty$-category $\D$ at a full $\infty$-subcategory $\C$ as an $(\infty, 1)$-adjoint pair:
                                    $$
                                        \begin{tikzcd}
                                        	\C & \D
                                        	\arrow[""{name=0, anchor=center, inner sep=0}, "L"', shift right=2, from=1-2, to=1-1]
                                        	\arrow[""{name=1, anchor=center, inner sep=0}, "R"', shift right=2, hook, from=1-1, to=1-2]
                                        	\arrow["\dashv"{anchor=center, rotate=-90}, draw=none, from=0, to=1]
                                        \end{tikzcd}
                                    $$
                                whose right-adjoint component is fully faithful. 
                                
                                A localisation that is \textit{exact} shall be called a \textbf{topological localisation}. 
                            \end{enumerate}
                        \item \textbf{(Accessibility and geometric embeddings/localisations):}
                            \begin{enumerate}
                                \item \textbf{(Accessibility):} Let $\kappa$ be some \href{https://ncatlab.org/nlab/show/regular+cardinal}{\underline{regular cardinal}}. An $\infty$-category $\C'$ is said to be \textbf{$\kappa$-accessible} if and only if it is equivalent to the $(\infty, 1)$-ind-completion of some $\kappa$-small $\infty$-category $\C$, i.e. one should be able to identify $\C'$ as a full $\infty$-subcategory of $\Psh_{(\infty, 1)}(\C)$ which is closed under all $\kappa$-small $(\infty, 1)$-filtered colimits.
                                \item \textbf{($(\infty, 1)$-geometric embeddings and $\infty$-topoi):} An \textbf{$\infty$-topos} \textit{\`a la} Rezk-Lurie is the topological localisation of some $\infty$-category of $(\infty, 1)$-presheaves at some \textit{accessible} full $\infty$-subcategory; the adjoint pair defining this localisation may either be referred to as an \textbf{$\infty$-geometric embedding} or an \textbf{$\infty$-geometric localisation}, depending on whether we wish to put emphasis on the left-adjoint or right-adjoint component.
                                
                                One very important thing to note is that every $(\infty, 1)$-presheaf $\infty$-category is trivially an $\infty$-topos; we refer to them as $(\infty, 1)$-presheaf $\infty$-topoi. 
                            \end{enumerate}
                    \end{enumerate}
                \end{definition}
                
                In algebraic geometry, one cares as much about sheaves as the topoi they span. Therefore, it would be nice if $\infty$-topoi were to behave as homotopical analogues of Grothendieck $1$-topoi (and let us recall that there is a geometric embedding from every Grothendieck $1$-topos into a presheaf $1$-topos), in that its objects could be realised as higher sheaves on higher sites. Luckily, such a characterisation of $\infty$-topos is available. We shall, however, need to roll our sleeves up a bit to obtain it.
                
                \begin{definition}[$\infty$-sites and $\infty$-descent theory] \label{def: infinity_sites}
                    \noindent
                    \begin{enumerate}
                        \item \textbf{($\infty$-coverages):} 
                            \begin{enumerate}
                                \item \textbf{($\infty$-sieves):} An $\infty$-sieve on an object $X$ of some $\infty$-category $\C$ is nothing more than an $(\infty, 1)$-subpresheaf of the representable presheaf $h_X$; i.e. for every test object $X_0$ of $\C_{/X}$ and every $\infty$-sieve $\U$ on $X$, one can identify $\U(X_0)$ as a full $\infty$-subcategory ($\infty$-subgroupoid, actually) of the $\infty$-groupoid $\C(X_0, X)$.
                                
                                An $\infty$-sieve $\U$ is said to \textbf{cover} an object $X$ of an $\infty$-category $\C$ if and only if $X$ is the $(\infty, 1)$-colimit of the \v{C}ech nerve $\U^{\bullet}_{/X}$. 
                                \item \textbf{(Axioms for $\infty$-coverages):} The following are conditions for a class of sieves on objects of some base $\infty$-category to qualify as an \textbf{$\infty$-coverage}:
                                    \begin{itemize}
                                        \item Representable $(\infty, 1)$-presheaves - viewed as $\infty$-sieves - cover the objects they represent.
                                        \item $(\infty, 1)$-pullbacks of covering $\infty$-sieves must also be covering $\infty$-sieves themselves. 
                                        \item Should the $(\infty, 1)$-pullback of an $\infty$-sieve be one that covers, then the first $\infty$-sieve must also be a covering $\infty$-sieve.
                                    \end{itemize}
                                An $\infty$-category equipped with an $\infty$-coverage is an \textbf{$\infty$-site.}
                            \end{enumerate}
                        \item \textbf{($\infty$-descent theory):} An $\infty$-prestack $\calY$ (cf. convention \ref{conv: infinity_prestacks}) on some $\infty$-category $\C$ equipped with an $\infty$-coverage $J$ (i.e. an $\infty$-site $(\C, J)$) is said to \textbf{satisfy $J$-descent} if and only if for all objects $X$ of $\C$ and for all $\infty$-sieves $\U$ that covers $X$, one has the following equivalence of $\infty$-groupoids:
                            $$\calY(X) \cong \calY\left({}^{(\infty, 1)}\colim \U^{\bullet}_{/X}\right)$$
                        In particular, $(\infty, 1)$-presheaves that satisfy descent are \textbf{$(\infty, 1)$-sheaves}.
                    \end{enumerate}
                \end{definition}
                
                \begin{definition}[Presentability] \label{def: presentable_infinity_categories}
                    Let $\kappa$ be regular cardinal. Then, a $\kappa$-accessible $\infty$-category is said to be \textbf{$\kappa$-presentable} if it is furthermore $\kappa$-small $(\infty, 1)$-cocomplete. In other words, a $\kappa$-accessible $\infty$-category is a \textit{full} $\infty$-subcategory of the $(\infty, 1)$-presheaf $\infty$-topos over some underlying $\infty$-category that is \textit{closed under all $\kappa$-small $(\infty, 1)$-colimits}; clearly, every object in a $\kappa$-presentable category \textit{can be built out of a $\kappa$-small set of objects using $(\infty, 1)$-colimits}.
                \end{definition}
                \begin{example}[Examples of presentable $\infty$-categories]
                    
                \end{example}
                
                \begin{definition}[Universal $(\infty, 1)$-colimits] \label{def: universal_colimits}
                    Given any finitely $(\infty, 1)$-complete $\infty$-category $\E$, one can build so-called pullback $\infty$-functors:
                        $$f^*: \E_{/y} \to \E_{/x}$$
                    via $(\infty, 1)$-pullbacks along arrows $f: x \to y$. If a $(\infty, 1)$-colimit in $\E$ is preserved by all these pullback $\infty$-functors, then we shall call it \textbf{universal}.
                \end{definition}
                
                \begin{definition}[Subobject classifiers in $\infty$-categories] \label{def: subobject_classifiers}
                    \noindent
                    \begin{enumerate}
                        \item \textbf{(Monomorphisms):} A morphism:
                            $$f: x \to y$$
                        in an $\infty$-category $\E$ is a \textbf{monomorphism} if and only if $\E(x, -)$ is an $(\infty, 1)$-subfunctor of $\E(y, -)$: that is to say, for all test objects $x_0$ of $\E$, $\E(x, x_0)$ is a full $\infty$-subcategory of the $\infty$-groupoid $\E(y, x_0)$. 
                        \item \textbf{(Subobject classifiers):} Let $\E$ be a $(\infty, 1)$-complete $\infty$-category and let us denote its terminal object by $1$ (note that such a terminal object exists as a consequence of the finite-completeness hypothesis). Then, a morphism:
                            $$t: 1 \to \Omega$$
                        of $\E$ will be called a \textbf{subobject classifier} if and only if it is terminal in the (non-full) $\infty$-subcategory of the $\infty$-category $\E^{[1]} \cong {}^{\infty}\Cat([1], \E)$ of arrows and commutative squares in $\E$ spanned by monomorphisms.
                        
                        Thanks to the assumption that $\E$ is finitely $(\infty, 1)$-complete, if a subobject classifier $t: 1 \to \Omega$ exists, one will be able to write any monomorphism $u \to x$ in $\E$ as the pullback of the subobject classifier along some \textit{unique} morphism $\chi_x: x \to \Omega$.
                    \end{enumerate}
                \end{definition}
                
                \begin{definition}[(De)looping] \label{def: looping} \index{Loop spaces} \index{Classifying spaces} 
                    \noindent
                    \begin{enumerate}
                        \item \textbf{(Loop spaces):} Let $\S$ be a finitely $(\infty, 1)$-complete $\infty$-category of \say{spaces} and let $X$ be some object thereof; also, let $*$ be a terminal object of $\S$. Then, the \textbf{loop space} $\loopspace X$ fits into the following $(\infty, 1)$-pullback square:
                            $$
                                \begin{tikzcd}
                                	{\loopspace X} & {*} \\
                                	{*} & X
                                	\arrow[from=1-1, to=2-1]
                                	\arrow[from=2-1, to=2-2]
                                	\arrow[from=1-1, to=1-2]
                                	\arrow[from=1-2, to=2-2]
                                	\arrow["\lrcorner"{anchor=center, pos=0.125}, draw=none, from=1-1, to=2-2]
                                \end{tikzcd}
                            $$
                        \item \textbf{(Classifying spaces):} Within the same framework, one can also define the so-called \textbf{classifying space} or \textbf{delooping space} of an object $H$ as the $(\infty, 1)$-pushout of the unique morphism $H \to *$ along itself:
                            $$
                                \begin{tikzcd}
                                	H & {*} \\
                                	{*} & {\bfB H}
                                	\arrow[from=1-1, to=2-1]
                                	\arrow[from=2-1, to=2-2]
                                	\arrow[from=1-1, to=1-2]
                                	\arrow[from=1-2, to=2-2]
                                	\arrow["\lrcorner"{anchor=center, pos=0.125, rotate=180}, draw=none, from=2-2, to=1-1]
                                \end{tikzcd}
                            $$
                        Note that because our ambient $\infty$-category is only finitely $(\infty, 1)$-complete, deloopings of its objects may or may not exist; should they do, however, they would guarantee the existence of all $(\infty, 1)$-coequalisers in $\E$ (we will use this fact in the proof of lemma \ref{lemma: building_infinity_topoi_out_of_colimits}).
                    \end{enumerate}
                    It is not too hard to see that:
                        $$\loopspace \bfB H \cong H$$
                    and if $(\infty, 1)$-pushouts were to exist in $\E$:
                        $$\bfB \loopspace X \cong X$$
                \end{definition}
                \begin{example}
                    \noindent
                    \begin{enumerate}
                        \item \textbf{((De)looping of spheres):} The simplest setting in which one can perform (de)looping on objects of a finitely $(\infty, 1)$-complete $\infty$-category is when $\S$ is the homotopy category $\Ho\Top$ of the category $\Top$ of topological spaces, which happens to be $(\infty, 1)$-cocomplete. For instance, for all positive integers $n$, the loop space of the $(n + 1)$-sphere is nothing but the $n$-sphere:
                            $$
                                \begin{tikzcd}
                                	{\bbS^n} & {*} \\
                                	{*} & {\bbS^{n + 1}}
                                	\arrow[from=1-1, to=2-1]
                                	\arrow[from=2-1, to=2-2]
                                	\arrow[from=1-1, to=1-2]
                                	\arrow[from=1-2, to=2-2]
                                	\arrow["\lrcorner"{anchor=center, pos=0.125, rotate=180}, draw=none, from=2-2, to=1-1]
                                \end{tikzcd}
                            $$
                        and conversely, the delooping of the $n$-sphere is the $(n + 1)$-sphere.
                        \item \textbf{(\v{C}ech nerves):} The delooping of the \v{C}ech nerve $u^{\bullet}_{/x}$ on some object $x$ of a finitely $(\infty, 1)$-complete $\infty$-category is exactly $x$. 
                    \end{enumerate}
                \end{example}
                
                \begin{lemma}[Building $\infty$-topoi out of colimits] \label{lemma: building_infinity_topoi_out_of_colimits}
                    Let $\kappa$ be a regular cardinal. Then, every $\kappa$-small $\infty$-topos is necessarily $\kappa$-presentable, has subobject classifiers, and the $(\infty, 1)$-colimits therein (which are all $\kappa$-small) are all universal.
                \end{lemma}
                    \begin{proof}
                        Fix a base $\kappa$-small $\infty$-category $\C$ and the following topological localisation defining an $\infty$-topos $\E$, which is necessarily $\kappa$-small by virtue of being a full $\infty$-subcategory of the $\kappa$-small $\infty$-category $\Psh_{(\infty, 1)}(\C)$:
                            $$
                                \begin{tikzcd}
                                	\E & {\Psh_{(\infty, 1)}(\C)}
                                	\arrow[""{name=0, anchor=center, inner sep=0}, "L"', shift right=2, from=1-2, to=1-1]
                                	\arrow[""{name=1, anchor=center, inner sep=0}, "R"', shift right=2, hook, from=1-1, to=1-2]
                                	\arrow["\dashv"{anchor=center, rotate=-90}, draw=none, from=0, to=1]
                                \end{tikzcd}
                            $$
                        \begin{enumerate}
                            \item \textbf{(Presentability):} Because $\kappa$-small $\infty$-topos are already $\kappa$-accessible by default (cf. definition \ref{def: infinity_topoi}), it shall suffice to show that they are $\kappa$-small $(\infty, 1)$-cocomplete. For this, we shall attempt to show that $\E$ has all $(\infty, 1)$-coproducts and all $(\infty, 1)$-coequalisers. 
                                \begin{enumerate}
                                    \item \textbf{(Coproducts):} By viewing our accessible $\infty$-category $\E$ as a quasi-category (which in particular, have underlying topological spaces), we can simply apply the Seifert-van Kampen Theorem to see how $(\infty, 1)$-coproducts ought to exist in $\E$ via the existence of liftings inside pushout squares of the following form: 
                                        $$
                                            \begin{tikzcd}
                                            	\varnothing & \bullet \\
                                            	\bullet & \bullet
                                            	\arrow[from=1-1, to=2-1]
                                            	\arrow[from=2-1, to=2-2]
                                            	\arrow[from=1-1, to=1-2]
                                            	\arrow[from=1-2, to=2-2]
                                            	\arrow[dashed, from=2-1, to=1-2]
                                            	\arrow["\lrcorner"{anchor=center, pos=0.125, rotate=180}, draw=none, from=2-2, to=1-1]
                                            \end{tikzcd}
                                        $$
                                    These liftings exist thanks to the fact that topological spaces have underlying sets, and one can always order their cardinalities. One thing to note here is that the Seifert-van Kampen Theorem can be applied here because disjoint spaces intersect at the empty set (of course!) and subsequently because the empty set is path-connected (recall that the Seifert-van Kampen Theorem can only be applied when the intersection is path-connected).
                                    \item \textbf{(Coequalisers):} $(\infty, 1)$-coequalisers are nothing but internal equivalence relations, and since these are nothing more than certain kinds of internal groupoids, let us show the stronger assertion that all internal groupoids in $\E$ can be delooped (cf. definition \ref{def: looping}). To that end, note that because $\E$ is $\kappa$-accessible, we can write this as a $\kappa$-small filtered $(\infty, 1)$-colimit of other internal groupoids $\left( s^{(i)}, t^{(i)}: H_1^{(i)} \toto H_0^{(i)} \right)$ in $\E$, which we might as well take to be deloopable:
                                        $$(s, t: H_1 \to H_0) \cong {}^{(\infty, 1)}\underset{i \in I}{\colim} \left( s^{(i)}, t^{(i)}: H_1^{(i)} \toto H_0^{(i)} \right)$$
                                    But we $(\infty, 1)$-colimits commute with one another, and so it is precisely because the groupoids $\left( s^{(i)}, t^{(i)}: H_1^{(i)} \toto H_0^{(i)} \right)$ are deloopable that we can also deloop the filtered $(\infty, 1)$-colimit $(s, t: H_1 \toto H_0)$. Thus, every internal groupoid in $\E$ is deloopable, which as stated above, implies that $\E$ has all $(\infty, 1)$-coequalisers.
                                \end{enumerate}
                            \item \textbf{(Universality of colimits):} Now that we have established the $(\infty, 1)$-cocompleteness of $\infty$-topoi, let us check if the $(\infty, 1)$-colimits therein are all universal. To that end, let $x$ be an object of $\E$ and let:
                                $$f: x \to y$$
                            be a morphism. Also, let:
                                $$F: \D \to \E_{/y}$$
                            be a diagram of shape $\D$ in $\E_{/y}$, which we note to necessarily be $\kappa$-small; a pullback of this diagram is just a lifting in ${}^{\infty}\Cat^{\leq \kappa}$ of the following form:
                                $$
                                    \begin{tikzcd}
                                    	& {\E_{/x}} \\
                                    	\D & {\E_{/y}}
                                    	\arrow["{f^*}"', from=2-2, to=1-2]
                                    	\arrow["F"', from=2-1, to=2-2]
                                    	\arrow["{f^*F}", dashed, from=2-1, to=1-2]
                                    \end{tikzcd}
                                $$
                            If the diagram $F$ were to be filtered, then the assertion would be a trivial consequence of the fact that filtered colimits commute with finite limits (recall that $f^*$ is given by pulling back along $f: x \to y$), so let us assume that is is not filtered; actually, we can simply assume that our diagram $F$ is discrete, since every diagram can built out of filtered and discrete subdiagrams using coproducts of functors. With this assumption in place, we would only need to show that every $(\infty, 1)$-coproduct in $\E$ is universal. 
                            \item \textbf{(Subobject classifiers):}
                        \end{enumerate}
                    \end{proof}
                \begin{proposition}[$\infty$-categories of $(\infty, 1)$-sheaves] \label{prop: infinity_categories_of_higher_sheaves}
                    Let $(\C, J)$ be a $\kappa$-small $\infty$-site, for some regular cardinal $\kappa$. Then, the category of $(\infty, 1)$-sheaves on $(\C, J)$, denoted by $\Sh_{(\infty, 1)}(\C, J)$, is a full $\infty$-subcategory of the $(\infty, 1)$-presheaf $\infty$-topos $\Psh_{(\infty, 1)}(\C, J)$. Furthermore, $\infty$-categories of $(\infty, 1)$-sheaves:
                        \begin{enumerate}
                            \item they are always $\kappa$-presentable (assuming that the underlying $\infty$-site is $\kappa$-small, of course),
                            \item they are $\infty$-categories where all $\kappa$-small $(\infty, 1)$-colimits are universal, and
                            \item they have subobject classifiers.
                        \end{enumerate}
                \end{proposition}
                    \begin{proof}
                        \noindent
                        \begin{enumerate}
                            \item \textbf{(Presentability):} 
                            \item \textbf{(Universality of colimits):}
                            \item \textbf{(Subobject classifiers):}
                        \end{enumerate}
                    \end{proof}
                \begin{corollary}[$\infty$-topoi are $\infty$-categories of $(\infty, 1)$-sheaves] \label{coro: infinity_topoi_are_sheaf_topoi}
                    Let:
                        $$
                            \begin{tikzcd}
                            	\E & \Psh_{(\infty, 1)}(\C)
                            	\arrow[""{name=0, anchor=center, inner sep=0}, "L"', shift right=2, from=1-2, to=1-1]
                            	\arrow[""{name=1, anchor=center, inner sep=0}, "R"', shift right=2, hook, from=1-1, to=1-2]
                            	\arrow["\dashv"{anchor=center, rotate=-90}, draw=none, from=0, to=1]
                            \end{tikzcd}
                        $$
                    be a topological localisation that defines some $\infty$-topos $\E$. Then, there exists an $\infty$-site $(\C, J)$ such that $\E$ is equivalent to the $\infty$-category of $(\infty, 1)$-sheaves over $(\C, J)$:
                        $$\E \cong \Sh_{(\infty, 1)}(\C, J)$$
                    Because of this, the left-adjoint component of the topological localisation $L \ladjoint R$ is usually known as the $(\infty, 1)$-sheafification functor with respect to the $\infty$-coverage on $(\C, J)$, and we shall denote it by ${}^{(\infty, 1), \sh}(-)_J$.
                \end{corollary}
            
                \begin{remark}[The $(\infty, 2)$-category of $(\infty, 1)$-topoi] \label{remark: (infinity, 1)_topoi_categories} \index{$\infty$-topoi} \index{$\infty$-topoi! ${}^{\infty}\Sh\Topos$}
                    We shall leave the verification of the following facts to the reader.
                    \\
                    $(\infty, 1)$-topoi naturally form a \textit{weak} $(\infty, 2)$-category, wherein:
                        \begin{itemize}
                            \item Objects are $\infty$-topoi, which we should note to be $(\infty, 1)$-categories.
                            \item $(\infty, 1)$-morphisms are $(\infty, 1)$-geometric morphisms, i.e. pairs of $(\infty, 1)$-adjoint $(\infty, 1)$-functors with the left-$(\infty, 1)$-adjoint component being left-$(\infty, 1)$-exact; also, compositions of these left-$(\infty, 1)$-adjoint components are merely associative up to invertible strict $(\infty, 1)$-natural transformations.
                            \item $(\infty, 1)$-morphisms are $(\infty, 1)$-natural transformations between the left-$(\infty, 1)$-adjoint components of weak $(\infty, 1)$-geometric morphisms. 
                        \end{itemize}
                    This weak $(\infty, 2)$-category shall be denoted by ${}^{\infty}\Sh\Topos$. It is finitely weakly $(\infty, 2)$-complete:
                        \begin{itemize}
                            \item \textbf{(Products):} It admits weak $(\infty, 2)$-products and weak $(\infty, 2)$-pullbacks, along with a terminal object, that being the $(\infty, 1)$-category ${}^{\infty}\Grpd$ of small $\infty$-groupoids, $(\infty, 1)$-(ana)functors between them, and $(\infty, 1)$-(ana)natural transformations between those functors. 
                            \item \textbf{(Monomorphisms):} Its \href{https://ncatlab.org/nlab/show/monomorphism+in+an+\%28infinity\%2C1\%29-category}{\underline{monomorphisms}} are topological localisations.
                        \end{itemize}
                \end{remark}
        
                \begin{convention}[\textcolor{red}{\underline{IMPORTANT}} $\infty$-prestacks and $\infty$-stacks] \label{conv: infinity_prestacks} \index{$\infty$-prestacks} \index{$\infty$-stacks}
                    \noindent
                    \begin{enumerate}
                        \item \textbf{(The descent-theoretic perspective):} \begin{enumerate}
                            \item \textbf{($\infty$-prestacks):} To us, a \textbf{prestack} on an $(\infty, 1)$-category $\C$ will always be a \textit{weak} $(\infty, 2)$-functor (i.e. a $(\infty, 2)$-functor that respects compositions only up to natural isomorphisms) from $\C^{\op}$ (viewed as a tautological bicategory) into the weak $(\infty, 2)$-category ${}^{(\infty, 1)}\Cat$ of small $(\infty, 1)$-categories, $(\infty, 1)$-(ana)functors between them, and $(\infty, 1)$-(ana-)natural transformations between these $(\infty, 1)$-(ana-)functors. In other terminologies, an $\infty$-prestack is an $(\infty, 1)$-pseudo-functor with values in ${}^{(\infty, 1)}\Cat$. Prestacks over a given $(\infty, 1)$-category $\C$ form a $(\infty, 1)$-category in an obvious manner; we shall denote it by ${}^{\infty}\Pre\Stk(\C)$. Actually, for all base $(\infty, 1)$-categories $\C$, the $(\infty, 1)$-category ${}^{\infty}\Pre\Stk(\C)$ can also be endowed with the structure of a weak $(\infty, 2)$-category, determined on the $(\infty, 2)$-categorical level by $(\infty, 2)$-morphisms which are \textit{strict} $(\infty, 2)$-natural transformations
                            
                            Additionally, we shall assume the Axiom of Choice (which incidentally, forces us to adopt definition \ref{def: internal_categories}). The advantage in this is that for all base $(\infty, 1)$-categories $\C$, we will automatically be given a \textit{weak} $2$-equivalence of \textit{weak} $(\infty, 2)$-categories:
                                $${}^{\infty}\Pre\Stk(\C) \cong \Fib_{(\infty, 1)}(\C)$$
                            between the weak $(\infty, 2)$-category of $\infty$-prestacks on $\C$ and that of fibred $(\infty, 1)$-categories on $\C$. Logicians might scoff at such a practice, but since choosing cleavages in algebraic geometry is mostly just asking for trouble, we shall try to bear the shame of having Choice.
                            \item \textbf{($\infty$-stacks):} Let us build upon the above notion of $\infty$-prestacks and declare that from this point on, the term \say{\textbf{$\infty$-stack}} shall mean \say{sheaf of $(\infty, 1)$-categories}, i.e. an $\infty$-stack is a fibred $(\infty, 1)$-category which satisfies $(\infty, 1)$-descent; concretely, an $\infty$-stack $\calX$ over a small \href{https://ncatlab.org/nlab/show/(infinity,1)-site}{\underline{$\infty$-site}} $(\C, J)$ is an $\infty$-prestack such that for all objects $X$ of $\C$ and all covering $J$-sieves $\U_{/X}$ thereon, one has the following equivalence of $(\infty, 1)$-categories:
                                $$\calX(X) \cong \calX\left({}^{(\infty, 1)}\underset{U \in \U_{/X}}{\colim} h_U\right)$$
                            $\infty$-stacks on $(\C, J)$ form a full weak $(\infty, 2)$-subcategory of the weak $(\infty, 2)$-category ${}^{\infty}\Pre\Stk(\C)$ of $\infty$-prestacks on $\C$, which shall be denoted by ${}^{\infty}\Stk(\C, J)$. Furthermore, $\infty$-stacks of $\infty$-groupoids (i.e. $(\infty, 1)$-categories fibred in $\infty$-groupoids that satisfy $(\infty, 1)$-descent) on small $\infty$-sites $(\C, J)$ form an $\infty$-topos which is written $\Sh_{(\infty, 1)}(\C, J)$; occasionally, we might refer to these as sheaves of $\infty$-groupoids; naturally, $\Sh_{(\infty, 1)}(\C, J)$ comes equipped with a \href{https://ncatlab.org/nlab/show/(infinity,1)-topos#AsAGeometricEmbedding}{\underline{$(\infty, 1)$-geometric embedding}} into the weak $(\infty, 2)$-category $\Psh_{(\infty, 1)}(\C)$ of $\infty$-prestacks of $\infty$-groupoids on $\C$.
                        \end{enumerate}
                        \item \textbf{(The internal point of view):} From the internal point of view (which in the opinion of the author, is a lot more intuitive; this is, however, purely personal), $\infty$-(pre)stacks are nothing but internal $(\infty, 1)$-categories inside $\infty$-(pre)sheaf topoi. Note that $\infty$-(pre)stacks fibred in $\infty$-groupoids are precisely $(\infty, 1)$-(pre)sheaves, so we do not to treat them as a cases of (pre)stacks which are not (pre)sheaves, unlike how $(2, 1)$-(pre)sheaves having to be considered as phenomena more general than (pre)sheaves of sets; this is thanks to the fact that the notion of $\infty$-groupoids subsumes both those of $1$-groupoids and sets (which may be viewed as $0$-groupoids): in particular, \textit{sets are $0$-truncated $\infty$-groupoids, and $1$-groupoids are $1$-truncated $\infty$-groupoids}. Incidentally, this is also a first glimpse into the myriads of reasons why $\infty$-categories might help us simplify instead of further complicating constructions in algebraic geometry, especially when it comes to taking colimits (recall how in general, quotients stacks are not sheaves of sets but rather stacks in groupoids).
                    
                        Readers may also have heard of mysterious entities known as \textbf{$\infty$-gerbes}. There are many definitions floating around, but again, let us fix one meaning for the term. To us, an $\infty$-gerbe on a small $\infty$-site $(\C, J)$ will always be an group object internal to ${}^{\infty}\Stk(\C, J)$ (i.e. an object of $\Sh_{{}^{\infty}\Grpd}(\C, J)$ whose connected component is the connected component of the terminal object of $\Sh_{{}^{\infty}\Grpd}(\C, J)$). Additionally, the notion of gerbes coincides with that of so-called principal $\infty$-bundles.
                    \end{enumerate}
                \end{convention}
                
                \begin{remark}[Derived affine schemes over general $\infty$-(pre)stacks] \label{remark: weak_(infinity, 2)_yoneda}
                    Instead of embedding a base $(\infty, 1)$-category $\C$ into the $(\infty, 1)$-category of $(\infty, 1)$-presheaves of $\infty$-groupoids $\Psh_{{}^{\infty}\Grpd}(\C)$ thereon, we shall be viewing $\C$ as a tautological weak $(\infty, 2)$-category, and shall instead be embedding it into the weak $(\infty, 2)$-category ${}^{\infty}\Pre\Stk(\C)$ of $\infty$-prestacks over $\C$ via a fully faithful $(\infty, 2)$-functor, known as the \textbf{$(\infty, 2)$-Yoneda embedding}. Via the $(\infty, 2)$-Yoneda embedding, one can view objects of $\C$ as $\infty$-prestacks fibred in $\infty$-groupoids (which we note to be $(\infty, 2)$-categories wherein all $(\infty, 1)$-cells and $(\infty, 2)$-cells are identities), and therefore, the notion of objects of $\C$ with mappings into an arbitrarily given base $\infty$-prestack $\calY \in {}^{\infty}\Pre\Stk(\C)$ is well-defined; in particular, one can meaningfully talk about so-called derived affine schemes over $\infty$-prestacks on any (symmetric monoidal) dg-category of commutative algebras ${}^{\dg, k/}\Comm\Alg^{\op}$ (more on this later). 
                    \\
                    For a detailed discussion of the weak $(\infty, 2)$-Yoneda embedding, the reader may consult \cite{nlab:yoneda_lemma_for_bicategories}.
                \end{remark}
                
            \subsubsection{Truncation; connectivity and co-connectivity}
                \begin{definition}[Truncated objects] \label{def: truncated_objects} \index{Truncated! objects}
                    Let $n \geq -2$ be an integer.
                    \begin{enumerate}
                        \item \textbf{(Truncated spaces):} An $\infty$-groupoid $H$ said to be $n$-truncated if:
                            \begin{itemize}
                                \item  
                                \item
                            \end{itemize}
                        \item \textbf{(Truncated objects \cite[Definition 5.5.6.1]{HTT}):} Let $n \geq -2$. An object $X$ of an $\infty$-category $\S$ is said to be $n$-truncated if
                    \end{enumerate}
                \end{definition}
    
        \subsection{Derived geometric stacks}
            \subsubsection{Generalities}
                \begin{definition}[Geometric $\infty$-stacks] \label{def: derived_geometric_stacks} \index{$\infty$-stacks! geometric}
                    \noindent
                    \begin{enumerate}
                        \item \textbf{(Representable morphisms):} A morphism:
                            $$f: \calX \to \calY$$
                        of $\infty$-stacks on some subcanonical small $\infty$-site $(\C, J)$ is called \textbf{representable} if and only if for all morphism:
                            $$\pi: h_V \to \calY$$
                        from a representable sheaves $h_V$, the $(\infty, 1)$-pullback $\calX \x_{f, \calY, \pi} h_V$ is also representable. When the underlying site is a site of commutative dg-algebras, representable morphisms are might also be referred to as \textbf{affine-schematic}.
                        \item \textbf{(Geometric $\infty$-stacks):} A \textbf{geometric $\infty$-stack} over some small $\infty$-site $(\C, J)$ is a quotient $\infty$-stack $\calX$ (i.e. an equivalence relation internal to $\Sh_{(\infty, 1)}(\C, J)$) whose diagonal morphism:
                            $$\Delta_{\calX}: \calX \to \calX \x^{(\infty, 1)} \calX$$
                        is representable. Geometric $\infty$-stacks over small $\infty$-sites of commutative dg-algebras are commonly known as algebraic $\infty$-stacks; in particular, algebraic $\infty$-stacks over small \'etale sites are colloquially known as \textbf{derived Deligne-Mumford stacks}, and those over small smooth $\infty$-sites (in the algebraic sense of the word \say{smooth}) are known as \textbf{derived Artin stacks}.   
                    \end{enumerate}
                \end{definition}
                
                \begin{lemma}[$\infty$-categories of geometric $\infty$-stacks] \label{lemma: derived_geometric_stack_categories}
                    Let $(\C, J)$ be a small $\infty$-site. Then, we have the following tower of full weak $(\infty, 2)$-subcategories that are all \textit{closed under arbitrary weak $(\infty, 2)$-limits and under finite weak $(\infty, 2)$-colimits}:
                        $$
                            \begin{tikzcd}
                            	{{}^{\infty}\Stk(\C, J)} \\
                            	{\Sh_{(\infty,1)}(\C, J)} \\
                            	{{}^{\infty}\Stk^{\geom}(\C, J)}
                            	\arrow[no head, from=3-1, to=2-1]
                            	\arrow[no head, from=2-1, to=1-1]
                            \end{tikzcd}
                        $$
                    wherein ${}^{\infty}\Stk^{\geom}(\C, J)$ is the weak $(\infty, 2)$-category of geometric $\infty$-stacks on $(\C, J)$.
                \end{lemma}
                    \begin{proof}
                        
                    \end{proof}
                
                \begin{theorem}[$(\infty, 1)$-sheafification of geometric $\infty$-prestacks] \label{theorem: (infinity,1)_sheafification_of_derived_geometric_prestacks}
                    Let $(\C, J)$ be a small $\infty$-site. Then, every weak $(\infty, 2)$-geometric embedding of the $\infty$-topos $\Sh_{(\infty, 1)}(\C, J)$ into the $(\infty, 1)$-presheaf $\infty$-topos $\Psh_{(\infty, 1)}(\C, J)$ can be restricted down to a weak $(\infty, 2)$-geometric embedding of the weak $(\infty, 2)$-category of geometric $\infty$-stacks $\Stk^{\geom}(\C, J)$ into the weak $(\infty, 2)$-category of geometric $\infty$-prestacks ${}^{\infty}\Pre\Stk^{\geom}(\C)$ (i.e. geometric $\infty$-stacks on $\C$ equipped with the chaotic topology); pictorially, one might think of this statement as the following diagram being commutative in the weak $(\infty, 2)$-category ${}^{(\infty, 2)}\Cat$ of weak $2$-categories: 
                        $$
                            \begin{tikzcd}
                            	{\Sh_{(\infty, 1)}(\C, J)} & {\Psh_{(\infty, 1)}(\C)} \\
                            	{{}^{\infty}\Stk^{\geom}(\C, J)} & {{}^{\infty}\Pre\Stk^{\geom}(\C)}
                            	\arrow[""{name=0, anchor=center, inner sep=0}, "{{}^{\sh}(-)}"', shift right=2, shorten <=2pt, from=1-2, to=1-1]
                            	\arrow[""{name=1, anchor=center, inner sep=0}, "i"', shift right=2, hook, from=1-1, to=1-2]
                            	\arrow[shift left=2, hook', from=2-1, to=1-1]
                            	\arrow[shift left=2, hook', from=2-2, to=1-2]
                            	\arrow[""{name=2, anchor=center, inner sep=0}, "{i^{\geom}}"', shift right=2, hook, from=2-1, to=2-2]
                            	\arrow[""{name=3, anchor=center, inner sep=0}, "{{}^{\sh}(-)^{\geom}}"', shift right=2, from=2-2, to=2-1]
                            	\arrow["\dashv"{anchor=center, rotate=-90}, draw=none, from=0, to=1]
                            	\arrow["\dashv"{anchor=center, rotate=-90}, draw=none, from=3, to=2]
                            \end{tikzcd}
                        $$
                    Note that the restricted geometric embedding $({}^{\sh}(-)^{\geom} \ladjoint i^{\geom})$ is well-defined thanks to categories of geometric (pre)stacks embedding fully faithfully into $(\infty, 1)$-(pre)sheaf $\infty$-topoi.
                \end{theorem}
                    \begin{proof}
                        
                    \end{proof}
                \begin{convention}[\textcolor{red}{\underline{IMPORTANT}} Categories of derived algebraic stacks] \label{conv: derived_algebraic_stacks_notations}
                    From now on, category of derived algebraic stacks determined by some Grothendieck topology $\tau$ on some small $\infty$-site of commuative $k$-algebras ${}^{\dg, k/}\Comm\Alg^{\op}$ will be denoted by ${}^{\infty}[\Spec k]^{\geom}_{\tau}$. For instance, the category of derived DM-stacks over $\Spec k$ is going to be denoted by ${}^{\infty}[\Spec k]^{\geom}_{\et}$. Also, let us also denote the category of general $\infty$-stacks on a site ${}^{\dg, \tau}\Comm\Alg^{\petit, \op}_{\tau}$ by ${}^{\infty}[\Spec k]_{\tau}$. When $\tau$ is the chaotic topology, we shall simply write ${}^{\infty}[\Spec k]$ for the $\infty$-category of $\infty$-prestacks on ${}^{\dg, k/}\Comm\Alg^{\op}$.
                    
                    $\infty$-topoi are also of great importance, as categories of geometric $\infty$-stacks embed fully faithfully via left-exact inclusions into them (cf. lemma \ref{lemma: derived_geometric_stack_categories}). As such, we shall dedicate the notation ${}^{(\infty, 1)}[\Spec k]^{\geom}_{\tau}$ for the $\infty$-topos of sheaves of $\infty$-groupoids over some algebraic $\infty$-site ${}^{\dg, k/}\Comm\Alg_{\tau}^{\petit, \op}$; as for $\infty$-presheaf topoi, they shall be denoted by ${}^{(\infty, 1)}[\Spec k]$. 
                \end{convention}
            
            \subsubsection{Connectivity and co-connectivity of derived geometric stacks}
        
        \subsection{Derived schemes and derived algebraic spaces}
            \subsubsection{As derived DM-stacks}
            
            \subsubsection{As structured spaces \textit{\`a la} Lurie}
        
    \section{Quasi-coherent modules} \label{section: qcoh}
        Let's be real: people - especially modern representation theorists disguised as geometers - care absolutely zilch about the \textit{actual} geometric objects in algebraic geometry. Instead, we sought after a notion of \say{local-global compatible} linear algebra, i.e. a way to probe geometric structures using the abundance of knowledge that has been accumulated through the centuries in the field of linear algebra. For instance, and to bring up representation theory again, by gaining and understanding of their linear representations, one effectively knows all there is to know about mysterious and exotic objects such as groups and Lie algebras, which (and especially the case of groups) are ultimately geometric in nature: groups are nothing but local versions of group schemes. Now that's all well and good, but how do we actually perform this linear-algebraic black magic ? The answer is via quasi-coherent modules.
        
        Quasi-coherent modules are quite literally everywhere in algebraic geometry. Classically (i.e. \`a la EGA or Stacks Project), one would define these objects as sheaves of modules on schemes/algebraic spaces/what-have-you or even just topological spaces and sites that satisfy certain local-to-global compatibility conditions (see \cite[\href{https://stacks.math.columbia.edu/tag/01BD}{Tag 01BD}]{stacks} for a reminder of this traditional formulation). This approach, while seemingly technically simple, has one flaw that is rather hard to ignore: the local-to-global transition is not very \say{categorical}, meaning that this seemingly perfectly usable definition of quasi-coherent modules will might bring home technical difficulties that at some point, might make us (read: the author) decide to throw in the towel and go downstairs for a biscuit instead. It is due to this reason that in this book, we are going to approach quasi-coherent modules from a $2$-categorical angle: namely, we shall be studying the symmetric monoidal (abelian) categories of quasi-coherent modules and continuous functors between them, instead of objects therein (in fact, we care little about the actual quasi-coherent modules themselves), how these categories are parametrised by underlying sites of schemes, and how everything is packaged together by so-called \say{stacks of quasi-coherent modules}, which form $2$-categories in a somewhat obvious manner. We should also note that such an approach is $2$-categorical due to the important fact that the associativity that compositions of base change functors between categories of quasi-coherent modules are \textit{supposed} to enjoy is only preserved up to natural equivalences; heuristically, one can think about how the functors:
            $$(- \tensor R) \tensor S$$
        and:
            $$- \tensor (R \tensor S)$$
        are only naturally isomorphic and not identical. 
        
        Admittedly, this is a very high-tech approach, and while the classical low-tech approach has been employed to produce astounding results such as Serre's Criterion for Affineness and the Grothendieck-Riemann-Roch Theorem, we wholeheartedly believe that it has merits. In particular:
            \begin{enumerate}
                \item The stack of quasi-coherent modules over site of schemes tells us intuively as well as \textit{formally} (i.e. \textit{sans} hand-wavy sheaf restriction) how local and global sections of quasi-coherent sheaves are related via change of scalar operations. 
                \item The fact that quasi-coherent module categories possess symmetric monoidal structures is given much emphasis in this formulation.
                \item The homotopy-isation of our setup shall be rather self-evident.
            \end{enumerate}
        With that being said, let us proceed by discussing how the category of quasi-coherent modules on a given scheme is defined via a \textit{stack} on whatever site one might associate to said scheme.
    
        \begin{convention}[Everything is derived!] \label{conv: schemes_2_everything_is_derived}
            \noindent
            \begin{itemize}
                \item From now on until the end of the chapter, everything will be assumed to be derived. In particular, this means that we shall defer to convention \ref{conv: algebraic_stacks_notations} rather than convention \ref{conv: derived_algebraic_stacks_notations}.
                \item By $\Cat^1$, or simply $\Cat$, we shall actually mean ${}^{(\infty, 1)}\Cat^1$, i.e. the $(\infty, 1)$-category of $(\infty, 1)$-categories and functors between them, and by $\Cat^2$ we will be referring to the $(\infty, 2)$-category of $(\infty, 1)$-categories, functors between them, and natural transformations between these functors. 
                
                Similarly, by $\Grpd^1$, or simply $\Grpd$, we will actually mean the $(\infty, 1)$-category of $\infty$-groupoids and functors between them, and by $\Grpd^2$, we shall mean the $(\infty, 2)$-category of $\infty$-groupoids, functors between them, and natural transformations between these functors.
                \item A subcategory of $\Cat$ this is of particular interest is $(\Cat^{\dg, \cont})^2$ (or simply $\Cat^{\dg, \cont}$), the $(\infty, 2)$-category of stable linear (i.e. differential-graded) $(\infty, 1)$-categories (see section \ref{section: homological_algebra} for the notion of stable $(\infty, 1)$-categories). Of course, we can also view $\Cat^{\dg, \cont}$ as a mere $(\infty, 1)$-category; when necessary, we shall write $(\Cat^{\dg, \cont})^1$ to put emphasis on the disregard of $2$-morphisms.
            \end{itemize} 
        \end{convention}
        
        \subsection{Generalities}
            \subsubsection{Prestacks of quasi-coherent modules}
                \begin{definition}[Quasi-coherent modules] \label{def: qcoh_def}
                    \noindent
                    \begin{enumerate}
                        \item \textbf{(Quasi-coherent modules):} Our goal is to construct categories of quasi-coherent sheaves on schemes and algebraic spaces/stacks in a manner that is as adaptable to the world of derived algebraic geometry as possible, because at the end of the day, one cares most about cohomologies of quasi-coherent modules, and these \say{things} naturally inhabit stable linear $(\infty,1)$-categories - whatever that means - and to that end, let us consider firstly a sketch of the theory. For every affine scheme $\Spec R$, let us \textit{declare} that:
                            $$\QCoh(\Spec R) \cong {}_R\Mod$$
                        and for prestacks of groupoids $\calY$ on $\Cring^{\op}$ (a class of objects which subsumes that of affine schemes; schemes and algebraic stacks are instances of prestacks on $\Cring^{\op}$), fitting into commutative diagrams of prestacks as follows:
                            $$
                                \begin{tikzcd}
                                	{\Spec R'} && {\Spec R} \\
                                	& \calY
                                	\arrow["{y'}"', from=1-1, to=2-2]
                                	\arrow["y", from=1-3, to=2-2]
                                	\arrow["f", from=1-1, to=1-3]
                                \end{tikzcd}
                            $$
                        suppose that there is a functor $f^*: \QCoh(\Spec R) \to \QCoh^*(\Spec R')$ such that for all objects $M_{\Spec R', y'} \in \QCoh(\Spec R')$, there exists an object $M_{\Spec R, y} \in \QCoh^*(\Spec R)$ so that:
                            $$M_{\Spec R', y'} \cong f^* M_{\Spec R, y}$$
                        We shall be calling the categories of the form $\QCoh^*(\calY)$ categories of \textbf{quasi-coherent modules} on $\calY$.
                        \item \textbf{(Categories of quasi-coherent modules):} More precisely, an object $M \in \QCoh^*(\calY)$ is a prestack (cf. convention \ref{conv: prestacks}):
                            $$M: (\Sch^{\aff}_{/\calY})^{\op} \to \Cat$$
                        which:
                            \begin{enumerate}
                                \item sends the affine schemes $y: \Spec R \to \calY$ to the categories $\QCoh(\Spec R)$.
                                \item and sends commutative diagrams in $\Sch^{\aff}_{/\calY}$ as below:
                                    $$
                                        \begin{tikzcd}
                                        	{\Spec R''} & {\Spec R'} & {\Spec R} \\
                                        	& \calY
                                        	\arrow["{y''}"', from=1-1, to=2-2]
                                        	\arrow["{y'}", from=1-2, to=2-2]
                                        	\arrow["{f'}", from=1-1, to=1-2]
                                        	\arrow["y", from=1-3, to=2-2]
                                        	\arrow["f", from=1-2, to=1-3]
                                        \end{tikzcd}
                                    $$
                                to diagrams in $\Cat$ as below:
                                    $$
                                        \begin{tikzcd}
                                        	{\QCoh(\Spec R'')} & {\QCoh(\Spec R')} & {\QCoh(\Spec R)}
                                        	\arrow["{M(f')}"', from=1-2, to=1-1]
                                        	\arrow["{M(f)}"', from=1-3, to=1-2]
                                        \end{tikzcd}
                                    $$
                                such that one has natural isomorphisms between $M(f' \circ f)$ and $M(f') \circ M(f)$ that are not necessarily the identity.
                            \end{enumerate}
                        and a morphism in $\QCoh(\calY)$ is just a \href{https://ncatlab.org/nlab/show/pseudonatural+transformation}{\underline{pseudo-natural transformation}} (i.e. a morphism of pseudo-functors).
                        \item \textbf{(Prestacks of quasi-coherent modules):} Having defined categories of quasi-coherent modules on categories of affine schemes over prestacks of groupoids on $\Cring^{\op}$, let us try to define the \textbf{prestack of quasi-coherent modules} on the category $[\Spec \Z]$ of prestacks on $\Cring^{\op}$, which we shall denote by:
                            $$\QCoh^*: [\Spec \Z]^{\op} \to \Cat$$
                        Such a prestack will associate to each commutative diagrams in $[\Spec \Z]$ as below:
                            $$
                                \begin{tikzcd}
                                	{\calY''} & {\calY'} & \calY
                                	\arrow["{\varphi'}", from=1-1, to=1-2]
                                	\arrow["\varphi", from=1-2, to=1-3]
                                \end{tikzcd}
                            $$
                        to diagrams in $\Cat$ as below:
                            $$
                                \begin{tikzcd}
                                	{\QCoh^*(\calY'')} & {\QCoh^*(\calY')} & {\QCoh^*(\calY)}
                                	\arrow["{\QCoh^*(\varphi')}"', from=1-2, to=1-1]
                                	\arrow["{\QCoh^*(\varphi)}"', from=1-3, to=1-2]
                                \end{tikzcd}
                            $$
                        such that one has natural isomorphisms between $\QCoh^*(\varphi' \circ \varphi)$ and $\QCoh^*(\varphi') \circ \QCoh^*(\varphi)$ that are not necessarily the identity. In short, $\QCoh^*(-)$ is a prestack that sends objects $\calY \in [\Spec \Z]^{\op}$ to $1$-categories $\QCoh^*(\calY)$ of prestacks $M$ that in turn assign to affine schemes over $\calY$ the appropriate module categories. 
                        
                        In the event that the prestacks $\calY'', \calY'$, and $\calY$ are affine schemes over some base prestack $\calY_0$, i.e. by replacing the category $[\Spec \Z]^{\op}$ with $\Sch^{\aff}_{/\calY_0}$, one recovers the prestack $\QCoh^*(\calY_0)$. 
                    \end{enumerate}
                \end{definition}
                \begin{remark}[The purpose of quasi-coherent modules]
                    Ultimately, we are trying to build a theory of quasi-coherent modules wherein categories thereof over schemes can be obtained via gluing together those on the covering affine schemes. In other words, the theory of quasi-coherent modules ought to look like a globalisation of the theory of modules over commutative rings. This goal shall be realised fully via descent theory (cf. \cite{vistoli_descent}).
                \end{remark}
                \begin{remark}[Why prestacks ?]
                    We should note that definition \ref{def: qcoh_def} is not standard. Most textbooks will define \textit{objects} of quasi-coherent module categories as modules over structure sheaves that satisfy certain cohomological conditions. The problem with this formulation is that even though abstract entities such as pseudo-functors are not needed for it, it relies very heavily on the idea of sheaf restriction, which is not very well-defined. Prestacks allow us to avoid that technical hiccup; they also highlight an important point: non-quasi-coherent modules are not very \say{algebraic}.
                \end{remark}
                
                Now that we have managed to write down the definition of what it means for a category to have quasi-coherent modules as objects, let us examine some basic properties of the prestack of quasi-coherent modules on $[\Spec \Z]$. The first of these is its universal property.
                \begin{proposition}[Universal property of quasi-coherent modules] \label{prop: qcoh_universal_property}
                    Fix a base commutative ring $k$. Then, the prestack:
                        $$\QCoh^*: [\Spec k]^{\op} \to \Cat$$
                    of quasi-coherent modules on the $1$-category $[\Spec k]$ of prestacks of groupoids on ${}^{k/}\Comm\Alg^{\op}$ is the \textit{right}-Kan extension of:
                        $$\QCoh^*|_{\Sch^{\aff}_{/\Spec k}}: \Sch^{\aff, \op}_{/\Spec k} \to \Cat$$
                    along the canonical embedding $\Sch^{\aff, \op} \hookrightarrow [\Spec k]^{\op}$. In particular, we have:
                        $$\QCoh^*(\calY) \cong \underset{S \in \Sch^{\aff}_{/\calY}}{\lim} \QCoh^*(S)$$
                    for all $\calY \in [\Spec k]$.
                \end{proposition}
                    \begin{proof}
                        
                    \end{proof}
                \begin{remark}[The correct definition of quasi-coherent modules]
                    To be quite honest, definition \ref{def: qcoh_def} should be thought of less as a proper definition of the prestack of quasi-coherent modules and more of a preliminary discussion. We have only granted it the title of \say{Definition} and the honour that goes along with it because proposition \ref{prop: qcoh_universal_property} is a rather non-trivial phenomenon. 
                \end{remark}
    
            \subsubsection{Stability and continuity}
                \begin{remark}[dg-categories of quasi-coherent sheaves] \label{remark: dg_qcoh_categories}
                    For every commutative ring $k$ and every prestacks of groupoids $\calY$ over ${}^{\dg, k/}\Comm\Alg^{\op}$, the $\infty$-category $\QCoh(\calY)$ is stable and $k$-linear (see definition \ref{def: stable_infinity_categories} along with section \ref{section: homological_algebra} altogether for the notion of stable $(\infty, 1)$-categories), i.e. $\QCoh(\calY)$ is a dg-category (via Lurie's version of the Dold-Kan Correspondence for stable $\infty$-categories; cf. \cite[Theorem 1.2.3.7]{HA}). If this seems a bit abstract, then keep in mind that one can imagine the stable $\infty$-category of quasi-coherent sheaves on a prestacks of groupoids to be the stable $\infty$-category of chain complexes of quasi-coherent sheaves on the underlying classical prestack; note that chain complexes of classical quasi-coherent sheaves are well-defined, as these classical quasi-coherent sheaves form abelian categories. 
                \end{remark}
                
                \begin{remark}[Discontinuity of $*$-pullbacks] \label{remark: *_pullbacks_discontinuity}
                    Let:
                        $$f: \calY_1 \to \calY_2$$
                    be a morphism of prestacks of groupoids on ${}^{k/}\Comm\Alg^{\op}$ (where $k$ is some base commutative ring). Then, the corresponding $*$-pullback:
                        $$f^*: \QCoh(\calY_2) \to \QCoh(\calY_1)$$
                    is expected to preserve (small) limits in general, as it is given by tensor products (cf. definition \ref{def: qcoh_def}), and it is why we need to develop the notion of $*$-pushforwards (see subsubsection \ref{subsubsection: qcoh_*_pushforwards}). To give an example of why discontinuous $*$-pullbacks might create issues for us, \todo{Finish example}
                \end{remark}
        
        \subsection{The pull-push yoga}
            \subsubsection{\texorpdfstring{$*$}{}-pushforwards of quasi-coherent modules} \label{subsubsection: qcoh_*_pushforwards}
                \begin{remark}[Existence of $*$-pushforwards] \label{remark: existence_of_*_pushforwards}
                    Let:
                        $$f: \calY_1 \to \calY_2$$
                    be a morphism of prestacks of groupoids on ${}^{k/}\Comm\Alg^{\op}$, for some commutative ring $k$. As categories of quasi-coherent modules are presentable (cf. definition \ref{def: presentable_infinity_categories}), we can apply Lurie's Adjoint Functor Theorem \cite[Corollary 5.5.2.9]{HTT} to the canonically induced pullback:
                        $$f^*: \QCoh(\calY_2) \to \QCoh(\calY_1)$$
                    and get back the following adjunction:
                        $$
                            \begin{tikzcd}
                            	{\QCoh(\calY_1)} & {\QCoh(\calY_2)}
                            	\arrow[""{name=0, anchor=center, inner sep=0}, "{f^*}"', shift right=2, from=1-2, to=1-1]
                            	\arrow[""{name=1, anchor=center, inner sep=0}, "{f_*}"', shift right=2, from=1-1, to=1-2, dashed]
                            	\arrow["\dashv"{anchor=center, rotate=-90}, draw=none, from=0, to=1]
                            \end{tikzcd}
                        $$
                    (specifically, $f^*$ preserves small colimits - which can all be built out of epimorphisms and coproducts - because $f^*(-) \cong - \tensor_{\calO_{\calY_1}} \calO_{\calY_2}$, and hence admits a right-adjoint). 
                    
                    The right-adjoint $f_*: \QCoh(\calY_1) \to \QCoh(\calY_2)$ is known as the \textbf{$*$-pushforward}, and we shall call the adjoint pair $(f^* \ladjoint f_*)$ a $*$-pull-push adjunction.
                \end{remark}
            
                \begin{definition}[Base change morphisms] \label{def: base_change_morphisms}
                    Let $k$ be an arbitrary commutative ring. Then, for every pullback square in $[\Spec k]$:
                        $$
                            \begin{tikzcd}
                            	{\calY_1'} & {\calY_1} \\
                            	{\calY_2'} & {\calY_2}
                            	\arrow["\lrcorner"{anchor=center, pos=0.125}, draw=none, from=1-1, to=2-2]
                            	\arrow["{f'}"', from=1-1, to=2-1]
                            	\arrow["g", from=2-1, to=2-2]
                            	\arrow["f", from=1-2, to=2-2]
                            	\arrow["{g'}", from=1-1, to=1-2]
                            \end{tikzcd}
                        $$
                    (which, let us note, is uniquely determined by the pair of arrows $f: \calY_1 \to \calY_2, g: \calY_2' \to \calY_2'$), there is a uniquely determined \textbf{base change morphism}, which is the following natural transformation coming from the various $*$-pull-push adjunctions at play:
                        $$
                            \begin{tikzcd}
                            	{\QCoh(\calY_1')} & {\QCoh(\calY_1)} \\
                            	{\QCoh(\calY_2')} & {\QCoh(\calY_2)}
                            	\arrow["{g^*}"', from=2-2, to=2-1]
                            	\arrow["{f'_*}"', from=1-1, to=2-1]
                            	\arrow["{f_*}", from=1-2, to=2-2]
                            	\arrow["{g'^*}"', from=1-2, to=1-1]
                            	\arrow[shorten <=14pt, shorten >=14pt, Rightarrow, from=2-2, to=1-1]
                            \end{tikzcd}
                        $$
                \end{definition}
                \begin{example}[Pathological behaviours of base change morphisms] \label{example: base_change_bad_behaviour}
                    One would hope that base change morphisms are all isomorphisms, but sadly, this is not the case. Consider the following situation, for instance:
                        $$
                            \begin{tikzcd}
                            	{\A^1 \x_{\pt} \coprod^{\aleph} \pt} & {\coprod^{\aleph} \pt} \\
                            	{\A^1} & \pt
                            	\arrow["{f'}"', from=1-1, to=2-1]
                            	\arrow["g", from=2-1, to=2-2]
                            	\arrow["{g'}", from=1-1, to=1-2]
                            	\arrow["f", from=1-2, to=2-2]
                            	\arrow["\lrcorner"{anchor=center, pos=0.125}, draw=none, from=1-1, to=2-2]
                            \end{tikzcd}
                        $$
                    where $\pt$ is the one-point scheme and $\aleph$ is some countable cardinal. There is no guarantee that the morphism:
                        $$f: \coprod^{\aleph} \pt \to \pt$$
                    would be affine-schematic, and hence we can not expect the base change morphism $g^* f_* \to f'_* g'^*$ to be an isomorphism (to prove this, we can use the fact that higher direct images of quasi-coherent modules vanish over affine schemes).
                \end{example}
                
                \begin{proposition}[Schematic base change] \label{prop: schematic_qc_base_change} 
                    This is \cite[Proposition I.2.2.2]{GR1}.
                    
                    Let:
                        $$f: \calY_1 \to \calY_2$$
                    be a schematic and quasi-compact morphism in $[\Spec k]$, for some commutative ring $k$. Then:
                        \begin{enumerate}
                            \item The $*$-pushforward $f_*: \QCoh(\calY_1) \to \QCoh(\calY_2)$ is continuous (i.e. it preserves all small limits).
                            \item For all morphism of prestack $g: \calY_2' \to \calY_2$, the base change morphism $g^* f_* \to f'_* g'^*$ shall be an isomorphism.
                        \end{enumerate}
                \end{proposition} 
                    \begin{proof}
                        Without loss of generality, we may assume that $\calY_1, \calY_2$ are merely affine schemes instead of general quasi-compact prestacks (as we can simply perform induction to recover the general case), and in particular, that $f: \calY_1 \to \calY_2$ is \textit{affine}-schematic, as schematic morphisms between quasi-compact prestacks are representable by finitely affine-schematic morphisms; let:
                            $$\calY_1 \cong \Spec R_1, \calY_2 \cong \Spec R_2$$
                            \begin{enumerate}
                                \item Thanks to the fact that quasi-coherent modules over affine schemes are just modules in the ordinary sense (cf. definition \ref{def: qcoh_def}), the $*$-pushforward along $f: \Spec R_1 \to \Spec R_2$ is, by the tensor-hom adjunction, nothing but the functor:
                                    $${}_{R_1}\Mod(R_2, -): {}_{R_1}\Mod \to {}_{R_2}\Mod$$
                                and hence preserves all small limits. 
                                \item This can easily be shown by evaluating the natural transformation $g^* f_* \to f'_* g'^*$ at objects of ${}_{R_2}\Mod$ and carrying out the routine computations.  
                            \end{enumerate}
                    \end{proof}
                    \begin{remark}[Quasi-coherent modules on schematic quasi-compact prestacks] \label{remark: qcoh_on_schematic_qs_prestacks}
                        Fix a base commutative ring $k$ and note that because compositions of schematic morphisms in $[\Spec k]$ are also schematic and compositions of quasi-compact morphisms remain quasi-compact, there is a (non-full) subcategory $[\Spec k]^{\sch, \qc}$ of $[\Spec k]$ spanned by schematic quasi-compact morphisms. Now, thanks to the fact that affine schemes are tautologically quasi-compact (and schematic!) and by proposition \ref{prop: schematic_qc_base_change}, there is a \say{restricted} prestack of quasi-coherent modules:
                            $$\QCoh_*|_{[\Spec k]^{\sch, \qc}}: [\Spec k]^{\sch, \qc} \to \Cat^{\dg, \cont}$$
                        which exists as the \textit{left}-Kan extension of $\QCoh_*|_{\Sch^{\aff}_{/\Spec k}}: \Sch^{\aff}_{/\Spec k} \to \Cat^{\dg, \cont}$ along the canonical inclusion $\Sch^{\aff}_{/\Spec k} \hookrightarrow [\Spec k]^{\sch, \qc}$. In particular, this tells us that:
                            $$\QCoh_*|_{[\Spec k]^{\sch, \qc}}(\calY_1 \to \calY_2) \cong \underset{S \in \Sch^{\aff}_{/\calY_2}}{\colim} \QCoh_*|_{\Sch^{\aff}_{/\Spec k}}(S \to \calY_2)$$
                        for all schematic and quasi-compact morphism of prestacks $\calY_1 \to \calY_2$ (the reader in encouraged to contrast this with proposition \ref{prop: qcoh_universal_property}).
                    \end{remark}
                    
                    \begin{proposition}[Smooth and quasi-compact quasi-separated base change] \label{prop: smooth_qcqs_base_change}
                        
                    \end{proposition}
                        \begin{proof}
                            
                        \end{proof}
                    
            \subsubsection{\texorpdfstring{$\QCoh$}{} is rigid symmetric monoidal}
        
    \section{Ind-coherent sheaves} \label{section: indcoh}
        \subsection{Categories of ind-coherent sheaves} \label{subsection: categories_of_ind_coherent_sheaves}
            \subsubsection{Ind-coherent sheaves over locally almost of finite type schemes}
                \begin{definition}[Ind-coherent sheaves] \label{def: ind_coherent_sheaves_on_laft_schemes}
                    Let $k$ be an arbitrary base commutative ring, $X$ be a scheme \textit{locally almost of finite type} over $\Spec k$. Then, the category $\Ind\Coh(X)$ of ind-coherent modules on $X$ is precisely the ind-completion of $\Coh(X)$. 
                \end{definition}
            
                Having stated the definition, let us now investigate the (desired) formal properties of ind-coherent sheaves over the simplest setup possible (at least from a homological point-of-view), namely that of locally almost of finite type schemes. 
                \begin{convention} \label{conv: indcoh_to_qcoh_functor}
                    $\Ind\Coh(\calY)$ is a cocomplete full subcategory of $\QCoh(\calY)$ for all prestacks $\calY$ locally almost of finite type. Typically, the evident fully faithful embedding is denoted by $\Psi_{\calY}: \Ind\Coh(\calY) \to \QCoh(\calY)$. 
                \end{convention}
                
                \paragraph{Homological characterisations of ind-coherent sheaves}
                    \begin{theorem}[Ind-coherent modules on classical schemes] \label{theorem: indcoh_on_classically__regular_and_smooth_schemes}
                        If a $0$-coconnective scheme $X$ is locally almost of finite type\footnote{Recall that locally almost of finite type $0$-coconnective schemes are just locally Noetherian in the usual sense.} and regular then $\Psi_X$ is an equivalence. This implication is an equivalence if and only if $X$ is smooth.
                    \end{theorem}
                        \begin{proof}
                            It is known that whenever $X$ is Noetherian as a $0$-coconnective scheme, $\QCoh(X)$ is compactly generated by its (small) subcategory of perfect complexes, i.e. $\QCoh(X) \cong \Ind(\QCoh(X)^{\perf})$. It is also known that over $0$-coconnective regular schemes $X$, one has $\Coh(X) \cong \QCoh(X)^{\perf}$. Thus, if $X$ is a $0$-coconnective regular scheme then one has an equivalence $\Psi_X: \Ind\Coh(X) \cong \QCoh(X)$.
                            
                            \todo{Finish this up}
                        \end{proof}
                        
                    Next, recall that for any scheme $X$, $\Coh(X)$ carries a natural t-structure (cf. definition \ref{def: t_structures}), which is a sub-t-structure of the t-structure of $\QCoh(X)$. As it happens, this t-structure gets passed along to $\Ind\Coh(X)$ whenever $X$ is locally almost of finite type.
                    \begin{lemma}[t-structures of ind-coherent sheaves] \label{lemma: t_structure_of_ind_coherent_sheaves}
                        Let $X$ be a locally almost of finite type scheme. Then:
                            \begin{enumerate}
                                \item $\Ind\Coh(X)$ inherits a $t$-structure from $\Coh(X)$ whose accompanying truncation functors commute with filtered colimits. 
                                \item the canonical fully faithful embedding $\Coh(X) \subset \Ind\Coh(X)$ is $t$-exact.
                            \end{enumerate}
                    \end{lemma}
                        \begin{proof}
                                        
                        \end{proof}
                    \begin{proposition}[Truncations of ind-coherent sheaves] \label{prop: truncations_of_ind_coherent_sheaves}
                        Let $X$ be a locally almost of finite type scheme. Then for all $n \in \Z$, the truncated canonical embeddings:
                            $$\Psi_X^{\geq n}: \Ind\Coh(X)^{\geq n} \hookrightarrow \QCoh(X)^{\geq n}$$
                        are actually equivalences of categories.
                    \end{proposition}
                        \begin{proof}
                                        
                        \end{proof}
                    \begin{corollary}[Connectivity of ind-coherent sheaves] \label{coro: connectivity_of_ind_coherent_sheaves}
                        \noindent
                        \begin{enumerate}
                            \item An ind-coherent module of over a locally almost of finite type scheme is connective if and only if it is so as a quasi-coherent module.
                            \item For every locally almost of finite type scheme, one has an equivalence $\Ind\Coh(X)^{\perf} \cong \Coh(X)$.
                        \end{enumerate}
                    \end{corollary}
                    
                    \begin{proposition}[$\QCoh$ is the left-completion of $\Ind\Coh$] \label{prop: qcoh_is_the_left_completion_of_indcoh}
                        Over a Notherian scheme $X$, one recognises $\QCoh(X)$ as the left-completion of $\Ind\Coh(X)$ in its t-structure.
                    \end{proposition}
                        \begin{proof}
                            
                        \end{proof}
                    
                \paragraph{An action of \texorpdfstring{$\QCoh(X)$}{} on \texorpdfstring{$\Ind\Coh(X)$}{}}
                    It is not hard to see that for any prestack locally almost of finite type $\calY$, the category $\Ind\Coh(\calY)$ of ind-coherent sheaves over it has a natural structure of a dg-category. What this means is that $\Ind\Coh(\calY)$ is an object of the $1$-category $(\Cat^{\dg, \cont})^1$ of dg-categories and continuous functors between them. On this category, there exists a natural monoidal structure inherited from the $1$-category of stable $\infty$-categories, which is the Lurie tensor product $\boxtimes$. 
                
                    Now, one thing to note is that for each locally almost of finite type scheme $X$, the evident embedding $\Psi_X: \Ind\Coh(X) \hookrightarrow \QCoh(X)$ is a continuous functor. In particular, this means one can construct continuous functors $\QCoh(X) \x \Ind\Coh(X) \to \Ind\Coh(X)$, and hence the Lurie tensor product $\QCoh(X) \boxtimes \Ind\Coh(X)$, which satisfies the following universal property:
                        $$
                            \begin{tikzcd}
                            	{\QCoh(X) \boxtimes \Ind\Coh(X)} & \calA \\
                            	{\QCoh(X) \x \Ind\Coh(X)}
                            	\arrow[from=2-1, to=1-2]
                            	\arrow[dashed, from=1-1, to=1-2]
                            	\arrow["\boxtimes", from=2-1, to=1-1]
                            \end{tikzcd}
                        $$
                    (wherein, of course, $\calA$ is an arbitrary dg-category and the arrows are continuous functors). As a consequence, we can construct so-called \textbf{actions} of $\QCoh(X)$ on $\Ind\Coh(X)$, which are just continuous functors:
                        $$\alpha: \QCoh(X) \boxtimes \Ind\Coh(X) \to \Ind\Coh(X)$$
                    via continuous functors $\underline{\alpha}: \QCoh(X) \x \Ind\Coh(X) \to \Ind\Coh(X)$
                        
                    \begin{proposition}[The canonical action of $\QCoh$ on $\Ind\Coh$] \label{prop: canonical_action_of_qcoh_on_indcoh}
                        Let $X$ be a locally almost of finite type scheme. Then, there exists a continuous functor $\bar{\alpha}_X: \QCoh(X) \x \Ind\Coh(X) \to \Ind\Coh(X)$ rendering the following diagram commutative:
                            $$
                                \begin{tikzcd}
                                	{\QCoh(X) \x \Ind\Coh(X)} & {\Ind\Coh(X)} \\
                                	{\QCoh(X) \x \QCoh(X)} & {\QCoh(X)}
                                	\arrow["{\id \x \Psi_X}"', from=1-1, to=2-1]
                                	\arrow["\tensor", from=2-1, to=2-2]
                                	\arrow["{\bar{\alpha}_X}", from=1-1, to=1-2]
                                	\arrow["{\Psi_X}", from=1-2, to=2-2]
                                \end{tikzcd}
                            $$
                    \end{proposition}
                        \begin{proof}
                            
                        \end{proof}
                    \begin{corollary} \label{coro: canonical_action_of_qcoh_on_indcoh}
                        For each Notherian scheme $X$, there exists a canonically defined $\QCoh(X)$-action on $\Ind\Coh(X)$ coming from $\bar{\alpha}_X$ as above; this means that this action is given by:
                            $$\alpha_X(\calF \boxtimes \E) \cong \calF \tensor \Psi_X(\E)$$
                    \end{corollary}
                
                \paragraph{Ind-coherent sheaves over eventually coconnective schemes}
                    \begin{definition}[Eventually coconnective schemes] \label{def: eventually_coconnective_schemes}
                        A scheme which is locally almost of finite type is said to be \textbf{eventually coconnective} if and only if it admits a Zariski atlas by affine schemes almost of finite type.
                        
                        Equivalently - and perhaps more succinctly - a scheme $X$ is eventually coconnective if and only if $\calO_X \in \Coh(X)$.
                    \end{definition}
                    \begin{remark}
                        We note that being eventually coconnective implies being locally almost of finite type, since the latter notion is only that the structure sheaf is Zariski-locally coherent. In particular, this means that one can still consider ind-coherent sheaves (which are only defined over schemes locally of finite type) over eventually coconnective schemes.
                    \end{remark}
                    
                    \begin{proposition}[A canonical adjunction] \label{prop: canonical_adjunction} 
                        Let $X$ be an eventually coconnective scheme. Then, there exists an adjunction as follows, wherein $\Xi_X$ is fully faithful:
                            $$
                                \begin{tikzcd}
                                	{\Ind\Coh(X)} & {\QCoh(X)}
                                	\arrow[""{name=0, anchor=center, inner sep=0}, "{\Psi_X}"', shift right=2, from=1-1, to=1-2]
                                	\arrow[""{name=1, anchor=center, inner sep=0}, "{\Xi_X}"', shift right=2, hook', from=1-2, to=1-1]
                                	\arrow["\dashv"{anchor=center, rotate=-90}, draw=none, from=1, to=0]
                                \end{tikzcd}
                            $$
                    \end{proposition}
                        \begin{proof}
                            
                        \end{proof}
                
            \subsubsection{Ind-coherent sheaves over schemes not locally of finite type}
            
        \subsection{Basic functoriality via \texorpdfstring{$*$}{}-pullbacks and \texorpdfstring{$*$}{}-pushforwards}
            The upshot of this subsection is that ind-coherent sheaves, via pullback and pushforward functors enjoy a certain flavour of functoriality. In subsection \ref{subsection: categories_of_ind_coherent_sheaves}, we have already studied the objects of these to-be categories, so now, let us figure out what the ($1$-)morphisms ought to be. In fact, the resulting category $\Ind\Coh$ will turn out to be fibred over $\Sch^{\aft}$, and moreover inherits many important properties from $\QCoh$.
        
            \begin{convention}
                We continue to work over schemes almost of finite type (read: Noetherian) throughout this subsection.
            \end{convention}
            
            \subsubsection{\texorpdfstring{$*$}{}-pushforwards}
                
                \begin{proposition}[Existence of $*$-pushforwards] \label{prop: indcoh_*_pushforwards}
                    Let $f: X \to Y$ be any morphism in $\Sch^{\aft}$. Then, there exists a corresponding uniquely defined t-exact functor:
                        $$f_*|_{\Ind\Coh}: \Ind\Coh(X) \to \Ind\Coh(Y)$$
                    fitting into the following commutative diagram in $(\Cat^{\dg, \cont})^1$:
                        $$
                            \begin{tikzcd}
                            	{\Ind\Coh(X)} & {\QCoh(X)} \\
                            	{\Ind\Coh(Y)} & {\QCoh(X)}
                            	\arrow["{\Psi_X}", from=1-1, to=1-2]
                            	\arrow["{f_*|_{\Ind\Coh}}"', from=1-1, to=2-1]
                            	\arrow["{f_*}", from=1-2, to=2-2]
                            	\arrow["{\Psi_Y}", from=2-1, to=2-2]
                            \end{tikzcd}
                        $$
                \end{proposition}
                    \begin{proof}
                        
                    \end{proof}
                \begin{corollary}[Compatibility of $*$-pushforwards and $\QCoh$-actions] \label{coro: *_pushforwards_and_qcoh_actions}
                    Let $f: X \to Y$ be any morphism in $\Sch^{\aft}$. Then, via the pushforward functor $f_*|_{\Ind\Coh}: \Ind\Coh(X) \to \Ind\Coh(Y)$, one obtains a natural $\QCoh(Y)$-action on $\Ind\Coh(X)$ that is compatible with the $\QCoh(Y)$-action on $\QCoh(X)$ induced by $f_*: \QCoh(X) \to \QCoh(Y)$.
                \end{corollary}
                
                Now, to establish a proper functoriality for ind-coherent sheaves (cf. theorem \ref{theorem: indcoh_functoriality}), we will need to go on a technical detour.
                \begin{convention}
                    \noindent
                    \begin{itemize}
                        \item Let $(\Cat^{\dg, \cont, \co\compl}_{\tstructure^{\pm}, \access(\geq 0), \comp\gen(+)})^1$ be the $1$-full subcategory of $(\Cat^{\dg, \cont})^1$ wherein:
                            \begin{itemize}
                                \item objects are \textit{cocomplete} dg-categories $\calA$ endowed with t-structures such that $\calA$ is compactly generated by $\calA^+$ and that $\calA^{\geq 0}$ is accessible, and
                                \item $1$-morphisms are \textit{continuous} functors $F: \calA \to \calB$ between such non-cocomplete dg-categories which are left-t-exact up a finite shift and such that the restriction $F|_{\calA^{\geq 0}}$ is accessible. 
                            \end{itemize}
                        \item Let $(\Cat^{\dg, \non\co\compl}_{\tstructure^+, \access(\geq 0)})^1$ be the $1$-full subcategory of $(\Cat^{\dg})^1$ wherein:
                            \begin{itemize}
                                \item objects are \textit{non-cocomplete} dg-categories $\calA$ endowed with t-structures such that:
                                    $$\calA \cong \calA^+$$
                                (i.e. the objects of $\calA$ are all eventually coconnective) and that $\calA^{\geq 0}$ is accessible, and
                                \item $1$-morphisms are functors $F: \calA \to \calB$ between such non-cocomplete dg-categories which are left-t-exact up a finite shift and such that the restriction $F|_{\calA^{\geq 0}}$ is accessible. 
                            \end{itemize}
                    \end{itemize}
                \end{convention}
                \begin{lemma} \label{lemma: selecting_a_compact_generator}
                    There exists a natural $1$-fully faithful embedding:
                        $$(\Cat^{\dg, \cont, \co\compl}_{\tstructure^{\pm}, \access(\geq 0), \comp\gen(+)})^1 \to (\Cat^{\dg, \non\co\compl}_{\tstructure^+, \access(\geq 0)})^1$$
                    defined via the assignment:
                        $$\calA \mapsto \calA^+$$
                \end{lemma}
                    \begin{proof}
                        Denote the assignment in question by $(-)^+$. Then, observe that for every object $\calA \in (\Cat^{\dg, \cont, \co\compl}_{\tstructure^{\pm}, \access(\geq 0), \comp\gen(+)})^1$, the corresponding object $\calA^+ \in (\Cat^{\dg, \non\co\compl}_{\tstructure^+, \access(\geq 0)})^1$ compactly generates $\calA$ by the very construction of $(\Cat^{\dg, \cont, \co\compl}_{\tstructure^{\pm}, \access(\geq 0), \comp\gen(+)})^1$. Also, note that any $\calA' \in (\Cat^{\dg, \non\co\compl}_{\tstructure^+, \access(\geq 0)})^1$ satisfies $\calA' \cong (\calA')^+$ by definition. By putting the two observations together one sees that there is a natural equivalence:
                            $$\Maps(\calA, \calB) \cong \Maps(\calA^+, \calB^+)$$
                        which implies that indeed, there exists a natural $1$-fully faithful embedding:
                            $$(\Cat^{\dg, \cont, \co\compl}_{\tstructure^{\pm}, \access(\geq 0), \comp\gen(+)})^1 \to (\Cat^{\dg, \non\co\compl}_{\tstructure^+, \access(\geq 0)})^1$$
                        defined via the assignment:
                            $$\calA \mapsto \calA^+$$
                    \end{proof}
                
                \begin{theorem}[Functoriality of $\Ind\Coh$] \label{theorem: indcoh_functoriality}
                    There exists a functor:
                        $$\Ind\Coh_*|_{\Sch^{\aft}}: \Sch^{\aft} \to (\Cat^{\dg, \cont})^1$$
                    along with a natural tranformation:
                        $$\Psi|_{\Sch^{\aft}}: \Ind\Coh_*|_{\Sch^{\aft}} \to \QCoh_*|_{\Sch^{\aft}}$$
                    which at the level of components, associates to each morphism $f: X \to Y$ in $\Sch^{\aft}$ a commutative diagram as follows:
                        $$
                            \begin{tikzcd}
                            	{\Ind\Coh(X)} & {\QCoh(X)} \\
                            	{\Ind\Coh(Y)} & {\QCoh(X)}
                            	\arrow["{\Psi_X}", from=1-1, to=1-2]
                            	\arrow["{f_*|_{\Ind\Coh}}"', from=1-1, to=2-1]
                            	\arrow["{f_*}", from=1-2, to=2-2]
                            	\arrow["{\Psi_Y}", from=2-1, to=2-2]
                            \end{tikzcd}
                        $$
                \end{theorem}
                    \begin{proof}
                        
                    \end{proof}
        
            \subsubsection{\texorpdfstring{$*$}{}-pullbacks}
        
        \subsection{\texorpdfstring{$\Ind\Coh$}{} as a functor out of the category of correspondences}
            \begin{convention}
                Since everything is derived (cf. convention \ref{conv: schemes_2_everything_is_derived}), by \say{$n$-category} we will actually mean \say{$(\infty, n)$-category}.
            \end{convention}
                
            \subsubsection{Ind-coherent sheaves via correspondences}
        
            \subsubsection{!-pullbacks and Grothendieck duality}
	    
	    \chapter{Dimension theory}
    \begin{abstract}
        
    \end{abstract}
    
    \minitoc

    \section{Fundamentals of dimension theory}
        \subsection{Polynomial rings and Hilbert polynomials}
            \subsubsection{Polynomial rings}
                \begin{proposition}[Krull dimensions of polynomial rings] \label{prop: dimensions_of_polynomial_rings}
                    Let $R$ be a Noetherian commutative ring and let $n$ be a natural number. Then:
                        $$\dim_{\Krull} R[x_1, ..., x_n] = \dim_{\Krull} R + n = \dim_{\Krull} R + {}_R\length R[x_1, ..., x_n]$$
                    for all $x_1, ..., x_n$ transcendental over $R$.
                \end{proposition}
                    \begin{proof}
                        
                    \end{proof}
                
            \subsubsection{Hilbert polynomials and their degrees}
        
        \subsection{Cases of dimension \texorpdfstring{$0$}{}}
        
        \subsection{Valutaion rings and (co)dimension \texorpdfstring{$1$}{}}
    
    \section{Depths and Cohen-Macaulay rings}
    
    \section{Elimination theory and dimensions of fibres}
	    
	    \chapter{Smoothness} \label{chapter: smoothness}
    \begin{abstract}
        
    \end{abstract}
    
    \minitoc
    
    \section{Differentials and smoothness}
        \subsection{The cotangent complex formalism}
            \subsubsection{K\"ahler differentials, reprised} \label{subsubsection: kahler_differentials}
                \paragraph{As objects generated by derivations}
                    \begin{definition}[(Pre)derivations and Leibniz algebras] \label{def: derivations}
                        Let $k$ be a ring (which need not be commutative) and let $(\O, \tensor, 1)$ be a $k$-linear monoidal category. 
                            \begin{enumerate}
                                \item \textbf{(Prederivations):} A \textbf{left/right/two-sided-prederivation} on an (not necessarily associative, commutative, nor unital) algebra $\left(\g, \nabla\right)$ internal to $\O$ is an endomorphism:
                                    $$D: \g \to \g$$
                                thereon that turns the triple $\left(\g, \nabla, D\right)$ into an \textbf{additive left/right/two-sided-Leibniz algebra}. That is to say, we require the following diagram to commute in $\O$:
                                    $$
                                        \begin{tikzcd}
                                        	{\g \tensor \g} & {\g} \\
                                        	{\g \tensor \g} & {\g}
                                        	\arrow["{D}", from=1-2, to=2-2]
                                        	\arrow["{\nabla}", from=2-1, to=2-2]
                                        	\arrow["{\nabla}", from=1-1, to=1-2]
                                        	\arrow["{D \tensor \id_{\g} + \id_{\g} \tensor D}"', from=1-1, to=2-1]
                                        \end{tikzcd}
                                    $$
                                \item \textbf{(Derivations):} If $\g$ also happens to be a unital algebra (with unit map $\eta: 1 \to \g$), then we require that the following diagram commutes:
                                    $$
                                        \begin{tikzcd}
                                        	{1} & {\g} \\
                                        	& {\g}
                                        	\arrow["{D}", from=1-2, to=2-2]
                                        	\arrow["{\eta}", from=1-1, to=1-2]
                                        	\arrow["{0}"', from=1-1, to=2-2]
                                        \end{tikzcd}
                                    $$
                                whererin $0$ is understood to be the additive identity in the $k$-module $\O(1, \g)$. In this situation, we call the quadruple $(\g, \nabla, D, \eta)$ a \textbf{linear Leibniz algebra}, and specifically, the prederivation $D$ will be referred to simply as a \textbf{derivation}.
                            \end{enumerate}
                    \end{definition}
                    \begin{remark}
                        If there exists a notion of elements inside objects of $\O$ (should $\O$ be the linear monoidal category of modules over a ring, for example), then the information that the diagrams in definition \ref{def: derivations} carry can also be understood as the following statement:
                            $$\forall x, y \in \g: D \nabla(x \tensor y) = \nabla(Dx \tensor y) + \nabla(x \tensor Dy)$$
                        and again, if $\g$ is unital, then:
                            $$\forall a \in 1: D(a) = 0$$
                        which are just conditions imposed upon derivations in more familiar definitions (see, for instance, \cite[\href{https://stacks.math.columbia.edu/tag/00RN}{Tag 00RN}]{stacks}). 
                    \end{remark}
                    
                    \begin{proposition}[Categories of Leibniz algebras] \label{prop: leibniz_algebra_categories}
                        Let $k$ be a ring and let $(\O, \tensor, 1)$ be a monoidal $k$-linear category. Additive and linear Leibniz algebras internal to $\O$ form subcategories of $\Alg(\O)$, which we shall repsectively denote by $\Leib\Alg(\O)$ and $\Assoc\Leib\Alg(\O)$. Furthermore, one has the following diagram of \textit{fully faithful} embeddings of categories:
                            $$
                                \begin{tikzcd}
                                	{\Alg(\O)} & {\Assoc\Alg(\O)} \\
                                	{\Leib\Alg(\O)} & {\Assoc\Leib\Alg(\O)}
                                	\arrow[hook', from=2-1, to=1-1]
                                	\arrow[hook', from=2-2, to=1-2]
                                	\arrow[hook, from=2-1, to=2-2]
                                	\arrow[hook, from=1-1, to=1-2]
                                \end{tikzcd}
                            $$
                    \end{proposition}
                        \begin{proof}
                            \noindent
                            \begin{enumerate}
                                \item \textbf{(Additive Leibniz algebras):} Let $(\g, \nabla, D)$ and $(\g', \nabla', D')$ be two additive Leibniz algebras. Then, let us declare that a morphism of additive Leibniz algebras internal to $\O$ is an algebra homomorphism $\phi: \g \to \g'$ (i.e. a morphism satisfying $\phi \circ \nabla = \nabla\ \circ (\phi \tensor \phi)$) such that:
                                    $$\phi \circ D = D' \circ \phi$$
                                It will suffice to show that the following diagram commutes:
                                    $$
                                        \begin{tikzcd}
                                        	& {\g' \tensor \g'} & {\g'} \\
                                        	& {\g' \tensor \g'} & {\g'} \\
                                        	{\g \tensor \g} & {\g} \\
                                        	{\g \tensor \g} & {\g}
                                        	\arrow["{\phi \tensor \phi}", from=3-1, to=1-2]
                                        	\arrow["{\phi}", from=3-2, to=1-3]
                                        	\arrow["{\phi}", from=4-2, to=2-3]
                                        	\arrow["{D'}", from=1-3, to=2-3]
                                        	\arrow["{\nabla'}", from=1-2, to=1-3]
                                        	\arrow["{D' \tensor \id_{\g'} + \id_{\g'} \tensor D'}"', from=1-2, to=2-2]
                                        	\arrow["{\nabla'}", from=2-2, to=2-3]
                                        	\arrow["{D}", from=3-2, to=4-2]
                                        	\arrow["{D \tensor \id_{\g} + \id_{\g} \tensor D}"', from=3-1, to=4-1]
                                        	\arrow["{\nabla}"', from=4-1, to=4-2]
                                        	\arrow["{\nabla}"', from=3-1, to=3-2]
                                        	\arrow["{\phi \tensor \phi}", from=4-1, to=2-2]
                                        \end{tikzcd}
                                    $$
                                if we are simply trying to show that additive Leibniz algebras form a full subcategory of $\O$. To that end, consider the following:
                                    $$
                                        \begin{aligned}
                                            \phi \circ D \circ \nabla & = \phi \circ \nabla \circ \left(D \tensor \id_{\g} + \id_{\g} \tensor D\right)
                                            \\
                                            & = \nabla' \circ (\phi \tensor \phi) \circ \left(D \tensor \id_{\g} + \id_{\g} \tensor D\right)
                                            \\
                                            & = \nabla' \circ \left((\phi \circ D) \tensor \phi + \phi \tensor (\phi \circ D)\right)
                                            \\
                                            & = \nabla' \circ \left((D' \circ \phi) \tensor \phi + \phi \tensor (D' \circ \phi)\right)
                                            \\
                                            & = \nabla' \circ \left(D' \tensor \id_{\g'} + \id_{\g'} \tensor D'\right) \circ (\phi \tensor \phi)
                                            \\
                                            & = D' \circ \nabla' \circ (\phi \tensor \phi)
                                        \end{aligned}
                                    $$
                                \item \textbf{(Linear Leibniz algebras):} Because linear Leibniz algebras are additive Leibniz algebras that also happen to be unital, it will be enough to prove that the following diagram, wherein $\eta$ and $\eta'$ are the units and $\phi$ is once again just some algebra homomorphism (note that every algebra homomorphism respects units \textit{a priori}), commutes:
                                    $$
                                        \begin{tikzcd}
                                        	&& {1} & {\g'} \\
                                        	{1} & {\g} && {\g'} \\
                                        	& {\g}
                                        	\arrow["{\phi}", from=2-2, to=1-4]
                                        	\arrow["{\phi}", from=3-2, to=2-4]
                                        	\arrow["{\eta}" description, from=2-1, to=2-2]
                                        	\arrow["{D}", from=2-2, to=3-2]
                                        	\arrow["{D'}", from=1-4, to=2-4]
                                        	\arrow["{\eta'}" description, from=1-3, to=1-4]
                                        	\arrow[Rightarrow, from=2-1, to=1-3, no head]
                                        	\arrow["{0}"', from=2-1, to=3-2]
                                        	\arrow["{0}"', from=1-3, to=2-4]
                                        \end{tikzcd}
                                    $$
                                To that end, consider the following:
                                    $$
                                        \begin{aligned}
                                            D' \circ \phi \circ \eta & = D' \circ \eta'
                                            \\
                                            & = D' \circ 0
                                            \\
                                            & = 0
                                            \\
                                            & = \phi \circ 0
                                            \\
                                            & = \phi \circ D \circ \eta
                                        \end{aligned}
                                    $$
                                By matching the terms in these equations with composition of arrows in the preceding two diagrams, we can see that the diagrams indeed commute.
                            \end{enumerate}
                        \end{proof}
                    \begin{example}[Examples of derivations and Leibniz algebras] \label{example: derivations}
                        \noindent
                        \begin{enumerate}
                            \item \textbf{(The adjoint representation):} Let $k$ be a ring and let $\left(\g, [-,-]\right)$ be a Lie algebra internal to the braided symmetric monoidal $k$-linear category ${}_k\Mod_k$ of two-sided $k$-modules. Then, notice that for all $x \in \g$, the adjoint operator $\ad(x) := [x,-]$ is a left-derivation and thus turns every triple $\left(\g, [-,-], \ad(x)\right)$ into an additive left-Leibniz algebra.
                            \item \textbf{(K\"ahler differentials and a \textit{raison d'\^etre}):} Let $A$ be a \textit{commutative} ring. Then, the linear Leibniz algebras on monoid objects $B$ of the $A$-linear symmetric monoidal category ${}_A\Mod$ are just modules of K\"ahler differentials on $B$ over $A$ (note the plurality of "modules"; we will prove that there is a universal one in theorem \ref{theorem: kahler_differentials_universal_property}). This might seem a bit strange, because usually, one would think of modules of K\"ahler differentials as mere modules instead of algebras equipped with a derivation. However, if we would recall that the module of K\"ahler differentials $\Omega^1_{B/A}$ defined with respect to some ring map $\phi: A \to B$ and some $A$-linear derivation $d$ on $B$ is just the one generated by the symbols $db$, which are subjected to the following relations:
                                $$\forall a \in A: d\left(\phi(a)\right) = 0$$
                                $$\forall (b, b') \in B \x B: d(b + b') = db + db'$$
                                $$\forall (b, b') \in B \x B: d(bb') = db \cdot b' + b \cdot db'$$
                            (see \cite[\href{https://stacks.math.columbia.edu/tag/00RM}{Tag 00RM}]{stacks} for more details), then we would see that every module of K\"ahler differentials $\Omega^1_{B/A}$ comes equipped with an $A$-linear map satisfying the above specifications from $B$:
                                $$d: B \to \Omega^1_{B/A}$$
                            By simply evaluating the formal symbols $db$ at elements of $B$, one would recover the notion of derivations as endomorphisms. Thus, commutative linear Leibniz algebras are the same thing as modules of K\"ahler differentials. 
                            
                            We shall package these observations neatly into the content of theorem \ref{theorem: kahler_differentials_universal_property}. 
                        \end{enumerate}
                    \end{example}
                    \begin{remark}[Commutative Leibniz algebras] \label{remark: commutative_leibniz_algebras}
                        Let $k$ be a ring and let $(\O, \tensor, 1)$ be a monoidal $k$-linear category. Then inside $\Assoc\Leib\Alg(\O)$, there lies a full subcategory $\Comm\Alg(\O)$ spanned by Leibniz algebras whose underlying algebra object is associative, unital, and furthermore, commutative.
                    \end{remark}
                    
                    We shall now state our main theorem, which establishes the existence and accompanying universal property of K\"ahler differentials \textit{internally}.
                    \begin{theorem}[\textcolor{red}{\underline{\textbf{IMPORTANT}}} Modules of K\"ahler differentials are free commutative Leibniz algebras] \label{theorem: kahler_differentials_universal_property}
                        Let $k$ be a ring and let $(\calA, \tensor, 1)$ be a two-sided $k$-linear monoidal category (i.e. a category enriched over the symmetric monoidal category ${}_k\Mod_k$ of two-sided $k$-modules) that is \textit{pre-abelian} (this is to ensure that kernels exist). Then, (two-sided) $k$-linear K\"ahler differentials arise naturally from the following forgetful-free adjunction:
                            $$
                                \begin{tikzcd}
                                	{\Comm\Leib\Alg(\calA)} & {\Comm\Alg(\calA)}
                                	\arrow[""{name=0, anchor=center, inner sep=0}, "\oblv"', shift right=2, hook, from=1-1, to=1-2]
                                	\arrow[""{name=1, anchor=center, inner sep=0}, "{\Omega^1}"', shift right=2, from=1-2, to=1-1]
                                	\arrow["\dashv"{anchor=center, rotate=-90}, draw=none, from=1, to=0]
                                \end{tikzcd}
                            $$
                        wherein:
                            \begin{enumerate}
                                \item $\oblv$ is the obvious forgetful functor, which also happens to be the obvious fully faithful embedding, and
                                \item $\Omega^1$ is the functor associating to commutative algebras $B$ the canonical derivation:
                                    $$d: B \to \calI_{B/A}/\calI_{B/A}^2$$
                                wherein $\calI_{B/A}$ is the kernel of the codiagonal/multiplication $\nabla_{B/A}: B \tensor B \to B$, which exists thanks to the assumption that $\calA$ is pre-abelian.
                            \end{enumerate}
                    \end{theorem}
                        \begin{proof}
                            \noindent
                            \begin{enumerate}
                                \item \textbf{(Unit):} 
                                \item \textbf{(Counit):} 
                            \end{enumerate}
                        \end{proof}
                    \begin{corollary}[Universal property of K\"ahler differentials] \label{coro: kahler_differential_universal_property}
                        Let $\calA$ be the (pre-)abelian linear symmetric monoidal category ${}_A\Mod$ of modules over some commutative ring $A$. Commutative algebras therein are just commutative $A$-algebras, and so for each such algebra:
                            $$\phi: A \to B$$
                        one gets a module of K\"ahler differential:
                            $$d: B \to \Omega^1_{B/A}$$
                        satisfying:
                            $$\forall a \in A: d\left(\phi(a)\right) = 0$$
                            $$\forall (b, b') \in B \x B: d(b + b') = db + db'$$
                            $$\forall (b, b') \in B \x B: d(bb') = db \cdot b' + b \cdot db'$$
                        Then, by general properties of adjoint functors, one can show that the above canonical choice of module of K\"ahler differentials given by the functor $\Omega^1: {}^{A/}\Comm\Alg \to {}^{A/}\Comm\Leib\Alg$ is actually universal, in the sense that it is initial in the category of all $A$-module homormophisms:
                            $$\delta: B \to M$$
                        satisfying:
                            $$\forall a \in A: \delta\left(\phi(a)\right) = 0$$
                            $$\forall (b, b') \in B \x B: \delta(b + b') = \delta b + \delta b'$$
                            $$\forall (b, b') \in B \x B: \delta(bb') = \delta b \cdot b' + b \cdot \delta b'$$
                        In other words, given any $A$-linear derivation $\delta: B \to M$ on $B$, there is the following unique factorisation:
                            $$
                                \begin{tikzcd}
                                	 & B \\
                                	{\Omega^1_{B/A}} && M
                                	\arrow["\delta", from=1-2, to=2-3]
                                	\arrow["d"', from=1-2, to=2-1]
                                	\arrow[dashed, from=2-1, to=2-3]
                                \end{tikzcd}
                            $$
                    \end{corollary}
                
            \paragraph{As square-zero extensions}    
                We will need to concern ourselves with (yet another) reprisal of the theory of K\"ahler differentials; this time, in the style of \cite[Sections 7.3 and 7.4]{HA}.
                    
                    \begin{definition}[The square-zero extension functor: existence] \label{def: the_square_zero_extension_functor_existence}
                        Let $\calA$ be a symmetric monoidal stable $\infty$-category (cf. definition \ref{def: stable_infinity_categories}) and let:
                            $$\oblv: \Comm\Alg(\calA) \to \calA$$
                        be the forgetful functor that ignores the multiplication on commutative $\calA$-algebras. One can then show, using a very routine and formal argument, that this forgetful functor preserves limits and (small) filtered colimits, and because $\calA$ is \textit{a priori} presentable, it therefore admits a left-adjoint, which we shall denote by:
                            $$\Split\SqZ: \calA \to \Comm\Alg(\calA)$$
                    \end{definition}
                    \begin{proposition}[The square-zero extension functor: explicit description] \label{prop: the_square_zero_extension_functor_explicit_description}
                        Let $(\calA, \tensor, \1)$ be a \textit{closed} symmetric monoidal stable $\infty$-category and let:
                            $$\oblv: \Comm\Alg(\calA) \to \calA$$
                        be the forgetful functor that forgets the multiplication on commutative $\calA$-algebras. Its left-adjoint:
                            $$\Split\SqZ: \calA \to \Comm\Alg(\calA)$$
                        is thus given by:
                            $$\g \mapsto \1 \oplus \g$$
                    \end{proposition}
                        \begin{proof}
                            First of all, we need to show that the object $\1 \oplus \g$, for any $\g \in \calA^{\leq 0}$, can be endowed with the structure of a commutative monoid. This is rather routine, however, so we will leave it up to our dear readers (simply check the relevant commutative diagrams defining a commutative monoid internal to a symmetric monoidal category). One thing we will note, however, that thanks to finite direct sums being the same as finite biproducts in abelian categories, there necessarily exists a canonical unit map $\1 \to \1 \oplus \g$; also, the monoidal closure assumption on $(\calA, \tensor, \1)$ guarantees that for all $M \in \calA$, the functor $M \tensor -$ is a left-adjoint and thus preserves colimits.
                            
                            Next, we can use the fact $\calA$ and $\Comm\Alg(\calA)$ are presentable categories (cf. definition \ref{def: presentable_infinity_categories}), in conjunction with the fact that $\1 \oplus \g$ is a coproduct (so in particular, a colimit) to show the functor $\1 \oplus -: \calA \to \Comm\Alg(\calA)$ must preserve colimits in $\calA$ without fail. From this, we can deduce that the functor $\1 \oplus -$ (i.e. $\Split\SqZ$) admits a right-adjoint.
                            
                            It now remains to show that the right-adjoint of $\1 \oplus -: \calA \to \Comm\Alg(\calA)$ is indeed the canonical forgetful functor $\oblv: \Comm\Alg(\calA) \to \calA$ omitting the algebra structures and returning the underlying object. For this, simply note that there exists the following unit and counit componenets:
                                $$\eta_{\g}: \g \to ( \oblv \circ (\1 \oplus -) )(\g)$$
                                $$\e_A: ( (\1 \oplus -) \circ \oblv )(A) \to A$$
                            for all $\g \in \calA$ and $A \in \Comm\Alg(\calA)$; from the existence of these maps, we obtain the unit-counit pair that defines $\1 \oplus -$ as the left-adjoint of $\oblv$. This concludeds the proof that $\Split\SqZ(\g)$ is given by $\1 \oplus \g$ for all $\g \in \calA$.
                        \end{proof}
                    \begin{corollary}[Internal abelian groups] \label{coro: internal_abelian_groups}
                        It is not hard to deduce from the proof of proposition \ref{prop: the_square_zero_extension_functor_explicit_description} that for $(\calA, \tensor, \1)$ any closed symmetric monoidal stable $\infty$-category, the essential image of the split square-zero extension functor:
                            $$\Split\SqZ: \calA \to \Comm\Alg(\calA)$$
                        is equivalent to the category $\Comm\Mon( {}^{\1/}\Comm\Alg(\calA)_{/\1} )$ of commutative (additive) monoids internal to the double-slice ${}^{\1/}\Comm\Alg(\calA)_{/\1}$ (which is actually equivalent to $\Comm\Alg(\calA)_{/\1}$, since every unital algebra $A$, by definition, comes equipped with a unit map $\1 \to A$). In other words, one might think of objects of $\calA$ as abelian groups internal to $\Comm\Alg(\calA)_{/\1}$; this also means that $\Split\SqZ$ embeds $\calA$ fully faithfully into $\Comm\Alg(\calA)_{/\1}$.
                    \end{corollary}
                        \begin{proof}
                            This is simply a matter of restricting the codomain category $\Comm\Alg(\calA)$ until the counit $\e: \Split\SqZ \circ \oblv \to \id$ of the adjunction $\Split\SqZ \ladjoint \oblv$ becomes a natural isomorphism (and also, using the fact that finite direct sums are finite biproducts).
                        \end{proof}
                    \begin{remark}[$\Split\SqZ$ preserves limits and colimits] \label{remark: square_zero_extension_functor_preserves_(co)limits}
                        It is not hard to see that for any closed symmetric monoidal stable $\infty$-category $\calA$, the corresponding evident functor of split square-zero extensions:
                            $$\Split\SqZ: \calA \to \Comm\Alg(\calA)$$
                        preserves both limits and colimits.
                    \end{remark}
                    
                    \begin{lemma}[Cotangent bundles] \label{lemma: cotangent_bundles} \index{Cotangent bundle}
                        Let $(\calA, \tensor, \1)$ be a \textit{closed} symmetric monoidal stable $\infty$-category and let:
                            $$\Split\SqZ: \calA \to \Comm\Alg(\calA)$$
                        be the corresponding evident functor of split square-zero extensions.
                            \begin{enumerate}
                                \item \textbf{(Existence and uniqueness):} This functor admits a left-adjoint of its own, which we shall denote by $\tangent^{\vee}$ and refer to as the \textbf{cotangent bundle functor}.
                                \item \textbf{(Explicit description):} One obtains via corollary \ref{coro: internal_abelian_groups} the following localisation of $\Comm\Alg(\calA)_{/\1}$:
                                    $$
                                        \begin{tikzcd}
                                        	\calA & {\Comm\Alg(\calA)_{/\1}}
                                        	\arrow[""{name=0, anchor=center, inner sep=0}, "\Split\SqZ_{/\1}"', shift right=2, hook, from=1-1, to=1-2]
                                        	\arrow[""{name=1, anchor=center, inner sep=0}, "{\tangent^{\vee}_{/\1}}"', shift right=2, from=1-2, to=1-1]
                                        	\arrow["\dashv"{anchor=center, rotate=-90}, draw=none, from=1, to=0]
                                        \end{tikzcd}
                                    $$
                                Then, one can show that the essential image of $\tangent^{\vee}_{/\1}$ inside $\calA$ is equivalent to the subcategory $\Comm\Leib\Alg(\calA)$ spanned by commutative Leibniz algebras (cf. proposition \ref{prop: leibniz_algebra_categories}) therein.
                            \end{enumerate}
                    \end{lemma}
                        \begin{proof}
                            \noindent
                            \begin{enumerate}
                                \item \textbf{(Existence and uniqueness):} Because both $\calA$ and $\Comm\Alg(\calA)$ are \textit{a priori} presentable and because $\Split\SqZ$ preserves all limit and colimits, the left-adjoint $\tangent^{\vee}$ exists naturally by the Special Adjoint Functor Theorem for presentable $\infty$-categories.
                                \item \textbf{(Explicit description):} Because $\Split\SqZ_{/\1}$ is already fully faithful, (which in particular, implies that the counit of the adjunction $\Split\SqZ_{/\1} \ladjoint \tangent^{\vee}_{/\1}$ is a natural isomorphism) and because we have an isomorphism of simplicial sets:
                                    $$\calA(\tangent^{\vee}_{/\1}(S), N) \cong \Comm\Alg(\calA)_{/\1}(S, \1 \oplus N)$$
                                for all $S \in \Comm\Alg(\calA)_{/\1}$ and $N \in \calA$, it will suffice to show that any arrow $\varphi \in \Comm\Alg(\calA)_{/\1}(S, \1 \oplus N)$ induces a derivation (cf. definition \ref{def: derivations}) on $\1 \oplus N$ (it is \textit{very important} to consider $\varphi$ as an arrow in the slice category $\Comm\Alg(\calA)_{/\1}$ instead of in $\Comm\Alg(\calA)$). We shall do this by checking that we actually have all the relevant commutative diagrams, as in definition \ref{def: derivations}. To that end, 
                            \end{enumerate}
                        \end{proof}
                    \begin{remark}[Cotangent bundles and K\"ahler differentials] \label{remark: cotangent_bundles_and_kahler_differentials}
                        Recall, from theorem \ref{theorem: kahler_differentials_universal_property}, that K\"ahler differentials taking values in some symmetric monoidal stable $\infty$-category $(\calA, \tensor, \1)$ arise through an adjunction:
                            $$
                                \begin{tikzcd}
                                	{\Comm\Leib\Alg(\calA)} & {\Comm\Alg(\calA)}
                                	\arrow[""{name=0, anchor=center, inner sep=0}, "\oblv"', shift right=2, hook, from=1-1, to=1-2]
                                	\arrow[""{name=1, anchor=center, inner sep=0}, "{\Omega^1_{-/\1}}"', shift right=2, from=1-2, to=1-1]
                                	\arrow["\dashv"{anchor=center, rotate=-90}, draw=none, from=1, to=0]
                                \end{tikzcd}
                            $$
                        wherein the canonical forgetful functor $\oblv: \Comm\Leib\Alg(\calA) \to \Comm\Alg(\calA)$ is fully faithful (cf. proposition \ref{prop: leibniz_algebra_categories}). When $(\calA, \tensor, \1)$ is furthermore monoidally closed, we get additionally by lemma \ref{lemma: cotangent_bundles} that $\Comm\Leib\Alg(\calA)$ is equivalent to the essential image of the cotangent bundle functor $\tangent^{\vee}_{/\1}: \Comm\Alg(\calA)_{/\1} \to \calA$. Due to this, we can also view $\tangent^{\vee}_{\1}$ as the functor assigning to each $A \in \Comm\Alg(\calA)_{/\1}$ the module of K\"ahler differentials $\Omega^1_{-/\1}$.
                    \end{remark}
                
            \subsubsection{Properties of modules of K\"ahler differentials} \label{subsubsection: properties_of_kahler_differentials}
                \begin{remark}[K\"ahler differentials and colimits] \label{remark: differentials_and_colimits}
                    This could be viewed as a corollary to theorem \ref{theorem: kahler_differentials_universal_property}. 
                    
                    Let $R$ be a base commutative ring and let $\{S_i\}_{i \in I}$ be a diagram of $R$-algebras $S_i$. Then, due to $\Omega^1_{-/R}$ being a left-adjoint (which means, in particular, that it would preserve colimits \textit{a priori}), one has the following identity:
                        $$\Omega^1_{\underset{i \in I}{\colim} S_i/R} \cong \underset{i \in I}{\colim} \Omega^1_{S_i/R}$$
                \end{remark}
                \begin{example}[K\"ahler differentials and localisations] \label{example: differentials_and_localisations}
                    An example of a colimit of an infinite diagram of modules of K\"ahler differentials is how these modules interact with localisations of commutative rings. Let $S$ be a (possibly infinite) commutative ring let $\q \in |\Spec S|$ be a prime ideal thereof, and let $R \to S$ be a ring map. Then:
                        $$\Omega^1_{S_{\q}/R} \cong (\Omega^1_{S/R})_{\q}$$
                    Note that this exhibits the commutativity of $\Omega^1_{-/R}$ with an \textit{infinite} colimit because:
                        $$S_{\q} \cong \underset{y \in S \setminus \q}{\colim} S[1/y]$$
                \end{example}
                Let us examine how K\"ahler differentials interact with colimits a bit closer through the following proposition, wherein we rely on the fact that finite colimits can be constructed out of finite coproducts and epimorphisms.
                \begin{proposition}[K\"ahler differentials and finite colimits] \label{prop: differentials_and_finite_colimits}
                    Let $R$ be a base commutative ring. 
                        \begin{enumerate}
                            \item \textbf{(Module of differentials of a surjection):} If $R \to S$ is a surjective ring homomorphism, then:
                                $$\Omega^1_{S/R} \cong 0$$
                            \item \textbf{(Module of differentials and base change):} Consider a pushout diagram of commutative rings such as the following one:
                                $$
                                    \begin{tikzcd}
                                    	{S'} & {R'} \\
                                    	S & R
                                    	\arrow[from=2-2, to=2-1]
                                    	\arrow[from=2-1, to=1-1]
                                    	\arrow[from=2-2, to=1-2]
                                    	\arrow[from=1-2, to=1-1]
                                    	\arrow["\lrcorner"{anchor=center, pos=0.125}, draw=none, from=1-1, to=2-2]
                                    \end{tikzcd}
                                $$
                            Then:
                                $$\Omega^1_{S'/R} \cong \Omega^1_{S/R} \oplus \Omega^1_{R'/R}$$
                        \end{enumerate}
                \end{proposition}
                    \begin{proof}
                        \noindent
                        \begin{enumerate}
                            \item \textbf{(Module of differentials of a surjection):} 
                                \begin{enumerate}
                                    \item First of all, we claim that $\Omega^1_{R/R} \cong 0$. To see why this is the case, recall firstly that $R$ is the initial object of ${}^{R/}\Comm\Alg$, the category of internal commutative and unital algebras in ${}_R\Mod$. The universal property of the initial object as the colimit of the empty diagram as well as the fact that $\Omega^1_{-/R}$ is a left-adjoint, then jointly imply that $\Omega^1_{R/R}$ must be initial in ${}_R\Mod$. Lastly, recall that $0$ is intial in ${}_R\Mod$: this implies that $\Omega^1_{R/R} \cong 0$. Note that this is a special case of $\Omega^1_{S/R} \cong 0$ whenever $R \to S$ is surjective, because the zero object $0 \in {}_R\Mod$ is also terminal (and also because ${}_R\Mod$ is an abelian category).
                                    \item By remark \ref{remark: differentials_and_colimits}, there exists a surjective $R$-module homomorphism:
                                        $$\Omega^1_{R/R} \to \Omega^1_{S/R}$$
                                    and because $\Omega^1_{R/R} \cong 0$, we can thus deduce that:
                                        $$\Omega^1_{S/R} \cong 0$$
                                    from the universal property of the zero object $0$ in ${}_R\Mod$ as the limit of the empty diagram.
                                \end{enumerate}
                            \item \textbf{(Module of differentials and base change):} This is completely trivial.
                        \end{enumerate}
                    \end{proof}
                    
                \begin{lemma}[Surjections between modules of differentials] \label{lemma: surjections_between_modules_of_differentials}
                    Consider the following commutative diagram in $\Cring$:
                        $$
                            \begin{tikzcd}
                            	{S'} & {R'} \\
                            	S & R
                            	\arrow[from=2-1, to=1-1]
                            	\arrow[from=2-2, to=1-2]
                            	\arrow[from=1-2, to=1-1]
                            	\arrow[from=2-2, to=2-1]
                            \end{tikzcd}
                        $$
                    Should the arrow $S \to S'$ be surjective, then the naturally induced $S$-module homomorphism $\Omega^1_{S/R} \to \Omega^1_{S'/R'}$ shall also be surjective.
                \end{lemma}
                    \begin{proof}
                        First of all, the $S$-module homomorphism $\Omega^1_{S/R} \to \Omega^1_{S'/R'}$ is well-defined as it comes from evaluating the natural transformation $\Omega^1_{-/R} \to \Omega^1_{-/R'}$ along the arrow $S \to S'$ in the following manner:
                            $$
                                \begin{tikzcd}
                                	& {} & {\Omega^1_{S'/R'}} & 0 \\
                                	{\Omega^1_{S'/R}} & {\Omega^1_{R'/R}} \\
                                	{\Omega^1_{S/R}} & 0
                                	\arrow[from=3-2, to=2-2]
                                	\arrow[from=3-2, to=3-1]
                                	\arrow[from=3-1, to=2-1]
                                	\arrow[from=2-2, to=2-1]
                                	\arrow[from=1-4, to=1-3]
                                	\arrow[from=2-1, to=1-3]
                                	\arrow[from=2-2, to=1-4]
                                \end{tikzcd}
                            $$
                        Next, note that the $S$-module homomorphism $\Omega^1_{S/R} \to \Omega^1_{S'/R}$ is trivially surjective via an application of proposition \ref{prop: differentials_and_finite_colimits}.
                    \end{proof}
                    
                \begin{proposition}[The canonical exact sequence] \label{prop: canonical_exact_sequence_of_differentials}
                    For $A \to B \to C$ a composition of ring maps, there exists a canonically associated right-exact sequence of $C$-modules:
                        $$C \tensor_B \Omega^1_{B/A} \to \Omega^1_{C/A} \to \Omega^1_{C/B} \to 0$$
                \end{proposition}
                    \begin{proof}
                        First of all, the morphisms $A \to B \to C$ gives rise to a natural transformations:
                            $$\Omega^1_{-/A} \to \Omega^1_{-/B} \to \Omega^1_{-/C}$$
                        In particular, this tells us that there are the following canonically defined commutative diagrams:
                            $$\Omega^1_{B/A} \to \Omega^1_{B/B}$$
                            $$\Omega^1_{C/A} \to \Omega^1_{C/B} \to \Omega^1_{C/C}$$
                        Second of all, recall that we know by proposition \ref{prop: differentials_and_finite_colimits} that:
                            $$\Omega^1_{B/B} \cong 0$$
                            $$\Omega^1_{C/C} \cong 0$$
                        Thus, there exists the following canonical commutative diagram of $C$-modules:
                            $$
                                \begin{tikzcd}
                                	{C \tensor_B \Omega^1_{B/A}} & {\Omega^1_{C/A}} & 0 \\
                                	0 & {\Omega^1_{C/B}} & 0
                                	\arrow[from=1-2, to=2-2]
                                	\arrow[from=1-1, to=2-1]
                                	\arrow[from=2-1, to=2-2]
                                	\arrow[from=1-1, to=1-2]
                                	\arrow[from=2-2, to=2-3]
                                	\arrow["{!}", from=1-2, to=1-3]
                                	\arrow[from=1-3, to=2-3]
                                \end{tikzcd}
                            $$
                        wherein:
                            \begin{itemize}
                                \item the horizontal arrows exist as a consequence of $C \tensor_B -: {}_B\Mod \to {}_C\Mod$ being a left-adjoint
                                \item $!: \Omega^1_{C/A} \to 0$ is the canonical terminal arrow, and
                                \item the arrows $0 \to \Omega^1_{C/B}$ and $\Omega^1_{C/B} \to 0$ are actually $C \tensor_B \Omega^1_{B/B} \to \Omega^1_{C/B}$ and $\Omega^1_{C/B} \to \Omega^1_{C/C}$, respectively.
                            \end{itemize}
                        An application of lemma \ref{lemma: surjections_between_modules_of_differentials} to the square:
                            $$
                                \begin{tikzcd}
                                	C & B \\
                                	C & A
                                	\arrow["{\id_C}", from=2-1, to=1-1]
                                	\arrow[from=2-2, to=1-2]
                                	\arrow[from=1-2, to=1-1]
                                	\arrow[from=2-2, to=2-1]
                                \end{tikzcd}
                            $$
                        (note that the identity morphism $\id_C: C \to C$ is trivially surjective) then helps us show the surjectivity of the map $\Omega^1_{C/A} \to \Omega^1_{C/B}$. This concludes the proof.
                    \end{proof}
                    
            \subsubsection{Cotangent complexes}
                \begin{convention}[Simplicial objects]
                    Let $\C$ be an arbitrary category. Then, the category of so-called \textbf{simplicial objects} of $\C$ is nothing but the functor category $\C^{\simp^{\op}}$, where $\simp$ is the category of finite simplicies. As $\simp$ is an $\infty$-category in a natural way, so is $\C^{\simp^{\op}}$.
                \end{convention}
                
                \begin{remark}[The Dold-Kan Correspondence] \label{remark: the_dold_kan_correspondence}
                    We will be assuming familiarity with the Dold-Kan Correspondence. In essence, it asserts that for $\calA$ any abelian category, there exists an equivalence of stable $\infty$-categories (cf. definition \ref{def: stable_infinity_categories}) between the category of simplicial objects of $\calA$ and that of projective resolutions/connective objects in $\calA$:
                        $$\calA^{\simp^{\op}} \cong {}^{\leq 0}\calA$$
                    For more details, see \cite{nlab:dold-kan_correspondence} and \cite[Subsection 1.2.3]{HA} (and in particular, \cite[Theorem 1.2.3.7]{HA}).
                \end{remark}
                
                \begin{remark}[Constructing cotangent complexes] \label{remark: constructing_cotangent_complexes}
                    Let $R$ be a base commutative ring and let $\e: P \to S$ be a surjective $R$-algebra homomorphism. We know by proposition \ref{prop: canonical_exact_sequence_of_differentials}, that there exists the following long exact sequence of $S$-modules:
                        $$S \tensor_P \Omega^1_{P/R} \to \Omega^1_{S/R} \to \Omega^1_{S/P} \to 0$$
                    and because $\e: P \to S$ is surjective (which implies - via proposition \ref{prop: differentials_and_finite_colimits} - that $\Omega^1_{S/P} \cong 0$), the sequence reduces down to:
                        $$S \tensor_P \Omega^1_{P/R} \to \Omega^1_{S/R} \to 0$$
                    Now, in knowing that $\e: P \to S$ is surjective, we have essentially been given a recipe for $S$ in terms of $P$: by the First Isomorphism Theorem, we have:
                        $$S \cong P/J$$
                    where $J := \ker \e$. This means that we can study $\Omega^1_{S/R}$ via studying $S \tensor_P \Omega^1_{P/R}$; the exactness of the sequence $S \tensor_P \Omega^1_{P/R} \to \Omega^1_{S/R} \to 0$ implies that the map $S \tensor_P \Omega^1_{P/R} \to \Omega^1_{S/R}$ is surjective, but since we do not know it explicitly, we will try to extend the sequence above to the left into a projective resolution instead of trying to find its kernel. Before attempting that, however, we should take a moment to verify that we actually can. There thus, to the forefront, comes the reason why we spend so much effort throughout subsubsection \ref{subsubsection: kahler_differentials} reformulating the construction of K\"ahler differentials: the notion is, per our definition, entirely internal to any symmetric monoidal linear category, such as ${}_R\Mod$, and one can therefore simply lift the entire setup up to the $\infty$-categorical level. To be more precise, because Leibniz algebras enjoy an internal definition (cf. proposition \ref{prop: leibniz_algebra_categories}), we can simply use the fact that the symmetric monoidal linear category ${}_R\Mod$ is abelian in conjunction with the Dold-Kan Correspondence (cf. remark \ref{remark: the_dold_kan_correspondence}) to construct the following $\infty$-adjunction:
                        $$
                            \begin{tikzcd}
                            	{{}^{\leq0, R/}\Comm\Leib\Alg} & {{}^{\leq0, R/}\Comm\Alg}
                            	\arrow[""{name=0, anchor=center, inner sep=0}, "\R\oblv"', shift right=2, hook, from=1-1, to=1-2]
                            	\arrow[""{name=1, anchor=center, inner sep=0}, "{\L\Omega^1_{-/R}}"', shift right=2, from=1-2, to=1-1]
                            	\arrow["\dashv"{anchor=center, rotate=-90}, draw=none, from=1, to=0]
                            \end{tikzcd}
                        $$
                    wherein:
                        \begin{itemize}
                            \item ${}^{\leq0, R/}\Comm\Alg$ is the $\infty$-subcategory of internal commutative monoids inside the stable $\infty$-category ${}_R^{\leq 0}\Mod$ of projective resolutions of $R$-modules (which, by the Dold-Kan Correspondence, is equivalent as a stable $\infty$-category to ${}_R\Mod^{\simp^{\op}}$),
                            \item ${}^{\leq0, R/}\Comm\Alg$ is the $\infty$-subcategory of commutative Leibniz algebras internal to ${}_R^{\leq 0}\Mod$, and
                            \item $\L\Omega^1_{-/R}$ and $\R\oblv$ are the obvious left and right-derived functors of $\Omega^1_{-/R}$ and $\oblv$ respectively (cf. theorem \ref{theorem: kahler_differentials_universal_property} for the definitions of $\Omega^1_{-/R}$ and $\oblv$).
                        \end{itemize}
                    Recovering the usual K\"ahler differential functor $\Omega^1_{-/R}$ is easy: it is simply the $0$-truncation of $\L\Omega^1_{-/R}$. One might could even make a case for the adjunction $\L\Omega^1_{-/R} \ladjoint \R\oblv$ being more fundamental and natural than the adjunction $\Omega^1_{-/R} \ladjoint \oblv$ from theorem \ref{theorem: kahler_differentials_universal_property}.
                \end{remark}
            
                \begin{definition}[\textcolor{red}{\underline{\textbf{IMPORTANT}}} Cotangent complexes] \label{def: cotangent_complexes} \index{Cotangent complex} \index{Conormal module}
                    Let $R$ be a base commutative ring and let $S$ be a simplicial $R$-algebra (which we shall view, through the lens of the Dold-Kan Correspondence, as a commutative monoids internal to the stable $\infty$-category ${}_R^{\leq 0}\Mod$); let $... \to P_1 \to P_0 \to S$ be a projective resolution of $S$ where the $P_i$'s are commutative $R$-algebras also. 
                    \begin{enumerate}
                        \item \textbf{(Cotangent complexes):} The \textbf{relative cotangent complex} over $R$ associated to $S$ is thus the application of the left-derived functor:
                            $$\L\Omega^1_{-/R}: {}^{\leq0, R/}\Comm\Alg \to {}^{\leq0, R/}\Comm\Leib\Alg$$
                        to the object $S \in {}^{\leq 0, R/}\Comm\Alg$. It is commonly denoted by $\bfL_{S/R}$.
                        \item \textbf{(Na\"ive cotangent complexes):} 
                            \begin{enumerate}
                                \item \textbf{(Na\"ive cotangent complexes):}  The \textbf{na\"ive cotangent complex} over $R$ associated to $S$ just the $(-1)$-truncation of the cotangent complex $\bfL_{S/R}$. It is usually denoted by $\NL_{S/R}$.
                                \item \textbf{(The conormal module):} The first homology of the (na\"ive) cotangent complex will frequently be of special interest (cf. theorem \ref{theorem: computing_naive_cotangent_complexes} and corollary \ref{coro: naive_cotangent_complex_of_separated_schemes_and_closed_subschemes}), so it shall be graced with a title: $H_1(\bfL_{S/R})$ shall henceforth be known as the \textbf{conormal module} over $R$ associated to $S$ and shall be denoted by $\calN^{\vee}_{S/R}$.
                            \end{enumerate}
                    \end{enumerate}
                \end{definition}
                \begin{remark}
                    It is not hard to see that by construction (cf. remark \ref{remark: constructing_cotangent_complexes}), one has:
                        $$\Omega^1_{S/R} \cong H_0(\bfL_{S/R})$$
                    for all simplicial commutative $R$-algebras $S$. Also, for $i \in \{0, 1\}$, one has that:
                        $$H_i(\bfL_{S/R}) \cong H_i(\NL_{S/R})$$
                    We will usually write $\NL_{S/R}$ in place of $\bfL_{S/R}$ when the homologies of degree $2$ or higher are of no relevance, such as in theorem \ref{theorem: computing_naive_cotangent_complexes} or corollary \ref{coro: naive_cotangent_complex_of_separated_schemes_and_closed_subschemes}.
                \end{remark}
                
                As any given cotangent complex arise through an $\infty$-left-adjoint, it will interact well with $\infty$-colimits, much like how modules of K\"ahler differentials interact well with ordinary $1$-categorical colimits. The following results are in perfect analogy with the ones presented in subsubsection \ref{subsubsection: properties_of_kahler_differentials}, and since the proofs of those results are entirely categorical, we shall be omitting proofs down below \footnote{We hope our readers are willing to accept the syntactic similarities between the theory of $1$-categories and that of $(\infty, 1)$-categories as an excuse for accepting the latter as black magic; if not (read: if you are a homotopy theorist) then we would like to offer our deepest apologies. Higher category theory, however, is not something one should attempt in public anyway, for great shame is usually brought upon those who are foolish (or brave) enough to do so.}. Our readers, of course, are more than welcome to use attempt these practice exercises.
                
                We would also like to emphasise that even though the results down below are phrased in terms of rings, they in fact remain true within any stable symmetric monoidal linear $\infty$-category (i.e. any symmetric monoidal dg-category). For instance, one may generalise these results - without any issue - by replacing simplicial commutative rings with commutative ring objects internal to some given $\infty$-topos (see corollary \ref{coro: canonical_exact_sequence_of_cotangent_complexes_of_ringed_topoi} for more details). It is simply a lot more convenient to state results this way \footnote{One usually restricts to the affine case in practice anyway.}. 
                
                \begin{convention}[Everything is derived!] \label{conv: cotangent_complex_everything_is_derived}
                    From now on until the end of the chapter, everything will be assumed to be derived. In particular, this means that all commutative rings and algebras over them shall be assumed to be simplicial (note that simplicial commutative rings are nothing but objects of $\Comm\Alg(\Ab^{\simp^{\op}})$) and also, that we shall confuse $\infty$-categorical with their ordinary categorical counterparts. The symbol \say{$\cong$} will also be understood to represent quasi-isomorphisms whenever the context is appropriate.
                \end{convention}
                
                \begin{remark}[Cotangent complexes and colimits] \label{remark: cotangent_complexes_and_colimits}
                    Let $R$ be a base commutative ring and let $\{S_i\}_{i \in I}$ be a diagram of $R$-algebras $S_i$. Then, due to $\bfL_{-/R}$ being a left-adjoint (which means, in particular, that it would preserve colimits \textit{a priori}), one has the following identity:
                        $$\bfL_{\underset{i \in I}{\colim} S_i/R} \cong \underset{i \in I}{\colim} \bfL_{S_i/R}$$
                \end{remark}
                \begin{example}[Cotangent complexes and localisations] \label{example: cotangent_complexes_and_localisations}
                    An example of a colimit of an infinite diagram of modules of K\"ahler differentials is how these modules interact with localisations of commutative rings. Let $S$ be a (possibly infinite) commutative ring let $\q \in |\Spec {}^{\leq 0}S|$ be a prime ideal thereof (${}^{\leq 0}S$ means the $0$-truncation of $S$), and let $R \to S$ be a ring map. Then:
                        $$\bfL_{S_{\q}/R} \cong (\bfL_{S/R})_{\q}$$
                    Note that this exhibits the commutativity of $\bfL_{-/R}$ with an \textit{infinite} colimit because:
                        $$S_{\q} \cong \underset{y \in {}^{\leq 0}S \setminus \q}{\colim} S[1/y]$$
                \end{example}
                Let us examine how cotangent complexes interact with colimits in closer details via the following proposition, wherein we rely on the fact that finite colimits can be constructed out of finite coproducts and epimorphisms.
                \begin{proposition}[Cotangent complexes and finite colimits] \label{prop: cotangent_complexes_and_finite_colimits}
                    Let $R$ be a base commutative ring. 
                        \begin{enumerate}
                            \item \textbf{(Module of differentials of a surjection):} If $R \to S$ is a surjective ring homomorphism, then:
                                $$\bfL_{S/R} \cong 0$$
                            \item \textbf{(Module of differentials and base change):} Consider a pushout diagram of commutative rings such as the following one:
                                $$
                                    \begin{tikzcd}
                                    	{S'} & {R'} \\
                                    	S & R
                                    	\arrow[from=2-2, to=2-1]
                                    	\arrow[from=2-1, to=1-1]
                                    	\arrow[from=2-2, to=1-2]
                                    	\arrow[from=1-2, to=1-1]
                                    	\arrow["\lrcorner"{anchor=center, pos=0.125}, draw=none, from=1-1, to=2-2]
                                    \end{tikzcd}
                                $$
                            Then:
                                $$\bfL_{S'/R} \cong \bfL_{S/R} \oplus \bfL_{R'/R}$$
                            where the direct is to be understood as the binary biproduct in the stable $\infty$-category ${}_R^{\leq 0}\Mod$.
                        \end{enumerate}
                \end{proposition}
                    
                \begin{lemma}[Surjections between cotangent complexes] \label{lemma: surjections_between_cotangent_complexes}
                    Consider the following commutative diagram of commutative rings:
                        $$
                            \begin{tikzcd}
                            	{S'} & {R'} \\
                            	S & R
                            	\arrow[from=2-1, to=1-1]
                            	\arrow[from=2-2, to=1-2]
                            	\arrow[from=1-2, to=1-1]
                            	\arrow[from=2-2, to=2-1]
                            \end{tikzcd}
                        $$
                    Should the arrow $S \to S'$ be surjective, then the naturally induced $S$-module homomorphism $\bfL_{S/R} \to \bfL_{S'/R'}$ shall also be surjective.
                \end{lemma}
                    
                \begin{proposition}[The canonical exact sequence] \label{prop: canonical_exact_sequence_of_cotangent_complexes}
                    For $A \to B \to C$ a composition of ring maps, there exists a canonically associated distinguished triangle (cf. definition \ref{def: triangulated_infinity_categories}) of $C$-modules:
                        $$C \tensor_B^{\L} \bfL_{B/A} \to \bfL_{C/A} \to \bfL_{C/B} \to (C \tensor_B^{\L} \bfL_{B/A})[1]$$
                    Note that it is possible for the right-most term $(C \tensor_B^{\L} \bfL_{B/A})[1]$ to be non-zero, as one can not guarantee that all the maps $B_j \to A_i$ are surjective (here, $... \to B_1 \to B_0 \to B$ and $... \to A_1 \to A_0 \to A$ are projective resolutions of $B$ and $A$, respectively). 
                \end{proposition}
                \begin{corollary}[The canonical exact sequence for morphisms of ringed topoi] \label{coro: canonical_exact_sequence_of_cotangent_complexes_of_ringed_topoi}
                    Let:
                        $$
                            \begin{tikzcd}
                            	{(\calX, \calO_{\calX})} & {(\calY, \calO_{\calY})} & {(\calZ, \calO_{\calZ})}
                            	\arrow["f", from=1-1, to=1-2]
                            	\arrow["g", from=1-2, to=1-3]
                            \end{tikzcd}
                        $$
                    be composition of geometric morphisms between (small) (simplicially) ringed sheaf topoi (each of which can be thought of as pairs consisting of an $\infty$-topos and a choice of internal commutative ring therein); these topoi could arise, for instance, as categories of sheaves (of simplical sets/$\infty$-groupoids/\say{spaces}) over a derived scheme or over a derived algebraic stack (cf. section \ref{section: schemes}). Then, through definition \ref{def: qcoh_def}, one obtains the following canonical distinguished triangle in the stable $\infty$-category of simplicial $\calO_X$-modules:
                        $$f^*\bfL_{\calY/\calZ} \to \bfL_{\calX/\calZ} \to \bfL_{\calX/\calY} \to (f^*\bfL_{\calY/\calZ})[1]$$
                \end{corollary}
                
            \subsubsection{Cotangent spaces} \label{subsubsection: cotangent_spaces}
                Now, let us look into how you, our dear reader, might actually be able to compute some na\"ive cotangent complexes for yourself, and thereby discern geometric interpretations of these algebraic beasts. 
                
                \begin{theorem}[\textcolor{red}{\underline{\textbf{IMPORTANT}}} Computing na\"ive cotangent complexes] \label{theorem: computing_naive_cotangent_complexes}
                    \noindent
                    \begin{enumerate}
                        \item Let $R$ be a base commutative ring and let $P \to S$ be a surjective map between $R$-algebras (viewed as a $(-1)$-truncated projective resolution), with kernel $J$. Then, the conormal module takes on the following form:
                            $$H_1(\NL_{S/R}) \cong J/J^2$$
                        which is to say that the conormal module $\calN^{\vee}_{S/R}$ is given by $J/J^2$.
                        \item If $P \to S$ has, in addition, a section (i.e a right-inverse), then:
                            $$H_2(\NL_{S/R}) \cong 0$$
                        Furthermore, the resulting short exact sequence $\bfL_{S/R}^{\geq -1}$, i.e.:
                            $$0 \to \calN^{\vee}_{S/R} \to S \tensor_P \Omega^1_{P/R} \to \Omega^1_{S/R} \to 0$$
                        splits.
                        \item Let $A$ be a $0$-connective commutative ring, let $B$ be a $0$-connective $A$-algebra, and consider the codiagonal/multiplication $\nabla_{B/A}: B \tensor_A B \to B$: this map is surjective and admits the canonical map $B \to B \tensor_A B$ as a section, so the na\"ive cotangent complex $\NL_{B/A}$ will take the on the form of the following splitting short exact sequence $\bfL_{B/A}^{\geq -1}$, i.e.:
                            $$0 \to \calI_{B/A}/\calI_{B/A}^2 \to B \tensor_{B \tensor_A B} \Omega^1_{(B \tensor_A B)/A} \to \Omega^1_{B/A} \to 0$$
                        where $\calI_{B/A}$ denotes the kernel $\ker \nabla_{B/A}$. Then, one can actually characterise $\Omega^1_{B/A}$ as $\calI_{B/A}/\calI_{B/A}^2$ via an isomorphism of $B$-modules:
                            $$\Omega^1_{B/A} \cong \calI_{B/A}/\calI_{B/A}^2$$
                    \end{enumerate}
                \end{theorem}
                    \begin{proof}
                        \noindent
                        \begin{enumerate}
                            \item 
                            \item 
                            \item This is a straightforward consequence of theorem \ref{theorem: kahler_differentials_universal_property}.
                        \end{enumerate}
                    \end{proof}
                \begin{corollary}[Na\"ive cotangent complex of separated schemes and of closed subschemes] \label{coro: naive_cotangent_complex_of_separated_schemes_and_closed_subschemes}
                    \noindent
                    \begin{itemize}
                        \item \textbf{(Differentials on separated schemes):} A separated (relative) scheme $X \to S$ is one that is closed inside its diagonal $X \x_S X$ (i.e. the diagonal map $\Delta_{X/S}: X \to X \x_S X$ is a closed immersion). This means that there exists a quasi-coherent ideal sheaf $\calI_{X/S} \subset \calO_{X \x_S X}$ such that:
                        $$\Spec_{X \x_S X/S} \calO_{X \x_S X}/\calI_{X/S} \cong X$$
                        or concretely, that the codiagonal:
                            $$\nabla_{X/S}: \calO_X \tensor_{\calO_S} \calO_X \to \calO_X$$
                        has $\Delta^*_{X/S} \calI_{X/S}$ as its kernel (note that we have $\Delta^*_{X/S} \calO_{X \x_S X} \cong \calO_X \tensor_{\calO_S} \calO_X$ by some general topos theory; we shall leave this verification up to the reader). We can then apply theorem \ref{theorem: computing_naive_cotangent_complexes} to obtain the following definition of the sheaf of (relative) K\"ahler differentials on $X$:
                            $$\Omega^1_{X/S} \cong \Delta^*_{X/S}(\calI_{X/S}/\calI_{X/S}^2)$$
                        It is not hard to see that $\Omega^1_{X/S}$ is necessarily quasi-coherent.
                        \item \textbf{(Differentials on closed subschemes):}
                            \begin{itemize}
                                \item \textbf{(Differentials on general closed subschemes):} By arguing similarly, one sees that for:
                                    $$j: Z \hookrightarrow X$$
                                a closed immersion of schemes, there exists a canonically defined associated (quasi-coherent) sheaf of differentials $\Omega^1_{Z/X}$, which is given by:
                                    $$\Omega^1_{Z/X} \cong j^*(\calI_{Z/X}/\calI_{Z/X}^2)$$
                                This, in fact, is a generalisation of the first case.
                                \item \textbf{(Zariski cotangent spaces):} One very interesting subcase that we will examine in closer details later is that of closed points $x: \Spec \kappa_x \to X$. For now, however, note that the na\"ive cotangent complex at $x \in |X|$ has the form:
                                    $$\m_x/\m_x^2 \to \kappa_x \tensor_{\calO_{Z, z}} \Omega^1_{X, x} \to \Omega^1_{\kappa_x/\calO_{X, x}} \to 0$$
                                Because there exists a canonical surjective ring homomorphism $\calO_{X, x} \to \kappa_x$, we have:
                                    $$\Omega^1_{\kappa_x/\calO_{X, x}} \cong 0$$
                            \end{itemize}
                    \end{itemize}
                \end{corollary}
    
        \subsection{Smoothness}
            Smoothness is a notion that, while being intuitively simple (or at least seemingly so), is extremely subtle and furthermore, has far-reaching consequences. Morally, one should imagine a smooth scheme (or for that matter, a smooth variety) as an algebro-geometric object that behaves as much like a smooth manifold as possible. For instance, there ought to be no singularities, as well as no funny business of dimension-hopping between tangent spaces at different points. Smooth varieties should also admit some sort of de Rham cohomology (see section \ref{section: algebraic_de_rham_cohomology_over_characteristic_0}) that returns results agreeing with those given by \'etale cohomology (which is somehow the \say{right} notion of singular comology for schemes). But what if one is looking for something a bit more technical ? Well, first of all, we are going to restrict ourselves to cases where a so-called \say{smooth} morphism is of finite presentation, which is because our first line of attack is going to be through Jacobian matrices: should these be of full rank, our schemes shall be \say{smooth}, and since Jacobians are only well-defined for functions between finite-dimensional spaces, \say{smooth} morphisms had better be of finite presentation in the first place (otherwise, there might be infinitely many components in our Jacobians). This, however, turns out to be a na\"ive attempt at tackling algebro-geometric smoothness, which is not to imply that one is unable to write down a meaningful definition of what it means for a scheme to be smooth, but instead, that such a definition is entirely impractical (this was pushed, for instance, by Michael Artin): the Jacobian criterion, or even the alternative definition involving the cotangent complex, while concrete, is just not easy to check at all, and worse, does not generalise well to more exotic settings such as those of derived schemes or perfectoid spaces. Due to this, we will start with what is called \say{formal smoothness}. A formally smooth morphism, roughly speaking, shall be one with all the qualitative properties that one would expect from a smooth morphism. We will subsequently introduce finiteness to the picture to obtain morphisms that are smooth in the technical sense. 
        
            \subsubsection{Formally smooth morphisms}
                \begin{definition}[Formal smoothness] \label{def: formall_smoothness} \index{Smoothness! formal}
                    \noindent
                    \begin{enumerate}
                        \item \textbf{(Formally smooth ring map):} A homomorphism of commutative rings:
                            $$\varphi: R \to S$$
                        is \textbf{formally smooth} if and only if for all $S$-algebras $B$ and nilpotent ideal $J$ thereof, the canonical map induced by the ring map $B \to B/J$:
                            $$\Spec S(B) \to \Spec R(B/J)$$
                        is surjective.
                        \item \textbf{(Formally smooth prestacks):} A morphism:
                            $$f: \calX \to \calY$$
                        of prestacks is said to be \textbf{formally smooth} if and only if it is represented by a formally smooth morphism of affine schemes.
                    \end{enumerate}
                \end{definition}
                \begin{remark}
                    Note that the so-called \say{canonical map} induced by $B \to B/J$ always exists; simply consider the following commutative diagram:
                        $$
                            \begin{tikzcd}
                            	{X(B)} & {X(B/J)} \\
                            	{Y(B)} & {Y(B/J)}
                            	\arrow[from=1-1, to=2-1]
                            	\arrow[from=2-1, to=2-2]
                            	\arrow[from=1-1, to=1-2]
                            	\arrow[from=1-2, to=2-2]
                            	\arrow[dashed, from=1-1, to=2-2]
                            \end{tikzcd}
                        $$
                \end{remark}
                
                \begin{proposition}[Formal smoothness is stable under base changes and compositions] \label{prop: compositions_and_base_changes_of_formally_smooth_morphisms}
                    \noindent
                    \begin{enumerate}
                        \item Let:
                            $$
                                \begin{tikzcd}
                                	A & B & C
                                	\arrow["\varphi", from=1-1, to=1-2]
                                	\arrow["\psi", from=1-2, to=1-3]
                                \end{tikzcd}
                            $$
                        be a composition of formally smooth ring homomorphisms. The composite map $A \to C$ is thus also formally smooth.
                        \item Let $\varphi: R \to S$ be a formally smooth ring map and $\psi: R \to R'$ be an arbitrary homomorphism of commutative rings. Then, the pushout $S \tensor_{\varphi, R, \psi} R'$ is formally smooth over $R'$ as well.
                    \end{enumerate}
                \end{proposition}
                    \begin{proof}
                        \noindent
                        \begin{enumerate}
                            \item 
                            \item 
                        \end{enumerate}
                    \end{proof}
                    
                \begin{lemma}[Splitting of the canonical short exact sequence] \label{lemma: canonical_short_exact_sequence_splits}
                    Let $\varphi: R \to S$ be a ring map and let $\pi: P \to S$ be a surjective homomorphism of $R$-algebras from a polynomial $R$-algebra $P$; additionally, write $J := \ker \pi$. Then, $\varphi: R \to S$ is smooth if and only if the canonically defined right-exact sequence:
                        $$J/J^2 \to \Omega^1_{P/R} \tensor_P S \to \Omega^1_{S/R} \to 0$$
                    is actually a short exact sequence that splits.
                \end{lemma}
                    \begin{proof}
                        \noindent
                        \begin{enumerate}
                            \item 
                            \item 
                        \end{enumerate}
                    \end{proof}
                
                \begin{proposition}
                    
                \end{proposition}
                    \begin{proof}
                        
                    \end{proof}
                    
                \begin{proposition}[Formal smoothness is a local property] \label{prop: formal_smoothness_is_local}
                    Let $\varphi: R \to S$ be a homomorphism between two commutative rings and let $\q$ be some prime ideal of $S$ (read: point of $|\Spec S|$). Then, $\varphi$ is formally smooth if and only if the induced maps $\varphi_{\q}: R \to S_{\q}$ are all formally smooth. 
                \end{proposition}
                    \begin{proof}
                        \noindent
                        \begin{enumerate}
                            \item Suppose first of all that $\varphi_{\q}: R \to S_{\q}$ is a formally smooth ring map for any prime $\q \in |\Spec S|$.  
                            \item 
                        \end{enumerate}
                    \end{proof}
                    
                \begin{proposition}[Formally smooth + finite type + local = flat] \label{prop: formally_smooth_finite_type_local_morphisms_are_flat}
                    Let $(R, \m)$ be a local ring, let $S$ be a finitely presented $R$-algebra, and consider a local homomorphism $(R, \m) \to (S_{\q}, \q)$. Then, should $R \to S_{\q}$ be formally smooth, it shall also be flat. 
                \end{proposition}
                    \begin{proof}
                        
                    \end{proof}
        
            \subsubsection{Smooth morphisms}
                \begin{definition}[Standard smoothness] \label{def: standard_smoothness} \index{Smoothness! standard}
                    \noindent
                    \begin{enumerate}
                        \item \textbf{(Standard smooth ring maps):} A map of commutative rings:
                            $$\varphi: R \to S$$
                        is called \textbf{standard smooth} if and only if it is of \textit{finite presentation} (i.e. there exists natural numbers $N, n$ such that:
                            $$S \cong R[x_1, ..., x_N]/(f_1, ..., f_n)$$
                        for some finite subset $\{f_i\}_{1 \leq i \leq n}$ of $R[x_1, ..., x_n]$) and the Jacobian of the vector-valued function $(f_1, ..., f_n)$ (mind the abuse of notation):
                            $$\Jac(f_1, ..., f_n) = \left(\nabla f_1, ..., \nabla f_n\right)^T = 
                                \begin{pmatrix}
                                    \del_{x_1} f_1 & ... & \del_{x_n} f_1
                                    \\
                                    \vdots & \ddots & \vdots
                                    \\
                                    \del_{x_1} f_n & ... & \del_{x_n} f_n
                                \end{pmatrix}
                            = (\del_{x_j} f_i)_{1 \leq i, j \leq n}$$
                        is \textit{full-rank} (i.e. of rank $n$ in this particular instance); alternatively, by basic module theory, one can require the determinant of the Jacobian to be \textit{invertible} in $S$.  
                        \item \textbf{(Standard smooth prestacks):} A morphism:
                            $$f: \calX \to \calY$$
                        of prestacks is said to be \textbf{standard smooth} if and only if it is represented by a standard smooth morphism of affine schemes.
                    \end{enumerate}
                \end{definition}
                \begin{remark}[Unpacking the definition] \label{remark: standard_smoothness}
                    Definition \ref{def: standard_smoothness} paints a rather conrete and down-to-earth picture depicting what it means for a ring map to supposedly be \say{smooth}. Essentially, what it is trying to say is that given a ring map of finite presentation:
                        $$\varphi: R \to S$$
                    with:
                        $$S \cong R[x_1, ..., x_N]/(f_1, ..., f_n)$$
                    then should the Jacobian - an $R$-linear operator on $S$ viewed as a finitely presented $R$-module - be of full rank, the aforementioned ring map $\varphi$ is going to be somehow \say{smooth} (the quotation marks are here because as it turns out, standard smooth morphisms are only cohomologically smooth - i.e. smooth in the \say{right} algebro-geometric way - if the associated universal module of K\"ahler differential is free; cf. proposition \ref{prop: smooth_iff_standard_smooth}). In other words, definition \ref{def: standard_smoothness} is nothing but an analogue of the Inverse Function Theorem from calculus. 
                \end{remark}
                \begin{remark}[Locality of (standard) smoothness]
                    One very important bit of information that can be inferred from definition \ref{def: standard_smoothness} is that standard smoothness (and as we shall see later on, cohomological smoothness as well) is a Zariski-local property: one checks whether or not some given scheme over a base commutative ring is standard smooth by checking if the affine patches covering it are so. 
                \end{remark}
                
                \begin{definition}[Cohomological smoothness] \label{def: cohomological_smoothness} \index{Smoothness! cohomological}
                    \noindent
                    \begin{enumerate}
                        \item \textbf{(Cohomologically smooth ring maps):} A homomorphism between commutative rings:
                            $$\varphi: R \to S$$
                        is called \textbf{cohomologically smooth} if and only if it is of finite presentation and its associated (na\"ive) cotangent complex is quasi-isomorphic to a finitely generated projective $S$-module placed in degree $0$.
                        \item \textbf{(Cohomologically smooth prestacks):} A morphism:
                            $$f: \calX \to \calY$$
                        of prestacks is said to be \textbf{cohomologically smooth} if and only if it is represented by a cohomologically smooth morphism of affine schemes.
                    \end{enumerate}
                \end{definition}
                \begin{remark}[Cotangent complex: na\"ive or nay ?]
                    Definition \ref{def: cohomological_smoothness} made reference to na\"ive cotangent complexes associated to ring maps, and how those of ring maps that are of finite presentation being quasi-isomorphic to certain complexes of modules concentrated in degree $0$ implies cohomological smoothness. On the surface this might seem like a rather sensible characterisation of smoothness, but dive a little deeper and one shall find one glaring problem: the na\"ive cotangent complex is incredibly awkward to work with. There is, however, a silver lining, which is that na\"ive cotangent complexes are actually nothing but $(-1)$-truncated cotangent complexes. Thus, we can simply remove the word \say{na\"ive} from definition \ref{def: cohomological_smoothness}. 
                \end{remark}
                
                \begin{proposition}[Cohomological smoothness is the same as standard smoothness] \label{prop: smooth_iff_standard_smooth}
                    A ring map $\varphi: R \to S$ of finite presentation is smooth if and only if it is standard smooth.
                \end{proposition}
                    \begin{proof}
                    
                    \end{proof} 
                \begin{convention}
                    Thanks to proposition \ref{prop: smooth_iff_standard_smooth}, it makes sense from this point on for us to do away with the specifications and refer to both standard smooth morphisms and cohomologically smooth ones as simply being \say{smooth}.
                \end{convention}
                    
                \begin{proposition}[Smoothness implies almost-finiteness of cotangent complex] \label{prop: smoothness_implies_almost_finiteness_of_cotangent_complex}
                    The cotangent complex associated to any smooth ring map $\varphi: R \to S$ is almost of finite type, and because the cotangent complex associated to any smooth ring map is quasi-isomorphic to a projective module placed in degree $0$, this is actually just asserting that the associated module of K\"ahler differentials $\Omega^1_{S/R}$ is a finitely generated projective module.
                \end{proposition}
                    \begin{proof}
                    
                    \end{proof}
                \begin{corollary}[Relative dimensions of smooth maps]
                    The relative dimension of a smooth ring map is the number of generators of its associated cotangent complex, which according to proposition \ref{prop: smoothness_implies_almost_finiteness_of_cotangent_complex}, had better be finite.
                \end{corollary}
                \begin{example}
                    A smooth ring map of the form:
                        $$\varphi: R \to R[x_1, ..., x_N]/(f_1, ..., f_n)$$
                    has relative dimension $N - n$. 
                \end{example}
                
                \begin{proposition}[Smooth maps are finitely presented formally smooth maps] \label{prop: smooth_iff_formally_smooth_and_of_finite_presentation}
                    A ring map of finite presentation is smooth if and only if it is formally smooth.
                \end{proposition}
                    \begin{proof}
                        
                    \end{proof}
                
                \begin{proposition}[Smoothness is a local property] \label{prop: smoothness_is_local}
                    Let $\varphi: R \to S$ be a ring map of finite presentation and let $\q$ be some prime ideal of $S$ (read: point of $|\Spec S|$). Then, $\varphi$ is smooth if and only if the induced maps $\varphi_{\q}: R \to S_{\q}$ are all smooth. 
                \end{proposition}
                    \begin{proof}
                        
                    \end{proof}
                \begin{corollary}[Fibre-wise smoothness] \label{coro: fibrewise_smoothness}
                    Let $X$ be a scheme over some base scheme $S$. Then, the structure morphism $X \to S$ is smooth if and only if all of its fibres are so, i.e. for all $s \in |S|$, the fibre $X_s \cong X \x_S \Spec \kappa_s$ is smooth over the residue field $\kappa_s$. In practice, this means that to check for smoothness, one can simply pullback to over a point and apply fibre-wise results on smoothness (such as proposition \ref{prop: dimensions_of_smoothn_morphisms_over_fields}).
                \end{corollary}
                    
                \begin{proposition}[Smoothness is stable under base changes and compositions] \label{prop: compositions_and_base_changes_of_smooth_morphisms}
                    \noindent
                    \begin{enumerate}
                        \item Let:
                            $$
                                \begin{tikzcd}
                                	A & B & C
                                	\arrow["\varphi", from=1-1, to=1-2]
                                	\arrow["\psi", from=1-2, to=1-3]
                                \end{tikzcd}
                            $$
                        be a composition of smooth ring homomorphisms, and suppose that $\varphi$ is of relative dimension $r$, and $\psi$ is of relative dimension $s$. Given these hypotheses, the relative dimension of $\psi \circ \varphi$ is $r + s$.
                        \item Let $\varphi: R \to S$ be a smooth ring map of relative dimension $d$ and $\psi: R \to R'$ be an arbitrary homomorphism of commutative rings. Then, the pushout $S \tensor_{\varphi, R, \psi} R'$ is smooth over $R'$, and of relative dimension $d$ as well. 
                    \end{enumerate}
                \end{proposition}
                    \begin{proof}
                        \noindent
                        \begin{enumerate}
                            \item According to definition \ref{def: standard_smoothness} and proposition \ref{prop: smooth_iff_standard_smooth}, we can write $B$ as a commutative $A$-algebra of the form $\frac{A[x_1, ..., x_N]}{(f_1, ..., f_n)}$ for some pair $N, n$ of natural numbers, and subsequently, $C$ as a commutative $B$-algebra (which should be viewed as an $\frac{A[x_1, ..., x_N]}{(f_1, ..., f_n)}$-algebra) of the form $\frac{\frac{A[x_1, ..., x_N]}{(f_1, ..., f_n)}[y_1, ..., y_M]}{(g_1, ..., g_m)} \cong \frac{A[x_1, ..., x_N, y_1, ..., y_M]}{(f_1, ..., f_n, g_1, ..., g_m)}$ for another pair $M, m$ of natural numbers. Notice that:
                                $$N - n = r, M - m = s$$
                            (also, recall that smooth morphisms are \textit{a priori} of finite presentation, which would imply that $n \leq N$ and $m \leq M$, so the above expressions are well-defined - we do not want negative dimensions, after all). It is then rather easy to see that the relative dimension of $\psi \circ \varphi$ had better be equal to $r + s$.
                            \item Suppose that for some pair of natural numbers $n, N$, we have:
                                $$S \cong \frac{R[x_1, ..., x_N]}{(f_1, ..., f_n)}$$
                            Then, by the fact that colimits commute, we have:
                                $$S \tensor_{\varphi, R, \psi} R' \cong \frac{R[x_1, ..., x_N]}{(f_1, ..., f_n)} \tensor_{\varphi, R, \psi} R' \cong \frac{R'[x_1, ..., x_N]}{(f_1, ..., f_n)}$$
                            which tells us that the pushout $S \tensor_{\varphi, R, \psi} R'$ is smooth as a commutative $R'$-algebra, and that it is of relative dimension $d = N - n$, much like $S$ is as an $R$-algebra.
                        \end{enumerate}
                    \end{proof}
                \begin{corollary}[Compositions and base changes of \'etale morphisms] \label{coro: compositions_and_base_changes_of_etale_morphisms}
                    Compositions of \'etale morphisms (see definition \ref{def: etale_morphisms} for the notion of \'etale-ness) are \'etale themselves, as these are smooth and of relative dimension $0$. Likewise, base changes of \'etale morphisms are also \'etale.
                \end{corollary}
                \begin{remark}[Preservation of smoothness and \'etale-ness of non-affine schemes]
                    As smoothness (and hence \'etale-ness) is a local notion (cf. proposition \ref{prop: smoothness_is_local}), proposition \ref{prop: compositions_and_base_changes_of_smooth_morphisms} and corollary \ref{coro: compositions_and_base_changes_of_etale_morphisms} generalise in a rather obvious manner to cases where one's schemes might not be affine. Namely:
                        \begin{enumerate}
                            \item should:
                                $$
                                    \begin{tikzcd}
                                        	X & Y & Z
                                        	\arrow["\varphi", from=1-1, to=1-2]
                                        	\arrow["\psi", from=1-2, to=1-3]
                                        \end{tikzcd}
                                $$
                            be any pair of composable smooth (or \'etale) morphisms of schemes, wherein $\varphi$ is of relative dimension $r$ and $\psi$ is of relative dimension $s$, then their composition $\psi \circ \varphi$ will be smooth and of relative dimension $r + s$, and
                            \item given any pullback square of schemes as follows:
                                $$
                                    \begin{tikzcd}
                                    	{Y'} & Y \\
                                    	{X'} & X
                                    	\arrow["\psi", from=2-1, to=2-2]
                                    	\arrow["\varphi", from=1-2, to=2-2]
                                    	\arrow[from=1-1, to=1-2]
                                    	\arrow[from=1-1, to=2-1]
                                    	\arrow["\lrcorner"{anchor=center, pos=0.125}, draw=none, from=1-1, to=2-2]
                                    \end{tikzcd}
                                $$
                            wherein $\varphi: Y \to X$ is smooth of relative dimension $d$ and $\psi: X' \to X$ is arbitrary, the canonical projection $Y \x_{\varphi, X, \psi} X' \to X'$ is also smooth and of relative dimension $d$ (when $d = 0$, one obtains the stability of \'etale-ness of under base changes).
                        \end{enumerate}
                \end{remark}
                
                \begin{proposition}[Relative and pure dimensions of smooth maps over fields] \label{prop: dimensions_of_smoothn_morphisms_over_fields}
                    Let $k$ be a field and let:
                        $$\pi: X \to \Spec k$$
                    be a scheme that is smooth over $\Spec k$. Then, the following are equivalent:
                        \begin{enumerate}
                            \item $\pi: X \to \Spec k$ is of relative dimension $d$. 
                            \item The Krull dimension of $X$ is $d$. 
                        \end{enumerate}
                \end{proposition}
                    \begin{proof}
                        Because smoothness, as a property of schemes, is Zariski-local, let us assume that $X$ is affine. Note that this is not at the detriment of generality. 
                        \begin{enumerate}
                            \item To start, let us assume \textbf{1}. Specifically, let us assume that for some pair of natural numbers $n, N$ such that $d = N - n$, we have:
                                $$X \cong \Spec \frac{k[x_1, ..., x_N]}{(f_1, ...,f_n)}$$
                            Then, it is simply a matter of finding the Krull dimension $\frac{k[x_1, ..., x_N]}{(f_1, ...,f_n)}$. By the Third Isomorphism Theorem, prime ideals of $\frac{k[x_1, ..., x_N]}{(f_1, ...,f_n)}$ are in bijective correspondence with those of $k[x_1, ..., x_n]$ that contain the ideal $(f_1, ..., f_n)$; the Krull dimension of $\frac{k[x_1, ..., x_N]}{(f_1, ...,f_n)}$ is thus, by definition, the supremum of the heights of such prime ideals. 
                            \item Let:
                                $$X \cong \Spec \frac{k[x_1, ..., x_N]}{(f_1, ..., f_n)}$$
                            and suppose that:
                                $$\dim_{\Krull} X = d$$
                            
                        \end{enumerate}
                    \end{proof}
        
        \subsection{Syntomicity and the art of building examples}
            \begin{proposition}[Smooth maps are syntomic] \label{prop: smooth_maps_are_syntomic}
                Smooth ring maps are syntomic (which in particular, implies that they are flat).
            \end{proposition}
                \begin{proof}
                    
                \end{proof}
     
    \section{de Rham cohomology over characteristic \texorpdfstring{$0$}{}} \label{section: algebraic_de_rham_cohomology_over_characteristic_0}
    
    \section{Generalities on singularities}
        \subsection{The \texorpdfstring{$\Proj$}{}-construction and blowups; resolution of curves}
            \subsubsection{\texorpdfstring{$\Proj$}{} of graded rings}
            
            \subsubsection{Blowups}
            
            \subsubsection{Application to resolving singular curves}
        
        \subsection{Smoothing out regular rings}
            \subsection{Singular ideals}
                \begin{definition}[Singular loci and singular ideals] \label{def: singular_loci}
                    The \textbf{singular locus} of a homomorphism of commutative rings $\varphi: R \to S$, denoted by $\frakS_{S/R}$ or $\frakS_{\varphi}$, is the following subset of $\Spec S$:
                        $$\frakS_{S/R} := \{\q \in \Spec S \mid \text{$\varphi$ is not smooth at $\q$}\}$$
                    The ideal associated to a singular locus is the \textbf{singular ideal}: in our case, the $S$-ideal $I(\frakS_{S/R})$ is the singular ideal associated to the ring map $\varphi: R \to S$.
                \end{definition}
                \begin{remark}[Unpacking the definition]
                    Definition \ref{def: singular_loci} is one of those definitions that are deceptively clean and simple. It asserts not much more than the fact that the singular locus of a ring map (or rather, of the corresponding morphism of affine schemes) is just the set of points/primes at which the map is not smooth. But it is precisely this reliance on smoothness (or lack thereof) that makes this definition so hard to work with. Smoothness, at its core, is a cohomological property: a ring map that is of finite presentation is smooth if and only if the associated (na\"ive) cotangent complex is quasi-isomorphic to a finitely generated projective module placed in degree $0$ \cite[\href{https://stacks.math.columbia.edu/tag/00T2}{Tag 00T2}]{stacks}. Thus, in checking if a ring map has a non-empty singular locus, one will need to make peace with having to examine this clunky cohomological property.  
                \end{remark}
                
                \begin{proposition}[Singular loci are radical ideals]
                    Let $\varphi: R \to S$ be a fixed homomorphism of commutative rings. Then, the singular ideal $I(\frakS_{S/R})$ is a radical ideal of $S$.
                \end{proposition}
                    \begin{proof}
                        By proposition \ref{prop: radical_properties}, radical ideals are precisely equal to the intersection of all primes containing it, i.e. an ideal $\a$ of $S$ is radical if and only if:
                            $$\a = \bigcap_{\p \in V(\a)} \p$$
                        so let us check if the singular ideal associated to a ring map satisfies this characterisation. 
                    \end{proof}
    
    \section{Resolutions of singularities}
        \subsection{Resolutions of singular surfaces}
        
        \subsection{Resolutions of singular varieties in higher dimensions}
        
    \section{Logarithmic geometry}
        \subsection{Log-schemes}
            \subsubsection{What do we mean by "logarithmic" ?}
            
            \subsubsection{Monoided topoi; immersions}
                \begin{definition}[Ideals] \label{def: ideals_in_symmetric_monoidal_categories}
                    Suppose that $\O$ is a symmetric monoidal category and denote its category of internal monoids by $\Mon(\O)$. An \textbf{ideal} of an object $A \in \Mon(\O)$, should it exist, shall then be nothing but a subobject of $A$ that is \textit{distinct} from $A$ and such that the canonical composite map $I \tensor A \to A \tensor A \to A$ (wherein the last factor map $A \tensor A \to A$ is the multiplication on $A$) admits an epi-mono factorisation:
                        $$
                            \begin{tikzcd}
                            	{I \tensor A} & A \\
                            	I
                            	\arrow[two heads, from=1-1, to=2-1]
                            	\arrow[tail, from=2-1, to=1-2]
                            	\arrow[from=1-1, to=1-2]
                            \end{tikzcd}
                        $$    
                \end{definition}
                \begin{remark}
                    As usual, there exists a natural partial order of ideals within any commutative monoid.
                \end{remark}
                
                \begin{definition}[Sums and products of ideals with other objects] \label{def: products_and_sums_of_ideals}
                    Let $(\O, \tensor, \1)$ be a symmetric monoidal category and let $I$ be an ideal of some commutative monoid $A \in \Comm\Mon(\O)$. 
                        \begin{itemize}
                            \item \textbf{(Products):} Its product with another element $M \in \O$ - henceforth denoted by $IM$ - shall thus be defined as the coequaliser of the kernel pair $I \tensor M \toto A \tensor M$ (of course, should the pullback and pushout, respectively, exist) induced by the inclusion $I \hookrightarrow A$.
                            \item \textbf{(Sums):} Let $\{M_s\}_{s \in S}$ be a family of objects of $\O$, and suppose that the coproduct $\coprod_{s \in S} M_s$ and product $\prod_{s \in S} M_s$ exist as objects of $\O$. Then, the sum $\sum_{s \in S} M_s$ shall be the coequaliser of the canonical kernel pair $\prod_{s \in S} M_s \toto \coprod_{s \in S} M_s$.  
                        \end{itemize}
                \end{definition}
                \begin{remark}
                    In the event that $(\O, \tensor, \1)$ is furthermore monoidally closed (i.e. it is enriched over itself and any tensoring functor $M \tensor -$ admits the internal hom $[M,- ]$ as its right-adjoint) and that $I$ is flat (i.e. the functor $I \tensor -$ is faithful), one has:
                        $$IM \cong I \tensor M$$
                    for all $M \in \O$.
                \end{remark}
                
                \begin{proposition}[Sums and products of ideals are ideals] \label{prop: products_and_sums_of_ideals_are_ideals}
                    Sums and products of ideals of a commutative monoid internal to any symmetric monoidal category (in the sense of definition \ref{def: products_and_sums_of_ideals}) are ideals of the same monoid.
                \end{proposition}
                    \begin{proof}
                                    
                    \end{proof}
                
                \begin{definition}[Prime ideals] \label{def: prime_ideals_in_symmetric_monoidal_categories}
                    Suppose that $\O$ is a symmetric monoidal category and denote its category of internal commutative monoids by $\Comm\Mon(\O)$. Then, an ideal $\p$ of some $A \in \Comm\Mon(\O)$ is said to be \textbf{prime} if and only if:
                        $$IJ \in \Ideals(\p) \implies J \in \Ideals(\p) \vee I \in \Ideals(\p)$$
                    where $\Ideals(-): \O \to \Sets$ is the functor assigning to objects $\O$ their (po)set of equivalence classes of subobjects. The set of prime ideals of a give commutative monoid $A$ is denoted by $\Spec A$.
                \end{definition}
                \begin{remark}[Maximal ideals are prime]
                    It is not hard to show that any maximal ideal of a commutative monoid is necessarily prime.
                \end{remark}
                
                For the next proposition, recall that for \textit{any} symmetric monoidal category $\O$, the subcategory $\Comm\Mon(\O)$ of commutative monoids internal to $\O$ is complete and cocomplete.
                \begin{proposition}[The Zariski topology] \label{prop: zariski_topology_on_symmetric_monoidal_categories}
                    Suppose that $\O$ is a symmetric monoidal category, and fix an object $A \in \Comm\Mon(\O)$. Then, there exists a topology on $\Spec A$ wherein sets given by:
                        $$V(I) := \{\p \in \Spec A \mid I \in \Ideals(\p)\}$$
                \end{proposition}
                    \begin{proof}
                        See proposition \ref{prop: zariski_closed_well_definiteness}.
                    \end{proof}
                \begin{corollary}[Small Zariski sites] \label{coro: small_zariski_sites_in_symmetric_monoidal_categories}
                    We can argue in a manner similar to remark \ref{remark: big_and_small_zariski_sites} to show that should $\O$ be a symmetric monoidal category then the full subcategory $\O^{\fp}$ of finitely presented objects would be small. Therefore, by proposition \ref{prop: zariski_topology_on_symmetric_monoidal_categories}, there naturally exists a small Zariski site $\Comm\Mon(\O)_{\Zar}^{\petit} := \Comm\Mon(\O^{\fp})_{\Zar}$. 
                \end{corollary}
                \begin{convention}[Spectral spaces]
                    Any topological space that is locally homeomorphic to the spectrum of a commutative monoid is said to be \textbf{spectral}. The Zariski sheaf topos over a spectral space $X$ shall be denoted by $\Sh(X_{\Zar})$. 
                \end{convention}
                
                \begin{definition}[Localisation of commutative monoids] \label{def: commutative_monoid_localisation}
                    Let $\O$ be a symmetric monoidal category with finite pullbacks and let $A$ be a commutative monoid, which we shall view as a symmetric monoidal category with one object (namely $A$) that is internal to $\O$ (see definition \ref{def: internal_categories} for the notion of internal categories, as well as an explanation of why we need to require that $\O$ has finite pullbacks). \say{Elements} of $A$ are thus simply endomorphisms on $A$, and thanks to the commutativity of the multiplication on $A$ (which means that the order in which one composes these endomorphisms does not matter), one can \textbf{localise} $A$ \textit{away} from some submonoid $S$ by formally inverting these arrows; the \textbf{localisation of $A$ away from $S$} is denoted by $A[S^{-1}]$. 
                \end{definition}
                \begin{remark}[Localisations are Zariski-open] \label{remark: localisations_of_monoids_are_open}
                    It is not hard to see that localisations are colimits that, while taken in the ambient symmetric monoidal categories, are legitimate colimits of monoids. From here, we can deduce that the prime spectrum of a localisation of a commutative monoid is necessarily open in the Zariski topology on said spectrum.
                \end{remark}
                
                \begin{definition}[Local monoids] \label{def: local_monoids}
                    A commutative monoid $A$ internal to a symmetric monoidal category $\O$ is said to be \textbf{local} if and only if it is isomorphic to every one of its localisation, i.e. for all submonoids $S$ of $A$, one has:
                        $$A[S^{-1}] \cong A$$
                \end{definition}
                \begin{remark}[Local monoids have unique maximal ideals]
                    It is not hard to show that any local commutative monoid must have one and only one unique maximal ideal.
                \end{remark}
                \begin{convention}[Localising monoids at prime ideal] \label{conv: localising_commutative_monoids_at_primes}
                    Let $\O$ be a symmetric monoidal category and let $\p \in \Spec A$ be a prime ideal of a commutative monoid $A$. Then, the localisation of $A$ \textbf{at} the prime ideal $\p$ (at the point $\p \in \Spec A$), denoted by $A_{\p}$, shall be the simultaneous localisation of $A$ at every submonoid that does \textit{not} contain $\p$. 
                \end{convention}
                \begin{lemma}[Localisations at primes are local] \label{lemma: localisations_at_primes_are_local}
                    Let $\O$ be a symmetric monoidal category and let $A$ be a commutative monoid therein. Then, any localisation $A_{\p}$ of $A$ at a prime ideal $\p$ shall be a local monoid.
                \end{lemma}
                    \begin{proof}
                        Use corollary \ref{coro: localisation_at_primes} as a guide.      
                    \end{proof}
                    
                \begin{lemma}[Monoids in topoi are local] \label{lemma: monoids_in_topoi_are_local}
                    Let $\E$ be a sheaf topos. Then, every commutative monoid internal to $\O$ is necessarily local.
                \end{lemma}
                    \begin{proof}
                    First of all fix a commutative monoid $A$ internal to $\E$. Next, note that because localisations are colimits (cf. remark \ref{remark: localisations_of_monoids_are_open}), and because the stalk of an object of a sheaf topos $\E$ is given by the left-adjoint component $x^*: \E \to \Sets$ of the geometric point:
                        $$
                            \begin{tikzcd}
                            	\Sets & \E
                            	\arrow[""{name=0, anchor=center, inner sep=0}, "{x_*}"', shift right=2, from=1-1, to=1-2]
                            	\arrow[""{name=1, anchor=center, inner sep=0}, "{x^*}"', shift right=2, from=1-2, to=1-1]
                            	\arrow["\dashv"{anchor=center, rotate=-90}, draw=none, from=1, to=0]
                            \end{tikzcd}
                        $$
                    we can simply work locally within the topos $\Sets$. Let us then break the proof down into two steps:
                        \begin{enumerate}
                            \item \textbf{(Non-units form a prime ideal):} Denote the subset of non-units of $A$ by $\m_A$ and consider $x, y \in A$ such that $xy \in \m_A$. Suppose then to the contrary that $\m_A$ is not prime, i.e. that neither $x$ nor $y$ are elements of $\m_A$. However, this would imply that $x$ and $y$ are both units (since $A^{\x} = A \setminus \m_A$ by construction), which means that $xy$ can not be an element of $\m_A$. This is a contradiction, which means that our assumption that neither $x$ nor $y$ are elements of $\m_A$ was wrong. We have thus shown that $\m_A$ is a prime ideal of $A$.  
                            \item \textbf{(The prime ideal of non-units is uniquely maximal):} Showing that $\m_A$ is maximal is easy: were it not maximal, there would exist a non-unit that would also not be an element of $\m_A$, which does not make sense because $\m_A$ contains all non-units. It now remains to show that $\m_A$ is the only maximal ideal of $A$. For this, suppose that there exist another maximal ideal $\n$ of $A$ which does not coincide with $\m_A$. Such an ideal can not contain any unit, of course, but this means that by virtue of containing only non-units, $\n$ is necessarily a subset of $\m_A$, and hence not maximal. $\m_A$ is therefore unique as a maximal ideal of $A$. 
                        \end{enumerate}
                    \end{proof}
                \begin{remark}[Local monoids and local rings] \label{remark: linearity_gives_rise_to_many_maximal_ideals}
                    One might ask: \say{But hold on, are commutative rings not internal commutative monoids too ? How can there be non-local rings then ?} and to that, we say: \say{It's because rings are monoids internal to linear symmetric monoidal categories, and more often than not, abelian symmetric monoidal categories. Topoi are neither of these things.} Vaguely speaking, the absence of linearity disallows monoids from having more than one maximal ideal. Explicitly,  
                \end{remark}
                \begin{example}
                    \noindent
                    \begin{itemize}
                        \item The only unit of $\N$ (viewed as an additive monoid internal to $\Sets$) is the element $0$, and so its unique maximal ideal is $\N^{\geq 1} := \N \setminus \{0\}$. This, however, is not its only prime ideal, since every monoid in $\Sets$ admits the empty set $\varnothing$ as the \say{zero} ideal. The space $\Spec \N$ thus has a generic point and a closed point, and nothing else.
                        \item Let $S^{-1}$ any additive subset (i.e. submonoid) of $\N$ and consider the additive monoid $\N[S^{-1}]$. It is clear that $\N[S^{-1}]^{\x} = \{0\} \cup S$, and hence the unique maximal ideal of $\N[S^{-1}]$ is $\N \setminus (\{0\} \cup S)$. 
                    \end{itemize}
                \end{example}
                \begin{definition}[Monoided topoi] \label{def: monoided_topoi}
                    A so-called \textbf{monoided topos} is the data of a pair $(\calX, \calM_{\calX})$ of:
                        \begin{itemize}
                            \item a topos $\calX$ (which we shall always take to be a sheaf topos, never an elementary topos) and
                            \item a distinguished commutative monoid object $\calM_{\calX}$ (which is guaranteed to be well-defined because sheaf topoi are symmetric-monoidal), called the \textbf{structure sheaf}.
                        \end{itemize}
                    Morphisms of monoided topoi are just the geometric morphisms between the underlying topoi. Additionally, note that because every commutative monoid internal to a topos is local, monoided topoi are tautologically locally monoided; however, there still exists a subcategory $\Spc^{\mon, \loc}$ spanned by geometric morphisms between monoided topoi whose stalks are local homomorphisms of concrete commutative monoids (i.e. commutative monoids internal to $\Sets$).
                \end{definition}
                
                \begin{definition}[Immersions] \label{def: immersions_of_monoided_spaces}
                    Let $(\Sh(Z), \calM_Z), (\Sh(X), \calM_X)$ be the monoided topos associated to topological spaces $Z$ and $X$ and let $j: Z \to X$ be a continuous function. Said continuous function will induce an \textbf{immersion} if and only if:
                        \begin{itemize}
                            \item it is injective, and 
                            \item the induced comorphism of structure sheaves:
                                $$j^{\sharp}: \calM_X \to j_*\calM_Z$$
                            is surjective.
                        \end{itemize}
                \end{definition}
                \begin{proposition}[Closed immersions are set-theoretically bijective] \label{prop: closed_immersions_of_monoided_spaces_are_bijective}
                    The continuous map that induces a given closed immersion of monoided spaces is necessarily a homeomorphism.  
                \end{proposition}
                    \begin{proof}
                                    
                    \end{proof}
                \begin{corollary}
                    Not every surjective monoid homomorphism induces a closed immersion.
                \end{corollary}
                
            \subsubsection{Log-structures}
                \begin{definition}[Logarithmic structures on ringed topoi] \label{def: log_structures}
                    Let $(\calX, \calO_{\calX})$ be a ringed topos.
                    \begin{enumerate}
                        \item \textbf{(Prelog-structures):}
                        \item \textbf{(Log-structures):}
                    \end{enumerate}
                \end{definition}
        
        \subsection{Deformations of log schemes}
	    
	    \chapter{Singularities}
    \begin{abstract}
        
    \end{abstract}
    
    \minitoc
    
    \section{The \texorpdfstring{$\Proj$}{}-construction and blowups; resolution of curves}
        \subsection{\texorpdfstring{$\Proj$}{} of graded rings}
        
        \subsection{Blowups}
        
        \subsection{Application to resolving singular curves}
    
    \section{Smoothing out regular rings}
        \subsection{Singular ideals}
            \begin{definition}[Singular loci and singular ideals] \label{def: singular_loci}
                The \textbf{singular locus} of a homomorphism of commutative rings $\varphi: R \to S$, denoted by $\frakS_{S/R}$ or $\frakS_{\varphi}$, is the following subset of $\Spec S$:
                    $$\frakS_{S/R} := \{\q \in \Spec S \mid \text{$\varphi$ is not smooth at $\q$}\}$$
                The ideal associated to a singular locus is the \textbf{singular ideal}: in our case, the $S$-ideal $I(\frakS_{S/R})$ is the singular ideal associated to the ring map $\varphi: R \to S$.
            \end{definition}
            \begin{remark}[Unpacking the definition]
                Definition \ref{def: singular_loci} is one of those definitions that are deceptively clean and simple. It asserts not much more than the fact that the singular locus of a ring map (or rather, of the corresponding morphism of affine schemes) is just the set of points/primes at which the map is not smooth. But it is precisely this reliance on smoothness (or lack thereof) that makes this definition so hard to work with. Smoothness, at its core, is a cohomological property: a ring map that is of finite presentation is smooth if and only if the associated (na\"ive) cotangent complex is quasi-isomorphic to a finitely generated projective module placed in degree $0$ \cite[\href{https://stacks.math.columbia.edu/tag/00T2}{Tag 00T2}]{stacks}. Thus, in checking if a ring map has a non-empty singular locus, one will need to make peace with having to examine this clunky cohomological property.  
            \end{remark}
            
            \begin{proposition}[Singular loci are radical ideals]
                Let $\varphi: R \to S$ be a fixed homomorphism of commutative rings. Then, the singular ideal $I(\frakS_{S/R})$ is a radical ideal of $S$.
            \end{proposition}
                \begin{proof}
                    By proposition \ref{prop: radical_properties}, radical ideals are precisely equal to the intersection of all primes containing it, i.e. an ideal $\a$ of $S$ is radical if and only if:
                        $$\a = \bigcap_{\p \in V(\a)} \p$$
                    so let us check if the singular ideal associated to a ring map satisfies this characterisation. 
                \end{proof}

    \section{Resolutions of singular surfaces}
    
    \section{Resolutions of singular varieties in higher dimensions}
	    
	    \chapter{Deformation theory}
    \begin{abstract}
        
    \end{abstract}
    
    \minitoc

    \section{Schlessinger's classical deformation theory}
        \subsection{Thickenings and deformation functors}
            \subsubsection{An obesity epidemic among Artinian rings}
                Since the notion is somewhat less popular than that of being Noetherian, let us first recall what it means for a commutative ring to be Artinian, and what the topological consequences of this condition are.
                
                \begin{definition}[Artinian rings] \label{def: artinian_rings} \index{Artinian rings}
                    A commutative ring $\Lambda$ is said to be \textbf{Artinian} if and only if there exist no non-terminating descending chain of ideals (up to bijections, of course). Alternatively, since ideals of commutative rings corespond to Zariski-closed sets, one can define Artinian rings $\Lambda$ as those whose prime spectra $|\Spec \Lambda|$ are topological spaces with no non-terminating \textit{ascending} chains of closed subsets. 
                    \\
                    It is not hard to see that for every given base commutative ring $R$ there is a category whose objects are local Artinian $R$-algebras and whose morphisms are local homomorphisms between them. We shall denote this category by ${}^{R/}\Comm\Alg^{\loc, \Art}$. Furthermore, for each $R$-algebra $k$ that is a field, there is a corresponding full subcategory of ${}^{R/}\Comm\Alg^{\loc, \Art}$, which we shall denote by ${}^{R/}\Comm\Alg_{/k}^{\loc, \Art}$ spanned by local Artinian $R$-algebras whose residue field is $k$. 
                \end{definition}
                
                \begin{proposition}[Basic properties of local Artinian rings] \label{prop: artinian_rings_properties}
                    In deformation theory, we are greatly interested in local Artinian rings, for reasons that we shall divulge later (in remark \ref{remark: why_local_artinian_rings}), so let us try to establish some elementary and (hopefully) geometrically intuitive properties of theirs.
                        \begin{enumerate}
                            \item Quotients and localisations of Artinian rings are also Artinian.
                            \item \cite[\href{https://stacks.math.columbia.edu/tag/00J6}{Tag 00J6}]{stacks} Finitely generated algebras over fields are Artinian.
                            \item A local Artinian ring $(\Lambda, \m)$ with residue field $\kappa$ is a finitely generated $\kappa$-algebra and admits the splitting:
                                $$\Lambda \cong \kappa \oplus \m$$
                            \item \cite[\href{https://stacks.math.columbia.edu/tag/00J7}{Tag 00J7}]{stacks} Artinian rings only have finitely many maximal ideals.
                            \item \cite[\href{https://stacks.math.columbia.edu/tag/00J8}{Tag 00J8}]{stacks} Let $A$ be an Artinian ring. Then, its Jacobson radical is nilpotent. In fact, its Jacobson radical shall be the same as its nilradical.
                            \item \cite[\href{https://stacks.math.columbia.edu/tag/00JA}{Tag 00JA}]{stacks} Any commutative ring with finitely many maximal ideals and locally nilpotent Jacobson radical (such as Artinian rings) can be decomposed into the direct sum of its localisations at the maximal ideals. Furthermore, any prime ideal in such a ring is automatically maximal.
                            \item \cite[\href{https://stacks.math.columbia.edu/tag/00JB}{Tag 00JB}]{stacks} A commutative ring $A$ is simultaneously Artinian and Noetherian if and only if $A$ has finite length as a module over itself. 
                        \end{enumerate}
                \end{proposition}
                    \begin{proof}
                        \noindent
                        \begin{enumerate}
                            \item Let $A$ be an Artinian ring, which we shall view as the Artinian toplogical space $|\Spec A|$. If we would recall that localisations and quotients of $A$ correspond to Zariski-open and Zariski-closed subsets of $|\Spec A|$ respectively (cf. lemma \ref{lemma: localisations_are_open}, corollary \ref{coro: localisation_at_primes}, and corollary \ref{coro: quotients_are_closed}), then we would see how localisations and quotients of Artinian rings being Artinian themselves is entirely evident as a topological phenomenon. 
                            \item Let $k$ be a field and let $A$ be a finitely generated $k$-algebra. Such an algebra is, first and foremost, a finite-dimensional $k$-vector space, and ideals of which are (necessarily finite-dimensional) vector subspaces. This tells us that all descending chains of $A$-ideals are just certains chains of finite-dimensional vector subspaces of the finite-dimensional vector space $A$; they therefore must all be finite length, and they thus all terminate. This means that $A$ is Artinian by definition.
                            \item First of all, we need to show that $\Lambda$ is an algebra over its residue field $\kappa$, i.e. that there exists a homomorphism of commutative rings:
                                $$\kappa \to \Lambda$$
                            
                            \item Suppose to the contrary that there is an Artinian ring $A$ with infinitely many distinct maximal ideal, and without loss of generality, let us also assume that it has \textit{countably} many maximal ideals; let us organise them into a sequence $\{\m_n\}_{n \in \N}$. From such a sequence, we can construct the following descending chain of $A$-ideals:
                                $$\m_0 \supseteq \m_0 \cap \m_1 \supseteq ... \supseteq \bigcap_{n \in \N} \m_n$$
                            But this is manifestly an \textit{infinite} chain of ideals (note that intersections of ideals are still ideals), which means that its existence violates the Artinian assumption on $A$. Thus, $A$ can have only finitely many maximal ideals.
                            
                            One thing to note is that this proof does not imply that every Artinian rings only have finitely many proper ideals. 
                            \item Recall firstly, that the Jacobson radical of a commutative ring $A$ is defined to be the intersection of all its maximal ideals (which we note to still be an ideal, as intersections of ideals are ideals):
                                $$\J(A) = \bigcap_{\m \in |\Spm A|} \m$$
                            and also, that the nilradical of a commutative ring is the same as the intersection of all its prime ideals (cf. proposition \ref{prop: radical_properties}):
                                $$\Nil(A) = \bigcap_{\p \in |\Spec A|} \p$$
                            Then, consider the following (for which we shall assume that the Law of Excluded Middle holds):
                                $$
                                    \begin{aligned}
                                        & x \in \Nil(A)
                                        \\
                                        \iff & x \in \bigcap_{\p \in |\Spec A|} \p
                                        \\
                                        \iff & \bigwedge_{\p \in |\Spec A|} (\p \ni x)
                                        \\
                                        \iff & \neg \neg \bigwedge_{\p \in |\Spec A|} (\p \ni x)
                                        \\
                                        \iff & \neg \bigvee_{\p \in |\Spec A|} \neg (\p \ni x)
                                        \\
                                        \iff & \bigwedge_{\p \in |\Spec A|} \neg \left(\p \in D_A(x)\right)
                                        \\
                                        \iff & \bigwedge_{\p \in |\Spec A|} \left(\p \in V_A(x)\right)
                                    \end{aligned}
                                $$
                            wherein the last line holds thanks to remark \ref{remark: basic_opens_complements}. We can also obtain the following through reasoning similarly:
                                $$x \in \J(A) \iff \bigwedge_{\m \in |\Spm A|} \left(\m \in V_A(x)\right)$$
                            and because maximal ideals are prime (which in particular means that $|\Spm A|$ is a subset of $|\Spec A|$), these tell us that:
                                $$\left(x \in \Nil(A)\right) \implies \left(x \in \J(A)\right)$$
                            i.e.:
                                $$\Nil(A) \subseteq \J(A)$$
                            Now, suppose to the contrary that the complement $\J(A) \setminus \Nil(A)$ is non-empty, and from the above analysis, we know that this would imply the existence of non-maximal primes in $V_A(x)$ for all $x \in \Nil(A)$.  
                            \item
                            \item 
                        \end{enumerate}
                    \end{proof}
                \begin{corollary}
                    \cite[\href{https://stacks.math.columbia.edu/tag/00JB}{Tag 00JB}]{stacks} Let $A$ be a commutative ring that is simultaneously Artinian and Noetherian. Then, all primes of $A$ are maximal ideals and there are only finitely many of them. As a consequence, $|\Spec A|$ is a totally disconnected set in the Zariski topology consisting merely of finitely many closed points.
                \end{corollary}
                    \begin{proof}
                        
                    \end{proof}
                
                \begin{definition}[Deformation categories] \label{def: deformation_categories}
                     
                \end{definition}
                \begin{remark}[Why local Artinian rings] \label{remark: why_local_artinian_rings}
                    
                \end{remark}
        
            \subsubsection{Deformation functors}
        
        \subsection{Schlessinger's criterion}
    
    \section{Deformations in derived algebraic geometry}
        \begin{convention}[Everything is derived!] \label{conv: deformation_theory_everything_is_derived}
            \noindent
            \begin{itemize}
                \item From now on until the end of the chapter, everything will be assumed to be derived. 
                \item By $\Cat^1$, or simply $\Cat$, we shall actually mean ${}^{(\infty, 1)}\Cat^1$, i.e. the $(\infty, 1)$-category of $(\infty, 1)$-categories and functors between them, and by $\Cat^2$ we will be referring to the $(\infty, 2)$-category of $(\infty, 1)$-categories, functors between them, and natural transformations between these functors. 
                
                Similarly, by $\Grpd^1$, or simply $\Grpd$, we will actually mean the $(\infty, 1)$-category of $\infty$-groupoids and functors between them, and by $\Grpd^2$, we shall mean the $(\infty, 2)$-category of $\infty$-groupoids, functors between them, and natural transformations between these functors.
                \item A subcategory of $\Cat$ this is of particular interest is $\Cat^{\dg, \cont}$, the $(\infty, 2)$-category of stable linear (i.e. differential-graded) $(\infty, 1)$-categories (see section \ref{section: homological_algebra} for the notion of stable $(\infty, 1)$-categories). Of course, we can also view $\Cat^{\dg, \cont}$ as a mere $(\infty, 1)$-category.
            \end{itemize}
        \end{convention}
        
        \subsection{Admittance of deformations}
            \subsubsection{Differential cohesiveness}
                
        
            \subsubsection{Spaces admitting deformations}
                \begin{definition}[Prestacks admitting deformations] \label{def: prestacks_admitting_deformations}
                    Let $k$ be an arbitrary base commutative ring. A prestack $\calX$ on ${}^{k/}\Comm\Alg^{\op}$ is said to \textbf{admit deformations} if it satisfies the following conditions:
                        \begin{itemize}
                            \item \textbf{(Convergence):} $\calX$ is convergent, i.e. for all affine schemes $S$ over $\Spec k$, one has:
                                $$\calX(S) \cong \underset{n \in \N}{\lim} \calX({}^{\leq n}S)$$
                            \item \textbf{(Admittance of a cotangent complex):} $\calX$ must admit a pro-cotangent complex. Should said pro-cotangetn complex be an actual cotangent complex, then we will say that $\calX$ \textbf{admits corepresentable deformations}.
                            \item \textbf{(Cohesiveness):} $\calX$ has to be differentially cohesive.
                        \end{itemize}
                \end{definition}
            
            \subsubsection{Consequences of admitting deformations}
            
    \section{Formal schemes and inf-schemes} \label{section: formal_schemes_and_inf_schemes}
        \subsection{Formal schemes}
            \subsubsection{The geomety of formal schemes}
                This subsubsection will, for the most part, rather straightforward and formal, owing to the fact that ind-schemes are defined rather simply.
                
                We start first of all with that simple definition of ind-schemes.
                \begin{definition}[Ind-schemes] \label{def: ind-schemes}
                    Let $k$ be an arbitrary base commutative ring and let $\kappa$ be a regular cardinal\footnote{Which we will never mention again beyond this definition}. The category of ind-schemes over $\Spec k$ is thus the $\kappa$-ind-completion $\Ind_{\kappa}({}^{< \infty}\Sch_{/\Spec k})$ of the category ${}^{< \infty}\Sch_{/\Spec k}$ of convergent schemes over $\Spec k$. This category shall be denoted by $\Ind\Sch_{/\Spec k}$.
                \end{definition}
                \begin{remark}[Ind-schemes vs. formal schemes] \label{remark: ind_schemes_vs_formal_schemes}
                    It should be noted while formal schemes are trivially ind-schemes, the converse statement is not necessarily true. This is because formal schemes are (small) filtered colimits of \textit{quasi-compact} schemes taken along \textit{closed immersions}. This use of terminologies is slightly contradictory to that of \cite[Definition I.2.1.1.2]{GR2}, wherein the authors define ind-schemes as what we refer to here as formal schemes. We have chosen to make this modification both to keep to a more traditional and popular etymological convention, but also, to put emphasis on the fact that unlike general ind-schemes, formal schemes are not \textit{just} filtered colimits of schemes, but rather certain special filtered colimits.
                    
                    For topological reasons (cf. proposition \ref{prop: topologically_complete_adic_modules}), we usually would want to work with Noetherian formal schemes. Note that because Noetherian topological spaces are \textit{a priori} quasi-compact, but there exist quasi-compact spaces that are not Noetherian (e.g. finite disjoint unions of non-Noetherian spaces), the category $(\Sch^{\wedge})^{\Noeth}$ of \textit{Noetherian} formal schemes are not quite the same as the \textit{sub}category of the category $\Ind\Sch^{\qc, \closed}$ of filtered colimits of closed immersions of quasi-compact schemes. The category spanned by closed immersions of \textit{locally} Notherian quasi-compact ind-schemes, however, is precisely equivalent to that of formal schemes. In short, one has the following chain of containment of categories:
                        $$\Ind\Sch^{\qc, \closed, \loc\Noeth} \cong (\Sch^{\wedge})^{\Noeth} \subset \Sch^{\wedge} \cong \Ind\Sch^{\qc, \closed} \subset \Ind\Sch^{\qc} \subset \Ind\Sch$$
                \end{remark}
                \begin{remark}[Morphisms of formal schemes] \label{remark: morphisms_of_formal_schemes}
                    Due to the fact that filtered colimits commute with finite limits, the category $\Ind\Sch^{\qc, \closed}$ spanned by filtered colimits of closed immersions of quasi-compact schemes (which is the same as the category $\Sch^{\wedge}$ of formal schemes), as well as any subcategories thereof, has only monomorphisms as arrows (recall that closed immersions are monomorphic). One can then also rather easily show that these monomorphisms of ind-schemes are actually closed immersions themselves. This, first of all, justifies the isomorphism:
                        $$\Sch^{\wedge} \cong \Ind\Sch^{\qc, \closed}$$
                    and second of all, tells us that the \href{https://ncatlab.org/nlab/show/skeleton}{\underline{skeleton}} of the category $\Sch^{\wedge}$ is a partial order, wherein the ordering is given by the canonical closed immersions. 
                \end{remark}
                
                \begin{definition}[Descriptors for formal scheme] \label{def: formal_schemes_descriptors}
                    There are many descriptors that one can use to describe formal schemes. Notable examples are:
                        \begin{itemize}
                            \item $n$-coconnective (for some natural number $n$),
                            \item affine,
                            \item (locally) almost of finite type,
                            \item classical,
                            \item reduced.
                        \end{itemize}
                    and so on. Let $\scrP$ be any one of these properties, or properties which imply any one of the above (e.g. smoothness, as it implies being of finite type). Then, the subcategory $(\Sch^{\wedge})^{\scrP}$ of formal schemes with property $\scrP$ shall be nothing but the ind-completion of $\Sch^{\qc, \closed, \scrP}$, the category of quasi-compact schemes with the same property $\scrP$ and closed immersions between them, i.e.:
                        $$(\Sch^{\wedge})^{\scrP} \cong \Ind(\Sch^{\qc, \closed, \scrP})$$
                \end{definition}
                \begin{remark}
                    Let $\scrP$ be a property as elaborated on above. Then, one can also characterise the category $(\Sch^{\wedge})^{\scrP}$ via:
                        $$(\Sch^{\wedge})^{\scrP} \cong \Sch^{\wedge} \cap (\Spec \Z)^{\scrP}$$
                    wherein the \say{intersection} is taken at both the level of objects and of (higher) morphisms.
                \end{remark}
                
                \begin{remark}[Other geometric facts about formal schemes] \label{remark: geometric_facts_about_formal_schemes}
                    \noindent
                    \begin{itemize}
                        \item \textbf{(Formal schemes are sheaves):} As schemes satisfy Zariski, \'etale, fppf, and fpqc descent, so do formal schemes (or more generally, ind-schemes). This is due to the fact that sheaf topoi are cocomplete.
                        \item \textbf{(Formal schemes preserve coconnectivity of quasi-compact schemes):} Let $\calX$ be a formal scheme and let $S \in {}^{\leq n}\Sch^{\qc}$ be an $n$-coconnective quasi-compact scheme. Then, the space $\calX(S)$ of $S$-points of $\calX$ will be $n$-truncated. To see why this ought to be true, note first of all that the truncation level of $\calX(S)$ has to be finite, as $\calX$ is convergent by definition. Second of all, 
                        
                        As a corollary, one sees that should $\calX$ be isomorphic to say, $\underset{i \in I}{\colim} X_i$, where $\{X_i\}_{i \in I}$ is small filtered diagram of closed immersions of quasi-compact schemes, then:
                            $$\calX(S) \cong \underset{i \in I}{\colim} X_i(S)$$
                        wherein $S$ is as above.
                    \end{itemize}
                \end{remark}
                
            \subsubsection{Deformations of formal schemes}
        
        \subsection{(Ind)-inf-schemes}
        
        \subsection{Ind-coherent sheaves on ind-(inf)-schemes}
            \subsubsection{Ind-coherent sheaves on formal schemes}
            
            \subsubsection{Ind-coherent sheaves on ind-(inf)-schemes}
        
    \section{Formal moduli}
        \begin{convention}[Everything is derived!] \label{conv: moduli_everything_is_derived}
            \noindent
            \begin{itemize}
                \item From now on until the end of the chapter, everything will be assumed to be derived. 
                \item By $\Cat^1$, or simply $\Cat$, we shall actually mean ${}^{(\infty, 1)}\Cat^1$, i.e. the $(\infty, 1)$-category of $(\infty, 1)$-categories and functors between them, and by $\Cat^2$ we will be referring to the $(\infty, 2)$-category of $(\infty, 1)$-categories, functors between them, and natural transformations between these functors. 
                
                Similarly, by $\Grpd^1$, or simply $\Grpd$, we will actually mean the $(\infty, 1)$-category of $\infty$-groupoids and functors between them, and by $\Grpd^2$, we shall mean the $(\infty, 2)$-category of $\infty$-groupoids, functors between them, and natural transformations between these functors.
                \item A subcategory of $\Cat$ this is of particular interest is $(\Cat^{\dg, \cont})^2$ (or simply $\Cat^{\dg, \cont}$), the $(\infty, 2)$-category of stable linear (i.e. differential-graded) $(\infty, 1)$-categories (see section \ref{section: homological_algebra} for the notion of stable $(\infty, 1)$-categories). Of course, we can also view $\Cat^{\dg, \cont}$ as a mere $(\infty, 1)$-category; when necessary, we shall write $(\Cat^{\dg, \cont})^1$ to put emphasis on the disregard of $2$-morphisms.
            \end{itemize} 
        \end{convention}
    
        \subsection{Formal moduli problems}
            \begin{definition}[Formal moduli problems] \label{def: formal_moduli_problems}
                Let $k$ be a commutative ring and let $\calX$ be an object of $[\Spec k]^{\laft}$ (i.e. a prestack over $\Spec k$ that is locally almost of finite type) and define the category $\Moduli_{/\calX}$ of \textbf{formal moduli problems over $\calX$} to be the full subcategory $[\Spec k]^{\laft, \defm}_{/\calX}$ spanned by nil-isomorphisms:
                    $$\calY \to \calX$$
                which are \textit{inf-schematic} (note that inf-schematic prestacks necessarily admit deformations).
            \end{definition}
        
        \subsection{Formal groupoids}
	
	\part{Algebraic number theory}
	    \chapter*{Introduction}
    \begin{abstract}
        
    \end{abstract}
    
    \minitoc
    
    \section{The Global Correspondence}
        Let $X$ be a curve that is smooth, proper, and geometrically connected algebraic curve (for instance, we can take $X$ be an elliptic curve or $\P^1$) and suppose that $G$ is a reductive group (think $\GL_n$ or $\SL_n$, or more concretely, $\GL_1$, or groups of diagonal matrices); both shall be over a field $k$ of characteristic $0$. Additionally, denote the function field of our curve $X$ by $K_X$, the completions of said field at (closed) points $x \in |X|$ by $K_{X, x}$, and we shall write $\scrO_{X, x}$ for the associated rings of integers (note how they coincide with the adic completions $\calO_{X, x}^{\wedge}$).
            
        The end goal for us, shall be to construct some semblance of an equivalence of derived/abelian/stable $\infty$-categories:
            $$\Dmod\left(\Bun_G(X)\right) \cong \Ind\Coh\left(\LocSys_F(X)^{\check{G}}\right)$$
        between:
            \begin{itemize}
                \item the category $\Dmod\left(\Bun_G(X)\right)$ of D-modules on the moduli stack $\Bun_G(X)$ of $G$-bundles on $X$, and
                \item the category $\Ind\Coh\left(\LocSys_F(X)^{\check{G}}\right)$ of ind-coherent sheaves (cf. section \ref{section: indcoh}) on the moduli stack of $\check{G}$-equivariant local systems on $X$ with coefficients in some implicitly understood suitable field $F$. 
            \end{itemize}
        When $G$ is a torus - i.e. when it is abelian - the above correspondence is a bit simpler:
            $$\Dmod\left(\Bun_G(X)\right) \cong \QCoh\left(\LocSys_F(X)^{\check{G}}\right)$$
        (notice how now, we can work with the entire category of quasi-coherent sheaves instead of having to restrict ourselves to ind-coherent sheaves). One thing that needs to be made clear right away, however, is that aside from a few very special cases such as $G = \GL_1$ and $G = \SL_2$, this equivalence is \textit{entirely conjectural}. Nevertheless, we do have a rough idea of how to eventually obtain a proper theorem from this vision:
            \begin{enumerate}
                \item The very first thing to do is to understand the construction of D-modules on (pre)stacks locally of finite type, and we can do this by learning about crystals (in the sense of Grothendieck) and their infinitesimal/crystalline cohomology over base fields of characteristic $0$ (crystalline cohomology over base fields of positive characteristics and the accompanying theory of arithmetic D-modules is significantly more complicated than their characteristic $0$ counterparts, which incidentally is why we have required that $\chara k = 0$).
                \item Then, we must know what $\check{G}$ actually is, i.e. we must understand Langlands duals. There is a tool for this, which is the Geometric Satake Equivalence. However, we are going to have to go through two substeps:
                    \begin{enumerate}
                        \item To begin, we shall need to understand what the affine Grassmannian is and its roles in the representation theory of algebraic groups.
                        \item We shall also have to know what it means to have a group act upon a (nice enough) category so as to be able to define the category of so-called \textbf{spherical D-modules}, which are certain kinds of equivariant D-modules.
                        \item We shall then establish the Geometric Satake Correspondence to be a Tannakian equivalence:
                            $$\Rep^{\heart}_F(\check{G}_{K_{X, x}}) \cong \Sph^{\heart}_{G, X, x}$$
                        between the hearts of the t-structures of the rigid monoidal derived categories of $F$-linear representations of the $K_{X, x}$-points of the Langlands dual group $\check{G}$ and of $G(\scrO_{X, x})$-equivariant/spherical D-modules over the local affine Grassmannian $\Gr_{G, X, x}$.
                    \end{enumerate}
                \item Lastly, we shall seek to understand the subtle technical differences between quasi-coherent sheaves and ind-coherent sheaves, and why restricting ourselves to the case of tori allows us to forego the ind-coherent sheaf machinery. 
            \end{enumerate}
        Of course, before embarking on this journey, we might also want to learn some (derived) algebraic geometry, which will help us understand $\Bun_G(X)$ and $\LocSys_F(X)^{\check{G}}$, what these categories have to do with the theory of Galois representations (because at the end of the day, the Langlands Programme is all about understanding higher reciprocity laws), or even simply why we have required that our curve $X$ is smooth (spoiler: smoothness helps us identify $\QCoh(X)$ with the category $\QCoh(X)^{\perf}$ of perfect complexes on $X$), proper, and geometrically connected, beyond wanting our machineries to be applicable to important classes of examples such as elliptic curves and abelian varieties. For details, see chapters \ref{chapter: schemes} and \ref{chapter: cohomology_and_derived_schemes}.
        
        We should also make some remarks about the above equivalence of categories as well. Thanks to Grothendieck's Galois theory, the left-hand side can be thought of as the \say{\textbf{Automorphic Side}} of the Langlands Correspondence, which holds information about Galois representations. Drawing inspiration from another one of Grothendieck's major contributions, $\ell$-adic \'etale cohomology, the right-hand side in turn can be thought of as the \say{\textbf{Spectral Side}}, which tells interesting stories\footnote{Fairy tales, really...} through harmonic analysis.
    
        \subsection{The Categorical-Geometric Langlands Correspondence for algebraic tori}
            This section, as the title suggests, shall be dedicated to outlining our hopes and dreams (or the lack thereof) for a Categorical-Geometric Langlands Correspondence for algebraic tori; specifically, we would like to present of a list of key results known to be involved in a proof of the Correspondence. We will also give run-down of the various technical tools used for establishing said key results. 
            
            \subsubsection{Equivariant local systems}
        
            \subsubsection{The Fourier-Muka\"i-Laumon Transform}
            
            \subsubsection{Factor-wise Langlands duality}
            
        \subsection{The Conjecture for non-abelian groups}
        
        \subsection{Outline of the proof for the case of \texorpdfstring{$G = \GL_2$}{}}
        
    \section{The Local Correspondence for complex loop groups}
        \subsection{The appearance of Langlands parameters}
            Consider the formal loop group $G(\!(t)\!)$ associated to some chosen connected complex reductive group $G$. 
            
            Let us start by describing the absolute Galois group of the field $\bbC(\!(t)\!)$. First of all, notice that:
                $$\Gal(\bbC(\!(t^{\frac1n})\!)/\bbC(\!(t)\!)) \cong \Z/n\Z$$
            and so:
                $$\Gal(\overline{\bbC(\!(t)\!)}/\bbC(\!(t)\!)) \cong \hat{\Z}$$
            which is a canonical homeomorphism of topological groups obtained via the Fundamental Theorem of Galois Theory. Now, one thing to note is that for some fixed power $q$ of a prime $p$, one also has:
                $$\Gal(\overline{\F_q}/\F_q) \cong \hat{\Z}$$
            but unlike the complex case, the group $\Gal(\overline{\F_q(\!(t)\!)}/\F_q(\!(t)\!))$ surjects (continuously) onto the non-trivial group $\Gal(\overline{\F_q}/\F_q)$ ($\bbC$ is algebraically closed so $\Gal(\bar{\bbC}/\bbC)$ is trivial), a fact known through local class field theory. As a consequence, describing the Weil group (and by extension, Weil-Deligne representations thereof) attached to $\bbC(\!(t)\!)$ will - hopefully - be somewhat simpler than that of $\F_q(\!(t)\!)$ and might therefore help us gain insight into the nature of the Langlands Correspondence. Better yet, we have via Grothendieck's Galois Theory, that:
                $$\Gal(\overline{\bbC(\!(t)\!)}/\bbC(\!(t)\!)) \cong \hat{\Z} \cong \pi_1^{\et}(\bbD^{\x}_{\bbC})$$
            wherein $\bbD^{\x}_{\bbC} \cong \Spec \bbC(\!(t)\!)$; through the discussion above, one sees that this is not the case for $\bbD^{\x}_{\F_q}$, i.e.:
                $$\Gal(\overline{\F_q(\!(t)\!)}/\F_q(\!(t)\!)) \not \cong \pi_1^{\et}(\bbD^{\x}_{\F_q})$$
            Since representations of the (\'etale) fundamental group correspond to certain D-modules, we essentially have access to the theory of D-modules in studying the Langlands Correspondence for the case of $G\!(t)\!)$, which roughly postulates a bijective relationship between homomorphisms $W_{\bbC(\!(t)\!)} \to \check{G}$ and certain representations of $G(\!(t)\!)$.
        
        \subsection{Representations of loop groups; Kac-Moody algebras}
    
    \section{Deformation quantisation of the Local Correspondence}
	
	    \chapter{\'Etale cohomology for schemes} \label{chapter: etale_cohomology_1}
    \begin{abstract}
        Galois theory via algebraic topology ? Game on!
    \end{abstract}
    
    \minitoc
    
    \section{Pr\'elude: Primes in integral extensions}
        \subsection{Behaviours of prime ideals in integral extensions}
            \subsubsection{Finite and integral extensions}
                \begin{definition}[Integral extensions] \label{def: integral_extensions} \index{Integral! extensions} \index{Integral! elements}
                    \noindent
                    \begin{enumerate}
                        \item \textbf{(Integral elements):} Let $A$ be a subring of a commutative ring $B$ (i.e. let their exist monic ring homomorphisms from $A$ to $B$). An element $b \in B$ will be called \textbf{integral} over $A$ if and only if $A[b]$ is a finitely generated commutative $A$-algebra; otherwise, it is called \textbf{transcendental}. When $A$ is a field, what one recovers are the notions of algebraic and transcendental elements (for instance, $\sqrt{2}$ is integral over $\Z$, as it is algebraic over $\Q$, whereas a formal variable $x$ is neither). 
                        \item \textbf{(Integral extensions):} A homomorphism between commutative rings $A \to B$ will be called an integral extension if all elements of $B$ are integral over $B$. Alternatively (and perhaps less confusingly), one may view an integral extension of a commutative ring $A$ as a tensor product $\bigotimes_{i \in I} A[b_i]$ wherein $\{b_i\}_{i \in I}$ is a (possibly infinite) set of elements that are integral over $A$; note that one has the following canonical isomorphism of commutative $A$-algebras:
                            $$\bigotimes_{i \in I} A[b_i] \cong A\left[\{b_i\}_{i \in I}\right]$$
                        thanks to the fact that left-adjoints (the polynomial ring free construction) commutes with colimits (tensor products of commutative algebras). Additionally, integral extensions are trivially injective ring homomorphisms.
                        \item \textbf{(Integral closures):} Let $\varphi: A \to B$ be a homomorphism of commutative rings. Then, the integral closure of $A$ inside $B$ (denoted by $\overline{A}$, $\overline{A_B}$, or $\overline{A_{\varphi}}$) is the subset of $B$ consisting of \textit{all} elements that are integral over $A$. It is not hard to show that integral closures are actually subrings of the codomains, and with this in mind, one can see that integral closures may be viewed as maximal integral extensions inside given commutative rings; this description can be succinctly summed up by the following expression:
                            $$\overline{A_{\varphi}} \cong \bigotimes_{\underset{\text{$b$ integral over $\im \varphi$}}{b \in B}} A[b]$$
                    \end{enumerate}
                \end{definition}
                \begin{example}[The ring of integers of a number field] \label{example: ring_of_integers}
                    \noindent
                    \begin{enumerate}
                        \item \textbf{(The global case):} Before we try to give a description of rings of integers inside global fields, let us fix a definition. To us, a global field is either a finite (hence \textit{a priori} algebraic) extension of $\Q$ or of $\F_p(t)$, the field of Laurent series with coefficients coming from the finite field of prime order $p$. With this definition in mind, let us then define the ring of integers of a global field $F$ as the maximal commutative ring (in terms of cardinality) that is integrally closed inside $F$. For example, $\Z$ is the ring of integers of $\Q$, but $\Z$ is not of $\R$ (not that $\R$ is a global field, nor can we even define the ring of integers of $\R$ anyway). This definition, while conceptually intuitive, is not very practical. That is because it is not entirely clear how one might trickle from a given global field down onto its largest subring that is integrally closed. Thus, one can define the ring of integers $\scrO_F$ of a global field $F$ alternatively as the set of all elements of $F$ that are integral over $\Z$ (in the event that $\chara F = 0$) or over $\F_p[t]$ (if $\chara F = p$, for some prime $p$). Per this definition, rings of integers are automatically closed inside their corresponding global fields. Furthermore, all global fields are equal to the field of fractions of their rings of integers.
                        \item \textbf{(The local case):} As above, let us first try to agree upon a notion of local fields: an \textit{archimedean} local field is a finite extension of $\R$ (so actually, just $\R$ and $\bbC$), and a \textit{non-archimedean} local field is either a finite extension of $\Q_p$ (i.e. a $p$-adic number field) - which we note to be of mixed characteristic $(0,p)$ - or a finite extension of $\F_p(\!(t))\!$ - which we note to be of equicharacteristic $(p, p)$ - i.e. the field of formal Laurent series over the finite field of order $p$. One can do some work to see that given a local field $K$, one can define a suitable sort of \say{absolute value} $|-|$ on it, and with respect to such an absolute value, one obtains either an archimedean metric topology or a non-archimedean ultrametric topology (hence the names). Then, consider the \textit{closed} unit ball inside $K$, i.e. the set:
                            $$\scrO_K := \{x \in K \mid |x| \leq 1\}$$
                        (for instance, $\Z_p$ and $\F_p[\![t]\!]$ are the closed unit balls inside $\Q_p$ and $\F_p(\!(t)\!)$ respectively, and $[-1, 1]$ is the the unit ball inside $\R$). In the non-archimedean case, this turns out to be a subring of $K$, which we dub the ring of integers of $K$. Interestingly, the ring of integers of a non-archimedean local field is integrally closed and one can show this by first showing that the \textit{open} unit ball inside $K$, i.e. the set:
                            $$\m_K := \{x \in K \mid |x| < 1\}$$
                        is the (necessarily unique) maximal ideal of $\scrO_K$; then, 
                        \\
                        Of course, one could also define the ring of integers of a non-archimedean local field $K$ as the set of all elements in $K$ that are integral over either $\Z_p$ or $\F_p[\![t]\!]$ (corresponding to $\chara K = 0$ and $\chara K = p$ respectively) and then show that such elements would have their absolute values bounded above by $1$. 
                    \end{enumerate}
                    
                    One interesting object that can be built out of global fields, local fields, along with rings of integers thereof are rings of a\`eles of global fields: the ring of ad\`eles of a global field $F$ is defined to be the following so-called \textbf{restricted product}:
                        $$\A_F := \hat{\prod_{v \in \Spec \scrO_F}} F_v := \underset{V \in \calP^{\fin}_{\Spec \scrO_F}}{\colim} \left(\prod_{v \in V} F_v \x \prod_{v \in \Spec \scrO_F \setminus V} \scrO_{F, v}\right)$$
                    wherein $\calP^{\fin}_{\Spec \scrO_F}$ is the poset of \textit{finite} subsets of $\Spec \scrO_F$, and for each place $v \in \Spec \scrO_F$, one writes $F_v$ for the $v$-adic completion of $F$ and $\scrO_{F, v}$ for the ring of integers of $F_v$. For instance, the ring of ad\`eles of $\Q$ is:
                        $$\A_{\Q} := \hat{\prod_{p \in \Spec \Z}} \Q_p := \underset{V \in \calP^{\fin}_{\Spec \Z}}{\colim} \left(\prod_{p \in V} \Q_p \x \prod_{q \in \Spec \scrO_F \setminus V} \Z_q\right)$$
                \end{example}
                \begin{example}[More instances of integrality]
                    \noindent
                    \begin{enumerate}
                        \item \textbf{(Dedekind domains):} Any Dedekind domain is integrally closed in its field of fractions. However, this is not the case for general integral domains, i.e. there are integral domains which are not 
                        \item \textbf{(Algebraic closures):} Because algebraic extensions are special cases of integral extensions, algebraic closures are nothing but instances of integral closures.
                        \item \textbf{(The ring of integers of a number field):} We have seen in example \ref{example: ring_of_integers} that given any number field $E$, the corresponding ring of integers $\scrO_E$ is its own integral closure in $E$. Let us now examine a few concrete instances of this phenomenon:
                            \begin{enumerate}
                                \item \textbf{(The Gaussian integers):} The ring of integers 
                                \item \textbf{(Quadratic extensions):}
                                \item \textbf{(Cyclotomic extensions):}
                                \item \textbf{(Algebraic integers):} The integral closure of $\Z$ inside the field $\overline{\Q}$ of algebraic numbers is the ring of algebraic integers (or in order to avoid tautological statements, the ring of integers of $\overline{\Q}$); one may draw the following diagram to understand the relationship between this example and that of $\Z$ inside $\Q$:
                                    $$
                                        \begin{tikzcd}
                                        	\overline{\Q} & {\scrO_{\overline{\Q}}} \\
                                        	\Q & \Z
                                        	\arrow[no head, from=2-1, to=1-1]
                                        	\arrow[no head, from=2-2, to=1-2]
                                        	\arrow[no head, from=1-1, to=1-2]
                                        	\arrow[no head, from=2-1, to=2-2]
                                        \end{tikzcd}
                                    $$
                            \end{enumerate}
                    \end{enumerate}
                \end{example}
                \begin{remark}[Finiteness and integrality] \label{remark: finite_implies_integral}
                    As it is the case with fields, finite extensions of commutative rings are integral, but the converse is not necessarily true. For instance, the extension $\Z[\{\sqrt{p}\}_{(p) \in \Spec \Z}]$ is certainly integral inside $\Q$, but definitely not finite. 
                \end{remark}
                \begin{convention}
                    From now on, integral extensions will be denoted like how field extensions are, i.e. as \say{quotients}.
                \end{convention}
                
                \begin{proposition}[Equivalent definitions of integrality]
                    Let $A$ be a subring of a commutative ring $B$ and let $b$ be an element of $B$. Then:
                        \begin{enumerate}
                            \item $b$ is integral over $A$ if and only if it is a root of a polynomial in $A[x]$. 
                            \item There exists a faithful $A[b]$-module that is finitely generated over $A$. 
                        \end{enumerate}
                \end{proposition}
                    \begin{proof}
                                    
                    \end{proof}
                
                \begin{proposition}
                    Compositions of integral extensions are themselves integral extensions. 
                \end{proposition}
                    \begin{proof}
                                    
                    \end{proof}
                
                \begin{proposition}[Integrality and localisations]
                    Let $A$ be a subring of a commutative ring $B$, let $\overline{A}$ denote the integral closure of $A$ inside $B$, and let $S$ be a multiplicative subset of $A$. Then, the integral closure of $S^{-1}A$ inside $S^{-1}$ is just $S^{-1}\overline{A}$. 
                \end{proposition}
                    \begin{proof}
                                    
                    \end{proof}
                
            \subsubsection{Lying Over, Going Up, and Going Down}
                \begin{definition}[Primes lying over one another]
                    Let $\pi: \Spec B \to \Spec A$ be a morphism of affine schemes. Then, a prime $\q \in |\Spec B|$ is said to lie over a prime $\p \in |\Spec A|$ if:
                        $$\q \in |\pi|^{-1}(\p)$$
                \end{definition}
            
                \begin{definition}[Going Up and Going Down]
                    Let $\varphi: A \to B$ be a homomorphism between commutative rings.
                        \begin{enumerate}
                            \item \textbf{(Going Up):} $\varphi$ is said to satisfy \textbf{Going Up} if for every pair of prime ideals $\p \subset \p'$ of $A$ and for every prime $\q$ lying over $\p$, there exists a prime ideal $\q'$ above $\p'$ such that $\q' \supset \q$.  
                            \item \textbf{(Going Down):} $\varphi$ is said to satisfy \textbf{Going Down} if for every pair of prime ideals $\p \subset \p'$ of $A$ and for every prime $\q'$ lying over $\p'$, there exists a prime ideal $\q$ above $\p$ such that $\q \subset \q'$. 
                        \end{enumerate}
                \end{definition}
                
                \begin{proposition}[Going Up and Going Down criteria] \label{prop: going_up_and_down_criteria}
                    \noindent
                    \begin{enumerate}
                        \item Integral (and hence finite; see remark \ref{remark: finite_implies_integral} for details) extensions satisfy Going Up.
                        \item Quotient maps satisfy Going Up.
                        \item Flat ring maps (and hence localisations; see \cite{stacks}, \href{https://stacks.math.columbia.edu/tag/00HT}{\underline{lemma 10.39.18}}; actually, we can prove this easily using the fact that left-adjoints commute with colimits) satisfy Going Down. 
                    \end{enumerate}
                \end{proposition}
                    \begin{proof}
                         
                    \end{proof}
                
            \subsubsection{Integral schemes, schemes of finite type, and normal schemes}
        
        \subsection{Ramification theory}
            \subsubsection{Extension of Dedekind domains and the splitting of primes in Galois extensions}
                \begin{lemma}[A ring of integers is a Dedekind domain]
                    Let $E$ be a number field (either local or global). Then, its ring of integers is a Dedekind domain. 
                \end{lemma}
                    \begin{proof}
                         
                    \end{proof}
                    
                \begin{theorem}[Extensions of Dedekind domains]
                    Let $L/K$ be a finite extension of fields. Then, there is an induced integral extension $\scrO_L/\scrO_K$ of Dedekind domains. In other words, $\scrO_L$ is the integral closure of $\scrO_K$ in $L$. 
                \end{theorem}
                    \begin{proof}
                        
                    \end{proof}
                \begin{corollary}[Splitting of primes in finite extensions] \label{coro: prime_splitting_finite_extensions}
                    Let $L/K$ be a finite field extension and let $\p$ be a prime ideal of $\scrO_K$. Then:
                        \begin{enumerate}
                            \item \textbf{(Primes splitting):} The ideal $\p\scrO_L$ factors uniquely into a \say{products} of primes $\q_1, ...\q_n$ of $\scrO_L$:
                                $$\p\scrO_L = \q_1^{e_1}...\q_n^{e_n}$$
                            (with the natural numbers $e_i$ being multiplicities, usually known as \textbf{ramification indices}).
                            \item \textbf{(Lying Over):} Thanks to the above unique factorisation 
                        \end{enumerate}
                \end{corollary}
                    \begin{proof}
                        
                    \end{proof}
                    
                \begin{definition}[Ramification indices and inertial degrees] \label{def: ramification_indices}
                    Let $L/K$ be a field extension of finite degree, and let $\p$ be a prime ideal of $\scrO_K$. Also, if there are no risks of confusion, let us write $\p$ instead of $\p\scrO_L$ from now on for the prime of $\scrO_L$ generated by $\p$.
                        \begin{enumerate}
                            \item \textbf{(Ramification indices):} The exponents of the prime ideals in the factorisation of $\p$ are called the \textbf{ramification indices} of said prime factors. Primes with ramification index $1$ are put into two further subclasses:
                                \begin{enumerate}
                                    \item \textbf{(Splitting primes):} Let $\p = \q_1^{e_1}...\q_n^{e_n}$. If $e_i = 1$ and $n > 1$, then we will say that $\p$ splits in $\scrO_L$.
                                    \item \textbf{(Inert primes):} However, if $e_i = 1$ for all $1 \leq i \leq n$ and $n = 1$ also, then we will say that $\p$ remains \textbf{inert} in $\scrO_L$.
                                    \item \textbf{(Ramifying primes):} Otherwise (i.e. if $e_i > 1$ for all $1 \leq i \leq n$ and $n Geq 1$), we will say that $\p$ \textbf{ramifies}, or that $\p$ is a \textbf{place of ramification}.
                                    \item If $e_i > 1$ for all $1 \leq i \leq n$ and $n = 1$ then we will say that $\p$ is a non-splitting prime that ramifies with index $e = e_i = e_1$.
                                \end{enumerate}
                            \item \textbf{(Inertial degrees):} Because $\p$ factors uniquely in $\scrO_L$ - say as $\q_1^{e_1}...\q_n^{e_n}$) - all the primes $\q_i$ are divisors of $\p$. Thus, any prime ideal $\p$ in a base Dedekind domain ($\scrO_K$ in this case) along with an integral extension of Dedekind domains (which is the canonical map $\scrO_K \to \scrO_L$ here) has prime divisors $\q_i$. To such prime divisors, there are associated \textbf{inertial degrees} $f_i$ that we are going to define as the degree of the field extension $(\scrO_L/\q_i)/(\scrO_K/\p)$, i.e.:
                                $$f_i := [\scrO_L/\q_i : \scrO_K/\p]$$
                            Often, we will just say that $f_i$ is the inertial degree of $\q_i$ over $\p$.
                        \end{enumerate}
                \end{definition}
                
                \begin{lemma}[Prime divisors lie over]
                    Let $L/K$ be a finite extension and let $\p$ be a prime of $\scrO_K$. Then, the following are equivalent:
                        \begin{enumerate}
                            \item A prime ideal $\q$ of $\scrO_L$ divides $\p\scrO_L$.
                            \item A prime ideal $\q$ of $\scrO_L$ contains $\p\scrO_L$.
                            \item The intersection of a prime ideal $\q$ of $\scrO_L$ with the subring $\scrO_K$ of $\scrO_L$ is $\p\scrO_L$.
                            \item The intersection of a prime ideal $\q$ of $\scrO_L$ with the subfield $K$ of $L$ is $\p\scrO_L$.
                        \end{enumerate}
                \end{lemma}
                    \begin{proof}
                        
                    \end{proof}
                    
                \begin{theorem}[The fundamental identity of ramification theory]
                    Let $L/K$ be a finite and separable field extension of degree $n$ and let $\p$ be a prime ideal of $\scrO_K$ that factors into primes of $\scrO_L$ as follows:
                        $$\p = \q_1^{e_1}...\q_n^{e_n}$$
                    and for each index $i$, let $f_i$ denote the inertial degree 
                \end{theorem}
                    \begin{proof}
                        
                    \end{proof}
                \begin{corollary}[An application to quadratic fields (\cite{christian_511_project}, proposition 2.15)] \label{coro: ramification_quadratic_fields}
                    Let $p$ be a prime, let $d$ be a square-free integer, and let $\Delta$ denote the discriminant of the quadratic field $\Q(\sqrt{d})$, and recall that this quantity is given by:
                        $$
                            \Delta := 
                            \begin{cases}
                                \text{$d$ if $d \equiv 1 \pmod{4}$}
                                \\
                                \text{$4d$ otherwise}
                            \end{cases}
                        $$
                    (essentially, $\Delta$ is the discriminant of the quadratic polynomial $x^2 - d$; the point is that the definition above comes from an application of the all-too-familiar quadratic formula to this polynomial). Also, let $\scrO_{\Delta} := \Z[\sqrt{d}]$ denote the ring of integers of $\Q(\sqrt{d})$. Then:
                        \begin{enumerate}
                            \item If $p \mid \Delta$ (i.e. if $p \mid d$ or $p = 2$) then $(p) \in \Spec \Z$ will be a non-splitting place of ramification of $\Q$ with ramification index $2$, i.e. there is a prime $\q$ of $\scrO_{\Delta}$ such that:
                                $$(p)\scrO_{\Delta} = \q^2$$
                            Also:
                                $$\scrO_{\Delta}/(p)\scrO_{\Delta} \cong \F_p[x]/(x^2)$$
                            \item When $p \ndiv d$, it is necessarily true that $p 
                            \not = 2$. We then obtain two subcases:
                                \begin{enumerate}
                                    \item If $\left(\frac{\Delta}{p}\right) = 1$ (see \href{https://ncatlab.org/nlab/show/quadratic+reciprocity+law}{\underline{here}} if a reminder of the definition of Legendre symbols is called for; note that the symbol $\left(\frac{\Delta}{p}\right)$ actually makes sense because $p$ has already been established to be an odd prime) then $(p)$ splits into two distinct prime factors $\q_1, \q_2$ inside $\scrO_{\Delta}$. Furthermore:
                                        $$\scrO_{\Delta}/\q_1\q_2 \cong \F_p \x \F_p$$
                                    \item If $\left(\frac{\Delta}{p}\right) = -1$ then $(p)$ remains inert in $\scrO_{\Delta}$, and:
                                        $$\scrO_{\Delta}/(p)\scrO_{\Delta} \cong \F_{p^2}$$
                                \end{enumerate}
                        \end{enumerate}
                \end{corollary}
                    \begin{proof}
                        
                    \end{proof}
                    
                \begin{convention}[Primes and places] \label{conv: places_and_primes} \index{Primes as places}
                    We probably should have mentioned this earlier, but since we have already got to this point without touching on it very often, this might be as good a time as any to discuss the terminologies \say{prime}, \say{prime ideals}, and \say{places}. Historically speaking, given a local field $K$, a \say{place} of $K$ is a valuation, and thanks to Ostrowski's theorem (\cite{koblitz_p_adic}, theorem I.1, pp.3) - which asserts that every non-trivial non-archimedean valuation on a local number field (i.e a $p$-adic field) is equivalent to the canonical $p$-adic valuation - equivalence classes thereof in the case where $K$ is a local number field. Thus, in the context of global number fields, a \say{place} is nothing but a prime ideal in the ring of integers, and hence it makes senses to interchange the words there. Furthermore, the terminology \say{place} helps us make sense of number-theoretic facts geometrically, as prime ideals are precisely points of affine schemes (spectra of rings of integers in this situation).
                    
                    As an example, consider $\Q$, a global number field in which a place is just a prime ideal of $\Z$, i.e. either $(0)$ or $(p)$, for some prime $p$. Below is an illustration wherein, by completing $\Q$ along a \textit{non-zero} prime $p$ of $\Z$, one gets a local number field $\Q_p$ that is complete with respect to the attached $p$-adic valuation, whereas by performing formal completion along $(0)$, one recovers $\Q$, which can be thought of as the corresponding \say{generic} number field due to its global nature:
                        \begin{figure}[H]
                            \centering
                            \includegraphics[width=\linewidth,height=\textheight,keepaspectratio]{Figures/places of Spec Z.png}
                            \caption{Places of $\Q$ (note that $(0)$ should be viewed as the generic place-at-infinity).}
                            \label{fig: places_of_Q}
                        \end{figure}
                \end{convention}
                
                \begin{proposition}[Local-global compatibility]
                
                \end{proposition}
                    \begin{proof}
                        
                    \end{proof}
                
                \begin{example}[Places of ramification of more general arithmetic schemes]
                    Below are examples of arithmetic schemes, i.e. schemes over $\Spec \Z$, on which there are primes lying over those of $\Spec \Z$ where ramifications take place.
                        \begin{enumerate}
                            \item \textbf{(The prime spectrum of the ring of integers of a number field):} Consider the following setting:
                                $$
                                    \begin{tikzcd}
                                    	{\Q(\sqrt{d})} & {\Z[\sqrt{d}]} \\
                                    	\Q & \Z
                                    	\arrow[no head, from=2-1, to=1-1]
                                    	\arrow[no head, from=2-2, to=1-2]
                                    	\arrow[no head, from=1-1, to=1-2]
                                    	\arrow[no head, from=2-1, to=2-2]
                                    \end{tikzcd}
                                $$
                            wherein we are consider the quadratic extension $\Q(\sqrt{d})/\Q$ along with the induced integral extension of Dedekind domains $\Z[\sqrt{d}]/\Z$; particularly, let us pick a prime ideal $\p$ of $\Z$ (i.e. either the zero ideal or an ideal generated by a prime number $p$). Now, do the primes of $\Z[\sqrt{d}]$ indeed lie over those of $\Z$ ? First of all, we will need to see what the prime ideals of $\Z[\sqrt{d}]$ actually look like, and luckily, we can apply corollary \ref{coro: prime_splitting_finite_extensions} to do this:
                                \begin{enumerate}
                                    \item Of course, the ideal $(0)\Z[\sqrt{d}]$ is just the zero ideal, and thus admits the trivial factorisation $(0) = (0)$. Because of this, let us only consider non-zero prime ideals of $\Z$ from now on.
                                    \item Then, there are three cases, according to corollary \ref{coro: ramification_quadratic_fields}:
                                        \begin{enumerate}
                                            \item 
                                            \item
                                            \item
                                        \end{enumerate}
                                \end{enumerate}
                            \item \textbf{(A conic over $\Spec \Z$):}
                            \item \textbf{(The line with double origin over $\Spec \Z$):}
                        \end{enumerate}
                    \end{example}
                    
            \subsubsection{Unramfied morphisms}
                \begin{definition}[Unramified morphisms] \label{def: unramified_morphisms}
                    A ring map is said to be \textbf{unramified} if it is of finite type and if the corresponding module of K\"ahler differentials is zero. 
                \end{definition}
                \begin{example}
                    \noindent
                    \begin{itemize}
                        \item \textbf{(\'Etale morphisms):} \'Etale morphisms (cf. definition \ref{def: etale_morphisms}) are trivially unramified. The converse statement is not necessarily true, since there are morphisms of finite type that are not of finite presentation, which is a necessary condition for \'etale-ness (or even just smoothness for that matter).
                        
                        As a concrete example, take any unramified finite extension $E/F$ (cf. definition \ref{def: ramification_indices}), and in the event that these fields have rings of integers (cf. example \ref{example: ring_of_integers}), the induced map:
                            $$\Spec E^{\circ} \to \Spec F^{\circ}$$
                        will also be unramified (in fact, they will be \'etale, since finite extensions come from finite presentations). As a result, for some fixed base scheme $S$, one can think of unramified morphisms $X \to S$ as coming from $S$-schemes $X$ such that preimages of points of $S$ consists merely of a single point. 
                        \item \textbf{(A counter example: non-\'etale smooth morphisms):} A smooth morphism of non-zero relative dimension can not be unramified, since its associated module of K\"ahler differential is not zero. 
                    \end{itemize}
                \end{example}
                
                \begin{proposition}[Locality of ramification] \label{prop: locality_of_ramification}
                    We say that 
                \end{proposition}
                
            \subsubsection{Extensions of discrete valuation rings}
                
            
            \subsubsection{Galois extensions and ramification}
    
    \section{\'Etale cohomology for schemes: The Chosen One}
        \subsection{\'Etale morphisms} \label{subsection: etale_morphisms}
            \begin{definition}[\'Etale morphisms] \label{def: etale_morphisms} \index{\'Etale-ness}
                An \'etale ring map is a smooth ring map whose cotangent complex is quasi-isomorphic to the zero complex. Equivalently, a ring homomorphism is \'etale if and only if it is smooth and of relative dimension $0$ (see proposition \ref{prop: smoothness_implies_almost_finiteness_of_cotangent_complex} for an explanation). 
            \end{definition}
            
            \begin{proposition}[The separable extension criterion for \'etale-ness] \label{prop: separable_criterion_for_etaleness} \index{\'Etale-ness! Separability Criterion}
                Let $\pi: F \to B$ be a ring map of finite presentation. Then, the following statements are equivalent:
                    \begin{enumerate}
                        \item $\pi$ is \'etale.
                        \item The field extension $\kappa_{\pi^{-1}(\q)}/\kappa_{\q}$ of the residue field at $\pi^{-1}(\q) \in |\Spec k|$ over that at $\q \in |\Spec B|$ is separable (and necessarily finite, as \'etale maps are \textit{a priori} of finite presentation) for all $\q \in |\Spec B|$. Note that the extension $\kappa_{\pi^{-1}(\q)}/\kappa_{\q}$ exists thanks to the fact that the stalk map:
                            $$\calO_{\Spec F, \pi^{-1}(\q)} \to \calO_{\Spec B, \q}$$
                        which is actually just: 
                            $$F_{\pi^{-1}(\q)} \to B_{\q}$$
                        is required to be a local homomorphism between local rings.
                        \item If $F$ is a field, then $B$, when viewed as an $F$-vector space, can be written as a (necessarily finite) direct sum of (necessarily finite) separable extensions of $F$ (of course, this assertion is only equivalent to the other two in the event that they are considered with $F$ a field as well).
                    \end{enumerate}
            \end{proposition}
                \begin{proof}
                    This is trivial when $\chara \kappa_{\q} = 0$, as every extension in characteristic $0$ is \textit{a priori} separable (for a proof, please consult \cite[\href{https://stacks.math.columbia.edu/tag/030Q}{Tag 030Q}]{stacks} and \cite[\href{https://stacks.math.columbia.edu/tag/030N}{Tag 030N}]{stacks}). Thus, let us assume that:
                        $$\chara \kappa_{\q} = p$$
                    for some prime $p$. Also, note that the extension $\kappa_{\pi^{-1}(\q)}/\kappa_{\q}$ decomposes into a separable part $\kappa_{\q}^{\sep}/\kappa_{\q}$ (with $\kappa_{\q}^{\sep}$ is the separable closure of $\kappa_{\q}$ inside $\kappa_{\pi^{-1}(\q)}$) and a purely inseparable part $\kappa_{\pi^{-1}(\q)}/\kappa_{\q}^{\sep}$ in the following manner \cite[\href{https://stacks.math.columbia.edu/tag/030K}{Tag 030K}]{stacks}:
                        $$
                            \begin{tikzcd}
                            	{\kappa_{\pi^{-1}(\q)}} \\
                            	{\kappa_{\q}^{\sep}} \\
                            	{\kappa_{\q}}
                            	\arrow[no head, from=3-1, to=2-1]
                            	\arrow[no head, from=2-1, to=1-1]
                            \end{tikzcd}
                        $$
                    \begin{enumerate}
                        \item 
                            \begin{enumerate}
                                \item \textbf{(1 implies 2):} Suppose firstly that \textbf{1} holds, i.e. that $\pi: F \to B$ is \'etale. According to definition \ref{def: etale_morphisms}, this tells us that $B$ is a smooth $F$-algebra that is of relative dimension $0$; in other words, we can write $B$ as $\frac{F[x_1, ..., x_n]}{(f_1, ..., f_n)}$ for some natural number $n$. Now, fix an arbitrary prime ideal $\q \in \left|\Spec \frac{F[x_1, ..., x_n]}{(f_1, ..., f_n)}\right|$, which should be noted to be nothing but a prime of $F[x_1, ..., x_n]$ containing the ideal $(f_1, ..., f_n)$, and note that:
                                    $$\left(\frac{F[x_1, ..., x_n]}{(f_1, ..., f_n)}\right)_{\q} \cong \frac{F[x_1, ..., x_n]_{\q}}{(f_1, ..., f_n)}$$
                                thanks to the fact that colimits commute. Now, let us suppose for the sake of deriving a contradiction, that $\kappa_{\pi^{-1}(\q)}/\kappa_{\q}$ is not a (finite) separable extension for our chosen prime $\q$, and observe that according to the preliminary discussion, this is the same as supposing that the purely inseparable extension $\kappa_{\pi^{-1}(\q)}/\kappa_{\q}^{\sep}$ is non-trivial. 
                                \item \textbf{(2 implies 1):} On the other hand, let us use \textbf{2} as our starting point. 
                            \end{enumerate}
                        \item 
                            \begin{enumerate}
                                \item \textbf{(2 implies 3):} Now, suppose that \textbf{2} is true and that $F$ is a field. Immediately, one sees that:
                                    $$\kappa_{\pi^{-1}{\q}} \cong \calO_{\Spec F, \pi^{-1}(\q)} \cong F_{\pi^{-1}(\q)} \cong F_{(0)} \cong F$$ 
                                which means that $F/\kappa_{\q}$ is a (finite) separable extension for all primes $\q \in \left|\Spec B\right|$. Also, thanks to the hypothesis whereby $F$ is a field, one can write the finitely presented $F$-algebra $B$ as some finite direct some of copies of $F$ (since algebras are first and foremost modules, and modules over fields are vector spaces, which are \textit{a priori} all free). Thus, the \'etale $F$-algebra $B$, when viewed as a vector space over $F$, can written as a finite direct sum of the finite separable extension $F$ of $\kappa_{\q}$, for all $\q \in |\Spec B|$. In other words, \textbf{2} implies \textbf{3}. 
                                \item \textbf{(3 implies 2):} Conversely, suppose that \textbf{3} is true, specifically that:
                                    $$B \cong F^{\oplus d}$$
                                for some natural number $d$, and suppose that the $F$-algebra of finite presentation $B$ is of the form $\frac{F[x_1, ..., x_N]}{(f_1, ..., f_n)}$, for some pair of natural numbers $n, N$. 
                            \end{enumerate}
                        Thus, we have managed to show that \textbf{1} is equivalent to \textbf{2}, and that \textbf{2} is in turn equivalent to \textbf{3}, and thus the three are jointly equivalent. 
                    \end{enumerate}
                \end{proof}
            \begin{corollary}[Finite separable extensions are \'etale] \label{coro: finite_separable_extensions_are_etale}
                Let $k$ be a field and let $B$ be a $k$-algebra of finite presentation. According to proposition \ref{prop: separable_criterion_for_etaleness}, $B$ is \'etale over $k$ if and only if when it is viewed as a $k$-vector space, $B$ is a direct sum of copies of some finite separable extension $K$ over $k$. But $K$ itself is a $k$-algebra, and thus every finite and separable field extension is an \'etale ring map. In fact, it is even better: via the adjoint equivalence:
                    $$
                        \begin{tikzcd}
                        	{{}^{k/}\Comm\Alg^{\op}} & {\Sch^{\aff}_{/\Spec k}}
                        	\arrow[""{name=0, anchor=center, inner sep=0}, "\Spec"', shift right=2, from=1-1, to=1-2]
                        	\arrow[""{name=1, anchor=center, inner sep=0}, "Gamma"', shift right=2, from=1-2, to=1-1]
                        	\arrow["\dashv"{anchor=center, rotate=-90}, draw=none, from=1, to=0]
                        \end{tikzcd}
                    $$
                any non-empty collection of finite separable extension $\{k \to K_{\alpha}\}_{\alpha \in A}$ of some given ground field $k$ corresponds to an \'etale covering sieve $\{\Spec K_{\alpha} \to \Spec k\}_{\alpha}$ (because spectra of fields are singletons, the joint surjection of presheaves:
                    $$
                        \begin{tikzcd}
                        	{\left(\coprod_{\alpha \in A} h_{\Spec K_{\alpha}}\right) \x_{h_{\Spec k}} \left(\coprod_{\alpha \in A} h_{\Spec K_{\alpha}}\right)} & {\coprod_{\alpha \in A} h_{\Spec K_{\alpha}}} & {h_{\Spec k}}
                        	\arrow["{\pr_2}"', shift right=2, from=1-1, to=1-2]
                        	\arrow["{\pr_1}", shift left=2, from=1-1, to=1-2]
                        	\arrow[dashed, from=1-2, to=1-3]
                        \end{tikzcd}
                    $$
                exists for all non-empty indexing sets $A$). In turn, this implies something remarkable, which is that for all Galois extensions $K/k$ (which are necessarily separable by definition, and hence \'etale) and for all \textit{sheaves} $\calF$ on ${}^{k/}\Comm\Alg^{\op, \petit}_{\et}$ (we refer the reader to paragraph \ref{paragraph: etale_descent} for the descent theory along \'etale morphisms) we have via the fact that finite colimits (in this case, the orbit space ) commute with filtered limits that and the Fundamental Theorem of Galois Theory that:
                    $$
                        \begin{aligned}
                            F(\Spec K)/Gal(K/k) & \cong \calF(\Spec E)/\left(\underset{E}{\lim} \Aut(E/k)\right)
                            \\
                            & \cong \underset{E}{\lim} \left(\calF(\Spec E)/\Aut(E/k)\right)
                            \\
                            & \cong \underset{E}{\lim} \left(\calF(\Spec k)/\Aut(E/k)\right)
                        \end{aligned}
                    $$
                wherein the limit is the filtered limit taken over the poset of finite separable subextensions $E/k$ of $K/k$; also, note that:
                    $$\calF(\Spec E) \cong \calF(\Spec k)$$
                for all finite separable extensions $E/k$ because sheaves satisfy descent, by definition.
            \end{corollary}
            \begin{remark}[Fibres over rational points of \'etale morphisms and inertia of primes] \label{remark: fibres_of_etale_maps_over_points}
                Let $k$ be a field and let $B$ be an \'etale $k$-algebra. From proposition \ref{prop: separable_criterion_for_etaleness}, we know that there exists a natural number $d$ such that:
                    $$B \cong K^{\oplus d} \cong (k^{\oplus [K : k]})^{\oplus d} \cong k^{\oplus ([K : k] \cdot d)}$$
                as $k$-vector spaces, where $K/k$ is some finite separable extension. Then, thanks to the fact that direct sums of algebra objects in the category $k\Vect$ of $k$-vector spaces are merely products in the commutative algebra category ${}^{k/}\Comm\Alg$, and by employing the adjoint equivalence:
                    $$
                        \begin{tikzcd}
                        	{{}^{k/}\Comm\Alg^{\op}} & {\Sch^{\aff}_{/\Spec k}}
                        	\arrow[""{name=0, anchor=center, inner sep=0}, "\Spec"', shift right=2, from=1-1, to=1-2]
                        	\arrow[""{name=1, anchor=center, inner sep=0}, "Gamma"', shift right=2, from=1-2, to=1-1]
                        	\arrow["\dashv"{anchor=center, rotate=-90}, draw=none, from=1, to=0]
                        \end{tikzcd}
                    $$
                one gets:
                    $$\Spec B \cong \Spec \prod_{i = 1}^d K \cong \coprod_{i = 1}^d \Spec K$$
                wherein the coproduct is taken in the category of locally ringed spaces over $\Spec k$, but lands in the category of affine schemes (also over $\Spec k$). Lastly, because spectra of fields are singletons, the \'etale $k$-algebra $B$ is nothing but a collection of $d$ disjoint points lying over the single point of $\Spec k$. 
                
                One implication of this is that given any scheme $X$ over $\Spec k$, its fibres over $k$-rational points $x \in X(k)$ are nothing but finite collections of points (even if the fibre is not affine, its affine patches would still necessarily be finite collections of points, as shown above, and thus the whole fibre would be so as well).
            \end{remark}
    
        \subsection{The interesting point: Galois cohomology}
            \subsubsection{Cohomology of profinite groups}
                Cohomology of profinite groups is nothing but group cohomology but for pro-objects in the category of finite groups (for us, cohomology shall always mean cohomology with abelian coefficients). Because of that, we shall pay less attention to the definition and general properties, and more on the special properties that cohomologies of profinite groups enjoy. By the end of it, we shall apply what we know to Galois groups, which are profinite thanks to the Fundamental Theorem of Galois Theory. 
                
                \paragraph{Group cohomology}
                    \begin{convention}
                        \noindent
                        \begin{itemize}
                            \item Throughout, we shall work with a sheaf topos $\E$ in which the category of internal abelian groups $\Ab(\E)$ has enough projectives (for instance, one can consider $\E \cong \Sets$ or the \'etale topos over any qcqs scheme). This essentially just means that we are assuming that the internal logic of $\E$ supports the Axiom of Choice (or more minimally, the Axiom of Presentation). 
                            
                            In particular, fix an abelian group $A \in \Ab(\E)$.
                            \item Additionally, fix a group object $G \in Grp(\E)$. Its group ring shall be denoted by $\Z[G]$ (which is not an abuse of notation, since a natural number object exists in $\E$, allowing us to define an integer object $\Z$ as the group completion of the monoid $\N$; alternatively, one can think of $\Z$ as the monoidal unit of $\Ab(\E)$); recall that this ring object is defined via the following forgetful-free adjunction, whose existence we shall let the reader prove as an exercise\footnote{Recall that $\Ring(\E)$ is the category of monoids internal to the symmetric monoidal category $\Ab(\E)$.}:
                                $$
                                    \begin{tikzcd}
                                    	{\Ring(\E)} & {\Grp(\E)}
                                    	\arrow[""{name=0, anchor=center, inner sep=0}, "\oblv"', shift right=2, from=1-1, to=1-2]
                                    	\arrow[""{name=1, anchor=center, inner sep=0}, "{\Z[-]}"', shift right=2, from=1-2, to=1-1]
                                    	\arrow["\dashv"{anchor=center, rotate=-90}, draw=none, from=1, to=0]
                                    \end{tikzcd}
                                $$
                                
                            We shall also want a $G$-action on $A$, which would make $A$ a $\Z[G]$-module.
                        \end{itemize}
                    \end{convention}
                    
                    \begin{definition}[Group cohomologies] \label{def: group_cohomologies}
                        The $n^{th}$ cohomology group of $G$ with coefficient in $A$ is defined as:
                            $$H^n(G, A) \cong \Ext^n_{\Z[G]}(\Z, A)$$
                        where we view $\Z$ as trivial $G$-representation on the right-hand side. 
                    \end{definition}
                    \begin{remark}
                        It is not hard to see via induction on the cohomological dimension $n \in \N$ that $H^n(G, A)$ (as in definition \ref{def: group_cohomologies}) actually has a $k$-module structure.
                    \end{remark}
                    
                    Let us now attempt to compute cohomologies of groups in low dimensions (namely, $n = 0, 1, 2$) as well as give meaning to these spaces, which we shall do using free (or at worst, projective) resolutions of the trivial $\Z[G]$-module $\Z$.
                    \begin{proposition}[Low-dimensional cohomologies of groups] \label{prop: low_dimensional_cohomologies_of_groups}
                        One has the following interpretations of the low-dimensional cohomologies of $G$ with coeffcients in $A$:
                        \begin{enumerate}
                            \item \textbf{($n = 0$: Invariants):}
                                $$H^0(G, A) \cong A^G$$
                            with $A^G$ denoting the space of $G$-fixed point of $A$.
                            \item \textbf{($n = 1$: Torsors):} 
                            \item \textbf{($n = 2$: Extensions):} 
                                $$H^2(G, A) \cong \{\text{extensions of $G$ by $A$}\}$$
                        \end{enumerate}
                    \end{proposition}
                        \begin{proof}
                            \noindent
                            \begin{enumerate}
                                \item \textbf{($n = 0$: Invariants):}
                                \item \textbf{($n = 1$: Torsors):}
                                \item \textbf{($n = 2$: Extensions):}
                            \end{enumerate}
                        \end{proof}
                    \begin{example}
                                    
                    \end{example}
                
                \paragraph{Properties of cohomologies of profinite groups}
            
            \subsubsection{Iwasawa theory}
    
        \subsection{\texorpdfstring{$\ell$}{}-adic cohomology}
            \subsubsection{\texorpdfstring{$\ell$}{}-adic cohomology for schemes: "This is where the fun begins!"}
                Let $X$ be a smooth projective scheme over a base field that is possibly of some prime characteristic $p$. Let $\ell$ be a prime different from $p$. By \say{$\ell$-adic cohomology}, we shall mean the cohomology theory whose cohomology groups are given by:
                    $$H^i_{\Q_{\ell}}(X) \cong \underset{n \in \N}{\lim} H_{\et}^i(X, \Z/\ell^{n + 1}\Z) \tensor_{\Z_{\ell}} \Q_{\ell}$$
                This means, among other things, that $\ell$-adic cohomology is effectively a \say{singular cohomology theory} for (smooth projective) schemes. This, however, is not the only amazing property that $\ell$-adic cohomology enjoys: it is also a Weil cohomology theory (see definition \ref{def: weil_cohomology_theories} for a description of what this means). At surface level, this might seem odd, since only \'etale cohomology with torsion coefficients gives satisfactory results. However, this is precisely why we take the limit over the torsion rings $\Z/\ell^{n + 1}\Z$ and then base change to $\Q_{\ell}$: $\Q_{\ell}$ is a field of characteristic $0$, and hence can serve as the field of a coefficients of a Weil cohomology theory (cf. definition \ref{def: weil_cohomology_theories}). 
                
                To check that $\ell$-adic cohomology is indeed a Weil cohomology theory, let us \say{simply} go through the axioms laid out in definition \ref{def: weil_cohomology_theories}. Before we can do that, however, we will first need to actually set up the theory of \'etale $\ell$-adic cohomology.
                
                Now, we are finally ready to verify that \'etale $\ell$-adic cohomology is a Weil cohomology theory.
                \paragraph{Finiteness}
            
                \paragraph{The K\"unneth Formula}
                
                \paragraph{Poincar\'e Duality}
            
                \paragraph{The Lefschetz Conditions}
        
            \subsubsection{The Artin comparison theorem: an \'etale GAGA theorem}
            
            \subsubsection{\texorpdfstring{$\ell$}{}-adic cohomology for algebraic stacks: "Another happy landing!"}
                \paragraph{Finiteness}
            
                \paragraph{The K\"unneth Formula}
                
                \paragraph{Poincar\'e Duality}
            
                \paragraph{The Lefschetz Conditions}
                
    
	    
	    \chapter{\'Etale cohomology and Galois representations} \label{chapter: etale_cohomology_2}
    \begin{abstract}
        
    \end{abstract}
    
    \minitoc
    
    \begin{convention}
        \noindent
        \begin{enumerate}
            \item From this point on, the absolute Galois group of any field $K$ shall be denoted by $\bfG_K$. 
            \item For now, we refer the reader to \cite[Definition 4.1.1 and Remark 4.1.3]{bhatt_scholze_2014_pro_etale} for discussions regarding the pro-\'etale topology.
        \end{enumerate}
    \end{convention}
    
    \section{\texorpdfstring{$\ell$}{}-adic sheaves and Galois representations} \label{section: l_adic_sheaves}
        \subsection{Grothendieck's Galois Theory} \label{subsection: grothendieck_galois_theory}
            Let us first review a bit of Grothendieck's (finite) Galois theory, mostly for the part of having definitions conveniently nearby for quick references and comparison with an infinite geometric Galois theory that we shall present after this interlude.
            
            \subsubsection{\'Etale fundamental groups of schemes}
                \begin{definition}[Noohi groups] \label{def: fintie_galois_categories}
                    \noindent
                    \begin{enumerate}
                        \item \textbf{(Finite Galois categories \cite[\href{https://stacks.math.columbia.edu/tag/0BMY}{Tag 0BMY}]{stacks}):} A \textbf{finite Galois category} is defined via the data contained in a pair $(\calG, F)$ consisting of an \textit{exact} functor $F: \calG^{\op} \to \Sets^{\fin}$ and a category $\calG$ such that:
                            \begin{itemize}
                                \item $\calG$ is finitely complete and finitely cocomplete.
                                \item Objects in $\calG$ can all be written as a (possibly empty but necessarily finite) coproduct of connected objects (objects $X \in \calG$ such that the functor $\calG(X, -)$ preserves all coproducts). 
                            \end{itemize}
                        Functors such as the functor $F$ above are commonly called \textbf{fibre functors}. 
                        \item \textbf{(Noohi groups):} In the sense of \cite[Theorem 2.16]{noohi_fundamental_group}, a so-called \textbf{Noohi group} is the group of natural automorphisms on the $\Sets^{\fin}$-valued functor defining a Galois finite category; that is to say, given a Galois finite category $(\calG, F)$, its Noohi group is $\Aut(F)$.  
                    \end{enumerate}
                \end{definition}
                
                \begin{lemma}[Profiniteness of Noohi groups] \label{lemma: profiniteness_of_noohi_groups}
                    Let $(\calG, F)$ be a finite Galois category. Then:
                        \begin{enumerate}
                            \item The associated Noohi group $\Aut(F)$ is profinite.
                            \item $\calG$ is equivalent to the category $[\bfB\Aut(F)^{\op}, \Sets^{\fin}]$ of $\Aut(F)$-equivariant finite sets.
                        \end{enumerate}
                \end{lemma}
                    \begin{proof}
                        \cite[Theorem 2.16]{noohi_fundamental_group}
                    \end{proof}
                    
                \begin{definition}[\'Etale fundamental group] \label{def: etale_fundamental_groups}
                    Recall first of all that for any given base scheme $X$, the category $\Sch_{/X, \fet}$ of schemes finite and \'etale over $X$ is a category wherein:
                        \begin{itemize}
                            \item all finite limits and all finite colimits exist, and
                            \item all objects can be written as a (possibly empty) finite coproduct of connected objects, which happen to be schemes that are \'etale over $X$.  
                        \end{itemize}
                    In other words, the category spanned by (possibly empty) finite coproducts of schemes \'etale over $X$ can serve as the underlying category of a finite Galois category. Let us then fix a geometric point:
                        $$\overline{x}: \Spec \overline{\kappa_x} \to X$$
                    (where $\overline{\kappa_x}$ denotes an algebraic closure of the residue field $\kappa_x$ at some point $x \in |X|$) of $X$ and define the following fibre functor:
                        $$F_{\overline{x}}: \Sch_{/X, \fet} \to \Sets^{\fin}$$
                    by the rule:
                        $$F_{\overline{x}}(f: Y \to X) \cong |Y \x_{f, X, \overline{x}} \Spec \overline{\kappa_x}|$$
                    The pair $(\Sch_{/X, \fet}, F_{\overline{x}})$ as above thus define a finite Galois category. Its Noohi group $\Aut(F_{\overline{x}})$ is commonly denoted by $\pi_1^{\fet}(X, \overline{x})$.
                \end{definition}
                \begin{remark}
                    Definition \ref{def: etale_fundamental_groups} is actually a bit subtle and honestly, somewhat ill-founded, as did not actually prove that $F_{\overline{x}}$ was an honest-to-Grothendieck fibre functor. It is certainly left-exact, by virtue of being defined via pullbacks, and it is right-exact because any \'etale algebra over a field can be written as a finite direct sum of finite extensions of that field \cite[\href{https://stacks.math.columbia.edu/tag/00U3}{Tag 00U3}]{stacks}, and direct sums are biproducts of vector spaces. However, the fact that the sets $|Y \x_{f, X, \overline{x}} \Spec \overline{\kappa_x}|$ are finite is not really trivial, although it is not too hard to prove either. Basically, this fact is also consequence \'etale algebras being isomorphic to finite direct sums of finite extensions: in our case, since $\overline{\kappa_x}$ is algebraically closed, the underlying vector space of \'etale $\overline{\kappa_x}$-algebras must be isomorphic to a finite direct sum of $\overline{\kappa_x}$ itself. In terms of schemes, this means that when both $Y$ and $X$ are affine, the pullback $Y \x_{f, X, \overline{x}} \Spec \overline{\kappa_x}$ would be nothing but a coproduct of finitely many copies of $\Spec \overline{\kappa_x}$, and hence the set $|Y \x_{f, X, \overline{x}} \Spec \overline{\kappa_x}|$ would have to be finite. Then, by using the fact the \'etale-ness is a local property, we can deduce that the set $|Y \x_{f, X, \overline{x}} \Spec \overline{\kappa_x}|$ must be finite regardless of whether $Y$ and $X$ are finite or not. The functor:
                        $$F_{\overline{x}}: \Sch_{/X, \fet} \to \Sets^{\fin}: (f: Y \to X) \mapsto |Y \x_{f, X, \overline{x}} \Spec \overline{\kappa_x}|$$
                    is therefore indeed a fibre functor.
                \end{remark}
                
                \begin{theorem}[Grothendieck's Galois theory] \label{theorem: grothendieck's_galois_theorem}
                    Let $X$ be a \textit{connected} base scheme and let:
                        $$\overline{x}: \Spec \overline{\kappa_x} \to X$$
                    be a geometric point therein. By lemma \ref{lemma: profiniteness_of_noohi_groups}, we have the following equivalence of categories:
                        $$\Sch_{/X, \fet} \cong [\bfB\pi_1^{\fet}(X, \overline{x})^{\op}, \Sets^{\fin}]$$
                    But this purely topological equivalence can be upgraded to an algebraic one via the the following canonical isomorphism of groups:
                        $$\pi_1^{\fet}(X, \overline{x}) \cong \Gal(k^{\sep}/k)$$
                    wherein $k^{\sep}$ is the unique separable closure inside $\overline{k}$, which holds if and only if $X$ is the spectrum of some field $k$ (note that in such a situation, the geometric point $\overline{x}$ is nothing but the canonical morphism $\Spec \overline{k} \to \Spec k$).
                \end{theorem}
                    \begin{proof}
                        
                    \end{proof}
                    
            \subsubsection{Properties of \'etale fundamental groups}
                \begin{theorem}[\'Etale fundamental groups are unique up to universal homeomorphisms] \label{theorem: etale_fundamental_groups_are_unique_up_to_universal_homeomorphisms}
                    Let $f: Y \to X$ be a universal homeomorphism. Then, one has the following equivalence of categories:
                        $$\Sch_{/X}^{\fet} \cong \Sch_{/Y}^{\fet}: (j: U \to X) \mapsto U \x_{j, X, f} Y$$
                    which in particular, implies that for any geometric point $\overline{x}$ of $X$, there is an isomorphism of \'etale fundamental groups:
                        $$\pi_1^{\fet}(X, \overline{x}) \cong \pi_1^{\fet}(Y, \overline{y})$$
                    where $\overline{y}$ is the geometric point of $Y$ lying over $\overline{x}$ (it is uniquely determined as $f: Y \to X$ is a universal homemomorphism). 
                \end{theorem}
                    \begin{proof}
                        
                    \end{proof}
                \begin{corollary}
                    Let $X \to X'$ be a morphism of schemes which is a homeomorphism at the level of the underlying topological spaces and enjoys the universal property of a  filtered colimit or that of a limit. This is a special case of a universal homeomorphism, and one thus has:
                        $$\pi_1^{\fet}(X) \cong \pi_1^{\fet}(X')$$
                    Examples include but certainly not limited to the following:
                        \begin{itemize}
                            \item $X \to X'$ is a thickening.
                            \item 
                        \end{itemize}
                \end{corollary}
    
        \subsection{The \texorpdfstring{$\ell$}{}-adic Monodromy Correspondence}
            Let us start with the notion of $\ell$-adic representations. 
            \begin{definition}[$\ell$-adic representations] \label{def: l_adic_representations}
                Let $K$ be a field and let $L/K$ be a Galois extension thereof. Additionally, let $F$ be a local field (we shall view finite fields as $0$-dimensional local fields) equipped with its natural topology (e.g. non-archimedean when $F$ is some sort of $\ell$-adic number field, archimedean when $F$ is $\R$ or $\bbC$, and discrete when $F$ is finite); also, we shall require that $\ell \not = \chara K$. An \textbf{$\ell$-adic representation of $\Gal(L/K)$} is thus a finite-dimensional \textit{continuous} $F$-linear representation of $\Gal(L/K)$, i.e. a continuous group homomorphism:
                    $$\rho: \Gal(L/K) \to \GL_n(F)$$
                for some natural number $n$. $\ell$-adic representations of \textit{absolute} Galois groups are known as \textbf{$\ell$-adic Galois representations}, or just Galois representations for short.
            \end{definition}
            \begin{remark}[It's actually a bit simpler than we've been led to believe]
                Definition \ref{def: l_adic_representations} can seem a bit complicated, but what it actually does is just giving names to certain continuous finite-dimensional $F$-linear representations of certain topological groups (recall how Galois groups naturally carry the profinite topology which reduces to the discrete topology in finite cases). The category of $\ell$-adic representations of a given Galois group is thus nothing special, from a categorical point of view, and a lot of the basic properties of $\ell$-adic representations are actually just abstract-nonsensical. 
            \end{remark}
            \begin{example} \label{example: l_adic_representations}
                Let $K$ be a field and let $L/K$ be a Galois extension thereof. Additionally, let $F$ be a local field (we shall view finite fields as $0$-dimensional local fields) equipped with its natural topology (e.g. non-archimedean when $F$ is some sort of $\ell$-adic number field, archimedean when $F$ is $\R$ or $\bbC$, and discrete when $F$ is finite); also, we shall require that $\ell \not = \chara K$.
                \begin{enumerate}
                    \item \textbf{($\ell$-adic representations that are not Galois):} 
                        \begin{itemize}
                            \item \textbf{(The trivial representation):} This is a bit of a silly example, but if $F$ were to be equipped with the discrete topology then any finite-dimensional $F$-linear representation of $\Gal(L/K)$ would be an $\ell$-adic representation for trivial reasons. Note that the trivial representation is a special case of this, since the trivial subgroup $1 \leq \GL_n(F)$ can not have any topology other than the discrete one. 
                            \item \textbf{(Finite Galois representations):} Any finite-dimensional $F$-linear represetation of a finite Galois group is trivially $\ell$-adic, due to the fact that every subset is defined to be open in the discrete topology. 
                        \end{itemize}
                    \item \textbf{(Galois representations):}
                        \begin{itemize}
                            \item \textbf{(Tate modules):} \index{Tate module} \index{Tate twist} Let $X$ be an abelian variety over $\Spec K$ and let us write $(\mu_{\ell^{\infty}})_{/X}$ for the base change:
                                $$(\mu_{\ell^{\infty}})_{/\Spec K} \x_{\Spec K} X$$
                            of the $\Spec K$-algebraic group:
                                $$(\mu_{\ell^{\infty}})_{/\Spec K} \cong \underset{n \in \N}{\lim} \Spec \frac{K[x]}{(x^{\ell^n} - 1)}$$
                            of all $\ell^{th}$-roots of unity to $X$, which is once more an commutative group scheme for trivial reasons. Then, the \textbf{$\ell$-adic Tate module} of $X$ is the abelian group of $\Spec K^{\sep}$-points of $\mu_{\ell^{\infty} /X}$; we shall denote it by $\T_{\ell}(X)$. Alternatively, one might define the $\ell$-adic Tate module of $X$ to be the filtered limit of all $\ell$-torsion subgroups of $X(\Spec K^{\sep})$, which form the following descending filtration: 
                                $$X[\ell] \supset X[\ell^2] \supset ... \supset \T_{\ell}(X)$$
                            wherein $X[\ell^n] \cong X(\Spec K^{\sep}) \tensor_{\Z} \Z/\ell^n\Z$ is the subgroup with $\ell^n$-torsion.
                                
                            Because $\ell$ is prime, $\T_{\ell}(X)$ is thus an abelian pro-$\ell$-group (i.e. a filtered limit of finite abelian $\ell$-groups), and hence isomorphic to a free $\Z_{\ell}$-module. 
                                \begin{enumerate}
                                    \item If $X$ were to be isomorphic to the multiplicative group scheme $(\G_m)_{/\Spec K}$ (i.e. the unique abelian variety of dimension $0$, up to isomorphisms) then:
                                        $$X[\ell^n] \cong (\G_m)_{/\Spec K}[\ell^n] \cong \Z/\ell^n\Z$$
                                    for all $n$, which would imply that:
                                        $$\T_{\ell}( (\G_m)_{/\Spec K} ) \cong \Z_{\ell}$$
                                    \item When $X$ is an elliptic curve (i.e. an abelian variety of dimension $1$; cf. definitions \ref{def: moduli_of_elliptic_curves} and \ref{def: abelian_varieties}), we can apply \cite[Corollary 6.4]{silverman_elliptic_curves} to get:
                                        $$\T_{\ell}(X) \cong \Z_{\ell} \oplus \Z_{\ell}$$
                                    
                                    More generally, one can make use of some \'etale homotopy theory to show that for all integers $N$ coprime with $p$, there exists the following decomposition of the $N$-torsion subgroup of $X(K)$:
                                        $$X[N] \cong \Z/N\Z \oplus \Z/N\Z$$
                                    First of all, it will have to be shown that every elliptic curve admits a finite \'etale covering which is also an elliptic curve. 
                                \end{enumerate}
                        \end{itemize}
                \end{enumerate}
            \end{example}
        
            \begin{definition}[Lisse sheaves] \label{def: lisse_sheaves}
                Let $F$ be a local field (we shall view finite fields as $0$-dimensional local fields) equipped with its natural topology (e.g. non-archimedean when $F$ is some sort of $\ell$-adic number field, archimedean when $F$ is $\R$ or $\bbC$, and discrete when $F$ is finite). We shall refer to functions into $F$ as being \say{$\ell$-adic} as typically, one takes $F$ to be $\Q_{\ell}$ or extensions thereof (the reason we are using $\ell$ instead of a simple \say{$p$} as our prime is historical: $\ell$-adic sheaves were first conceived for the purposes of the Riemann Hypothesis on varieties over characteristics $p$).
                \begin{enumerate}
                    \item \textbf{(Lisse $\ell$-adic functions):} Let $X$ be a \textit{totally disconnected} topological space (typically just locally profinite, although there are interesting non-profinite examples such as $\Q$). An $\ell$-adic function $f: X \to F$ shall then be called \textbf{lisse} if and only if it is \textit{compactly supported} and \textit{locally constant} (this terminology is suppose to be a nod to the notion of smooth functions on locally profinite spaces). The space of lisse $\ell$-adic functions on any open subset $U \subseteq X$ is denoted by $C^{\infty}_c(U, F)$ or simply $C^{\infty}_c(U)$ when $F$ is understood.
                    
                    One thing to note is that when $F = \bbC$, this notion does \textit{not} coincide with that of smoothness, since totally disconnected spaces can not admit any sort of archimedean metric. This is another reason why we opted for \say{lisse functions} instead of \say{smooth functions} or \say{bump functions}.
                    \item \textbf{(Lisse $\ell$-adic sheaves):} In analogy with the above notion of lisse $\ell$-adic functions, let us define a \textbf{lisse $\ell$-adic sheaf} as a \textit{finite-dimensional} $F$-linear local system over some pro-\'etale site $X_{\proet}$ of a given base scheme $X$. It is not hard to see that lisse sheaves on $X$ form a category, which we shall denote by $F\-\LocSys(X_{\proet})^{\fin}$.
                \end{enumerate}
            \end{definition}
        
            \begin{theorem}[The $\ell$-adic Monodromy Correspondence] \label{theorem: l_adic_monodromy_correspondence}
                Let $\ell$ be a prime, let $F$ be a local field (we shall view finite fields as $0$-dimensional local fields) equipped with its natural topology (e.g. non-archimedean when $F$ is some sort of $\ell$-adic number field, archimedean when $F$ is $\R$ or $\bbC$, and discrete when $F$ is finite). Also, let $X$ be a \textit{connected} base scheme. There is then the following equivalence of rigid symmetric monoidal categories:
                    $$F\-\LocSys(X_{\proet})^{\fin} \cong \Rep_F^{\cont}(\pi_1^{\fet}(X))$$
            \end{theorem}
                \begin{proof}
                    
                \end{proof}
            \begin{corollary}[Continuous Galois representations as lisse sheaves] \label{coro: continuous_galois_representations_as_lisse_sheaves}
                Let $K$ be a field whose characteristic is different from $\ell$. Then, we have the following equivalence of rigid symmetric monoidal categories:
                    $$F\-\LocSys(*_{\proet})^{\fin} \cong \Rep_F^{\cont}(\bfG_K)$$
            \end{corollary}
            \begin{example}[\'Etale cohomologies as geometric Galois representations] \label{example: etale_cohomologies_as_galois_representations}
                Let $K$ be a separably closed field and let $X$ be a smooth and proper scheme over $\Spec K$. Since $\ell$-adic cohomology is a Weil cohomology theory, the $\ell$-adic cohomologies $H^i_{\Q_{\ell}}(X)$ are, in particular, finite-dimensional. Since we wish to show that these cohomologies are naturall Galois representations, it then remains to verify that $\bfG_K$ indeed acts on them.
                            
                For this, let us first use \cite[\href{https://stacks.math.columbia.edu/tag/0BUM}{Tag 0BUM}]{stacks} along with the assumption that $X$ is a scheme over a separably closed field, we get that:
                    $$\pi^{\fet}(X) \cong \bfG_k$$
                Then, note that the chain complex $H^*_{\Q_{\ell}}(X)$ is actually the same as the \textit{finite-dimensional} chain complex of pro-\'etale cohomologies $H^*_{\proet}(X, \Q_{\ell})$. An appliction of theorem \ref{theorem: l_adic_monodromy_correspondence} then gives us the desired Galois action on the cohomologies $H^i_{\Q_{\ell}}(X)$.
            \end{example}
            
        \subsection{\texorpdfstring{$\ell$}{}-adic representations of finite fields}
        
        \subsection{\texorpdfstring{$\ell$}{}-adic representations of local fields}
    
    \section{The Weil Conjectures over finite fields}
        \subsection{The trace formula}
    
        \subsection{Rationality of the zeta functions}
        
        \subsection{The functional equation via Frobenii}
            \subsubsection{L-functions}
                Let us start the discussion by defining so-called $L$-functions  and examine some of the relevant properties that they exhibit. 
                    
                \begin{definition}[Selberg $L$-functions] \label{def: selberg_L_functions} \index{L-functions}
                    A \textbf{Selberg $L$-function} or simply, an \textbf{$L$-function} is a \textit{\href{https://en.wikipedia.org/wiki/Analytic_continuation}{\underline{meromorphic continuation}}} $F(s)$ to the entire complex plane (minus poles, of course) of a complex series of the form:
                        $$l(s) = \sum_{n = 1}^{+\infty} \frac{a_n}{s^n}$$
                    which is \textit{absolutely convergent} on the half-plane $\{s \in \bbC \mid \Re(s) > 1 \}$ and satisfies the following list of properties:
                        \begin{enumerate}
                            \item \textbf{(Meromorphy):} $F(s)$ should have at most one pole, and in the event that it does, the only pole should be $s = 1$. In other words, $F(s)$ should admit analytic continuations to $\bbC \setminus \{1\}$.
                            \item \textbf{(Ramanujan conjecture):} For all $\e > 0$, it should (conjecturally) be the case that:
                                $$a_1 = 1$$
                            and:
                                $$a_n \ll n^{\e}$$
                            for all $n > 1$. 
                            \item \textbf{(Functional equation):} Let $\Gamma(z)$ be the \href{https://en.wikipedia.org/wiki/Gamma_function}{\underline{Gamma function}} and let us require that every $L$-function $F(s)$ admit a so-called \textbf{gamma factor} $\gamma(s)$:
                                $$\gamma(s) := Q^s \prod_{j = 1}^N \Gamma(\omega_j s + \mu_j)$$
                            wherein $Q, \omega_j > 0$ and $\mu_j \in \{z \in \bbC \mid \Re(z) \geq 0\}$, along with a so-called \textbf{root number} $\alpha$ on the unit circle (i.e. a rotational factor) such that there exists a functional $\Phi \in \Func\left(\calM^1(\bbC, \{1\}), \bbC\right)$ (with $\calM^1(\bbC, \{1\})$ the set of all meromorphic functions with the only \textit{possible} pole at $1$) satisfying the following equation for all $s \in \bbC \setminus \{1\}$:
                                $$\Phi[F](s) = \alpha \overline{\Phi[F](1 - \overline{s})}$$
                            This can be thought of as a sort of symmetry/harmonicity condition imposed upon $L$-function.
                            \item \textbf{(Euler factorisation):} Over the half-plane $\{s \in \bbC \mid \Re(s) > 1 \}$, the $L$-function $F(s)$ (now simply the series $l(s)$) should also be factorisable into the following factors indexed by a certain set of prime numbers:
                                $$l(s) = \prod_{\text{$p$ prime}} l_p(s) = \prod_{\text{$p$ prime}} \exp\left( \sum_{n = 1}^{+\infty} \frac{b_{p^n}}{p^{n s}} \right)$$
                            wherein $b_{p^n} = O(p^{n\theta})$ for some $\theta < \frac12$. This factorisation is known as the \textbf{Euler factorisation}, and is the crucial bridge between complex analysis and number theory.
                        \end{enumerate}
                \end{definition}
                \begin{example}
                    Let $\h_{> 1}$ denote the half-place $\{s \in \bbC \mid \Re(s) > 1\}$. Also, a warning: \textit{$L$-functions are in no way, shape, or form simple creatures!}
                    \begin{enumerate}
                        \item \textbf{(Dirichlet series):} \index{L-functions! Dirichlet series} Every partial $L$-function, or in other words, every \href{https://en.wikipedia.org/wiki/Dirichlet_series}{\underline{Dirichlet series}} that is absolutely convergent on the half plane $\h_{> 1}$, is tautologically an $L$-function. 
                        \item \textbf{(The Riemann zeta functions):} \index{L-functions! Zeta functions} The (in)famous Riemann zeta function, which is the analytic continuation of the Dirichlet series:
                            $$\zeta(s) := \sum_{n = 1}^{+\infty} \frac{1}{n^s}$$
                        The perceptive reader might have noticed that we have not specified the domain of analytic continuation, and they should have. The only reason that we have not done as we ought to, is because we would like to give our dear readers a chance to attempt the famous exercise commonly referred to as \say{The Riemann Hypothesis}, which ask whether or not the only poles of the Riemann zeta function are the negative even integers and complex numbers with real part $\frac12$. Also, unlike most homework problems which would only earn the student a measly grade, this one actually has a rather sweet small prize of $1$ million dollars attached to it. That's \textit{the} way to earn enough money to buy a house doing maths research if you ask me.
                        
                        Let us actually try to show that the meromorphic continuation of the infinite series $\zeta(s)$ is in fact, an $L$-function.
                            \begin{enumerate}
                                \item \textbf{(Absolute convergence on $\h_{> 1}$):} Set $s = x + iy$ and consider the following:
                                    $$\sum_{n = 1}^{+\infty} \left|\frac{1}{n^s}\right| = \sum_{n = 1}^{+\infty} |e^{- \log(n) s}| = \sum_{n = 1}^{+\infty} |e^{- \log(n) (x + iy)}|  = \sum_{n = 1}^{+\infty} \left|\frac{1}{n^x} e^{- i \log(n) y}\right| = \sum_{n = 1}^{+\infty} \frac{1}{n^x}$$
                                Clearly, the series $\sum_{n = 1}^{+\infty} \frac{1}{n^x}$ converges if and only if $x > 1$, i.e. if and only if $\Re(s) > 1$. This proves that $\zeta(s)$ converges absolutely on $\h_{> 1}$.
                                \item \textbf{(Meromorphy):}
                                    \begin{enumerate}
                                        \item \textbf{(Holomorphy on $\h_{> 1}$):}
                                        \item \textbf{(Poles):} Let $\e > 0$ be arbitrary and let $s_0$ be a complex number such that:
                                            $$\Re(s_0) = 1 + \e$$
                                        At such a point in the half-plane $\h_{> 1}$ (which we note to be open in $\bbC$), we can evaluate the holomorphic function $\zeta(s)$ by evaluating Cauchy's integral formula around a contour $\gamma(\theta) = s_0 + \delta e^{i\theta}$ where $\delta > 0$:
                                            $$
                                                \begin{aligned}
                                                    \zeta(s_0) & = \frac{1}{2\pi i} \oint_{\gamma} \frac{\zeta(s)}{s - s_0} ds
                                                    \\
                                                    & = \frac{1}{2\pi i} \oint_{\gamma} \frac{\sum_{n = 1}^{+\infty} \frac{1}{n^s}}{s - s_0} ds
                                                    \\
                                                    & = \frac{1}{2\pi i} \sum_{n = 1}^{+\infty} \oint_{\gamma} \frac{e^{- \log(n) s}}{s - s_0} ds
                                                    \\
                                                    & = \frac{1}{2\pi i} \sum_{n = 1}^{+\infty} \int_0^{2\pi} \frac{e^{-\log(n) (s_0 + \delta e^{i\theta})}}{(s_0 + \delta e^{i\theta}) - s_0} ie^{i\theta} d\theta 
                                                    \\
                                                    & = \frac{1}{2\pi i} \sum_{n = 1}^{+\infty} \int_0^{2\pi} \frac{e^{-\log(n) (s_0 + \delta e^{i\theta})}}{\delta e^{i\theta}} ie^{i\theta} d\theta
                                                    \\
                                                    & = \frac{1}{2\pi \delta} \sum_{n = 1}^{+\infty} \int_0^{2\pi} e^{-\log(n) (s_0 + \delta e^{i\theta})} d\theta
                                                \end{aligned}
                                            $$
                                    \end{enumerate}
                                \item \textbf{(Ramanujan conjecture):}
                                \item \textbf{(Functional equation):}
                                \item \textbf{(Euler factorisation):}
                            \end{enumerate}
                    \end{enumerate} 
                \end{example}
        
        \subsection{The Riemann Hypothesis over finite fields}
        
    
	    
	    \chapter{Arithmetic differential geometry} \label{chapter: crystals}
    \begin{abstract}
        In this chapter, we introduce firstly the notion of divided power algebras over general commutative rings (also know in French circles, especially to those influenced by Pierre Berthelot, as PD-algebras, with \say{PD} standing for \say{puissances-divis\'ees}), which shall subsequently be employed in discussions regarding topics, notable among which is crystalline cohomology, a sort of analogue of the theory of vector bundles with flat connections in prime characteristics. These machineries play fundamental roles in the construction of crystalline cohomology, as well as in the theory of complete intersections. 
    \end{abstract}
    
    \minitoc
    
    \section{Divided power algebras}
        \subsection{Divided power rings}
            \subsubsection{Divided powers}
                \begin{definition}[Divided powers]
                    Let $R$ be a commutative ring and fix some $R$-ideal $I$. Then, a PD-structure on $R$ is give by a family of maps $\{\gamma_n: I \to I\}_{n \in \N}$ satisfying the following properties, for all $m \in \N$, $x, y \in I$, and $a \in R$
                        \begin{enumerate}
                            \item \textbf{(Raising to the zeroth power):} $\gamma_0(x) = 1$.
                            \item \textbf{(Raising to the first power):} $\gamma_1(x) = x$.
                            \item \textbf{(Addition of exponents):} $\gamma_m(x)\gamma_n(x) = \frac{(m + n)!}{m!n!} \gamma_{m + n}(x)$.
                            \item \textbf{(Multiplication of exponents):} $(\gamma_n \circ \gamma_m)(x) = \frac{(mn)!}{(m!)^n n!} \gamma_{mn}(x)$
                            \item \textbf{(Powers of multiples):} $\gamma_n(ax) = a^n\gamma_n(x)$.
                            \item \textbf{(Binomial expansion):} $\gamma_n(x + y) = \sum_{i=0}^n \gamma_i(x)\gamma_{n - i}(y)$
                        \end{enumerate}
                    A ring $R$ equipped with PD-structure $\left(I, \gamma := \{\gamma_n\}_{n \in \N}\right)$ is called a PD-ring. 
                \end{definition}
                \begin{remark}[Well-definiteness of PD-structures]
                    Note that all of the expressions are well-defined, since the fractions $\frac{(m + n)!}{m!n!}$ and $\frac{(mn)!}{(m!)^n n!}$ are in fact integers: to show that they are indeed integers, note, respectively, that:
                        $$\frac{(m + n)!}{m!n!} = \binom{m + n}{n} = \binom{m + n}{m}$$
                    (which implies that $(m + n)!$ is divisible by $m!n!$) and that:
                        $$\frac{(mn)!}{(m!)^n n!} = \binom{mn}{n}$$
                \end{remark}
                
                \begin{proposition}[Properties of PD-structures]
                    Let $(A, I, \gamma)$ be a triple consisting of a $\Z$-torsion-free ring $R$, an ideal $I$ of said ring, and a collection $\gamma := \{\gamma_n\}_{n \in \N}$ of endofunctions $\gamma_n: I \to I$. Then, we have the following for all $x \in I$ and all $n \in \N$:
                        \begin{enumerate}
                            \item If $(I, \gamma)$ is a PD-structure then $n! \gamma_n(x) = x^n$.
                            \item $(I, \gamma)$ is unique as a PD-structure on $R$.
                            \item $(I, \gamma)$ is a PD-structure if and only if $n! \gamma_n(x) = x^n$. 
                            \item $(I, \gamma)$ is a PD-structure on $R$ if and only if there exists a set of generators $x_{\alpha}$ of $I$ such that $x_{\alpha}^n \in n! I$. 
                        \end{enumerate}
                \end{proposition}
                    \begin{proof}
                        \noindent
                        \begin{enumerate}
                            \item Clearly:
                                $$0! \gamma_0(x) = 1 = x^0$$
                                $$1! \gamma_1(x) = x = x^1$$
                            and these form our inductive base case. Then, assume that $k! \gamma_k(x) = x^k$ for some $k \in \N$ (which we can do thanks to the established base case and the assumption that $R$ is $\Z$-torsion-free) and note that by the definition of PD-structures, we have the following:
                                $$x^{k + 1} = x^k \cdot x = k!\gamma_k(x) \cdot 1!\gamma_1(x) = k!\frac{(k + 1)!}{k! 1!} \gamma_{k + 1}(x) = (k + 1)!\gamma_{k + 1}(x)$$
                            Thus, we have completed the inductive step, and hence have shown that $n!\gamma_n(x) = x^n$ for all $x \in I$ and all $n \in \N$.
                            \item Suppose that to a fixed $R$-ideal $I$, one can associate two distinct PD-structures $\gamma := \{\gamma_n\}_{n \in \N}$ and $\theta := \{\theta_n\}_{n \in \N}$. Then, according to \textbf{1}, one has the following for all $x \in I$ and all $n \in \N$:
                                $$x^n = n!\gamma_n(x) = n!\theta_n(x)$$
                            which implies that $\gamma_n = \theta_n$ for all $n \in \N$, since $R$ is $\Z$-torsion-free.
                            \item We have already shown in \textbf{1} that $(I, \gamma)$ being a PD-structure implies that $x^n = n!\gamma_n(x)$ for all $n \in \N$ and all $x \in I$. Thus, it suffices to demonstrate that if $x^n = n!\gamma_n(x)$ for all $n \in \N$ and all $x \in I$ then $\gamma := \{\gamma_n\}_{n \in \N}$ is a PD-structure. Let us do this by checking the axioms defining PD-structures one by one.
                                \begin{enumerate}
                                    \item \textbf{(Raising to the zeroth power):} $1 = x^0 = 0! \gamma_0(x)$.
                                    \item \textbf{(Raising to the first power):} $x = x^1 = 1! \gamma_1(x)$.
                                    \item \textbf{(Addition of exponents):} Note that we have the following for all $m, n \in \N$ and all $x \in I$:
                                        $$m!\gamma_m(x) \cdot n!\gamma_n(x) = x^m \cdot x^n = x^{m + n} = (m + n)!\gamma_{m + n}(x)$$
                                    and as established above, the fraction $\frac{(m + n)!}{m!n!}$ is actually an integer for all $m, n \in \N$, and thus this implies that:
                                        $$\gamma_m(x)\gamma_n(x) = \frac{(m + n)!}{m!n!}\gamma_{m + n}(x)$$
                                    \item \textbf{(Multiplication of exponents):} Consider the following:
                                        $$
                                            \begin{aligned}
                                                \frac{(mn)!}{(m!)^n n!}\gamma_{mn}(x) & = \frac{1}{(m!)^n n!} x^{mn}
                                                \\
                                                & = \frac{1}{(m!)^n n!} \left((-)^n \circ (-)^m\right)(x)
                                                \\
                                                & = \left(\frac{1}{n!}(-)^n \circ \frac{1}{m!}(-)^m\right)(x)
                                                \\
                                                & = (\gamma_n \circ \gamma_n)(x)
                                            \end{aligned}
                                        $$
                                    Again, note that everything is well-defined here, since $\frac{(mn)!}{(m!)^n n!}$ is actually an integer.
                                    \item \textbf{(Powers of multiples):} This is rather straightforward:
                                        $$(ax)^n = a^n x^n = a^n n! \gamma(ax) = n!\gamma(ax)$$
                                    \item \textbf{(Binomial expansion):} Let $x, y$ be elements of $I$ and consider the following for all powers $n \in \N$:
                                        $$
                                            \begin{aligned}
                                                n!\gamma_n(x + y) & = (x + y)^n
                                                \\
                                                & = \sum_{i=0}^n \binom{n}{i} x^i y^{n - i}
                                                \\
                                                & = \sum_{i=0}^n \frac{n!}{(n - i)! i!} x^i y^{n - i}
                                                \\
                                                & = n!\sum_{i=0}^n \gamma_i(x)\gamma_{n - i}(y)
                                            \end{aligned}
                                        $$
                                    Thus:
                                        $$\gamma_n(x + y) = \sum_{i=0}^n \gamma_i(x)\gamma_{n - i}(y)$$
                                    for all $n \in \N$ and all $x, y \in I$. 
                                \end{enumerate}
                            \item 
                        \end{enumerate}
                    \end{proof}
                    
                \begin{example}
                    \noindent
                    \begin{enumerate}
                        \item \textbf{(The trivial PD-structure):} On the zero ideal of any (not even necessary commutative) ring, there exists an obviously trivial PD-structure given by the zero morphism in the abelian category of rings. 
                        \item \textbf{(Divided power structures on $\Z_{(p)}$-algebras):} Let $p$ be a prime and let $R$ be a $\Z_{(p)}$-algebra, i.e. a commutative ring in which all integers not divisible by $p$ are declared to be invertible, and consider the ideal $I := pR$. One can then endow this ideal with a PD-structure $\gamma = \{\gamma_n\}_{n \in \N}$ whose components $\gamma_n$ are given by:
                            $$\forall x \in I: \forall n \in \N: \gamma_n(x) := \frac{1}{n!}x^n$$
                        (which we note, first and foremost, to be well-defined, since every integer not divisible by $p$ is invertible in $R$, and in the event that $n$ is divisible by $p$, the fraction $\frac{x^n}{n!} = \frac{(pa)^n}{n!}$ would still be an integer; one may show this, for instance, by using the fact that within the prime factorisation of $n!$, the exponent of $p$ is $\sum_{r=1}^{+\infty} \left\lfloor \frac{n}{p^r} \right\rfloor$); we leave the verification of the axioms defining Pd-structures to the readers (alternatively, one can simply notice that for all $x \in I$ and all $n \in \N$, one has that $x^n = n!\gamma_n(x)$, which we know implies that $\gamma$ is a PD-structure on $I$).
                        \item \textbf{(Divided power structures on $\Q$-algebras):} Given any $\Q$-algebra $R$ and any $R$-ideal $I$, there is an obvious PD-structure $\gamma := \{\gamma_n\}_{n \in \N}$ on $I$ given by:
                            $$\gamma_n(x) := \frac{1}{n!} x^n$$
                    \end{enumerate}
                \end{example}
                
                \begin{proposition}
                    Let $R$ be a ring and let $(I, \gamma), (J, \delta)$ be a PD-structures on $R$-ideals $I$ and $J$. Then:
                        \begin{enumerate}
                            \item The PD-structures $\gamma$ and $\delta$ agree on $IJ$. In other words, one has a (unique) PD-structure $(IJ, \theta)$ given by:
                                $$(IJ, \theta) = (IJ, \gamma) = (IJ, \delta)$$
                            \item If $\gamma$ and $\delta$ agree on $I \cap J$ then they coincide with the restriction of a (unique) PD-structure $\e$ on $I + J$ down onto $I \cap J$.
                        \end{enumerate}
                \end{proposition}
                    \begin{proof}
                        For the sake of reference, let us recall first of all that for any commutative ring $R$ and all $R$-ideals $I$ and $J$, we have the following definitions:
                            $$IJ := \left\{\sum_{1 \leq i, j \leq N} x_iy_j \: \bigg| \: x \in I, y \in y, N \in \N \right\}$$
                            $$I + J := \{x + y \mid x \in I, y \in J\}$$
                            $$I \cap J := \{x \in R \mid (x \in I) \wedge (x \in J)\}$$
                        \begin{enumerate}
                            \item Let $x$ and $y$, respectively, be two arbitrary elements of $I$ and $J$ and consider the following, wherein $n$ is an arbitrary natural number:
                                $$\delta_n(xy) = x^n\delta_n(y) = n!\gamma_n(x) \frac{1}{n!}y^n = \gamma_n(x)y^n = \gamma_n(xy)$$
                            This shows that $\delta$ and $\gamma$ coincide on products of elements of $I$ and $J$. Now, because elements of $IJ$ are finite sums of such products, let us simply consider a sum of two such products:
                                $$\delta_n(xy + x'y') = \sum_{i = 0}^n \delta_i(xy)\delta_{n - i}(x'y') = \sum_{i = 0}^n \gamma_i(xy)\gamma_{n - i}(x'y') = \gamma_n(xy + x'y')$$
                            Thus, $\delta$ and $\gamma$ indeed coincide on finite sums of products of $I$ and of $J$, i.e. on all elements of $IJ$. In other words, they coincide on $IJ$. One can then let $\theta$ be the restriction of $\delta$ (or equivalently, of $\gamma$) down onto $IJ$ (which we note to be a sub-ideal of both $I$ and $J$), and it is necessarily unique by virtue of being a PD-structure.
                            \item Assume firstly that $\gamma$ and $\delta$ conincide with the restriction of a PD-structure $\e$ on $I + J$ down onto $I \cap J$. 
                        \end{enumerate}
                    \end{proof}
                    
                \begin{proposition}[$p$-powers in PD-rings] \label{prop: p_powers_in_PD_rings}
                    Let $p$ be a prime and let $(R, I , \gamma)$ be a PD-ring. Also, assume that $p$ is nilpotent in $R/I$. Then, the ideal $I$ is locally nilpotent if and only if $p$ is nilpotent in $R$. 
                \end{proposition}
                    \begin{proof}
                        Before we start proving the proposition, let us remind ourselves with the fact that an element $x \in R$ is nilpotent in $R/I$ if and only if there exists a natural number $N$ such that $x^N \in I$. 
                        \begin{enumerate}
                            \item Suppose first of all, that the ideal $I$ is locally nilpotent, i.e. that for all elements $x \in I$, there exists a natural number $N_x$ such that $x^{N_x} = 0$. Then, we can use the fact that $\gamma$ is a PD-structure on $I$ if and only if $x^n = n!\gamma_n(x)$ for all $x \in I$ and all $n \in \N$; in combination with the local nilpotency hypothesis, this implies that for all $x \in I$, there exists $N_x \in \N$ such that:
                                $$0 = x^{N_x} = N_x!\gamma_{N_x}(x)$$
                            \item
                        \end{enumerate}
                    \end{proof}
                    
            \subsubsection{Categories of divided power algebras over a ring}
                \begin{definition}[PD-homomorphisms] \label{def: PD_homomorphisms}
                    Let $(A, I , \gamma)$ and $(B, J, \delta)$ be two PD-rings and let $\phi: A \to B$ be a ring homomorphism. Then, $\phi$ is a homomorphism of PD-rings if and only if $\phi(I) \subseteq J$ and or all natural numbers $n$, one has the following commutative diagrams:
                        $$
                            \begin{tikzcd}
                            	{I} & {I} \\
                            	{J} & {J}
                            	\arrow["{\phi}"', from=1-1, to=2-1]
                            	\arrow["{\phi}", from=1-2, to=2-2]
                            	\arrow["{\delta}", from=2-1, to=2-2]
                            	\arrow["{\gamma}", from=1-1, to=1-2]
                            \end{tikzcd}
                        $$
                    Often, given a PD-homomorphism $\phi: (A, I, \gamma) \to (B, J, \delta)$, one says that $B$ is a PD-algebra over $A$. 
                \end{definition}
                \begin{remark}[Categories PD-algebras]
                    Via this definition of PD-homomorphism, for each $\Z_{(p)}$-algebra $A$ (we are consider $\Z_{(p)}$-algebras first because they have been shown to come equipped with canonical PD-sturctures), one gets directly the category of PD-algebras over $A$, denoted by ${}^{A/}\Comm\Alg^{\pd}$. It is a subcategory of ${}^{A/}\Comm\Alg$ that is \textbf{not full}, as not all homomorphisms of (commutative) $A$-algebras are PD-homomorphisms. Note that $\Q$-algebras are $\Z_{(p)}$-algebra (as $\Q \cong \Z_{(p)}[1/p]$), and so categories of PD-algebras over $\Q$-algebras exist in a similar fashion.
                    \\
                    Note that it is not the case that only PD-algebras over $\Z_{(p)}$ (for some prime $p$) form categories. However, if the base PD-ring is not a $\Z_{(p)}$-algebra, then there need not exist a terminal PD-ring (and as a consequence, no \say{absolute} category of PD-rings like ${}^{\Z_{(p)}/}\Comm\Alg$ or ${}^{\Q/}\Comm\Alg$), as there is no canonical way to associate PD-structures to ideals of general commutative rings.  
                \end{remark}
                
                \begin{proposition}[(Co)completeness of PD-algebra categories]
                    Let $p$ be a prime. For any $\Z_{(p)}$-algebra $A$, the category of PD-algebras over $A$ is both complete and cocomplete. Moreover, limits of PD-$A$-algebras agree with those of their underlying commutative $A$-algebras; the colimits, however, need not coincide with those of the same shape but taken in ${}^{A/}\Comm\Alg$. 
                \end{proposition}
                    \begin{proof}
                        \noindent
                        \begin{enumerate}
                            \item \textbf{(Completeness):} Because limits can be built out of products and equalisers (see \cite{maclane}, theorem V.2.1), it will suffice to show that ${}^{A/}\Comm\Alg^{\pd}$ has all products and all equalisers.
                            \\
                            First, let us show that ${}^{A/}\Comm\Alg^{\pd}$ has all products, and to that end, let:
                                $$\left\{\left(B^{(i)}, \b^{(i)}, \gamma^{(i)}\right)\right\}_{i \in I}$$
                            be a small discrete diagram of PD-algebras over $A$ (understood to be equipped with the canonical PD-structure on $\Z_{(p)}$-algebras). Now, consider the ideal $\prod_{i \in I} \b^{(i)}$ of the ring $\prod_{i \in I} B^{(i)}$, and note that because addition and multiplication on products of rings are determined component-wise, there is a natural PD-structure given by:
                                $$\prod_{i \in I} \gamma^{(i)} = \left\{\prod_{i \in I} \gamma_n^{(i)}\right\}_{n \in \N}$$
                            Thus, the category ${}^{A/}\Comm\Alg^{\pd}$ has arbitrary products. 
                            \\
                            Now, let us prove that ${}^{A/}\Comm\Alg^{\pd}$ has all equalisers, which we can do by checking if any diagram consisting of a pair of parallel PD-homomorphisms of PD-$A$-algebras $\phi, \psi: B \toto C$ has a limit that is also a PD-$A$-algebra.  
                            \item \textbf{(Cocompleteness):}
                        \end{enumerate}
                    \end{proof}
                \begin{corollary}[Free PD-algebras] \label{coro: free_PD_algebras}
                    Let $p$ be a prime and let $A$ be any $\Z_{(p)}$-algebra. 
                        \begin{enumerate}
                            \item The forgetful functor:
                                $$\oblv: {}^{A/}\Comm\Alg^{\pd} \to {}^{A/}\Comm\Alg$$
                            does not preserve colimits in general (finite or otherwise). 
                            \item However, $\oblv$ does admit a left-adjoint.
                        \end{enumerate}
                \end{corollary}
                    \begin{proof}
                        \noindent
                        \begin{enumerate}
                            \item Let $I$ be the shape of some diagram of PD-algebras over $A$ (again, understood to be equipped with the canonical PD-structure on $\Z_{(p)}$-algebras). As shown above, the colimit taken in ${}^{A/}\Comm\Alg^{\pd}$ over $I$ need not coincide with that taken in ${}^{A/}\Comm\Alg$, and thus the forgetful functor from ${}^{A/}\Comm\Alg^{\pd}$ to ${}^{A/}\Comm\Alg$ need not preserve colimits in general. The rest follows directly. 
                            \item Because limits of PD-algebras are just limits of the underlying commutative algebras, the forgetful functor:
                                $$\oblv: {}^{A/}\Comm\Alg^{\pd} \to {}^{A/}\Comm\Alg$$
                            must preserve limits, and because ${}^{A/}\Comm\Alg^{\pd}$ is locally small (by virtue of being a subcategory of the locally small category ${}^{A/}\Comm\Alg$), complete, and cocomplete (which implies that ${}^{A/}\Comm\Alg^{\pd}$ is presentable), it therefore admits a left-adjoint by the Special Adjoint Functor Theorem \cite[Theorem V.8.2]{maclane}: in essence, we have a well-defined notion of free PD-algebras over a given base commutative ring.   
                        \end{enumerate}
                    \end{proof}
                    
                \begin{definition}[PD-envelopes] \label{def: PD_envelopes}
                    Let $p$ be a prime and fix a $\Z_{(p)}$-algebra. Then, the construction of free PD-$A$-algebras that is left-adjoint to the forgetful functor $\oblv: {}^{A/}\Comm\Alg^{\pd} \to {}^{A/}\Comm\Alg$ (cf. corollary \ref{coro: free_PD_algebras}) shall be called the \textbf{PD-enveloping functor} over $A$; we denote it by:
                        $${}^{\pd}(-): {}^{A/}\Comm\Alg \to {}^{A/}\Comm\Alg^{\pd}$$
                    or ${}^{\pd}(-)_A$ when the base ring $A$ needs emphasis.
                \end{definition}
            
            \subsubsection{Induced divided power structures}
                \begin{definition}[Induced PD-structures]
                    Let $(A, I, \gamma)$ be a PD-ring and let $B$ be an $A$-algebra. Then, we say that the PD-structure $\gamma$ extends to a PD-structure $\overline{\gamma}$ on the ideal $IB$ of $B$ if there exists a homomorphism of PD-rings from $(A, I, \gamma)$ to $(B, IB, \overline{\gamma})$. Sometimes, we might refer to PD-structures such as $\overline{\gamma}$ above as induced PD-stuctures. 
                \end{definition}
                
                \begin{proposition}[Existence and uniqueness of induced PD-structures] \label{prop: induced_PD_structures_existence_and_uniqueness}
                    Let $(A, I, \gamma)$ be a PD-ring and let $B$ be an $A$-algebra. Then, $\gamma$ extends to a PD-structure if at least one of the following conditions is satisfied:
                        \begin{enumerate}
                            \item $IB = 0$.
                            \item $I$ is a principal ideal.
                            \item $B$ is flat as an $A$-module.
                        \end{enumerate}
                    Furthermore, if $\gamma$ does indeed extend to a PD-structure on $IB$, then said induced PD-structure will be unique.
                \end{proposition}
                    \begin{proof}
                        \noindent
                        \begin{enumerate}
                            \item \textbf{(Existence):} 
                                \noindent
                                \begin{enumerate}
                                    \item We have already seen that the zero ideal of any ring possesses a canonical PD-structure, so if $IB = 0$ then $\gamma$ trivially extends to a (unique) PD-structure on $IB$.
                                    \item Now, suppose that $I$ is a principal ideal, say, generated by some element $a \in A$. Suppose also, that $B$ is given by the ring homomorphism $\phi: A \to B$.
                                    \item Consider the following short exact sequence of $A$-modules:
                                        $$0 \to I \to A \to A/I \to 0$$
                                    Applying the functor $- \tensor_A B$ then gives the following commutative diagram of short sequences of $B$-modules, which we note to both be exact due to the assumption that $B$ is flat over $A$ (and less significantly, due to the fact that the left-adjoint $- \tensor_A B$ preserves finite colimits):
                                        $$
                                            \begin{tikzcd}
                                            	{0} & {I \tensor_A B} & {B} & {A/I \tensor_A B} & {0} \\
                                            	{0} & {IB} & {B} & {B/I} & {0}
                                            	\arrow["{\cong}", from=1-2, to=2-2]
                                            	\arrow[Rightarrow, from=1-3, to=2-3, no head]
                                            	\arrow["{\cong}", from=1-4, to=2-4]
                                            	\arrow[from=1-2, to=1-3]
                                            	\arrow[from=1-3, to=1-4]
                                            	\arrow[from=2-2, to=2-3]
                                            	\arrow[from=2-3, to=2-4]
                                            	\arrow[from=2-4, to=2-5]
                                            	\arrow[from=1-4, to=1-5]
                                            	\arrow[from=1-1, to=1-2]
                                            	\arrow[from=2-1, to=2-2]
                                            \end{tikzcd}
                                        $$
                                    
                                \end{enumerate}
                            \item \textbf{(Uniqueness):} Assume that at least one of the conditions guaranteeing that an induced PD-structure $\overline{\gamma}$ exists is satisfied, and suppose to the contrary that there exist two distinct induced PD-structures on $IB$, say $\overline{\gamma}$ and $\tilde{\gamma}$.
                        \end{enumerate}
                    \end{proof}
                    
        \subsection{Crystalline sites}
            \subsubsection{PD-thickenings}
                \begin{definition}[PD-thickenings] \label{def: PD_thickenings}
                    Let $p$ be a prime number, let $(A, I, \gamma)$ be a PD-algebra over $\Z_{(p)}$, and let $C$ be an $A$-algebra such that $IC = 0$ and such that $p$ is nilpotent in $C$ (for why we require these two conditions, refer to propositions \ref{prop: p_powers_in_PD_rings} and \ref{prop: induced_PD_structures_existence_and_uniqueness}). 
                        \begin{enumerate}
                            \item \textbf{(PD-thickenings):} Within the above context, we say that a \textbf{PD-thickening} over $(A, I, \gamma)$ of $C$ is a PD-homomorphism:
                                $$\phi: (A, I, \gamma) \to (B, J, \delta)$$
                            over $\Z_{(p)}$ (cf. definition \ref{def: PD_homomorphisms}) that fits into a commutative pentagon in ${}^{\Z_{(p)}/}\Comm\Alg$ as follows:
                                $$
                                    \begin{tikzcd}
                                    	A && B \\
                                    	{A/I} && {B/J} \\
                                    	& C
                                    	\arrow[two heads, from=1-1, to=2-1]
                                    	\arrow[two heads, from=1-3, to=2-3]
                                    	\arrow["\phi", from=1-1, to=1-3]
                                    	\arrow[from=2-1, to=3-2]
                                    	\arrow[from=3-2, to=2-3]
                                    \end{tikzcd}
                                $$
                            where $A \to A/I$ and $B \to B/J$ are the canonical quotient maps (note that the composition:
                                $$A \to A/I \to C$$
                            exists precisely because $IC = 0$, and the map:
                                $$B/J \to C$$
                            exists because $p$ is nilpotent in $C$). 
                            \item \textbf{(PD-thickening homomorphisms):} A \textbf{homomorphism between two PD-thickenings}:
                                $$\phi: (A, I, \gamma) \to (B, J, \delta)$$
                            and:
                                $$\phi': (A, I, \gamma) \to (B', J', \delta')$$
                            of $C$ over $(A, I, \gamma)$ is nothing more than a mapping of commutative pentagons in ${}^{\Z_{(p)}/}\Comm\Alg$ of the following form:
                                $$
                                    \begin{tikzcd}
                                    	&& A &&&& {B'} \\
                                    	&& {A/I} && C && {B'/J'} \\
                                    	A &&&& B \\
                                    	{A/I} && C && {B/J}
                                    	\arrow[two heads, from=3-1, to=4-1]
                                    	\arrow[two heads, from=3-5, to=4-5]
                                    	\arrow["\phi", from=3-1, to=3-5]
                                    	\arrow[from=4-1, to=4-3]
                                    	\arrow[from=4-3, to=4-5]
                                    	\arrow["{\phi'}", from=1-3, to=1-7]
                                    	\arrow["\cong", from=3-1, to=1-3]
                                    	\arrow["\cong", from=4-1, to=2-3]
                                    	\arrow["\psi", from=3-5, to=1-7]
                                    	\arrow[two heads, from=1-7, to=2-7]
                                    	\arrow[two heads, from=1-3, to=2-3]
                                    	\arrow[from=4-5, to=2-7]
                                    	\arrow[from=2-3, to=2-5]
                                    	\arrow[from=2-5, to=2-7]
                                    	\arrow["\cong", from=4-3, to=2-5]
                                    \end{tikzcd}
                                $$
                            wherein $\psi: (B, J, \delta) \to (B', J', \delta')$ is a PD-homomorphism.
                        \end{enumerate}
                \end{definition}
            
            \subsubsection{Crystalline sites}
                \begin{definition}[Crystalline affine schemes] \label{def: crystalline_affine_schemes}
                    Using the notion of homomorphisms of PD-thickenings given in definition \ref{def: PD_thickenings}, we can define so-called \say{categories of crystalline affine schemes}. Specifically, the category of $C$-crystalline affine schemes over $\Spec A$, denote by:
                        $$\Sch(C)^{\aff, \crys, \gros}_{/\Spec A}$$
                    is dual to the category of PD-thickenings of $C$ over $(A, I, \gamma)$ and homomorphisms between them, which is denoted by:
                        $${}^{(A, I, \gamma)/}\Comm\Alg(C)^{\crys, \gros}$$
                    or simply:
                        $${}^{A/}\Comm\Alg(C)^{\crys, \gros}$$
                    when the PD-structure on $A$ is fixed.
                \end{definition}
                \begin{remark}[Big and small categories of crystalline affine schemes] \label{remark: big_and_small_categories_of_crystalline_affine_schemes}
                    
                \end{remark}
        
        \subsection{Differential-graded divided power algebras}
            \subsubsection{Divided power polynomial algebras}
            
            \subsubsection{Tate resolutions}
            
        \subsection{Applications to complete intersections}
            \subsubsection{Local complete intersections}
            
            \subsubsection{Smooth ring maps and diagonals}
            
            \subsubsection{Freeness of the conormal bundle}
        
    \section{Crystalline cohomology and arithmetic D-modules}
        \subsection{Arithmetic jets}
            \subsubsection{Arithmetic derivations}
                \paragraph{Delta-rings}
                    \begin{definition}[$p$-derivations] \label{def: p_derivations}
                        \noindent
                        \begin{enumerate}
                            \item A set-map $\delta_p: B \to B$ is called an \textbf{absolute $p$-derivation} if for all $b, b' \in B$ and all $a \in \Z$, we have:
                                $$\delta_p(a) = 0$$
                                $$\delta_p(b + b') = \delta_p(b) + \delta_p(b') + \left(-\frac1p\sum_{i=0}^{p-1} \binom{p}{i} b^ib'^{p-i}\right)$$
                                $$\delta_p(bb') = b^p\delta_p(b') + \delta_p(b)b'^p + p\delta_p(b)\delta_p(b')$$
                            Note that in order for the binomial coefficients $\binom{p}{i} = \frac{p!}{i! (p - i)!}$ (or even $\frac1p \binom{p}{i} = \frac{(p - 1)!}{i! (p - i)!}$ for that matter) to always be well-defined, we will usually want to require $B$ to be a commutative algebra over $\Z_{(p)}$ instead of simply being any commutative ring. 
                            \item Let $A$ be a $\Z_{(p)}$-algebra and suppose that $B$ is an $A$-algebra. Then, a \textbf{$p$-derivation on $B$ with coefficients in $A$} is a set-map:
                                $$\delta_p: B \to B$$
                            satisfying
                                $$\delta_p(a) = 0$$
                                $$\delta_p(b + b') = \delta_p(b) + \delta_p(b') + \left(-\frac1p\sum_{i=0}^{p-1} \binom{p}{i} b^ib'^{p-i}\right)$$
                                $$\delta_p(bb') = b^p\delta_p(b') + \delta_p(b)b'^p + p\delta_p(b)\delta_p(b')$$
                            for all $a \in A$ and $b, b' \in B$.
                        \end{enumerate}
                    \end{definition}
                    \begin{convention}
                        For the sake of convenience, let us from now on say that $p$-derivations satisfy $p$-linearity and the $p$-Leibniz rule.
                    \end{convention}
                    
                    \begin{proposition}[$p$-derivations and Frobenius lifts] \label{prop: p_derivations_and_frobenius_lifts}
                        This is \cite[Remark 2.2]{bhatt_scholze_prisms}.
                        
                        If $\delta_p: B \to B$ is a $p$-derivation, then the endomorphism:
                            $$\phi^p: B \to B: b \mapsto b^p + p\delta_p(b)$$
                        will be a Frobenius lift on $B$. Conversely, if we have any Frobenius lift $\phi^p: B \to B$, and if the ring $B$ is $p$-torsion-free (i.e. $p$ is not a zero-divisor in $B$), then we will get a $p$-derivation $\delta_p$ on $B$:
                            $$\delta_p: B \to B: b \mapsto \frac{\phi^p(b) - b^p}{p}$$
                        We say that the $p$-derivation $\delta_p$ and the lift of Frobenius $\phi^p$ are attached to one another.
                    \end{proposition}
                        \begin{proof}
                            Firstly, suppose that $\delta_p$ is a $p$-derivation on $B$. Reducing modulo $p$ the map:
                                $$\phi^p: B \to B: b \mapsto b^p + p\delta_p(b)$$
                            clearly gives the $p^{th}$-power Frobenius, and so $\phi^p$ is a Frobenius lift. 
                            \\
                            Conversely, suppose that $\phi^p: B \to B$ is a Frobenius lift, and that $B$ is $p$-torsion-free. Then, it is simply a matter of manually checking the axioms defining a $p$-derivation $\delta_p$ on $B$. Note that we require $B$ to be $p$-torsion-free so that the expression for $\delta_p$ would be well-defined in $B$.
                        \end{proof}
                    \begin{example}[Examples of $p$-derivations] \label{example: p_derivations}
                        \noindent
                        \begin{enumerate}
                            \item A prototypical example of a $p$-derivation is the Fermat quoient $\del_p$ on $\Z$. This is the set-map given by:
                                $$\del_p z := \frac{z - z^p}{p}$$
                            for all $z \in \Z$. We can see that it is attached to the Frobenius lift:
                                $$\phi^p(z) := z^p + p \frac{z - z^p}{p} = z$$
                            i.e. $\phi^p = \id_{\Z}$. Note that $\del_p z$ is indeed an integer for any $z \in \Z$, as Fermat's little theorem tells us that:
                                $$z^p \equiv z \pmod{p}$$
                            \item One could also extend the definition of the Fermat quotient to any $p$-torsion-free commutative ring $A$ (and especially, any $\Q$-algebra $A$; note that $\Q = \Z_{(p)}[1/p]$), which we will denote by $\del_{p,A}$. For any element $a \in A$, we have:
                                $$\del_{p,A}a = \frac{a - a^p}{p}$$
                            and the lift of Frobenius is $\id_A$.
                        \end{enumerate}
                    \end{example}
                        
                    \begin{proposition}
                        Let $\delta_p$ be a $p$-derivation on some ring $B$, and let $I$ be an ideal of $B$ such that:
                            $$\delta_p(I) \subseteq I$$
                        Then $\delta_p(I^n) \subseteq I^n$ for all $n \in \N$. 
                    \end{proposition}
                        \begin{proof}
                            This is an immediate consequence of the definition of $p$-derivations.
                        \end{proof}
                        
                    \begin{definition}[$\delta$-rings] \label{def: delta_rings}
                        This is \cite[Definition 2.1]{bhatt_scholze_prisms}.
                        \begin{enumerate}
                            \item A ring equipped with a $p$-derivation is called a \textbf{$\delta$-ring}, or when $p$ is not fixed, a $\delta_p$-ring. A morphism of $\delta$-rings is a homomorphism $B \to B'$ of rings that commute with the $p$-derivations $\delta_p$ and $\delta'_p$ on them, i.e. one gets the following commutative diagram in $\Sets$:
                                $$
                                    \begin{tikzcd}
                                        B' \arrow[r, "\delta_p'"]         & B'          \\
                                        B \arrow[u] \arrow[r, "\delta_p"] & B \arrow[u]
                                    \end{tikzcd}
                                $$
                            \item Relatively, one could define \textbf{$A$-$\delta$-algebras} as ring homomorphisms $A \to B$, with $A$ a $\Z_{(p)}$-algebra, equipped with $p$-derivations $\delta_p: B \to B$ on $B$ with coefficients in $A$. In other words, $A$-$\delta$-algebras are commutative diagrams in $\Sets$ as follows:
                                $$
                                    \begin{tikzcd}
                                        B \arrow[r, "\delta_p"]               & B           \\
                                        A \arrow[u] \arrow[r, "{\del_{p,A}}"] & A \arrow[u]
                                    \end{tikzcd}
                                $$
                            A morphism of $A$-$\delta$-algebras are commutative diagrams in $\Sets$ of the following form:
                                $$
                                    \begin{tikzcd}
                                        B' \arrow[r, "\delta_p'"]         & B'          \\
                                        B \arrow[u] \arrow[r, "\delta_p"] & B \arrow[u] \\
                                        A \arrow[u] \arrow[r, "\del_{p, A}"]             & A \arrow[u]
                                    \end{tikzcd}
                                $$
                            \item $\delta$-rings form a category, which we will denote by $\delta\Cring$, or when the prime $p$ is not clear from the context, $\delta_p\Cring$, whose objects are $p$-derivations:
                                $$\delta_p: B \to B$$
                            and whose morphisms are commutative diagrams in $\Sets$:
                                $$
                                    \begin{tikzcd}
                                        B' \arrow[r, "\delta_p'"]         & B'          \\
                                        B \arrow[u] \arrow[r, "\delta_p"] & B \arrow[u]
                                    \end{tikzcd}
                                $$
                        \end{enumerate}
                    \end{definition}
                    
                    \begin{remark}[Arithmetic and algebraic derivations] \label{remark: arithmetic_and_algebraic_derivations}
                        \noindent
                        \begin{enumerate}
                            \item Note that when $\chara B \not = p$, $p$-derivations are not derivations in the usual sense. In characteristic $p$, we could use Fermat's little theorem to see that:
                            $$\delta_p(b + b') = \delta_p(b) + \delta_p(b')$$
                            $$\delta_p(bb') = b^p\delta_p(b') + \delta_p(b)b'^p \equiv b\delta_p(b') + \delta_p(b)b' \pmod{p}$$
                            The first equation also implies that $\delta_p$ is $\Z$-linear in characteristic $p$. Thus, when $\chara B = p$, $p$-derivations are actually derivations, or $\Z$-derivations for that matter, in the usual sense.
                            \item More generally, if $B$ is an $A$-algebra (for some commutative $\Q$-algebra $A$), then $p$-derivations on $B$ with coefficients in $A$ are derivations in the usual sense if and only if $\chara B = p$.
                            \item More functorially, we recognise that a $\delta_p$-ring $B$ in characteristic $p$ is simply an $\F_p$-algebra, and so $p$-derivations on $\F_p$-algebras are just derivations in the usual sense. Furthermore, this is not just a $\Z$-derivation, but an $\F_p$-derivation. Relatively, a $p$-derivation $\delta_p$ on $B$ with coefficients in some $\Z\left[\frac1p\right] \tensor_{\Z} \F_p$-algebra $A$ is just an $A$-derivation in the usual sense. Note furthermore that $\Z\left[\frac1p\right] \tensor_{\Z} \F_p \cong \F_p[\frac1p]$. This is due to the fact that the free functor:
                                $$\Z[-]: \Sets \to \Cring$$
                            as the left-adjoint of the forgetful functor $\Cring \to \Sets$, commutes with colimits, $-\tensor_{\Z} \F_p$ in this instance; that is to say, we have the following natural isomorphisms:
                                $$\Z[-] \tensor_{\Z} \F_p \cong (\F_p \tensor_{\Z} \Z)[-] \cong \F_p[-]$$
                        \end{enumerate}
                    \end{remark}
                    \begin{example}[$p$-derivations that are (not) actually derivations] \label{example: p_derivations_and_derivations}
                        \noindent
                        \begin{enumerate}
                            \item Consider the algebra $B := \F_p[t]$, with $t \not \in p\Z$. There, $\del_{p, \F_p}$ is defined by:
                                $$\del_{p, \F_p}a = 0$$
                                $$\del_{p, \F_p}(af + a'f') = a\del_{p, \F_p}f + a'\del_{p, \F_p}f'$$
                                $$\del_{p, \F_p}(ff') = f\del_{p, \F_p}f' + (\del_{p, \F_p}f) f'$$
                            for all $a, a' \in A$ and $f, f' \in \F_p[t]$.
                            \item $p$-derivations on neither $\Z_p$ nor $\Q_p$ are actual derivations, as these rings are of characteristic $0$ 
                        \end{enumerate}
                    \end{example}
                     
                    \begin{proposition}[Properties of $\delta\Cring$] \label{prop: (co)limits_of_delta_rings}
                        Let $p$ be any prime number. The following statements come from example 2.6 and remark 2.7 in \cite{bhatt_scholze_prisms}.
                        \begin{enumerate}
                            \item $\delta_p\Cring$ has $(\Z, \del_p)$ as the initial object, with $\del_p$ the Fermat quotient. Due to this, one could define the coslice category ${}^{(A, \del_{p,A})/}\delta_p\Comm\Alg$ for any $\Z_{(p)}$-algebra $A$. 
                            \item The category $\delta_p\Cring$ is both complete and cocomplete. Furthermore, said (co)limits commute with the forgetful functor:
                                $$\oblv_p: \delta_p\Cring \to \Cring$$
                            \item The forgetful functor:
                                $$\oblv_p: \delta_p\Cring \to \Cring$$
                            has both left and right-adjoints. The left-adjoint is of course the free construction, and the right-adjoint is the Witt vector functor $\W$; the latter point implies that rings of Witt vectors are naturally $\delta$-rings.
                        \end{enumerate}
                    \end{proposition}
                    
                \paragraph{Arithmetic Leibniz algebras}
                    \begin{definition}[$p$-(pre)derivations and $p$-Leibniz algebras] \label{def: arithmetic_leibniz_algebras}
                        Let $k$ be a \textbf{$p$-torsion-free} ring and let $(\V, \tensor, 1)$ be a monoidal $k$-linear category. 
                            \begin{enumerate}
                                \item A left/right-$p$-prederivation on an (not necessarily commutative and unital) algebra $\left(\g, \nabla\right)$ internal to $\V$ is a morphism of objects in $\V$:
                                    $$\delta: \g \to \g$$
                                that turns the triple $\left(\g, \nabla, \delta\right)$ into an \textbf{$p$-additive} left/right-$p$-Leibniz algebra. That is to say, we require the following diagram to commute in $\V$:
                                    $$
                                        \begin{tikzcd}
                                        	{\g \tensor \g} & {\g} \\
                                        	{\g \tensor \g} & {\g}
                                        	\arrow["{\delta}", from=1-2, to=2-2]
                                        	\arrow["{\nabla}", from=2-1, to=2-2]
                                        	\arrow["{\nabla}", from=1-1, to=1-2]
                                        	\arrow["{\delta \tensor \Frob_{\g} + \Frob_{\g} \tensor \delta + p \cdot \delta \tensor \delta}"', from=1-1, to=2-1]
                                        \end{tikzcd}
                                    $$
                                wherein $\Frob_{\g}$ is the usual $p^{th}$-power map.
                                \item If $\g$ also happens to be a unital algebra (with unit map $\eta: 1 \to \g$), then we require that the following diagram commutes:
                                    $$
                                        \begin{tikzcd}
                                        	{1} & {\g} \\
                                        	& {\g}
                                        	\arrow["{\delta}", from=1-2, to=2-2]
                                        	\arrow["{\eta}", from=1-1, to=1-2]
                                        	\arrow["{0}"', from=1-1, to=2-2]
                                        \end{tikzcd}
                                    $$
                                whererin $0$ is understood to be the additive identity in the $k$-module $\V(1, \g)$. In this situation, we call the quadruple $(\g, \nabla, \delta, \eta)$ a \textbf{$p$-linear} $p$-Leibniz algebra, and specifically, the $p$-prederivation $D$ will be referred to simply as a $p$-derivation.
                            \end{enumerate}
                    \end{definition}
                    \begin{example}
                        Let us keep notations as in definition \ref{def: arithmetic_leibniz_algebras}.
                        \begin{enumerate}
                            \item \textbf{($\delta$-rings)} If we take $k$ to be commutative and $\V$ to be the category of $k$-modules, then we can see that $p$-linear $p$-Leibniz algebras internal to $\Mod_k$ are just $\delta$-rings (cf. definition \ref{def: delta_rings}).  
                            \item \textbf{($p$-Lie algebras)} A $p$-Lie algebra is just a (non-unital and non-associative) $p$-additive $p$-Leibniz algebra internal to a braided symmetric monoidal $k$-linear category whose multiplication is a Lie bracket.
                        \end{enumerate}
                    \end{example}
                    
                    \begin{claim}[$p$-Leibniz algebras form categories]
                        Let $k$ be a $p$-torsion-free ring and let $(\V, \tensor, 1)$ be a monoidal $k$-linear category. Then, $p$-additive $p$-Leibniz algebras internal to $\V$ form a subcategory that admits that of $p$-linear $p$-Leibniz algebras as a subcategory of its own. We shall denote these two categories, repsectively, by $\delta_p\Alg(\V)$ and $\delta_p\Assoc\Alg(\V)$.
                    \end{claim}
                        \begin{proof}
                            Let $(\g, \nabla, \delta)$ and $(\g', \nabla', \delta')$ be two $p$-additive $p$-Leibniz algebras. Then, let us declare that a morphism of $p$-Leibniz algebras internal to $\V$ is an algebra homomorphism $\phi: \g \to \g'$ (i.e. a morphism satisfying $\phi \circ \nabla = \nabla\ \circ (\phi \tensor \phi)$) such that:
                                $$\phi \circ \delta = \delta' \circ \phi$$
                            Then, it will suffice to show that the following diagram commutes:
                                $$
                                    \begin{tikzcd}
                                    	& {\g' \tensor \g'} & {\g'} \\
                                    	& {\g' \tensor \g'} & {\g'} \\
                                    	{\g \tensor \g} & {\g} \\
                                    	{\g \tensor \g} & {\g}
                                    	\arrow["{\phi \tensor \phi}", from=3-1, to=1-2]
                                    	\arrow["{\phi}", from=3-2, to=1-3]
                                    	\arrow["{\phi}", from=4-2, to=2-3]
                                    	\arrow["{\delta'}", from=1-3, to=2-3]
                                    	\arrow["{\nabla'}", from=1-2, to=1-3]
                                    	\arrow["{\delta' \tensor \Frob_{\g'} + \Frob_{\g'} \tensor \delta' + p \cdot \delta' \tensor \delta'}"', from=1-2, to=2-2]
                                    	\arrow["{\nabla'}", from=2-2, to=2-3]
                                    	\arrow["{\delta}", from=3-2, to=4-2]
                                    	\arrow["{\delta \tensor \Frob_{\g} + \Frob_{\g} \tensor \delta + p \cdot \delta \tensor \delta}"', from=3-1, to=4-1]
                                    	\arrow["{\nabla}"', from=4-1, to=4-2]
                                    	\arrow["{\nabla}"', from=3-1, to=3-2]
                                    	\arrow["{\phi \tensor \phi}", from=4-1, to=2-2]
                                    \end{tikzcd}
                                $$
                            if we are simply trying to show that additive Leibniz algebras form a subcategory of $\V$. For the second assertion, we will, in addition, need to prove that the following diagram, wherein $\eta$ and $\eta'$ are the unit maps, commutes:
                                $$
                                    \begin{tikzcd}
                                    	&& {1} & {\g'} \\
                                    	{1} & {\g} && {\g'} \\
                                    	& {\g}
                                    	\arrow["{\phi}", from=2-2, to=1-4]
                                    	\arrow["{\phi}", from=3-2, to=2-4]
                                    	\arrow["{\eta}" description, from=2-1, to=2-2]
                                    	\arrow["{\delta}", from=2-2, to=3-2]
                                    	\arrow["{\delta'}", from=1-4, to=2-4]
                                    	\arrow["{\eta'}" description, from=1-3, to=1-4]
                                    	\arrow[Rightarrow, from=2-1, to=1-3, no head]
                                    	\arrow["{0}"', from=2-1, to=3-2]
                                    	\arrow["{0}"', from=1-3, to=2-4]
                                    \end{tikzcd}
                                $$
                            To these ends, consider the following:
                                $$
                                    \begin{aligned}
                                        \phi \circ \delta \circ \nabla & = \phi \circ \nabla \circ \left(\delta \tensor \Frob_{\g} + \Frob_{\g} \tensor \delta + p \cdot \delta \tensor \delta\right)
                                        \\
                                        & = \nabla' \circ (\phi \tensor \phi) \circ \left(\delta \tensor \Frob_{\g} + \Frob_{\g} \tensor \delta + p \cdot \delta \tensor \delta\right)
                                        \\
                                        & = \nabla' \circ \left((\phi \circ \delta) \tensor \phi + \phi \tensor (\phi \circ \delta) + p \cdot (\phi \circ \delta) \tensor (\phi \circ \delta)\right)
                                        \\
                                        & = \nabla' \circ \left((\delta' \circ \phi) \tensor \phi + \phi \tensor (\delta' \circ \phi) + p \cdot (\delta' \circ \phi) \tensor (\delta' \circ \phi)\right)
                                        \\
                                        & = \nabla' \circ \left(\delta' \tensor \Frob_{\g'} + \Frob_{\g'} \tensor \delta' + p \cdot \delta' \tensor \delta'\right) \circ (\phi \tensor \phi)
                                        \\
                                        & = \delta' \circ \nabla' \circ (\phi \tensor \phi)
                                    \end{aligned}
                                $$
                            and in the unital case, also the following:
                                $$
                                    \begin{aligned}
                                        \delta' \circ \phi \circ \eta & = \delta' \circ \eta'
                                        \\
                                        & = \delta' \circ 0
                                        \\
                                        & = 0
                                        \\
                                        & = \phi \circ 0
                                        \\
                                        & = \phi \circ \delta \circ \eta
                                    \end{aligned}
                                $$
                            By matching the terms in these equations with composition of arrows in the preceding two diagrams, we can see that the diagrams indeed commute.
                        \end{proof}
                
                    \begin{proposition}[Properties of categories of Leibniz algebras]
                        Let $k$ be a $p$-torsion-free ring and let $(\V, \tensor, 1)$ be a monoidal $k$-linear category. Also, we shall be writing $\delta_p\Assoc\Alg(\V)$ for the category of associative (and unital) $p$-linear $p$-Leibniz algebras internal to $\V$. 
                            \begin{enumerate}
                                \item $\delta_p\Assoc\Alg(\V)$ has $(k, \del_p)$ as the initial object, with $\del_p$ the Fermat quotient. Due to this, one could define the coslice category $\delta_p\Assoc\Alg(\V)_A$ for any $k$-algebra $A$. Then, obviously, $(A, \del_{p,A})$ is initial as an object of $\delta_p\Assoc\Alg(\V)_A$.
                                \item The category $\delta_p\Assoc\Alg(\V)$ is both complete and cocomplete. Furthermore, said (co)limits commute with the forgetful functor:
                                    $$U_p: \delta_p\Assoc\Alg(\V) \to \Assoc\Alg(\V)$$
                                \item The forgetful functor:
                                    $$U_p: \delta_p\Assoc\Alg(\V) \to \Assoc\Alg(\V)$$
                                has a left-adjoint, namely the free construction.
                            \end{enumerate}
                    \end{proposition}
    
        \subsection{Plethory, \texorpdfstring{$\Lambda$}{}-structures, and Witt vectors}
            The purpose of this subsection is two-fold:
                \begin{enumerate}
                    \item First of all, we would like to be able to understand the notion of Witt vectors via various universal properties instead of via the (frankly) \textit{ad hoc} formulas that have traditionally been used to define rings of ($p$-typical) Witt vectors. This is not to say that the traditional approach is \say{wrong} in any way: many important results have been proven using the classical notion of $p$-typical Witt vectors. However, rings of Witt vectors are rather pathological objects, at least to commutative algebraists. The only known \say{nice} case are Witt vectors over perfect domains of positive characteristics: if $B$ is a \textit{perfect} $\F_q$-domain (for some power $q$ of a prime $p$) with field of fractions $K$, then we have the following natural characterisation of the ring of $p$-typical Witt vectors over $K$ (which we note to be trivially perfect as an $\F_q$-algebra):
                        $$(\W(B)_{(p)})^{\wedge} \cong \W(K)$$
                    In particular, $\W(B)_{(p)}$ is an unramified extension of $\W(\F_q)$; we refer the reader to \cite[Proposition 5.2]{shimomoto2014witt} for a proof. Beyond this, all we know is that these are horrible objects. For instance, the ring $\W(\F_p[x])$ is not even Noetherian. 
                    \item Second of all, there is the de Rham-Witt complex to be understood (cf. subsection \ref{subsection: de_rham_witt_complex}). Its cohomology is crystalline cohomology in a very natural way, so obviously, we care very much about it. Allegedly, it is a certain quotient of the de Rham complex associated to a smooth affine scheme over a perfect base of characteristic $p$, but why on Earth one would think to consider such a construction is not at all evident. 
                \end{enumerate}
        
            \begin{convention}[Rings and algebras] \label{conv: commutative_rings_noncommutative_algebras}
                \noindent
                \begin{itemize}
                    \item All rings shall still be assumed to be commutative.
                    \item However, algebras over rings will not be taken to be commutative, although we will still be working with associative and unital algebras. Given a base commutative ring $k$, let us denote the category of left/right/two-sided algebras over $k$ by ${}_k\Assoc\Alg, \Assoc\Alg_k$, and ${}_k\Assoc\Alg_k$. 
                    \item \say{Bialgebras} are not to be confused with \say{two-sided algebras}. The former refers to objects in monoidal categories that are simultaneously monoids and comonoids, whereas the latter refers to commutative monoids internal to monoidal categories (note how we made no assumption on the symmetry of monoidal structures here). 
                \end{itemize}
            \end{convention}
            
            \subsubsection{Plethories}
                \begin{definition}[Plethories] \label{def: plethories}
                    Fix a base commutative ring $k$. 
                \end{definition}
                
            \subsubsection{A reconstruction theorem}
            
            \subsubsection{Amplification}
    
        \subsection{The de Rham-Witt complex} \label{subsection: de_rham_witt_complex}
        
        \subsection{The Hard Lefschetz Condition for crystalline cohomology}
        
    \section{Arithmetic D-modules and rigid cohomology}
        \subsection{Isocrystals, arithmetic D-modules, and rigid cohomology}
            In chapters \ref{chapter: etale_cohomology_1} and \ref{chapter: etale_cohomology_2}, and specifically in section \ref{section: l_adic_sheaves}, we have demonstrated how for $k$ a perfect field of some positive characteristic $p$, and for $\ell \not = p$ an auxiliary prime, $\ell$-adic \'etale cohomology is a Weil cohomology theory for (smooth and projective) varieties over $\Spec k$ (cf. definition \ref{def: weil_cohomology_theories}). Within said cohomology theory, the role of coefficients are played by so-called lisse $\ell$-adic sheaves (cf. definition \ref{def: lisse_sheaves}). To construct an analogue that works for the case where $\ell = p$, we will be replacing these lisse sheaves with the so-called \textbf{isocrystals}.
            
            \begin{convention}
                Throughout, we will be working with a fixed smooth (and perhaps projective) variety over a perfect field $k$ of some positive characteristic $p$. Also, fix a formal lift $\calX$ of $X$ to over the Witt vectors $\W(k)$, and let $\calX^{\rig}$ be its adic generic fibre, i.e. the pullback along the canonical morphism $\eta: \Spa\W(k)[1/p] \to \Spf\W(k)$, taken inside the category of adic spaces; by a theorem of Raynaud, $\calX^{\rig}$ is a rigid space, hence the notation. Lastly, denote the absolute Frobenius on $X$ by $\Frob_X$ and fix a lift $\sigma_{\calX}: \calX \to \calX$ thereof; this lift of Frobenius can then be pulled back to the generic fibre $\calX^{\rig}$ to get a so-called rigid lift $\sigma_{\calX^{\rig}}$.
            \end{convention}
            
            \subsubsection{Isocrystals and arithmetic D-modules}
               \begin{definition}[Isocrystals] \label{def: isocrystals}
                    An \textbf{isocrystal} on $\calX^{\rig}$ shall be a vector bundles $\E$ on $\calX^{\rig}$, which is:
                        \begin{itemize}
                            \item \textit{Frobenius-equivariant}, i.e. one has:
                                $$\sigma_{\calX^{\rig}}* \E \cong \E$$
                            and
                            \item equipped with an \textit{integrable connection}, i.e. an $\calO_{X^{\rig}}$-module homomorphism:
                                $$\nabla_{\E}: \E \tensor_{\calO_{X^{\rig}}} \calO_{X^{\rig}} \to \E \tensor_{\calO_{X^{\rig}}} \Omega^1_{X/k}$$
                            defined by:
                                $$\nabla_{\E}(x \tensor y) = x \tensor \nabla_{\E}(y) + x \tensor dy$$
                            such that:
                                $$\nabla_{\E} \circ \nabla_{\E} = 0$$
                        \end{itemize}
               \end{definition}
               \begin{proposition}[Categories of isocrystals] \label{prop: categories_of_isocrystals}
                    Isocrystals over $\calX^{\rig}$ form a full subcategory $\Isoc(\calX^{\rig})$ of the category $\Vect(\calX^{\rig})^{\nabla}_{\dR}$ of integrable connections on vector bundles on $\calX^{\rig}$.
               \end{proposition}
                \begin{proof}
                    Let us first check that $\Isoc(X)$ is indeed a category. We propose that the morphisms inside $\Isoc(X)$ are nothing but commutative diagrams of $\calO_{X^{\rig}}$-modules of the following form:
                        $$
                            \begin{tikzcd}
                            	{\E \tensor_{\calO_{X^{\rig}}} \calO_{X^{\rig}}} & {\E \tensor_{\calO_{X^{\rig}}} \Omega^1_{X/k}} \\
                            	{\E' \tensor_{\calO_{X^{\rig}}} \calO_{X^{\rig}}} & {\E' \tensor_{\calO_{X^{\rig}}} \Omega^1_{X/k}}
                            	\arrow["{f \tensor \id_{\Omega^1_{X/k}}}", from=1-2, to=2-2]
                            	\arrow["{f \tensor \id_{\calO_{X^{\rig}}}}"', from=1-1, to=2-1]
                            	\arrow["{\nabla_{\E'}}", from=2-1, to=2-2]
                            	\arrow["{\nabla_{\E}}", from=1-1, to=1-2]
                            \end{tikzcd}
                        $$
                    which is to say, they are just morphisms of (integrable) connections; such morphisms compose in the obvious manner, and said composition is trivially associative and unital. In fact, this tells us that $\Isoc(X)$ ought to be a full subcategory of $\Vect(\calX^{\rig})^{\nabla}_{\dR}$. 
                    
                    The only thing left to check is whether or not these diagrams are \textit{a priori} Frobenius-equivariant. 
                \end{proof}
                
            \subsubsection{Rigid cohomology}
        
        \subsection{Rigid D-modules}
	    
	    \chapter{Non-archimedean analytic geometry} \label{chapter: valuations} 
    \begin{abstract}
        Let's do some analysis!
    \end{abstract}
    
    \minitoc
    
    
    
    \section{Adic spaces}
        \subsection{Analysis with valuations}
            \subsubsection{Valuation rings}
                \begin{definition}[Valuation rings] \label{def: valuation_rings}
                    \noindent
                    \begin{enumerate}
                        \item \textbf{(Domination):} Let $K$ be a field, let $(B, \m_B)$ be \textit{local} subrings of $K$, and let $(A, \m_A)$ be a \textit{local} subring of $B$. Within such a setup, we say that \textbf{$B$ dominates $A$} if and only if:
                            $$\m_A = \m_B \cap A$$
                        Note that within every field, local subrings form a (possibly empty and possibly uncountable) poset of dominations, which can be roughly depicted by the following tower:
                            $$
                                \begin{tikzcd}
                                	K \\
                                	\vdots \\
                                	{(B, \m)} \\
                                	{(B_1, \m_1)} \\
                                	\vdots \\
                                	{(B_n, \m_n)} \\
                                	\vdots
                                	\arrow[no head, from=5-1, to=4-1]
                                	\arrow[no head, from=6-1, to=5-1]
                                	\arrow[no head, from=7-1, to=6-1]
                                	\arrow[no head, from=3-1, to=2-1]
                                	\arrow[no head, from=2-1, to=1-1]
                                	\arrow[no head, from=4-1, to=3-1]
                                \end{tikzcd}
                            $$
                        Observe that for all fixed local subring $(B, \m)$ and any local subring $(B_n, \m_n)$ dominated by $(B, \m)$, it is inductively true that:
                            $$\m_n = \m \cap \bigcap_{j \leq n} B_j$$
                        \item \textbf{(Valuation rings):} A local integral domain $(\scrV, \m)$ is called a \textbf{valuation ring} if and only if it is maximal among the poset of dominations between local subrings of its field of fractions (should such a maximal element even exist). Note that there is no mention of uniqueness nor universality of valuation rings as a maximal local subring of its field of fractions.
                        \item \textbf{(Centering):} A valuation ring $(\scrV, \m)$ is \textbf{centered} if and only if there are \textit{proper} subrings $B$ of its field of fractions containing $(\scrV, \m)$. In cruder terms, a valuation ring is centered if one can manage to \say{squeeze} rings in between it and its field of fractions.
                    \end{enumerate}
                \end{definition}
                
                Alright, we will admit it: it is entirely unclear how valuation rings as deifned in definition \ref{def: valuation_rings} might have anything to do with valuations (i.e. \say{generalised absolute values}) whatsoever. Worry not, as these notions are intimately related, as their names suggest. However, establishing this link is not so much of a trivial process, which we shall subdivide into a few steps.
                
                Let us start with the existence and uniqueness of valuation rings within fields. 
                \begin{lemma}[Existence and uniqueness of valuation rings] \label{lemma: valuation_rings_existence_and_uniqueness}
                    \noindent
                    \begin{enumerate}
                        \item \textbf{(Existence):} Let $K$ be a field and let $(A, \m_A)$ be a local subring. Then, there exists a valuation ring $(\scrV, \m)$ with fraction field $K$ that dominates $(A, \m_A)$.
                        \item \textbf{(Uniqueness):} Let $(\scrV, \m)$ be a valuation ring with field of fractions $K$. Then, given any $x \in K$, then:
                            $$\forall x \in K: (x \in \scrV) \vee (x^{-1} \in \scrV)$$
                        Conversely, given any local subring $(A, \m)$ of a field $K$ such that:
                            $$\forall x \in K: (x \in A) \vee (x^{-1} \in A)$$
                        then such a local subring is a valuation ring. This is to say, that the uniqueness of valuation rings within their field of fractions is up to the above condition.
                    \end{enumerate}
                \end{lemma}
                    \begin{proof}
                        \noindent
                        \begin{enumerate}
                            \item \textbf{(Existence):} Suppose for the sake of deriving a contradiction, that there exists a field $K$ whose poset of local subrings is non-empty and without maximal elements, and not that by definition, this is the same as suppose that our field $K$ does not contain a local subring that is a valuation ring (such a valuation ring, should it exist, would always dominate other local subrings of $K$; cf. definition \ref{def: valuation_rings}). Now, when we view the poset of local subrings of $K$ as a diagram category, we shall see that the morphisms therein are nothing but monomorphisms of commutative rings (which are local \textit{a priori}). Because of this, the union taken over this diagram (i.e. the filtered colimit of local subrings of $K$) must also be a local subring of $K$, owing to the fact that finite limits commute with filtered colimits. But hold on a minute, we have just built $K$ to not contain a maximal local strict subring, so now, it shall suffice to show that the above union of local subrings of $K$ is not $K$ itself.
                            \item \textbf{(Uniqueness):}
                        \end{enumerate}
                    \end{proof}
                    
                \begin{example}[Some obvious valuation rings] \label{example: valuation_rings}
                    \noindent
                    \begin{enumerate}
                        \item \textbf{($p$-adic integers):}
                        \item \textbf{($p$-torsion-free rings):}
                        \item \textbf{(Related: Pr\"ufer domains):}
                        \item \textbf{(Fields):} Fields are trivially valuation rings. 
                    \end{enumerate}
                \end{example}
                
                \begin{proposition}[Colimits of valuation rings] \label{prop: colimits_of_valuation_rings}
                    \noindent
                    \begin{enumerate}
                        \item A filtered colimit of valuation ring is itself a valuation ring.
                        \item Localisations and quotients of valuation rings at primes are again valuation rings. 
                    \end{enumerate}
                \end{proposition}
                    \begin{proof}
                        
                    \end{proof}
                    
                \begin{proposition}[Extensions of valuation rings] \label{prop: extensions of valuation rings}
                    Let $(\scrV', \m_{\scrV'})$ be a valuation with fraction field $K'$, and let $K$ be an arbitrary subfield of $K'$. Then, the valuation ring $(\scrV', \m_{\scrV'})$ extends down to a (necessarily unqiue) valuation ring $(\scrV, \m_{\scrV})$ of $K$, which is given by:
                        $$\scrV = \scrV' \cap K$$
                \end{proposition}
            
            \subsubsection{Value groups}
                \begin{definition}[Valuations] \label{def: valuations}
                    Let $\scrV$ be a valuation ring with field of fractions $K$. Then, a \textbf{valuation} on $K$ is an \textit{injective} group homomorphism:
                        $$\nu: K^{\x} \to \Gamma$$
                    into a \textit{totally ordered} abelian group $(\Gamma, \leq)$ (like $\Z$ or $\R$, for instance) such that:
                        $$\nu(x + y) \geq \min(\nu(x), \nu(y))$$
                    for all $x, y \in K^{\x}$. So-called \textbf{discrete valuations} are those taking values in $\Z$. 
                \end{definition}
                
                \begin{lemma}[Valuations attached to valuation rings]
                    Attached to every valuation ring is a valuation, which needs not be unique ($\Q$ for instance, has many associated valuations). 
                \end{lemma}
                    \begin{proof}
                        
                    \end{proof}
                \begin{theorem}[\textcolor{red}{\underline{IMPORTANT}} Principality and locality of valuation rings] \label{theorem: principality_and_locality_of_valuation rings}
                    A commutative ring $\scrV$ is a valuation ring if and only if it is a local domain wherein every finitely generated ideal is principal.
                \end{theorem}
                    \begin{proof}
                        \noindent
                        \begin{enumerate}
                            \item  
                            \item 
                        \end{enumerate}
                    \end{proof}
                    
            \subsubsection{Valuative spectra}
                \begin{definition}[Valuative spectra] \label{def: valuative_spectra}
                    The \textbf{valuative spectrum} of a commutative ring $A$, denoted by $\Spv A$ is the set of all equivalence classes of (continuous) valuations on $A$. We should note that by \say{valuations}, we actually mean the corresponding ultranorms, but these are uniquely determined via some fixed choice of exponentiation anyway.
                \end{definition}
                
                \begin{proposition}[The adic topology] \label{prop: the_adic_topology}
                    Fix a commutative ring $A$. Then, we can equip $\Spv A$ with a so-called \textbf{adic topology} generated by open subsets of the form:
                        $$D_{\Spv A}(f/g) := \{\nu: \Spv A \mid \left(\nu(f) \leq \nu(g)\right) \wedge (\nu(g) \not = 0)\}$$
                    Note that the \say{quotient} $f/g$ is purely symbolic.
                \end{proposition}
                    \begin{proof}
                        To prove that a certain collection of subsets form a topology, we shall need to verify that the empty set and the whole space are open, that arbitrary unions are open, and that finite intersections are open.
                            \begin{enumerate}
                                \item \textbf{(Empty set and whole space are open):} 
                                    \begin{enumerate}
                                        \item \textbf{(Whole space):} Any subset of $\Spv A$ that is of the form $D_{\Spv A}(0/g)$ is the same as the whole space, since:
                                            $$0 = \nu(0) \leq \nu(g)$$
                                        for all $g \in A$ such that $\nu(g) \not = 0$. Thus, the whole space $\Spv A$ is open. 
                                        \item \textbf{(Empty set):} The empty set is the complement of the whole space, which tells us that:
                                            $$
                                                \begin{aligned}
                                                    & \nu' \in \varnothing 
                                                    \\
                                                    \iff & \nu' \in \Spv A \setminus \Spv A
                                                    \\
                                                    \iff & \neg \bigvee_{g \in A} \left(\nu' \in D_{\Spv A}(0/g)\right)
                                                    \\
                                                    \iff & \bigwedge_{g \in A} \neg \left(\nu' \in D_{\Spv A}(0/g)\right)
                                                    \\
                                                    \iff & \bigwedge_{f \in A} \left(\nu' \in D_{\Spv A}(f/0)\right)
                                                    \\
                                                    \iff & \nu' \in \bigcap_{f \in A} D_{\Spv A}(f/0)
                                                    \\
                                                    \iff & \nu' \in \bigcap_{f \in A} \varnothing
                                                    \\
                                                    \iff & \nu' \in \varnothing
                                                \end{aligned}
                                            $$
                                        Hence, the empty set is also open.
                                    \end{enumerate}
                                \item \textbf{(Unions are open):} Let $\calF$ and $\calG$ be two \textit{arbitrary} subsets of $A$ and consider the following:
                                    $$
                                        \begin{aligned}
                                            & \nu' \in \bigcup_{f \in \calF} \bigcup_{g \in \calG} D_{\Spv A}(f/g)
                                            \\
                                            \iff & \bigvee_{f \in \calF} \bigvee_{g \in \calG} (\nu' \in D_{\Spv A}(f/g))
                                            \\
                                            \iff & \bigvee_{f \in \calF} \bigvee_{g \in \calG} \left((\nu'(f) \leq \nu'(g)) \wedge (\nu'(g) \not = 0)\right)
                                            \\
                                            \iff & \bigvee_{f \in \calF} \bigvee_{g \in \calG} \left(\neg(\nu'(f) \geq \nu'(g)) \wedge \neg(\nu'(g) = 0)\right)
                                            \\
                                            \iff & \bigvee_{f \in \calF} \bigvee_{g \in \calG} \neg \left((\nu'(f) \geq \nu'(g)) \vee (\nu'(g) = 0)\right)
                                            \\
                                            \iff & \neg \bigwedge_{f \in \calF} \bigwedge_{g \in \calG} \left((\nu'(f) \geq \nu'(g)) \vee (\nu'(g) = 0)\right)
                                        \end{aligned}
                                    $$
                                \item \textbf{(Finite intersections are open):}
                            \end{enumerate}
                    \end{proof}
                \begin{remark}[Comparison with the Zariski topology] \label{remark: adic_vs_zariski}
                    
                \end{remark}
        
        \subsection{Adic spaces}
            \subsubsection{Huber rings}
                \begin{definition}[Huber rings] \label{def: huber_rings}
                    \noindent
                    \begin{enumerate}
                        \item \textbf{(Adic rings):} To avoid terminology confusions that might arise from the somewhat liberal use of the word \say{adic} in various differing contexts, let us declare that an \textbf{adic ring} is a commutative ring $A$ that carries an $\a$-adic topology, for some ideal $\a \subset A$. The archetypal examples are $\Z_p$ and $\F_p[\![t]\!]$ (for some prime $p$); these are complete with respect to the obvious $p$-adic and $t$-adic topologies respectively.
                        \item \textbf{(Huber rings):} A topological commutative ring $A$ is said to be a \textbf{Huber ring} if and only if there exists a \textit{finitely generated} $A$-ideal $\a$ along with an $\a$-adic open subring $A_0 \subset A$. The ring $A_0$ is called the \textbf{subring of definition} and the $A_0$-ideal $\a$ is called the \textbf{ideal of definition}.
                        \item \textbf{(Tate rings):} A so-called \textbf{Tate ring} is a Huber ring with a topologically nilpotent unit, commonly called a \textbf{pseudo-uniformiser}. 
                    \end{enumerate}
                \end{definition}
                
                \begin{convention}[Adic completions of rings] \label{conv: adic_completion_of_rings}
                    When necessary, we shall denote the $\a$-adic completion of a commutative ring $A$ by $(A, \a)^{\wedge}$. 
                \end{convention}
                
                \begin{definition}[Power-bounded elements] \label{def: power_bounded_elements}
                    \noindent
                    \begin{enumerate}
                        \item \textbf{(Power-boundedness):} 
                            \begin{enumerate}
                                \item \textbf{(Boundedness):} A subset $S$ of a topological ring $R$ is said to be bounded if and only if for all open neighbourhoods $U \ni 0$, there exists another open neighbourhood $V \ni 0$ such that:
                                    $$VS \subseteq U$$
                                where $VS = \{v s \mid (v \in V) \wedge (s \in S)\}$. Note that per this definition, the ideal of definition of any adic ring (or for that matter, any finite power thereof) is trivially bounded. 
                                \item \textbf{(Power-bounded elements):} An element $x$ of a topological ring $R$ is said to be power-bounded if and only if the sequence $\{x^n\}_{n \in \N}$ is bounded as a subset of $R$.
                                \item \textbf{(Uniform Huber rings):} Any Huber ring whose subset of power-bounded elements is bounded is known as being \textbf{uniform}.
                            \end{enumerate}
                        \item \textbf{(Linear topologies):} 
                            \begin{itemize}
                                \item Let $R$ be a topological ring. If its subset of power-bounded elements $R^{\circ}$ is a subring instead of simply being a subset (such as the case of $\Z_p \subset \Q_p$), then we shall say that $R$ is \textbf{linearly topologised}. 
                                \item Every adic ring is trivially linearly topologised (in fact some older literatures refer to adic rings as \say{linearly topologised rings}). Furthermore, if $(A, \a)$ is an $\a$-adic ring then every subset $S$ of the subring $A^{\circ}$ of power-bounded elements is bounded, since:
                                    $$\a^n S \subseteq \a^n$$
                                and $\a^n$ is a neighbourhood of $0$ for every $n \in \N$. 
                            \end{itemize}
                    \end{enumerate}
                \end{definition}
                
                \begin{example} \label{example: huber_rings}
                    \noindent
                    \begin{enumerate}
                        \item \textbf{(A few more adic rings and some basic properties):} Fix a prime $p$.
                            \begin{itemize}
                                \item The ring $\Z_p[\![T]\!]$ is complete with respect to the $(p, T)$-adic topology, and hence \textit{a fortiori} $(p, T)$-adic. 
                                \item For all fields $k$, the power series ring $k[\![x_1, ..., x_n]\!]$ is complete with respect to the $(x_1, ..., x_n)$-adic topology.
                                \item If $E/\Q_p$ is a finite extension, then the ring of integers $E^{\circ}$ is $p$-adically complete. 
                                \item Every ring that is adic with respect to a \textit{finitely generated ideal} could be turned into a Huber ring if we were to take the subring of definition to be the whole ring, and the ideal of definition to be the ideal of definition of that adic ring. In fact, any commutative ring, if viewed as being $0$-adically complete, is a Huber ring. This fact will become useful when we wish to speak of Zariski-closed subsets of adic spectra.
                                \item Let $A$ be any commutative ring and let $\a$ be an arbitrary $A$-ideal. Then, any quotient of the form $A/\a^n$ is $\a$-adic. In fact, they correspond to quasi-compact subsets $|\Spec A/\a^n|$ of $|\Spf (A, \a)^{\wedge}|$ (recall how there is a canonical descending filtration $|\Spf (A, \a)^{\wedge}| \cong |\Spec A/\a| \supseteq |\Spec A/\a^2| \supseteq ...$).
                                
                                For instance, $\F_p$ is a $p$-adic ring (even though it is a field, we shall not refer to it as a $p$-adic field, as that terminology is reserved for finite extensions of $\Q_p$) whose $p$-adic completion is also $\F_p$. Another example is $k[T]/T^n$, for any field $k$: it is $T$-adic and its $T$-adic completion is also itself, as it is the case with $\F_p$. 
                                
                                In fact, \textit{all adic rings are complete with respect to their associated adic topologies}, thanks to the fact that completions, by virtue of being filtered limits, commute with quotients and localisations at finitely many variables, which are finite colimits. This is an algebraic version of the fact that every compact metric space is complete.  
                                \item Finite tensor products of adic rings are also adic rings. In particular, if $(A, \a)$ and $(B, \b)$ are adic algebras over some base commutative ring $k$, then the tensor product $A \tensor_k B$ is an $(\a + \b)$-adic ring: for instance, $\Z_p \tensor_{\Z} \Z[\![T]\!]$ is $(p, T)$-adic (in fact, this tensor product is $\Z_p[\![T]\!]$). More generally, if $M$ is an $(A, \a)$-adic module and $N$ is a $(B, \b)$-adic module, for $(A, \a), (B, \b)$ adic $k$-algebras, then the tensor product $M \tensor_k N$ will be $(\a + \b)$-adic; this follows from the fact that every module admits a presentation and again, the fact that filtered limits commute with finite colimits. 
                                
                                Sometimes we might write $M \hat{\tensor}_k N$ to emphasise the adic completeness, although this is unnecessary in the algebraic context (it is necessary, however, for say, general locally convex vector spaces). 
                            \end{itemize}
                        \item \textbf{(Non-trivial Huber rings from adic rings):}
                            \begin{itemize}
                                \item If $B$ is a \textit{perfect} $\F_q$-domain (for some power $q$ of a prime $p$) with field of fractions $K$, then we have the following natural characterisation of the ring of $p$-typical Witt vectors over $K$ (which we note to be trivially perfect as an $\F_q$-algebra):
                                    $$(\W(B)_{(p)})^{\wedge} \cong \W(K)$$
                                which implies, in particular, that $\W(K)$ is a Huber ring in the trivial manner. We refer the reader to \cite[Proposition 5.2]{shimomoto2014witt} for a proof. 
                                \item \textbf{(Counter-examples):} Constructing non-trivial Huber rings out of adic rings turns out to be somewhat non-trivial (\textit{badum tsss!}). For instance, $\Q_p[\![T]\!]$ will not be a Huber ring when the subring of definition is $\Z_p[\![t]\!]$ and ideal of definition $(p, T)$, as $\Z_p[\![t]\!]$ is not open in $\Q_p[\![T]\!]$ (one can use the Gauss norm induced by the $p$-adic valuation on $\Q_p$ to show that this is the case). A similar analysis applies to rings such as $\F_p(\!(x)\!)[\![y]\!]$: indeed, $\F_p[\![x, y]\!]$ is a closed subring. 
                                \item \textbf{(A non-uniform Huber ring):} Consider the ring $\Q_p[T]/T^2$ equipped with the $(p, T)$-adic topology. \todo{Finish the example}
                            \end{itemize}
                        \item \textbf{(Tate rings):} 
                            \begin{itemize}
                                \item For more or less trivial reasons, all complete non-archimedean fields are Tate rings where the subring of definition is the subring of power-bounded elements. This means that fields such as $\Q_p$ or $\F_p(\!(t)\!)$ are Tate.
                                \item Let $(K, |\cdot|)$ be a complete non-archimedean field, let $\varpi \in K^{\circ \circ}$ be a pseudo-uniformiser, and let $T_1, ..., T_n$ be finitely many variables which are transcendental over $K$. Then, the ring $K\<T_1, ..., T_n\>$ of convergent power series in these $n$ variables and with coefficients in $K$ is a Tate ring. Its subring of definition is the subring $K^{\circ}\<T_1, ..., T_n\>$ consisting of convergent power series with power-bounded coefficients (with respect to $|\cdot|$ of course), and via the Gauss norm:
                                    $$\left\|\sum_{k = -\infty}^{+\infty} \left(a_k \prod_{j = 1}^n T_j^{d_{j, k}}\right)\right\| = \sup_{k \in \Z} |a_k|$$
                                this subring is complete with respect to the adic topology induced by the ideal $(\varpi, T_1, ..., T_n)$; in fact:
                                    $$K^{\circ}\<T_1, ..., T_n\> \cong K^{\circ}[\![T_1, ..., T_n]\!]$$
                                (also, it is clear that the pseudo-uniformiser of $K\<T_1, ..., T_n\>$ is $\varpi$).
                                
                                By performing a similar analysis as above, we can also see that the usual power series ring $(K[\![T_1, ..., T_n]\!], \|\cdot\|)$, where $\|\cdot\|$ is the Gauss norm, is also Tate. Its subring of definition is also $K^{\circ}[\![T_1, ..., T_n]\!]$ (which is complete with respect to the $(\varpi, T_1, ..., T_n)$-topology), and its pseudo-uniformiser is also $\varpi$.         
                            \end{itemize}
                        \item \textbf{(Huber rings that are not Tate):} Equip an arbitrary \textit{non-zero} commutative ring $R$ with the trivial (\textit{a priori} discrete) valuation. Then, even though $R[\![t]\!]$ is Huber, it is not Tate, as there exist no non-zero pseudo-uniformiser in non-zero discrete rings.  
                    \end{enumerate}
                \end{example}
                \begin{remark}[A canonical norm for complete Tate rings] \label{remark: canonical_norm_for_tate_rings}
                    The norm that we shall define is a bit \textit{ad hoc}, so we will need to motivate its definition a bit first.
                    \begin{enumerate}
                        \item Let $R$ be any commutative ring, let $f$ be a a non-zero-divisor, and consider the adic completion $(R, (f))^{\wedge}$. It is then not hard to see that $\left((R, (f))^{\wedge}[1/f], (R, (f))^{\wedge}\right)$ is the data of a Huber ring. In fact, $f$ is topologically nilpotent and this data hence defines a Tate ring. 
                        \item Let $(A, (A_0, \a), \varpi)$ be the data of a \textit{complete} Tate ring. Then, we can put the following canonical \textit{continuous} valuation onto $A$ to turn it into a commutative Banach ring:
                            $$\nu: A \to \R: x \mapsto \sup\{n \in \N \mid \varpi^n x \in A_0\}$$
                    
                    \end{enumerate}
                \end{remark}
                
                \begin{proposition}[A criteria for Huber rings] \label{prop: huber_criteria}
                    A topological ring is Huber if and and only if it has an open and bounded subring. 
                \end{proposition}
                    \begin{proof}
                        \noindent
                        \begin{enumerate}
                            \item If $A$ is a Huber ring with $(A_0, \a)$ the adic subring of definition, then the implication is easy, as $(A_0, \a)$ is \textit{a priori} bounded (cf. definition \ref{def: power_bounded_elements}), and it is open by definition. 
                            \item Conversely, suppose that $A$ has an open and bounded subring $A_0$. 
                                \begin{enumerate}
                                    \item \textbf{(Step 1: Open and bounded imply linearly topologised):}
                                    \item \textbf{(Step 2: Open and linearly topologised imply adic):}
                                \end{enumerate}
                        \end{enumerate}
                    \end{proof}
                \begin{corollary}[A uniformity criterion] \label{coro: uniformity_criterion}
                    A Huber ring is uniform (cf. definition \ref{def: power_bounded_elements}) if and only if it admits its subset of power-bounded elements as a ring of definition (i.e. it is linearly topologised). 
                \end{corollary}
                    \begin{proof}
                        
                    \end{proof}
                    
            \subsubsection{Adic spectra}
                \begin{definition}[\textcolor{red}{\underline{IMPORTANT}} Adic spectra] \label{def: adic_spectra}
                    \noindent
                    \begin{enumerate}
                        \item \textbf{(Huber pairs):} A Huber pair is a pair $(A, A^+)$ consisting of a Huber ring $A$ and an integrally closed subring $A^+ \subseteq A$, called the \textbf{ring of integral elements} of $A$.
                        \item \textbf{(Adic spectra):} The \textbf{adic spectrum} of a Huber pair $(A, A^+)$, denoted by $\Spa (A, A^+)$, is the set of equivalence classes of continuous valuations on $\nu: A \to \Gamma \cup \{0\}$ such that for all $f \in A^+$, we are guaranteed that:
                            $$\nu(f) \leq 1$$
                        (with $1$ denoting the unit of the multiplicative totally ordered abelian group $(\Gamma, \leq)$). More generally, one might define the adic spectrum of any pair $(A, \Sigma)$ (which we shall call \textbf{pre-Huber pairs}) where $\Sigma$ is merely some subset of $A$ such that $\nu(f) \leq 1$ for all $f \in \Sigma$. 
                    \end{enumerate}
                \end{definition}
                \begin{remark}[Functoriality (or the lack thereof)] \label{remark: adic_spectrum_functoriality}
                    Let $A$ be a Huber ring and let $\Sigma' \supseteq \Sigma$ be subsets of $A$. Then:
                        $$\Spa(A, \Sigma) \subseteq \Spa(A, \Sigma')$$
                    since for all ultranorms $\nu$ on $A$ such that:
                        $$\nu(g) \leq 1$$
                    for all $g \in \Sigma'$, it must be true as a consequence of $\Sigma$ being a subset of $\Sigma'$ (and hence every element of $\Sigma$ being inside $\Sigma'$) that:
                        $$\nu(f) \leq 1$$
                    for all $f \in \Sigma$. 
                \end{remark}
        
        \subsection{Morphisms of adic spaces}
            \subsubsection{Morphisms of finite type and (quasi-)finite morphisms}
                \begin{definition}[A gazillion finite types] \label{def: finite_type_morphisms_between_adic_spaces} \index{Morphism between adic spaces! of weak finite type} \index{Morphism between adic spaces! of integrally weak finite type} \index{Morphism between adic spaces! of finite type} \index{Morphism between adic spaces! of finite presentation}
                    Let $f: X \to Y$ be an adic morphism of adic spaces. We have the following hierachy of morphisms of finite type:
                        \begin{enumerate}
                            \item \textbf{(Weak finite types):} The given morphism $f: X \to Y$ is said to be \textbf{locally of weak finite type} if for all $x \in |X|$, there exists an affinoid open neighbourhood $U \ni x$ around $x$ along with an affinoid open subspace $V \subseteq Y$ such that $f(U)$ immerses into $V$ (which exists thanks to the adic assumption on $f$) and such that the corresponding comorphism of Huber rings:
                                $$f^{\sharp}: \calO_Y(V) \to \calO_X(U)$$
                            is \textit{topologically of finite type} (cf. \cite[\href{https://stacks.math.columbia.edu/tag/0ANS}{Tag 0ANS}]{stacks}) (note that because $f(U)$ immerses into $V$, we have that $f^{-1}(V) \cong U$ and hence $f_*\calO_X(V) \cong \calO_X(U)$). 
                            \item \textbf{(Integrally weak finite types):} If $f: X \to Y$ is already \textit{locally of weak finite type} then it is furthermore \textbf{locally of integrally weak finite type} if for any choice of integral structure subsheaves $\calO_X^+$ and $\calO_Y^+$, and for all $x \in |X|$ and all affinoid open neighbourhood $U \ni x$ and all affinoid open subspace $V \subseteq Y$ into which $f(U)$ immerses itself, there exists a \textit{finite} set $E$ which contains the image of $\calO_Y^+(V)$ under $f^{\sharp}: \calO_Y(V) \to \calO_X(U)$. 
                            \item \textbf{(Finite types and integrally finite types):} 
                                \begin{enumerate}
                                    \item If $f: X \to Y$ is already locally of (integrally) \textit{weak} finite type then it is \textbf{locally of (integrally) finite type} if for all $x \in |X|$, there exists an affinoid open neighbourhood $U \ni x$ around $x$ along with an affinoid open subspace $V \subseteq Y$ such that $f(U)$ immerses into $V$ and such that the corresponding comorphism of integral subrings:
                                        $$f^{\sharp +}: \calO_Y^+(V) \to \calO_X^+(U)$$
                                    is topologically of finite type. Because morphisms locally of (integrally) finite types are \textit{a priori} locally of \textit{weak} finite type, the last statement is equivalent to us saying that the morphism of Huber pairs:
                                        $$(f^{\sharp}, f^{\sharp +}): \left(\calO_Y(V), \calO_Y^+(V)\right) \to \left(\calO_X(U), \calO_X^+(U)\right)$$
                                    is topologically of finite type. Also, note that the comorphism is well-defined at the level of integral structure subsheaves $\calO_Y^+, \calO_X^+$ because any choice of integral structure subpresheaf of a given structure sheaf of an adic space is \textit{a priori} a sheaf (in other words, Huber subpairs of sheaf Huber pairs are sheafy themselves). 
                                    \item Suppose now that $f: X \to Y$ is \textit{locally} of (integrally) finite type. Then, it is \textbf{(integrally) of finite type} if and only if it is \textit{quasi-compact} in addition.
                                \end{enumerate}
                            \item \textbf{(Finite presentations):} A morphism of (integrally) finite type is \textbf{(integrally) finite presentation} if the corresponding comorphism of Huber pairs is topologically of finite presentation.
                        \end{enumerate}
                \end{definition}
            
            \subsubsection{Separatedness and properness}
            
            \subsubsection{Ramification, smoothness, and \'etaleness}
            
            \subsubsection{Dimension theory for adic spaces}
            
    \section{Non-archimedean GAGA theorems}
                
	    
	    \chapter{Perfectoid spaces}
    \begin{abstract}
        
    \end{abstract}
    
    \section{"The perfect(oid) space doesn't exi-"} \label{section: perfectoid_spaces}
        \subsection{Perfectoid spaces}
            \subsubsection{Some almost-ring theory}
                \paragraph{Almost modules}
                    To be able to understand the consequences of perfectoid fields being deeply ramified, we will be needing the language of almost-ring theory. Our main reference shall be \cite[Chapter 14]{gabber_ramero_almost_ring_theory}.
                    
                    \begin{convention}[\say{Basic setups}] \label{conv: basic_setups}
                        A \textbf{basic setup} for us shall always be a pair $(\scrV, \m)$ consisting of a commutative ring $\scrV$ along with an idempotent ideal $\m$ (i.e. one such that $\m^2 = \m$; note that this ideal need not be maximal, despite what our notation might suggest). One often speaks also of \textbf{flat basic setups}, which are basic setups $(\scrV, \m)$ such that $\m$ is flat over $\scrV$; such a specification can be made to ensure that the functor $- \tensor_{\scrV} \m$ is exact.
                    \end{convention}
                    
                    \begin{definition}[Almost-zero modules] \label{def: almost_zero_modules}
                        Let $(\scrV, \m)$ be a basic setup (cf. convention \ref{conv: basic_setups}) and fix a $\scrV$-algebra. Then, a $\scrV$-module $M$ (or should the situation calls for speification, a $(\scrV, \m)$-module $M$) is said to be \textbf{almost-zero} if and only if it is annihilated by $\m$, i.e.:
                            $$\m M = 0$$
                    \end{definition}
                    \begin{lemma} \label{lemma: tensor_powers_of_flat_modules}
                        Let $R$ be a commutative ring and let $N$ be a flat $R$-module. Then, $N \tensor_R N$ is also flat. Also, for all $R$-ideals $\m$, one has:
                            $$\m \tensor_R M \cong \m M$$
                    \end{lemma}
                        \begin{proof}
                            If $N$ is flat over $R$ then the functor $- \tensor_R N$ will be left-exact. The composition of two left-exact is again left-exact for trivial reasons. Thus $N \tensor_R N$ is flat over $R$.
                            
                            As for the assertion that $\m \tensor_R M \cong \m M$ whenever $\m$ is flat, we can rely on homological algebra. Specifically, because $\m$ is flat, we have:
                                $$\m M \cong \Tor_R^1(\m, M) \cong 0$$
                        \end{proof}
                    \begin{proposition}[Another definition of almost-zero modules] \label{prop: almost_zero_module_alt_def}
                        Let $(\scrV, \m)$ be a \textit{flat} basic setup (cf. convention \ref{conv: basic_setups}) and fix a $\scrV$-algebra. Then, a $\scrV$-module $M$ is said to be \textbf{almost-zero} if and only if:
                            $$\m \tensor_{\scrV} M  \cong0$$
                    \end{proposition}
                        \begin{proof}
                            \noindent
                            \begin{enumerate}
                                \item Suppose first of all that $M$ is \textit{non-zero} and almost-zero, i.e. that $\m M \cong 0$. Then, a direct application of lemma \ref{lemma: tensor_powers_of_flat_modules} (which is appropriate because $\m$ is flat over $\scrV$) tells us that:
                                    $$0 \cong \m M \cong \m \tensor_{\scrV} M$$
                                \item Conversely, suppose that $\m \tensor_{\scrV} M \cong 0$. Because $\m$ is flat over $\scrV$, this tells us that there exists the following short exact sequence:
                                    $$0 \to \m \tensor_{\scrV} \m M \to 0 \to M/\m M \tensor_{\scrV} \m \to 0$$
                                Using some abstract nonsense, we can then see that:
                                    $$\m \tensor_{\scrV} \m M \cong 0$$
                                We can apply the flatness assumption on $\m$ again, which gives:
                                    $$0 \cong \m \tensor_{\scrV} \m M \cong (\m \tensor_{\scrV} \m) M \cong \m^2 M$$
                                Lastly, because $\m$ is idempotent, the above implies that:
                                    $$0 \cong \m^2 M = \m M$$
                                i.e. $M$ is almost-zero as a $\scrV$-module.
                            \end{enumerate}
                        \end{proof}
                        
                    \begin{proposition}[Thick subcategories of almost-zero modules] \label{prop: thick_subcategories_of_almost_zero_modules}
                        Let $(\scrV, \m)$ be a flat basic setup. Then, almost-zero $\scrV$-modules form a thick subcategory of $\scrV\mod$, which we shall denote by $(\scrV, \m)\mod^0$.
                    \end{proposition}
                        \begin{proof}
                            \noindent
                            \begin{enumerate}
                                \item \textbf{(Full subcategories of almost-zero modules):} Let $M, N$ be two almost-zero $(\scrV, \m)$-modules. Then, because:
                                    $$\m M \cong \m N \cong 0$$
                                we have:
                                    $$M \cong M/\m M, N \cong N/\m N$$
                                and so any diagram of the following form, wherein the rows are short exact sequences, would commute:
                                    $$
                                        \begin{tikzcd}
                                        	0 & {\m M} & M & {M/\m M} & 0 \\
                                        	0 & {\m N} & N & {N/ \m N} & 0
                                        	\arrow[from=1-1, to=1-2]
                                        	\arrow[from=1-2, to=1-3]
                                        	\arrow[from=1-3, to=1-4]
                                        	\arrow[from=1-4, to=1-5]
                                        	\arrow[from=2-1, to=2-2]
                                        	\arrow[from=2-2, to=2-3]
                                        	\arrow[from=2-3, to=2-4]
                                        	\arrow[from=2-4, to=2-5]
                                        	\arrow[from=1-2, to=2-2]
                                        	\arrow[from=1-3, to=2-3]
                                        	\arrow[from=1-4, to=2-4]
                                        \end{tikzcd}
                                    $$
                                From this, we can infer that morphisms of almost-zero module are just module homomorphisms, which means that there exists a full subcategory $(\scrV, \m)\mod^0$ of $\scrV\mod$ spanned by almost-zero $(\scrV, \m)$-modules.
                                \item \textbf{(Thickness):} A \textit{full} subcategory $\S$ of a triangulated category $\T$ (cf. definition \ref{def: triangulated_infinity_categories}; our triangulated category is $\scrV\mod$) is said to be \textbf{thick} whenever it is closed under extensions, which is to say, for all short exact sequences:
                                    $$0 \to M' \to M \to M'' \to 0$$
                                should $M$ be an object of $\S$, then so must $M'$ and $M''$ too, and vice versa (one might also think of a thick subcategory as the homotopy category of a stable full $\infty$-subcategory of a triangulated $\infty$-category; cf. remark \ref{remark: elementary_properties_of_triangulated_categories}). We already have fullness, so it remains to show that $(\scrV, \m)\mod^0$ is closed under extensions. To this end, consider a short exact sequence of $\scrV$-modules, such as the following:
                                    $$0 \to M' \to M \to M'' \to 0$$
                                wherein $M$ is almost-zero. Then, because binary direct sums of modules are biproducts and because $\m$ is flat over $\scrV$, one can construct the following diagram with short exact rows out of the short exact sequence above:
                                    $$
                                        \begin{tikzcd}
                                        	0 & {\m \tensor_{\scrV} M'} & {\m \tensor_{\scrV} (M' \oplus M'')} & {\m \tensor_{\scrV} M''} & 0 \\
                                        	0 & {\m \tensor_{\scrV} M'} & {\m \tensor_{\scrV} M} & {\m \tensor_{\scrV} M''} & 0 \\
                                        	0 & {\m \tensor_{\scrV} M'} & {\m \tensor_{\scrV} (M' \oplus M'')} & {\m \tensor_{\scrV} M''} & 0
                                        	\arrow[from=1-1, to=1-2]
                                        	\arrow[from=1-2, to=1-3]
                                        	\arrow[from=1-3, to=1-4]
                                        	\arrow[from=1-4, to=1-5]
                                        	\arrow[from=2-1, to=2-2]
                                        	\arrow[from=2-2, to=2-3]
                                        	\arrow[from=2-3, to=2-4]
                                        	\arrow[from=2-4, to=2-5]
                                        	\arrow[from=3-1, to=3-2]
                                        	\arrow[from=3-2, to=3-3]
                                        	\arrow[from=3-3, to=3-4]
                                        	\arrow[from=3-4, to=3-5]
                                        	\arrow["{=}"{description}, from=2-2, to=3-2]
                                        	\arrow[dashed, tail, from=1-3, to=2-3]
                                        	\arrow[dashed, two heads, from=2-3, to=3-3]
                                        	\arrow["{=}"{description}, from=1-4, to=2-4]
                                        	\arrow["{=}"{description}, from=2-4, to=3-4]
                                        	\arrow["{=}"{description}, from=1-2, to=2-2]
                                        \end{tikzcd}
                                    $$
                                We can reuse the flatness assumption on $\m$ again (specifically, lemma \ref{lemma: tensor_powers_of_flat_modules}, which implies that $\m \tensor_{\scrV} M' \cong \m M'$ and $\m \tensor_{\scrV} M'' \cong \m M''$) as well as the assumption that $M'$ and $M''$ are almost-zero, and the fact that tensor products commute with finite direct sums, to condense the above diagram into:
                                    $$
                                        \begin{tikzcd}
                                        	0 & 0 & 0 & 0 & 0 \\
                                        	0 & 0 & {\m M} & 0 & 0 \\
                                        	0 & 0 & 0 & 0 & 0
                                        	\arrow[from=1-1, to=1-2]
                                        	\arrow[from=1-2, to=1-3]
                                        	\arrow[from=2-1, to=2-2]
                                        	\arrow[from=2-2, to=2-3]
                                        	\arrow[dashed, tail, from=1-3, to=2-3]
                                        	\arrow[dashed, two heads, from=2-3, to=3-3]
                                        	\arrow[from=1-2, to=2-2]
                                        	\arrow[from=2-2, to=3-2]
                                        	\arrow[from=3-1, to=3-2]
                                        	\arrow[from=3-2, to=3-3]
                                        	\arrow[from=1-3, to=1-4]
                                        	\arrow[from=1-4, to=1-5]
                                        	\arrow[from=2-3, to=2-4]
                                        	\arrow[from=2-4, to=2-5]
                                        	\arrow[from=1-4, to=2-4]
                                        	\arrow[from=2-4, to=3-4]
                                        	\arrow[from=3-3, to=3-4]
                                        	\arrow[from=3-4, to=3-5]
                                        \end{tikzcd}
                                    $$
                                We can now easily deduce that $\m M \cong 0$, i.e. that $M$ is almost-zero. This tells us that the full subcategory $(\scrV, \m)\mod^0$ is closed under extensions, and thus thick by definition.
                            \end{enumerate}
                        \end{proof}
                    \begin{remark}[Almost-zero modules over non-flat basic setups] \label{remark: almost_zero_modules_over_non_flat_basic_setups}
                        \noindent
                        \begin{itemize}
                            \item \textbf{(Removing the flatness assumption):} Proposition \ref{prop: thick_subcategories_of_almost_zero_modules} allows us to pull off quite a stunt, and for that matter, not even with much difficulty: the flatness assumption on $\m$ can be completely removed! Of course, the trade-off is that now, only the $\infty$-categorical version of proposition \ref{prop: thick_subcategories_of_almost_zero_modules} would hold: via the Dold-Kan Correspondence, one has that for $(\scrV, \m)$ a basic setup, $\scrV$-modules annihilated by $\m$ span a stable full $\infty$-subcategory $(\scrV, \m)\mod^0$ (or perhaps ${}^{\leq 0}_{(\scrV, \m)}\mod^0$) of the triangulated $\infty$-category ${}^{\leq 0}_{\scrV}\mod$ of projective resolutions of $\scrV$-modules. This descends naturally down to the level of derived categories. One can thus generalise proposition \ref{prop: almost_zero_module_alt_def} too: a $\scrV$-module is almost-zero if and only if:
                                $$\m \tensor_{\scrV}^{\L} M \cong_{\qis} 0$$
                            (i.e. the chain complex $\Tor_{\scrV}^*(\m, M)$ is homotopic to $0$).
                            \item \textbf{(Abelian categories of almost-zero modules):} If we extract the hearts of the t-structures out of ${}^{\leq 0}_{(\scrV, \m)}\mod^0$ and ${}^{\leq 0}_{\scrV}\mod$, we will be able to establish ${}^{\leq 0}_{(\scrV, \m)}\mod^{\almost, \heart}$ as a Serre subcategory of ${}^{\leq 0}_{\scrV}\mod^{\heart}$, and since hearts of t-structure are spanned by objects concentrated in degree $0$, one can furthurmore recognise ${}^{\leq 0}_{(\scrV, \m)}\mod^0$ as a Serre subcategory of $\scrV\mod$ whenever the basic setup $(\scrV, \m)$ is \textit{flat}. In particular, we have that the thick subcategories of almost-zero modules are \textit{abelian}.
                        \end{itemize}
                    \end{remark}
                    
                    \begin{definition}[Almost modules] \label{def: almost_modules}
                        Let $(\scrV, \m)$ be a basic setup. Then, the essential image of the functor:
                            $$\m \tensor_{\scrV}^{\L} -: {\scrV}^{\leq 0}\mod \to {\scrV}^{\leq 0}\mod$$
                        (i.e. the category spanned by objects of the form $\m \tensor_{\scrV}^{\L} M$, where $M$ is a $\scrV$-module) shall be called the category of \textbf{almost modules}. We shall denote this category by ${(\scrV, \m)}^{\leq 0}\mod^{\almost}$.
                    \end{definition}
                    \begin{remark}[Categories of almost-zero modules are kernels] \label{remark: categories_of_almost_zero_modules_are_kernels}
                        It is not hard to see that for $(\scrV, \m)$ a basic setup, the category ${(\scrV, \m)}^{\leq 0}\mod^0$ of almost-zero $(\scrV, \m)$-modules is the kernel or the essentially surjective functor:
                            $$\m \tensor_{\scrV}^{\L} -: {\scrV}^{\leq 0}\mod \to {(\scrV, \m)}^{\leq 0}\mod^{\almost}$$
                        Note that this kernel is well-defined because $\m \tensor_{\scrV}^{\L} -$ is an idempotent functor (cf. remark \ref{remark: almost_zero_modules_over_non_flat_basic_setups}).
                    \end{remark}
                    
                    \begin{proposition}[Localising at almost modules] \label{prop: localising_at_almost_modules}
                        Let $(\scrV, \m)$ be a basic setup (which need not be flat). Then, the tensor-hom adjunction establishes ${(\scrV, \m)}^{\leq 0}\mod^{\almost}$ as a localisation. This is to say, there exists the following $\infty$-adjunction:
                            $$
                                \begin{tikzcd}
                                	{{(\scrV, \m)}^{\leq 0}\mod^{\almost}} & {{\scrV}^{\leq 0}\mod}
                                	\arrow[""{name=0, anchor=center, inner sep=0}, "{j_*}"', shift right=2, hook, from=1-1, to=1-2]
                                	\arrow[""{name=1, anchor=center, inner sep=0}, "{j^*}"', shift right=2, from=1-2, to=1-1]
                                	\arrow["\dashv"{anchor=center, rotate=-90}, draw=none, from=1, to=0]
                                \end{tikzcd}
                            $$
                        wherein $j^* \cong \m \tensor_{\scrV}^{\L} -$ and $j_*(-) \cong \R\Hom_{\scrV}(\m, -)$.
                    \end{proposition}
                        \begin{proof}
                            The only thing to prove here is that $j_*$ ought to be fully faithful, which we can do via showing that the counit of the adjunction is naturally isomorphic to the identity; explicitly, this means showing that for all almost-zero $(\scrV, \m)$-modules $M$, one has the following isomorphism:
                                $$j^* j_* M \cong M$$
                            However, this is an automatic consequence of the fact that ${(\scrV, \m)}^{\leq 0}\mod^{\almost}$ is the essential image of $j^*: {\scrV}^{\leq}\mod \to {\scrV}^{\leq}\mod$.
                        \end{proof}
                    
                    \begin{theorem}[Four functors for almost modules] \label{theorem: four_functors_for_almost_modules}
                        Let $(\scrV, \m)$ be a \textit{flat} basic setup. Then, the usual four-functor pull-push yoga is applicable to almost modules over \textit{flat} basic setups, in the sense that one has the following adjunction quadruple:
                            $$
                                \begin{tikzcd}
                                	{{\scrV}^{\leq 0}\mod} & {{(\scrV, \m)}^{\leq 0}\mod^{\almost}}
                                	\arrow[""{name=0, anchor=center, inner sep=0}, "{j_! \ladjoint j^* \ladjoint j_* \ladjoint j^!}"', shift right=5, from=1-2, to=1-1]
                                	\arrow[""{name=1, anchor=center, inner sep=0}, shift right=5, from=1-1, to=1-2]
                                	\arrow[""{name=2, anchor=center, inner sep=0}, shift left=2, from=1-2, to=1-1]
                                	\arrow[""{name=3, anchor=center, inner sep=0}, shift left=2, from=1-1, to=1-2]
                                	\arrow["\dashv"{anchor=center, rotate=-90}, draw=none, from=0, to=3]
                                	\arrow["\dashv"{anchor=center, rotate=-90}, draw=none, from=3, to=2]
                                	\arrow["\dashv"{anchor=center, rotate=-90}, draw=none, from=2, to=1]
                                \end{tikzcd}
                            $$
                        wherein $j^* \cong \m \tensor_{\scrV}^{\L} -$ and $j_*(-) \cong \R\Hom_{\scrV}(\m, -)$.
                    \end{theorem}
                        \begin{proof}
                            We have already been provided with the middle adjunction $j^* \ladjoint j_*$, so let us just prove that the $!$-pushforwards and $!$-pullbacks exist and fit into the following adjunctions:
                                $$j_! \ladjoint j^*$$
                                $$j_* \ladjoint j^!$$
                            Although, we should note that we can have $j^* \cong \m \tensor_{\scrV}^{\L} -$ and $j_*(-) \cong \R\Hom_{\scrV}(\m, -)$ as two of the basic \href{https://ncatlab.org/nlab/show/six+operations}{\underline{four operations}} because ${(\scrV, \m)}^{\leq 0}\mod^{\almost}$ is the essential image of a functor out of ${\scrV}^{\leq 0}\mod$ (cf. proposition \ref{prop: localising_at_almost_modules}). 
                                \begin{enumerate}
                                    \item \textbf{($!$-pushforward):} To show that the further left-adjoint $j^!$ exists, we will be using Lurie's Adjoint Functor Theorem \cite[Corollary 5.5.2.9]{HTT}, which states that should $\C$ and $\D$ be presentable $\infty$-categories and $R: \C \to \D$ be a functor between them, then there exists an adjunction $L \ladjoint R$ if and only if $R$ is accessible (i.e. $R$ preserves all filtered colimits) and left-exact. For this, we will first need to show that ${(\scrV, \m)}^{\leq 0}\mod^{\almost}$ is a presentable $\infty$-category (it is well-known that module categories like ${\scrV}^{\leq 0}\mod$ are presentable, so we will not be providing a proof).
                                        \begin{enumerate}
                                            \item \textbf{(${(\scrV, \m)}^{\leq 0}\mod^{\almost}$ is presentable):} Recall that a presentable stable $\infty$-category is the same as an $\infty$-category that is:
                                                \begin{itemize}
                                                    \item accessible (cf. proposition \ref{prop: accessible_stable_infinity_categories}; accessible $\infty$-category is simply one that is its own ind-completion) and
                                                    \item equivalent to a left-exact localisation (i.e a left-exact functor with a full faithful right-adjoint) of a stable $\infty$-category.
                                                \end{itemize}  
                                                
                                            The second condition is an automatic consequence of proposition \ref{prop: localising_at_almost_modules} and the assumption that $\m$ is flat (which means, by definition, that $j^*(-) \cong \m \tensor_{\scrV} -$ is a left-exact functor), and so it remains to prove the first      condition. 
                                            
                                            For this, consider a filtered colimit:
                                                $$\underset{i \in I}{\colim} (\m \tensor_{\scrV}^{\L} M_i)$$
                                            of almost $(\scrV, \m)$-modules $\m \tensor_{\scrV}^{\L} M_i$ (see proposition \ref{prop: localising_at_almost_modules} for why almost modules are of this form). We can then apply the fact that $\m \tensor_{\scrV}^{\L} -$ commutes with all colimits to get:
                                                $$\m \tensor_{\scrV}^{\L} \underset{i \in I}{\colim} (\m \tensor_{\scrV}^{\L} M_i) \cong \m \tensor_{\scrV}^{\L} \underset{i \in I}{\colim} M_i$$
                                            Because ${\scrV}^{\leq 0}\mod$ is a cocomplete category (which implies, in particular, that $\underset{i \in I}{\colim} M_i$ is a $\scrV$-module), this tells us that ${(\scrV, \m)}^{\leq 0}\mod^{\almost}$ is closed under filtered colimits. 
                                            
                                            The category ${(\scrV, \m)}^{\leq 0}\mod^{\almost}$ is thus accessible by definition, and we have therefore shown that ${(\scrV, \m)}^{\leq 0}\mod^{\almost}$ is presentable as an $\infty$-category.
                                            \item \textbf{($j^*$ is accessible):} Because tensor products commute with all colimits, $j^*$ is trivially accessible by virtue of being naturally isomorphic to $\m \tensor_{\scrV}^{\L} -$. 
                                            \item \textbf{($j^*$ is left-exact):} This is an automatic consequence of the flatness assumption on $\m$.
                                        \end{enumerate}
                                    \item \textbf{($!$-pullback):} We have already shown that ${(\scrV, \m)}^{\leq 0}\mod^{\almost}$ is a presentable $\infty$-category, so we can apply Lurie's Adjoint Functor Theorem again, which tells us that we will only need to show that $j_*$ is right-exact. However, because corepresentable functors preserve colimits \textit{a priori} (cf. \cite{nlab:hom-functor_preserves_limits}), the functor:
                                        $$j_* \cong \R\Hom_{\scrV}(\m, -)$$
                                    is trivially right-exact.
                                \end{enumerate}
                        \end{proof}
                        
                    \begin{proposition}[Tensor products of almost modules] \label{prop: tensor_products_of_almost_modules}
                        Let $(\scrV, \m)$ be a basic setup. Then, ${(\scrV, \m)}^{\leq 0}\mod^{\almost}$ is a rigid monoidal category; in particular, there is a compatibility of tensor products in the following manner for all $\scrV$-modules $M$ and $N$:
                            $$j^*M \tensor^{\L}_{\scrV} j^*N \cong j^*(M \tensor^{\L}_{\scrV} N)$$
                        where $j^*$ is as in proposition \ref{prop: localising_at_almost_modules} and theorem \ref{theorem: four_functors_for_almost_modules}; the monoidal unit is $\m$.
                    \end{proposition}
                        \begin{proof}
                            
                        \end{proof}
                    \begin{convention}[Almost tensor products] \label{conv: almost_tensor_products}
                        Let $(\scrV, \m)$ be a basic setup. Then for all $M, N \in {\scrV}^{\leq 0}\mod$, let us abbreviate the quasi-isomorphism:
                            $$j^*M \tensor^{\L}_{\scrV} j^*N \cong j^*(M \tensor^{\L}_{\scrV} N)$$
                        by:
                            $$M^{\almost} \tensor^{\L}_{\scrV} N^{\almost} \cong (M \tensor^{\L}_{\scrV} N)^{\almost}$$
                        whenever the underlying basic setup $(\scrV, \m)$ is understood.
                    \end{convention}
                    \begin{remark}[(In)compatibility of monoidal structures] \label{remark: incompatible_monoidal_structures_almost_modules}
                        Let $(\scrV, \m)$ be a basic setup, which is possible non-flat.
                        \begin{itemize}
                            \item \textbf{(Tensor products of almost modules):} ${\scrV}^{\leq 0}\mod$ does \textit{not} admit ${(\scrV, \m)}^{\leq 0}\mod^{\almost}$ as a \textit{monoidal} subcategory, as the monoidal structures thereon are formed differently; in particular, their monoidal units ($\scrV$ and $\m$ respectively) do not coincide.
                            \item \textbf{(Tensor products of almost-zero modules):} The story is not quite the same for almost-zero modules, which form a subcategory of ${\scrV}^{\leq 0}\mod$. Thanks to the assumption that $\m$ is idempotent (in the sense that $\m \tensor_{\scrV}^{\L} \m \cong \m$; cf. convention \ref{conv: basic_setups} and remark \ref{remark: almost_zero_modules_over_non_flat_basic_setups}), one has the following for all almost-zero modules $M, N$:
                                $$\m \tensor_{\scrV}^{\L} (M \tensor_{\scrV}^{\L} N) \cong (\m \tensor_{\scrV}^{\L} \m) \tensor_{\scrV}^{\L} (M \tensor_{\scrV}^{\L} N) \cong (\m \tensor_{\scrV}^{\L} M) \tensor_{\scrV}^{\L} (\m \tensor_{\scrV}^{\L} N) = 0 \tensor_{\scrV}^{\L} 0 \cong 0$$
                            which proves that ${(\scrV, \m)}^{\leq 0}\mod^0$ is a monoidal subcategory of ${\scrV}^{\leq 0}\mod$. 
                        \end{itemize}
                    \end{remark}
                    
                    \begin{definition}[Almost flatness] \label{def: almost_flatness}
                    
                    \end{definition}
                
                \paragraph{Deep ramifications}
        
            \subsubsection{Perfectoid fields and rings}
                We begin by introducing the notion of perfectoid rings and affinoid perfectoid (pre)adic spaces.
            
                \begin{definition}[Perfectoid fields and perfectoid algebras over them] \label{def: perfectoid_fields}
                    A \textit{complete} Tate ring $R$ is \textbf{perfectoid} if and only if:
                        \begin{itemize}
                            \item \textbf{(Uniformity):} The subring of topologically bounded elements $R^{\circ}$ is itself bounded (in the sense that for all open neighbourhoods $U$ of $0$, there exist another open neighbourhood $V$ of $0$ such that $VR^{\circ} \subseteq U$), and 
                            \item \textbf{(Surjectivity of Frobenius):} There exists a pseudo-uniformiser $\varphi \in R^{\circ \circ}$ such that $\varphi^p \mid p$ and that the Frobenius on $R^{\circ}/\varphi$ is \textit{surjective}.
                        \end{itemize}
                \end{definition}
                \begin{remark}[Perfectoid fields contain all $p$-power roots] \label{remark: perfectoid_fields_have_p_power_roots}
                    The requirement that residual Frobenii are surjective is an especially important one for perfectoid spaces. There is a myriad of facts that it implies, but they are all further consequences of one: that perfectoid fields contain lots of $p^{th}$ power roots. Interestingly, this is a straightforward consequence of the definition of surjectivity, which tells us that for all perfectoid fields $K$, one has:
                        $$\forall x \in K^{\circ}/p: \exists y \in K^{\circ}/p: y^p = x$$
                    which is the same as:
                        $$\forall x \in K^{\circ}/p: \exists y \in K^{\circ}/p: y = x^{\frac1p}$$
                    and from this one gets the existence of roots of higher powers $x^{\frac{1}{p^n}}$. 
                    
                    This detail will be very important when we try to construct examples of perfectoid fields, as well as in lemma \ref{lemma: perfectoid_tilts_are_perfect} where we establish the fact that tilts of perfectoid fields are perfect (thereby justifying the name \say{perfectoid}).
                \end{remark}
                \begin{example}[Perfectoid rings and fields] \label{example: perfectoid_rings_and_fields}
                    Standard examples of perfectoid fields are the topological completions of $\Q_p(p^{1/p^{\infty}}), \Q_p(\mu_{p^{\infty}}), \Q_p^{\alg}$, and $\F_p(\!(t^{1/p^{\infty}})\!)$.
                \end{example}
                
                \begin{definition}[Tilts] \label{def: perfectoid_tilts}
                    Let $R$ be a perfectoid ring and let $\varpi \in R^{\circ \circ}$ be some pseudo-uniformiser. Then, first of all, one defines its multiplicative tilt $R^{\flat}$, where one takes the limit in the category of commutative monoids. Next, one gives $R^{\flat}$ the structure of a ring via:
                        $$R^{\flat} \cong (R^{\circ}/\varpi)^{\flat}[1/\varpi^{\flat}]$$
                \end{definition}
                \begin{example}
                    Different perfectoid fields can have isomorphic tilts, e.g. the tilts of both $\Q_p(p^{1/p^{\infty}})$ and $\Q_p(\mu_{p^{\infty}})$ are both isomorphic to $\F_p(\!(t^{1/p^{\infty}})\!)$. Later on, we will see that this gives rise to the moduli space of untilts (i.e. the Fargues-Fontaine Curve).
                \end{example}
                
                \begin{lemma}[Tilts are perfectoid] \label{lemma: perfectoid_tilts_are_perfect}
                    \noindent
                    \begin{enumerate}
                        \item The tilt of any perfectoid ring of residue characteristic $p > 0$ is a perfectoid ring of characteristic $p$. 
                        \item A complete Tate ring of prime characteristic $p$ is perfectoid if and only if it is perfect. 
                    \end{enumerate}
                \end{lemma}
                    \begin{proof}
                        
                    \end{proof}
                \begin{corollary}[Tilts are perfect] \label{coro: perfectoid_tilts_are_perfect}
                    The tilt of a perfectoid ring is a perfectoid ring that is perfect.
                \end{corollary}
                
                \begin{lemma}[Algebraic extensions of perfectoid fields]
                    Finite extensions and algebraic closures of perfectoid fields are also perfectoid.  
                \end{lemma}
                    \begin{proof}
                        
                    \end{proof}
                \begin{theorem}[The Tilting Equivalence for perfectoid fields] \label{theorem: tilting_equivalence_for_perfectoid_fields}
                    For any perfectoid field $K$, there exists a canoncial adjoint-equivalence of \href{https://stacks.math.columbia.edu/tag/0BMQ}{\underline{Galois categories}}\footnote{Recall that the category of finite extensions of any field $k$ is precisely $\Sch_{/\Spec k}^{\fet}$ (which also happens to be $\Sch_{/\Spec k}^{\aff, \fet}$, since fields are of relative dimension $0$ over one another), the category of schemes finite \'etale over $\Spec k$ (see \cite[\href{https://stacks.math.columbia.edu/tag/0BL6}{Tag 0BL6}]{stacks} and \cite[\href{https://stacks.math.columbia.edu/tag/00U3}{Tag 00U3}]{stacks}).} as follows:
                        $$
                            \begin{tikzcd}
                            	{\{\text{Finite extensions of $K$}\}} & {\{\text{Finite extensions of $K^{\flat}$}\}}
                            	\arrow[""{name=0, anchor=center, inner sep=0}, "{(-)^{\flat}}"', shift right=2, from=1-1, to=1-2]
                            	\arrow[""{name=1, anchor=center, inner sep=0}, "\W"', shift right=2, from=1-2, to=1-1]
                            	\arrow["\dashv"{anchor=center, rotate=-90}, draw=none, from=1, to=0]
                            \end{tikzcd}
                        $$
                    wherein $\W$ denotes the Witt vector functor.
                \end{theorem}
                    \begin{proof}
                        
                    \end{proof}
                \begin{corollary}[Fontaine-Wintenberger for perfectoid fields]
                    The absolute Galois groups of any perfectoid fields is \textit{canonically} isomorphic to that of its tilt.
                \end{corollary}
                \begin{example}
                    The classical version of theorem \ref{theorem: tilting_equivalence_for_perfectoid_fields} (due to Fontaine and Wintenberger) asserts that there is a canonical isomorphism $\bfG_{ \Q_p(p^{1/p^{\infty}})^{\wedge} } \cong \bfG_{ \F_p(\!(t^{1/p^{\infty}})\!)^{\wedge} }$.
                \end{example}
                
            \subsubsection{Perfectoid spaces}
                \begin{definition}[Perfectoid Huber pairs] \label{def: perfectoid_huber_pairs}
                    A Huber pair $(R, R^+)$ is \textbf{perfectoid} if and only if $R$ is a perfectoid ring. 
                \end{definition}
                
                \begin{lemma}[Perfectoid Huber pairs are sheafy] \label{lemma: perfectoid_huber_pairs_are_sheafy}
                    Any perfectoid Huber pair $(R, R^+)$ is sheafy.
                \end{lemma}
                    \begin{proof}
                        
                    \end{proof}
                \begin{definition}[Perfectoid spaces] \label{def: perfectoid_spaces}
                    A \textbf{perfectoid space} is an adic space that is locally isomorphic to affinoid perfectoid spaces $\Spa(R, R^+)$. 
                    
                    The category of perfectoid spaces is denoted by $\Perfd$.
                \end{definition}
                \begin{proposition}[Pullbacks and products of perfectoid spaces] \label{prop: pullbacks_products_of_perfectoid_spaces}
                    Finite pullbacks and finite products exist in $\Perfd$, but there are no terminal objects (which means that products are not pullbacks). 
                \end{proposition}
                    \begin{proof}
                        
                    \end{proof}
                
                \begin{lemma}[Tilts of perfectoid Huber pairs] \label{lemma: tilts_of_perfectoid_huber_pairs}
                    Let $(R, R^+)$ be a perfectoid Huber pair and let $\varpi \in R^{\circ \circ}$ be a choice of pseudo-uniformiser. Then, $(R^+)^{\flat} \cong (R^+/\varpi)^{\flat}$. Furtheremore $R^{\circ \flat} \cong R^{\flat \circ}$.
                \end{lemma}
                    \begin{proof}
                        
                    \end{proof}
                \begin{convention}
                    Usually, one writes $R^{\flat +}$ in place of $(R^+)^{\flat}$.
                \end{convention}
                \begin{proposition}[Functoriality of tilting] \label{prop: tilting_functoriality}
                    \noindent
                    \begin{enumerate}
                        \item \textbf{(Relative affinoid tilting functors)} Fix an affinoid perfectoid space $S$. Then, there exists a functor $(-)^{\flat, \affd}_{/S}: \Perfd_{/S}^{\affd} \to \Perfd_{/S^{\flat}}^{\affd}$. This functor admits a left-adjoint, that being the relative Witt vector functor over $S$. 
                        \item \textbf{(The absolute affinoid tilting functor):} There also exists a functor $(-)^{\flat, \affd}: \Perfd^{\affd} \to \Perfd^{\affd}$.
                    \end{enumerate}
                \end{proposition}
                    \begin{proof}
                        
                    \end{proof}
                    
                \begin{lemma}[The Affinoid Tilting Equivalence] \label{lemma: the_affinoid_tilting_equivalence}
                    Let $S$ be an affinoid perfectoid space. Then, one has an equivalence:
                        $$\Perfd^{\affd, \fet}_{/S} \cong \Perfd^{\affd, \fet}_{/S^{\flat}}$$
                    (via the tilting functor of proposition \ref{prop: tilting_functoriality}) of \'etale sites fibred over the absolute \'etale site $\Perfd^{\affd, \fet}$ of affinoid perfectoid spaces and finite \'etale morphisms between them.
                \end{lemma}
                    \begin{proof}
                        
                    \end{proof}
                \begin{theorem}[The Tilting Equivalence] \label{theorem: the_tilting_equivalence}
                    Let $X$ be a perfectoid space. Then, one has an equivalence:
                        $$(\Perfd_{/X})_{\et} \cong (\Perfd_{/X^{\flat}})_{\et}$$
                    (via the tilting functor of proposition \ref{prop: tilting_functoriality}) of \textit{small} \'etale sites fibred over the absolute \'etale site $\Perfd_{\et}$.
                \end{theorem}
                    \begin{proof}
                        
                    \end{proof}
                \begin{corollary}[The Tilting Equivalence for fibred \'etale topoi]
                    Let $X$ be a perfectoid space. Then, one has an equivalence:
                        $$X_{\et} \cong X^{\flat}_{\et}$$
                    (via the tilting functor of proposition \ref{prop: tilting_functoriality}) of \textit{small} \'etale topoi fibred over the absolute \'etale site $\Perfd_{\et}$. 
                \end{corollary}
                    
        \subsection{Perfectoid spaces arising from arithemtic jet spaces}
                
            \subsubsection{Attaching perfectoid spaces to smooth schemes}
                
            \subsubsection{Attaching perfectoid spaces to formal schemes}
        
    \section{Diamonds} \label{section: diamonds}
        \subsection{Stacks on perfectoid spaces}        
            \subsubsection{Diamonds}
                \begin{definition}[Diamonds] \label{def: diamonds}
                    Let $p$ be a prime and let $X$ be a base perfectoid space of characteristic $p$. Also, let $\kappa$ be a regular cardinal. A ($\kappa$-small) \textbf{diamond} is thus a quotient stack $[Y/R]$ internal to $X_{\proet}^{< \kappa}$ (see definition \ref{def: quotient_stacks} for what this means), where $Y$ is representable.
                \end{definition}
                
                \begin{proposition}[Categories of diamonds] \label{prop: categories_of_diamonds}
                    Let $p$ be a prime and let $X$ be a base perfectoid space of characteristic $p$. Also, let $\kappa$ be a regular cardinal.
                        \begin{enumerate}
                            \item \textbf{(Morphisms of diamonds):} A morphism of diamonds on $X$ is simply a morphism of groupoids internal to $X_{\proet}^{< \kappa}$. A full subcategory of $X_{\proet}^{< \kappa}$ spanned by ($\kappa$-small) diamonds on $X$ thus naturally exists, and we shall denote it by $\Dia_{/X}^{< \kappa}$. 
                            \item \textbf{(Limits and colimits of diamonds):}
                                \begin{enumerate}
                                    \item \textbf{(Limits):} Finite products and finite pullbacks exist in $\Dia_{/X}^{< \kappa}$. These limts are taken as limits of pro-\'etale sheaves on $X$. 
                                    \item \textbf{(Colimits):} 
                                \end{enumerate}
                        \end{enumerate}
                \end{proposition}
                    \begin{proof}
                        \noindent
                        \begin{enumerate}
                            \item \textbf{(Morphisms of diamonds):}
                            \item \textbf{(Limits and colimits of diamonds):}
                                \begin{enumerate}
                                    \item \textbf{(Limits):}
                                    \item \textbf{(Colimits):}
                                \end{enumerate}
                        \end{enumerate}
                    \end{proof}
            
            \subsubsection{Perfectoid Artin stacks}
                \begin{definition}[Perfectoid Artin stacks] \label{def: perfectoid_artin_stacks} \index{Perfectoid Artin stacks}
                    Let $p$ be a prime and let $X$ be a perfectoid space of characteristic $p$. A \textbf{perfectoid Artin v-stack} (or simply an \textbf{Artin v-stack}, since we have only defined the v-topology for perfectoid spaces) is thus a 
                \end{definition}
                
            \subsubsection{An example: \texorpdfstring{$\Bun_G$}{}}
        
        \subsection{Smoothness of diamonds}
            \subsubsection{Formal smoothness}
            
            \subsubsection{A Jacobian criterion}
            
    \section{The Fargues-Fontaine Curve} \label{section: the_fargues_fontaine_curve}
        \subsection{Construction of The Fargues-Fontaine Curve}
        
        \subsection{A GAGA theorem for The Fargues-Fontaine Curve}
	    
	    \chapter{Purity}
    \begin{abstract}
        
    \end{abstract}
    
    \minitoc
    
    \section{Zariski-Nagata purity}
        \subsection{Local purity}
            \begin{definition}[Galois covers] \label{def: galois_covers}
                \noindent
                \begin{enumerate}
                    \item \textbf{(Degree of a morphism):} An \textit{affine} morphism $f: Y \to X$ is said to be \textbf{finite locally free of degree $d$} if:
                        $$f^{\sharp}: \calO_X \to f_*\calO_Y$$
                    identifies $f_*\calO_Y$ as a finite locally free $\calO_X$-module that is of rank $d$; in such a situation one writes:
                        $$\deg f = d$$
                    \item \textbf{(Galois covers):} Let $X$ be a connected base scheme and let $f: Y \to X$ be a finite \'etale morphism (i.e. an object of $\Sch_{/X}^{\fet}$) from a \textit{connected} scheme $Y$. Then, $Y$ is a \textbf{Galois cover} of $X$ if and only if:
                        $$|\Aut_X(Y)| = \deg f$$
                    (note that the notion of degree here makes sense because finite \'etale morphisms are \textit{a priori} finite locally free).
                \end{enumerate}
            \end{definition}
        
        \subsection{Purity of ramification loci}
    
    \section{Falting's almost purity}
	    
	    \chapter{\texorpdfstring{$p$}{}-adic \'etale cohomology}
    \begin{abstract}
        
    \end{abstract}
    
    \minitoc
    
    \section{Pro-\'etale cohomology for schemes: "Another fine cohomology theory to add to my collection"}
        Despite its successes - particularly its involvement in the proofs of the Weil Conjectures - \'etale cohomology has its shortcomings. For one, \'etale cohomology modules are not actually computed over \'etale sites, but rather, as scalar extensions (to $\overline{\Q_{\ell}}$ in the case of $\ell$-adic cohomology for instance) of the actually \'etale cohomology modules over \'etale sites. Pro-\'etale cohomology theories exist so that such issues could be rectified. 
        
        \begin{remark}[What is the pro-\'etale topology ?]
            For now, we refer the reader to \cite[Definition 4.1.1 and Remark 4.1.3]{bhatt_scholze_2014_pro_etale}.
        \end{remark}
        
        \subsection{The pro-\'etale topology for schemes}
        
        \subsection{Infinite Galois categories and pro-\'etale fundamental groups}
            \subsubsection{The abstract framework}
                \begin{definition}[Infinite Galois categories] \label{def: infinite_galois_categories}
                    \noindent
                    \begin{itemize}
                        \item \textbf{(Infinite Galois categories):} An \textbf{infinite Galois category} is defined via the data contained in a pair $(\calG, F)$ consisting of:
                        \begin{itemize}
                            \item a \textit{cocomplete and \textit{finitely} complete} small category $\calG$, wherein objects can all be written as a coproduct of connected objects, and
                            \item a functor $F: \calG \to \Sets$, called the \textbf{fibre functor}, which we shall require to reflect and preserve all colimits and all finite limits\footnote{In \cite[Definition 7.2.1]{bhatt_scholze_2014_pro_etale}, it is furthermore required that the fibre functor would be faithful. However, because any functor that preserves and reflect monomorphisms must also be faithful, we need not make this requirement.}.
                        \end{itemize}
                        \item \textbf{(Galois objects):} An object $Y$ of an infinite Galois category $\calG$ is a \textbf{Galois object} if and only if $Y/\Aut(Y)$ is terminal as an object of $\calG$ (which exists as $\calG$ is finitely complete by definition).
                        \item \textbf{(Galois functors):} A \textbf{Galois functor} is a functor $\Phi: \calG \to \calG'$ between infinite Galois categories $(\calG, F), (\calG', F')$ which preserves connected objects, colimits, finite limits, and commute with the fibre functors in the following manner:
                            $$
                                \begin{tikzcd}
                                	\calG && {\calG'} \\
                                	& \Sets
                                	\arrow["F"', from=1-1, to=2-2]
                                	\arrow["{F'}", from=1-3, to=2-2]
                                	\arrow["\Phi", from=1-1, to=1-3]
                                \end{tikzcd}
                            $$
                    \end{itemize}
                \end{definition}    
                \begin{definition}[Fundamental groups of infinite Galois categories] \label{def: fundamental_groups_of_infinite_galois_categories}
                    The \textbf{fundamental group} of a given infinite Galois category $(\calG, F)$, denoted by $\pi_1(\calG, F)$, is defined to be the automorphism group $\Aut(F)$\footnote{Compare this to definition \ref{def: fundamental_groups_of_finite_galois_categories} and note the lack of profinite completion the current situation.}.
                \end{definition}
                Arbitrary infinite Galois categories turn out to be rather pathological, so we shall introduce a more refined notion, namely that of infinite Galois categories which are \textbf{tame}.  
                \begin{definition}[Tame infinite Galois categories] \label{def: tame_infinite_galois_categories}
                    \footnote{Note that this is slightly different from \cite[Definition 7.2.4]{bhatt_scholze_2014_pro_etale}, in that we require furthermore that the fibre functor preserves connectedness instead of merely that $\pi_1(\calG, F)$ acts transitively on $F(X)$ whenever $X$ is connected.} One says that an infinite Galois category $(\calG, F)$ is \textbf{tame}\footnote{The terminology comes from the fact given any Galois extension $L/K$, the corresponding Galois group $\Gal(L/K)$ acts transitively on $L$ (cf. \cite[Proposition I.9.1]{neukirch_2010_algebraic_number_theory}).} whenever the fibre functor $F: \calG \to \Sets$ preserves connectedness\footnote{Compare this to corollary \ref{coro: fundamental_groups_of_finite_galois_categories_act_transitively_on_connected_objects}.} and $\pi_1(\calG, F)$ acts transitively on $F(X)$ for all $X$ connected.
                \end{definition}
                
                Let us now introduce the notion of Noohi groups, which shall serve in this new theory of infinite Galois categories as replacements for the more restrictive notion of profinite groups. The idea is that, similar to how fundamental groups of finite Galois categories, the fundamental group of a \textit{tame} infinite Galois category is Noohi. We shall also see that the notion of Noohi groups subsumes the notion of profinite groups and as such, so too does the notion of tame infinite Galois categories subsume that of finite Galois categories.
                \begin{definition}[Noohi groups] \label{def: noohi_groups}
                    \cite[Defintion 7.1.1]{bhatt_scholze_2014_pro_etale} A \textbf{Noohi group} is a Hausdorff topological group $G$ such that $G \cong \Aut(\oblv_G)$ \textit{as topological groups} (here, $\Aut(\oblv_G)$ carries the compact-open topology), with $\oblv_G: G\-\Sets \to \Sets$ being the forgetful functor from the category of sets with continuous $G$-actions into the category of sets. In addition, an (in)finite Galois category that is Galois-equivalent to $G\-\Sets$ for some Noohi group $G$ shall be called a \textbf{Noohi category}.
                \end{definition}
                \begin{remark}[Completions of topological groups] \label{remark: completions_of_topological_groups}
                    If $X$ is a topological space then we shall denote its Cauchy completion by $X^+$ (i.e. $X^+$ is the minimal topological space containing $X$, in which all Cauchy filters converge). 
                        \begin{itemize}
                            \item By \cite[Theorem 3.6.10]{topological_groups_and_related_structures}, we know that such a completion exists and is unique for all topological groups; furthermore, every topological group is continuously isomorphic to a dense subgroup of its Cauchy completion. 
                            \item \cite[Theorem 3.6.22]{topological_groups_and_related_structures} Any small product of complete topological groups is also a complete topological group.
                            \item Any closed subgroup of a complete topological group is complete.
                        \end{itemize}
                \end{remark}
                \begin{proposition}[Noohi groups are complete] \label{prop: noohi_groups_are_complete}
                    \cite[Proposition 7.1.5]{bhatt_scholze_2014_pro_etale} Let $G$ be a Hausdorff topological group with a basis generated by open subsets. Then, there is an homeomorphic group isomorphism $\psi: \Aut(\oblv_G) \to G^+$. In fact, a Hausdorff topological group is a Noohi group if and only if it is complete and has a basis generated by open subgroups.
                \end{proposition}
                \begin{lemma}[Fundamental groups of infinite Galois categories are Noohi] \label{lemma: fundamental_groups_of_infinite_galois_categories_are_noohi}
                    A topological group is Noohi if and only if it is the fundamental group of an infinite Galois category.
                \end{lemma}
                    \begin{proof}
                        If $G$ is a Noohi group then by definition $\Aut(\oblv_G) \cong G$. Thus, it shall suffice to demonstrate that the pair $(G\-\Sets, \oblv_G)$ is an infinite Galois category; this is routine (cf. \cite[Section 3]{nlab:category_of_G_sets}), so we shall leave it up to our readers as an exercise.
                    
                        Conversely, if $(\calG, F)$ is an infinite Galois category then it shall suffice to show that $\pi_1(\calG, F)$ will be a closed subgroup of $\prod_{Y \in \Ob(\calG)} \Aut(F(Y))$ (which is a well-defined group since it is a small product of groups), as we can demonstrated this product group to be Noohi. For this, we use the fact that $\Aut(S)$ for any set $S$ is complete in the compact-open topology (cf. \cite[Lemma 7.1.4]{bhatt_scholze_2014_pro_etale}), and that $\Aut(S)$ admits a basis generated by open subgroups (cf. \cite[\href{https://stacks.math.columbia.edu/tag/0BMC}{Tag 0BMC}]{stacks}): by combining this with the fact that products of complete groups are complete (cf. remark \ref{remark: completions_of_topological_groups}), one infers that $\prod_{Y \in \Ob(\calG)} \Aut(F(Y))$ endowed with the product topology is complete and admits a basis generated by open subgroups, meaning that it is Noohi (cf. proposition \ref{prop: noohi_groups_are_complete}). Then, to show that $\pi_1(\calG, F)$ is a closed subgroup of $\prod_{Y \in \Ob(\calG)} \Aut(F(Y))$, we shall show that it is complete (see remark \ref{remark: completions_of_topological_groups}). For this, see \cite[\href{https://stacks.math.columbia.edu/tag/0BMR}{Tag 0BMR}]{stacks}.
                    \end{proof}
                \begin{corollary}[Noohi categories are tame] \label{coro: noohi_categories_are_tame}
                    Noohi categories, as in definition \ref{def: noohi_groups}, are tame infinite Galois categories. 
                \end{corollary}
                    
                \begin{remark}[Action of fundamental groups on fibres] \label{remark: action_of_fundamental_groups_on_fibres_infinite_galois_categories}
                    By arguing as in remark \ref{remark: action_of_fundamental_groups_on_fibres_finite_galois_categories}, one sees that the fibre functor $F: \calG \to \Sets$ of any infinite Galois category $(\calG, F)$ must factor through $\pi_1(\calG, F)\-\Sets$ in the following manner:
                        $$
                            \begin{tikzcd}
                            	\calG && {\pi_1(\calG, F)\-\Sets} \\
                            	& \Sets
                            	\arrow[dashed, from=1-1, to=1-3]
                            	\arrow["F"', from=1-1, to=2-2]
                            	\arrow["{\oblv_{\pi_1(\calG, F)}}", from=1-3, to=2-2]
                            \end{tikzcd}
                        $$
                    As a consequence, functions $F(X) \to F(Y)$ coming from morphisms $X \to Y$ in $\calG$ are automatically $\pi_1(\calG, F)$-equivariant. 
                \end{remark}
                \begin{theorem}[The Infinite Galois Correspondence] \label{theorem: infinite_categorical_galois_correspondence}
                    \cite[Theorem 7.2.5(3)]{bhatt_scholze_2014_pro_etale} Any tame infinite Galois categories $(\calG, F)$, the functor $F: \calG \to \pi_1(\calG, F)\-\Sets$ is a Galois equivalence\footnote{As such, every tame infinite Galois category is Noohi.} (we equip $\pi_1(\calG, F)\-\Sets$ with the forgetful functor to $\Sets$ to make it a Noohi category; cf. corollary \ref{coro: noohi_categories_are_tame}).
                \end{theorem}
                    \begin{proof}
                        Because $F$ reflects monomorphisms (or more generally, all finite limits, as it is an isomorphism-reflecting left-exact functor; cf. definition \ref{def: infinite_galois_categories}) and because any function $f: S \to T$ can be viewed as a unique monomorphism $\Gamma_f: S \to S \x T$ such that $\pr_1 \circ \Gamma_f = \id_S$, any morphism $p: X \to Y$ in $\calG$ can be viewed as a unique monomorphism $\Gamma_p: X \to X \x Y$ such that $\pr_1 \circ \Gamma_p = \id_X$, which comes from. By combining this with the fact that $F$ commutes with finite limits, and that functions $F(X) \to F(Y)$ coming from morphisms $X \to Y$ in $\calG$ are automatically $\pi_1(\calG, F)$-equivariant (cf. remark \ref{remark: action_of_fundamental_groups_on_fibres_infinite_galois_categories}), we obtain the following bijections, which proves that $F: \calG \to \pi_1(\calG, F)\-\Sets$ is fully faithful:
                            $$
                                \begin{aligned}
                                    \calG(X, Y) & \cong \{\text{Monomorphisms $\Gamma_p: X \to X \x Y$ such that $\pr_1 \circ \Gamma_p = \id_X$}\}
                                    \\
                                    & \cong \{\text{Monomorphisms $\Gamma_f: F(X) \to F(X) \x F(Y)$ such that $\pr_1 \circ \Gamma_f = \id_{F(X)}$}\} 
                                    \\
                                    & \cong \pi_1(\calG, F)\-\Sets(F(X), F(Y))
                                \end{aligned}
                            $$
                        As $\pi_1(\calG, F)\-\Sets$ is an infinite Galois category, every object $S \in \pi_1(\calG, F)\-\Sets$ admits a decomposition $S \cong \coprod_{i \in I_S} S_i$ into connected objects $S_i \in \pi_1(\calG, F)\-\Sets$. But the functor $F: \calG \to \pi_1(\calG, F)\-\Sets$ reflects colimits by definition, so such a coproduct gives rise to a corresponding coproduct $Y_S := \coprod_{i \in I_S} Y_i$ in $\calG$, and since $\calG$ has all colimits, $Y_S$ is an object thereof. As such, for any $S \in \pi_1(\calG, F)\-\Sets$, there exists a corresponding object $Y_S \in \calG$ such that $F(Y_S) \cong S$. This implies that the functor $F: \calG \to \pi_1(\calG, F)\-\Sets$ is essentially surjective, which when combined with the fact that it is fully faithful (as shown above), implies that this functor is an equivalence of categories. 
                        
                        It remains to show that this equivalence of categories between $\calG$ and $\pi_1(\calG, F)\-\Sets$ is a Galois equivalence between Noohi categories. For this, observe that because $\pi_1(\calG, F)$ is a Noohi group (cf. lemma \ref{lemma: fundamental_groups_of_infinite_galois_categories_are_noohi}), and because $G\-\Sets$ is a Noohi category for all Noohi groups $G$ (cf. corollary \ref{coro: noohi_categories_are_tame}), $F: \calG \to \pi_1(\calG, F)\-\Sets$ is a Galois functor, by virtue of being a functor between (tame) infinite Galois categories that preserves colimits, finite limits, and connected objects.
                    \end{proof}
                
            \subsubsection{Pro-\'etale fundamental groups}
        
    \section{Finiteness of \'etale cohomology for smooth proper rigid-analytic varieties}
    
    \section{Interlude: Condensed mathematics} \label{section: condensed_mathematics}
        \begin{remark}[Strong limit cardinals]
            We will be using the notion of strong limit cardinals often. For details on the notion, see definition \ref{def: limit_cardinal}.
        \end{remark}
    
        \subsection{Basics of condensed mathematics}
            \subsubsection{Condensed sets}
                \begin{definition}[Condensation] \label{def: condensation}
                    Let $\kappa$ be a fixed strong limit cardinal and let $\C$ be a hypercomplete $\infty$-category with enough $\kappa$-small limits and enough $\kappa$-small filtered colimits. We then define so-called \textbf{condensed objects} of $\C$ to be $\C$-valued sheaves over the $\kappa$-small pro-\'etale site of a point (i.e. the pro-\'etale site of the spectrum of a field). 
                    
                    Clearly condensed objects of a given $\infty$-category $\C$ satisfying the above conditions form a category. We shall denote it by $\C^{\cond}$.
                \end{definition}
                \begin{remark}
                    For now, we refer the reader to \cite[Definition 4.1.1 and Remark 4.1.3]{bhatt_scholze_2014_pro_etale} for the definition of pro-\'etale coverages. In particular, recall that the $\kappa$-small pro-\'etale site of a point is equivalent to the site $\Pro_{\kappa}(\Sets^{\fin})$ of $\kappa$-small profinite sets (\textit{viewed as totally disconnected $\kappa$-small compact Hausdorff spaces}), whose coverage is generated by jointly surjective families. 
                \end{remark}
                \begin{example}
                    \noindent
                    \begin{itemize}
                        \item \textbf{(\textit{Small} condensed sets):} The pro-\'etale topos $\Sh(*_{\kappa\-\proet})$ over a point is, by definition, the category of sheaves of sets on the pro-\'etale site of a point. Therefore, this topos is the category of \textbf{$\kappa$-small condensed sets}. Whenenver we wish to put emphasis on the fact that $\Sh(*_{\kappa\-\proet})$ is actually the category of $\kappa$-small condensed sets, we will write $(\Sets^{\cond})^{< \kappa}$ instead.
                        \item \textbf{(Condensed abelian groups and modules):} If $R$ is a condensed commutative ring, then the category $R\mod^{\cond}$ of condensed $R$-modules is a \href{https://ncatlab.org/nlab/show/Grothendieck+category}{\underline{Grothendieck category}}; on the other hand, topological abelian groups even fails to form an abelian category, and as we well know: no abelian categories means no homological algebra. This is a biggy, so we shall bestow upon it the dignity of theorem-hood (see theorem \ref{theorem: abelian_categories_of_condensed_modules}).
                    \end{itemize}
                \end{example}
                \begin{remark}[Condensation and profiniteness] \label{remark: condensation_and_profiniteness}
                    \noindent
                    \begin{enumerate}
                        \item Fix a strong limit cardinal $\kappa$. The $\kappa$-pro-\'etale site of a point is equivalent to the category $\Pro_{\kappa}(\Sets^{\fin})$ of $\kappa$-small profinite sets equipped with the coverage generated by jointly surjective finite families of surjective functions. One can show this using the fact that the pro-\'etale site of a point is the same as the pro-\'etale site of the spectrum of a field, and subsequently, the Fundamental Theorem of Galois Theory, namely the fact that the Galois group of any Galois extension $L/K$ is the filtered limit over the galois groups $\Gal(E/K)$ over all \textit{finite} Galois subextensions $E/K$.
                        \item One important fact to keep in mind is that profinite sets are compact and Hausdorff. This will be used in lemma \ref{lemma: sheaves_over_compact_hausdorff_spaces} to show that the sheaf tops over the category of small compact Hausdorff spaces is the same as the sheaf topos of $\kappa$-small condensed sets.
                    \end{enumerate}
                \end{remark}
                \begin{remark}[Set-theoretic technicalities] \label{remark: condensed_sets_set_theoretic_issues}
                    It might seem as though the fixture of a strong limit cardinal $\kappa$ is an unnecessary gimmick and that the issues that one might run into when removing this cardinal bound are purely philosophical. However, because the unbounded pro-\'etale site $*_{\proet}$ (or for that matter, the category of all profinite sets) is large and sheaves on large sites may not form topoi, and because we rely crucially on the premise that the category of condensed sets would be a sheaf topos, we really do need to take these set-theoretic issues seriously and pre-suppose that we are only working over the $\kappa$-small pro-\'etale site of a point.
                \end{remark}
                
                \begin{lemma}[Sheaves over compact Hausdorff spaces] \label{lemma: sheaves_over_compact_hausdorff_spaces}
                    Fix a strong limit cardinal $\kappa$ and denote by $\Sh(\Comp^{< \kappa})$ the sheaf topos over the site of $\kappa$-small compact Hausdorff topological spaces\footnote{This is sometimes referred to as the topos of $\kappa$-small pyknotic sets.} (with coverage given by jointly surjective families of continuous functions). Then, one has the following equivalence of topoi:
                        $$\Sh(\Comp^{< \kappa}) \cong (\Sets^{\cond})^{< \kappa}$$
                    between the aforementioned sheaf topos and the topos of $\kappa$-small condensed sets.
                \end{lemma}
                    \begin{proof}
                        We can make use of the fact that the $\kappa$-small pro-\'etale site of a point is equivalent to the category of $\kappa$-small profinite sets to see that there exist a functor:
                            $$\beta: *_{\kappa\-\proet} \to \Comp^{< \kappa}$$
                        that is naturally isomorphic to the Stone-\v{C}ech Compactification functor restricted from $\Top^{< \kappa}$ down to $\Pro_{\kappa}(\Sets^{\fin})$. 
                    \end{proof}
                
                \begin{definition}[Extremally disconnected sets] \label{def: extrememly_disconnected_sets}
                    Extremally disconnected compact Hausdorff spaces are projective objects (cf. definition \ref{def: projective_and_injective_objects} and proposition \ref{prop: projectives_and_injectives_lifting_property}) in the category of compact Hausdorff spaces.
                \end{definition}
                \begin{example}
                    \noindent
                    \begin{enumerate}
                        \item \textbf{(Stone-\v{C}ech compactifications):}
                        \item \textbf{(A counter-example: the $p$-adics):} For a fixed prime $p$, the $p$-adic rationals $\Q_p$ is only totally disconnected, not extremally disconnected. 
                    \end{enumerate}
                \end{example}
                
                \begin{lemma}[Small condensed sets and extremally disconnected sets] \label{lemma: small_condensed_sets_and_extremally_disconnected_sets}
                    Fix a strong limit cardinal $\kappa$ and denote the topos of sheaves of sets on the site of $\kappa$-small extremally disconnected sets with coverage given by \textit{finite} jointly surjective families by $\Sh(\sfExt^{< \kappa})$. Then, one has the following equivalence of topoi:
                        $$\Sh(\sfExt^{< \kappa}) \cong (\Sets^{\cond})^{< \kappa}$$
                    between the aforementioned sheaf topos and the topos of $\kappa$-small condensed sets.
                \end{lemma}
                    \begin{proof}
                        
                    \end{proof}
                    
                \begin{lemma}[Enlargements of condensed sets] \label{lemma: large_condensed_sets}
                    Fix strong limit cardinals $\kappa < \lambda$ (cf. definition \ref{def: limit_cardinal}). The natural embedding of $(\Sets^{\cond})^{< \kappa}$ into $(\Sets^{\cond})^{< \lambda}$ is compatible with pro-\'etale sheafification. This is to say, the following diagram of topoi commutes:
                        $$
                            \begin{tikzcd}
                            	{(\Sets^{\cond})^{< \kappa}} & {\Psh(*_{\kappa\-\proet})} \\
                            	{(\Sets^{\cond})^{< \lambda}} & {\Psh(*_{\lambda\-\proet})}
                            	\arrow[hook, from=1-1, to=2-1]
                            	\arrow[hook, from=1-2, to=2-2]
                            	\arrow["{{}^{\sh}(-)}"', from=1-2, to=1-1]
                            	\arrow["{{}^{\sh}(-)}"', from=2-2, to=2-1]
                            \end{tikzcd}
                        $$
                \end{lemma}
                    \begin{proof}
                        
                    \end{proof}
                
            \subsubsection{Abelian categories of condensed objects}
                \begin{theorem}[Abelian categories of condensed modules] \label{theorem: abelian_categories_of_condensed_modules}
                    Fix a regular cardinal $\kappa$. Categories of modules $R\mod^{\cond}$ over $\kappa$-condensed commutative rings $R$ satisfy the following Grothendieck homological axioms:
                        \begin{enumerate}
                            \item They are abelian categories.
                            \item \textbf{($AB3$ \& $AB3^*$):} They are $\kappa$-small complete and $\kappa$-small cocomplete.
                            \item \textbf{($AB3$ \& $AB4^*$):} Coproducts of monics remain monic, and dually, products of epics remain epic.
                            \item \textbf{($AB5$):} Filtered colimits of exact sequences remain exact. Furthermore, categories of condensed modules are compactly generated: this is to say, its generator, call it $\Lambda$ is a compact object (i.e. the copresheaf $R\mod^{\cond}(\Lambda, -)$ preserves filtered colimits).
                            \item \textbf{($AB6$):} $\kappa$-small products commute with $\kappa$-small filtered colimits.
                        \end{enumerate}
                \end{theorem}
                    \begin{proof}
                        \noindent
                        \begin{enumerate}
                            \item Categories of condensed modules are categories of internal modules - specifically to the pro-\'etale topos over a point - and so are trivially abelian. 
                            \item \textbf{($AB3$ \& $AB3^*$):} Again, condensed modules are internal modules, and thus the categories they form are \textit{a priori} $AB5$, and hence $AB3$.
                            \item \textbf{($AB4$ \& $AB4^*$):} 
                            \item \textbf{($AB5$):} We have already shown that categories of condensed modules are $AB5$, so it remains to show that the generator of $R\mod^{\cond}$ is compact.
                            \item \textbf{($AB6$):} 
                        \end{enumerate}
                    \end{proof}
                \begin{corollary}[Properties of categories of condensed modules] \label{coro: condensed_modules_properties}
                    By virtue of being a Grothendieck category (i.e. an $AB5$-category with a generator), any condensed module category $R\mod^{\cond}$ enjoy the following properties:
                        \begin{enumerate}
                            \item 
                                \begin{enumerate}
                                    \item If a presheaf $F: (R\mod^{\cond})^{\op} \to \Sets$ preserves $\kappa$-small limits, then it is representable (i.e. the Yoneda embedding on $R\mod^{\cond}$ preserves all $\kappa$-small limits, not just the finite ones).
                                    \item If a presheaf $F: (R\mod^{\cond})^{\op} \to \Sets$ commutes with all $\kappa$-small colimits, then it possesses a right-adjoint. 
                                \end{enumerate}
                            \item By the \href{https://ncatlab.org/nlab/show/Gabriel-Popescu+theorem}{\underline{Gabriel-Popescu Theorem}}, we can realise $R\mod^{\cond}$ as a reflective localisation of some category of modules over a commutative ring. 
                            \item $R\mod^{\cond}$ is presentable. 
                        \end{enumerate}
                \end{corollary}
                
            \subsubsection{Cohomology of condensed modules}
            
        \subsection{Symmetric monoidal structures; solidity}
            \begin{convention}[Condensed local systems]
                From now on, if $L \in \Sets$ is a set then the corresponding condensed local system (i.e. pro-\'etale local system over a point) shall be suggestively denoted by $L^{\cond}$.
            \end{convention}
        
            First of all, let us clarify that it is not that tensor products of condensed modules do not exist. However, such tensor products will usually end up being topologically pathological or just outright nonsensical. Take for instance, the local system $\Z_p^{\cond} \in \Sets^{\cond}$. Its tensor product with other non-archimedean local systems can be easily topologised via formal completion, but if we were to consider say, $\Z_p^{\cond} \tensor \R^{\cond}$ or $\Z_p^{\cond} \tensor \Z_{\ell}^{\cond}$ (where $\ell \not = p$ is another prime), then it is not very clear what the corresponding topological completion should be.  
            
        \subsection{Analyticity}
    
    \section{\'Etale cohomology of diamonds}
    
    \section{Cohomology of solid pro-\'etale sheaves}
        
    \section{\texorpdfstring{$p$}{}-adic monodromy}
        \subsection{The Weight-Monodromy Conjecture}
            \subsubsection{Toric varieties}
            
            \subsubsection{The Weight-Monodromy Conjecture}
            
        \subsection{Around \texorpdfstring{$p$}{}-adic differential equations}
	    
	    \chapter{Motives}
    \begin{abstract}
        
    \end{abstract}
    
    \minitoc
    
    \section{Weil cohomology theories}
        \subsection{Fantastic cohomology theories and where to find them}
            \subsubsection{Some intersection theory}
                \begin{remark}[Categories of smooth projective schemes] \label{remark: categories_of_smooth_projective_schemes}
                    Fix a ground field $k$. 
                    
                    Inspired by Serre's GAGA Theorem - which relates smooth projective varieties over $\Spec \bbC$ to (compact) complex analytic manifolds - we shall try to set up the so-called Weil cohomology theories so that they would behave well over smooth projective (algebraic) schemes over fields such as $k$ (not that we would object to these cohomology theories working over more general schemes, but one should also be reasonable with one's expectations). For that, we shall need to first see if smooth projective algebraic schemes form a category (otherwise, what even is the point ?); also, notice how even with two additional adjectives, the class smooth projective algebraic schemes still includes a lot of important examples, notable among which are elliptic curves and higher dimensional abelian varieties.
                    
                    Thankfully, we do have a category of smooth projective algebraic schemes over any given base field $k$, which we denote by $\Sch_{/\Spec k}^{\smooth, \proj}$. To see why this is the case, recall firstly that thanks to the universal property of the $\Proj$-construction, a projective schemes $X$ is nothing but an $\N$-filtration of schemes (i.e. a diagram:
                        $$X: \N \to \Sch$$
                    of shape $\N$ whose transition maps are \href{https://stacks.math.columbia.edu/tag/01L1}{\underline{monomorphism of schemes}}; note in particular, that any immersion, closed or open, is a monomorphism \cite[\href{https://stacks.math.columbia.edu/tag/01L7}{Tag 01L7}]{stacks}); by abstract nonsense, $X$ is thus simply a functor that preserves monomorphisms, as every arrow in $\N$ is a monomorphism.  
                \end{remark}
                
                \begin{definition}[Algebraic cycles] \label{def: algebraic_cycles}
                    Let $S$ be a Noetherian base scheme and let $f: X \to S$ be an $S$-scheme of finite type. Recall also that Noetherian schemes form a full subcategory of $\Sch$; let us denote it by $\Sch^{\Noeth}$. 
                        \begin{enumerate}
                            \item \textbf{(Relative cycles):} 
                            \item \textbf{(The Chow functors):} A \textbf{presheaf of $d$-dimensional relative cycles} over $f: X \to S$ (or a \textbf{$d$-dimensional Chow functor} over $f: X \to S$) shall be a functor:
                                $$\Chow_{X/S}(d, -): \Sch_{/S}^{\Noeth} \to \Sets$$
                            that associates to Noetherian $S$-schemes $g: T \to S$ the set $\Chow_{X/S}(d, T)$ of $d$-dimensional relative cycles on the $T$-scheme $X \x_{f, S, g} T \to T$.
                        \end{enumerate}
                \end{definition}
                
            \subsubsection{Weil cohomology theories}    
                \begin{definition}[Weil cohomology theories] \label{def: weil_cohomology_theories}
                    Let $k$ be an arbitrary base field, and let $F$ be a field of characteristic $0$, which will serve as a so-called \say{field of coefficients}. A \textbf{Weil cohomology theory} over $k$ is thus a contravariant functor:
                        $$\H^*: \Sch_{/\Spec k}^{\smooth, \proj, \op} \to [\Z, {}_F\Vect]$$
                    from the category of smooth projective algebraic schemes over $\Spec k$ into the category of $\Z$-graded $F$-vector spaces (which are just diagrams of shape $\Z$ of $F$-vector spaces) that satisfies the following axioms:
                        \begin{enumerate}
                            \item \textbf{(Finiteness):} Given a smooth projective algebraic scheme $X$ over $\Spec k$, we shall want all the vector spaces in the diagram:
                                $$
                                    \H^*(X) =
                                    \left(
                                        \begin{tikzcd}
                                        	\cdots & {\H^{-1}(X)} & {\H^0(X)} & {\H^1(X)} & \cdots
                                        	\arrow[from=1-2, to=1-3]
                                        	\arrow[from=1-3, to=1-4]
                                        	\arrow[from=1-4, to=1-5]
                                        	\arrow[from=1-1, to=1-2]
                                        \end{tikzcd}
                                    \right)
                                $$
                            to be \textit{finite-dimensional}. Additionally, we would like to require that:
                                $$
                                    \dim_F \H^i(X) = 
                                    \begin{cases}
                                        \text{$n_i \not = 0$ if $0 \leq i \leq 2\dim X$}
                                        \\
                                        \text{$0$ otherwise}
                                    \end{cases}
                                $$
                            \item \textbf{(Poincar\'e Duality):} This is to say that there is an isomorphism, called the \textbf{trace map}:
                                $$\int_X(-): \H^{2 \dim X}(X) \cong F$$
                            and that for each $i \in \Z$, there exists a non-degenerate bilinear pairing:
                                $$\<\cdot \mid \cdot\>: \H^i(X) \x \H^{2\dim X - i}(X) \to \H^{2 \dim X}(X)$$
                            which establishes an isomorphism between $\H^i(X) \x \H^{2\dim X - i}(X)$ and $F$ via the trace map $\int_X$. 
                            \item \textbf{(K\"unneth Formula/Monoidality):} For all $i \in \Z$, we have:
                                $$\H^i(X \x_{\Spec k} Y) \cong \H^i(X) \tensor_F \H^i(Y)$$
                            \item \textbf{(Algebraic cycles):}
                        \end{enumerate}
                \end{definition}
                \begin{remark}[Why these axioms ?] \label{remark: motivation_for_motives}
                    For the most part, the axioms laid out in definition \ref{def: weil_cohomology_theories} are there because we want Weil cohomology theories to behave how we have come to expect reasonable \say{geometric} cohomology theories to. In particular, we want for (smooth and projective) schemes cohomology theories that act more or less like singular cohomology or the classical de Rham cohomology for manifolds. In fact, most famous examples of Weil cohomology theories were conceived in the images of singular and de Rham cohomologies: \'etale cohomology (cf. chapter \ref{chapter: etale_cohomology_1}) is supposed to be the topological cohomology theory that works for schemes - especially those in positive characteristics - and crystalline cohomology exists so that we might have a \say{differential geometry} of schemes (cf. chapter \ref{chapter: crystals}). 
                \end{remark}
                \begin{remark}[What about the Lefschetz Conditions ?] \label{remark: lefschetz_axioms}
                    
                \end{remark}
                \begin{example}
                    \noindent
                    \begin{enumerate}
                        \item \textbf{(Over characteristic $0$):} 
                            \begin{enumerate}
                                \item \textbf{(Betti cohomology):}
                                \item \textbf{(de Rham cohomology):}
                                \item \textbf{(A counter-example: Zariski cohomology):}
                            \end{enumerate}
                        \item \textbf{(Over positive characteristics):}
                            \begin{enumerate}
                                \item \textbf{($\ell$-adic cohomology):}
                                \item \textbf{(Crystalline cohomology):}
                            \end{enumerate}
                    \end{enumerate}
                \end{example}
        
        \subsection{Motives}
    
    \section{Complex Hodge theory}
        \subsection{Hodge structures}
            \subsubsection{Pure Hodge structures on compact complex analytic manifolds}
                \begin{theorem}[Grothendieck's algebraic de Rham cohomology] \label{theorem: de_rham_cohomology}
                    Let $X$ be a smooth scheme over $\Spec \bbC$ and let $X^{\an}$ denote the analytic manifold whose underlying set is $X(\bbC)$. Then, one has the following comparison quasi-isomorphism between algebraic and complex-analytic de Rham cohomologies:
                        $$H^*_{\dR}(X) \cong_{\qis} H^*_{\dR}(X^{\an})$$
                \end{theorem}
            
            \subsubsection{Mixed Hodge structures}
        
        \subsection{Periods}
        
        \subsection{Moduli of Hodge structures and complex Shimura varieties}
        
    \section{Motivic \texorpdfstring{$p$}{}-adic Hodge theory}
        \subsection{The Ax-Tate-Sen Theorem}
        
        \subsection{Period rings and period sheaves}
    
        \subsection{The Hodge-Tate Decomposition and the \'etale-de Rham comparison}
            \subsubsection{The Hodge-Tate Decomposition via perfectoid spaces}
                \paragraph{The case for abelian schemes with good reductions}
                    \begin{convention}
                        From this point on, fix a prime $p$ along with a $p$-adic number field $K/\Q_p$. Also, fix a completion $C$ of an algebraic closure of $K$. Lastly, let $A_{/K^{\circ}}$ be an abelian scheme over $\Spec K^{\circ}$ with generic fibre $A_{/K}$.
                    \end{convention}
                
                \paragraph{The general case}
            
            \subsubsection{The \'etale-de Rham comparison}
        
        \subsection{The \'etale-crystalline comparison}
        
        \subsection{Perfectoid Shimura varieties}
        
	    
	    \chapter{\texorpdfstring{$p$}{}-adic Hodge theory}
    \begin{abstract}
        
    \end{abstract}
    
    \minitoc
    
    \section{Motivic \texorpdfstring{$p$}{}-adic Hodge theory}
        \subsection{Introduction}
            The subject that is nowadays known as \say{Hodge theory} is - as least as far as its motivic aspect is concerned - essentially the study of cohomology theories (\'etale, de Rham, crystalline, etc.) as topological invariants of (algebraic and analytic) varieties along with the various relationships between said cohomology theories. As it stands currently, Hodge theory consists of two parallel subfields, namely the theories over the complex numbers $\bbC$ and that over non-archimedean fields such as $\Q_p$, bot of which could be seen as probational steps towards a more grander all-unifying incarnation of Hodge theory over number fields, which for various technical reasons might be the key technical toolbox for tackling some of the deepest networks of conjectures in modern algebraic number theory, like the Global Langlands Correspondence over $\Q$. For algebraic varieties over the complex numbers $\bbC$ (or more precisely, their associated complex manifolds), the story is relatively simple: due to there being only two interesting Grothendieck topologies, namely the \'etale topology on smooth varieties and the complex-analytic topology on their corresponding complex manifolds, there exists only one fundamental comparison theorem, that being the Riemann-Hilbert Correspondence, which more-or-less relates \'etale cohomology of local systems and de Rham cohomology of their associated complex manifolds. Over non-archimedean local fields such as $\Q_p$ or $\F_p(\!(t)\!)$, however, the story is much more complicated; here, there is a plethora of cohomology theories, and since we would like for there to exist pairwise comparison theorems amongst them, we will have to look beyond attempting to simply compare the various Grothendieck topologies. Moreoever, due to $p$-adic Hodge theory being composed of two different facades, namely the theories over mixed characteristic fields such as $\Q_p$ and equicharacteristic fields such as $\F_p(\!(t)\!)$, we will also have to worry about how certain cohomology theories behave poorly over positive characteristics (e.g. de Rham cohomology) and as such require replacements (e.g. crystalline cohomology), which might be full of their own stock of shortcomings and technical difficulties. Luckily, geometry over non-archimedean fields (and particularly those of positive characteristics) is a much richer theory than its complex counterpart and as such, there are many intermediary objects that one might try to make use of in order to \say{interpolate} the various cohomoloy theories and thereby comparing them indirectly; this, incidentally, is an approach to $p$-adic Hodge theory that has now gained mainstream popularity.
    
    \section{Robba rings and \texorpdfstring{$\varphi$}{}-modules}
        \subsection{Robba rings}
        
        \subsection{\texorpdfstring{$\varphi$}{}-modules}
    
        \subsection{\texorpdfstring{$(\varphi, \Gamma)$}{}-modules}
            \subsubsection{Pseudo-coherent sheaves}
            
            \subsubsection{The homological algebra of \texorpdfstring{$(\varphi, \Gamma)$}{}-modules}
            
    \subsection{The Ax-Tate-Sen Theorem}
    
        \subsection{The Hodge-Tate Decomposition and the \'etale-de Rham comparison}
            \subsubsection{The Hodge-Tate Decomposition via perfectoid spaces}
                \paragraph{Scholze's de Rham period sheaf}
                    \begin{definition}[de Rham period sheaves] \label{def: de_rham_period_sheaves}
                        Let $K$ be a perfectoid field of mixed characteristic $(0, p)$ and let $X$ be a perfectoid space over $\Spa K$. Additionally, recall that the integral subsheaf $\calO_{X^{\flat}}^+$ is \textit{a priori} perfect (over characteristic $p$). We are interested in the following commutative ring objects of the pro-\'etale topos $X_{\proet}$:
                            \begin{enumerate}
                                \item \textbf{(Fontaine's infinitesimal period rings):} There is first of all the so-called \textbf{infinitesimal period ring of Fontaine}, usually denoted by $\A_{\inf}$, and is defined to be the ring of Witt vectors $\Witt(\calO_{X^{\flat}}^+)$. One also would usually be interested in its rational localisation, namely $\B_{\inf} := \A_{\inf}[1/p]$. 
                                
                                We care also about the canonical map $\theta: \A_{\inf} \to \calO_X^+$, which extends naturally to a map $\Theta: \B_{\inf} \to \calO_X$.
                                \item \textbf{(Scholze's de Rham period rings):} Scholze, in \cite[Definition 6.1]{scholze2012padic} then defined his \textbf{de Rham period ring} as the formal completion $(\B_{\inf}, \ker \Theta)^{\wedge}$; we will denote this period ring by $\B_{\dR}^+$. There exists also a rational localisation of this period ring, though this is a less than trivial fact.
                            \end{enumerate}
                    \end{definition}
                    
                    \begin{proposition}[$\theta$ is surjective] \label{prop: theta_is_surjective}
                        Let $K$ be a perfectoid field of characteristic $0$ and let $X = \Spa(R, R^+)$ be an affinoid perfectoid space over $\Spa K$.
                        \begin{enumerate}
                            \item The canonical map $\theta: \A_{\inf} \to R^+$ is a surjective continuous ring homomorphism.
                            \item $\ker \Theta$ is a non-zero principal ideal of $\B_{\inf}$.
                        \end{enumerate}
                    \end{proposition}
                        \begin{proof}
                            \noindent
                            \begin{enumerate}
                                \item \textbf{(Surjectivity of $\theta$):} 
                                \item \textbf{(Principality of $\ker \Theta$):} First of all, because localisation is a colimit, the map $\Theta: \B_{\inf} \to \calO_X$ must also be surjective as a consequence of $\theta: \A_{\inf} \to \calO_X^+$ being surjective; this ensures that the ideal $\ker \Theta$ is not trivial.
                            \end{enumerate}
                        \end{proof}
                    \begin{corollary}[Rational de Rham period rings] \label{coro: rational_de_rham_period_rings}
                        \noindent
                        \begin{enumerate}
                            \item Proposition \ref{prop: theta_is_surjective} applies also to non-affinoid perfectoid spaces.
                            \item Additionally, not only is the completion $\B_{\dR}^+ := (\B_{\inf}, \ker \Theta)^{\wedge}$ adic, but also, it admits a rational localisation: if $t \in \ker \Theta$ is any generator, then we can define $\B_{\dR} := \B_{\dR}^+[1/t]$. 
                        \end{enumerate}
                    \end{corollary}
                    
                    \begin{remark}
                        
                    \end{remark}
                
                \paragraph{The Hodge-Tate Decomposition}
            
            \subsubsection{The \'etale-de Rham comparison}
        
        \subsection{The \'etale-crystalline comparison}
    
    \section{Period sheaves and \texorpdfstring{$\varphi$}{}-modules over them}
        \subsection{Perfect period sheaves}
        
        \subsection{Imperfect period sheaves}
	    
    \part{The Categorical-Geometric and Quantum Langlands Correspondences}
        \chapter*{Introduction}
    \begin{abstract}
        
    \end{abstract}
    
    \minitoc
    
    \section{The Global Correspondence}
        Let $X$ be a curve that is smooth, proper, and geometrically connected algebraic curve (for instance, we can take $X$ be an elliptic curve or $\P^1$) and suppose that $G$ is a reductive group (think $\GL_n$ or $\SL_n$, or more concretely, $\GL_1$, or groups of diagonal matrices); both shall be over a field $k$ of characteristic $0$. Additionally, denote the function field of our curve $X$ by $K_X$, the completions of said field at (closed) points $x \in |X|$ by $K_{X, x}$, and we shall write $\scrO_{X, x}$ for the associated rings of integers (note how they coincide with the adic completions $\calO_{X, x}^{\wedge}$).
            
        The end goal for us, shall be to construct some semblance of an equivalence of derived/abelian/stable $\infty$-categories:
            $$\Dmod\left(\Bun_G(X)\right) \cong \Ind\Coh\left(\LocSys_F(X)^{\check{G}}\right)$$
        between:
            \begin{itemize}
                \item the category $\Dmod\left(\Bun_G(X)\right)$ of D-modules on the moduli stack $\Bun_G(X)$ of $G$-bundles on $X$, and
                \item the category $\Ind\Coh\left(\LocSys_F(X)^{\check{G}}\right)$ of ind-coherent sheaves (cf. section \ref{section: indcoh}) on the moduli stack of $\check{G}$-equivariant local systems on $X$ with coefficients in some implicitly understood suitable field $F$. 
            \end{itemize}
        When $G$ is a torus - i.e. when it is abelian - the above correspondence is a bit simpler:
            $$\Dmod\left(\Bun_G(X)\right) \cong \QCoh\left(\LocSys_F(X)^{\check{G}}\right)$$
        (notice how now, we can work with the entire category of quasi-coherent sheaves instead of having to restrict ourselves to ind-coherent sheaves). One thing that needs to be made clear right away, however, is that aside from a few very special cases such as $G = \GL_1$ and $G = \SL_2$, this equivalence is \textit{entirely conjectural}. Nevertheless, we do have a rough idea of how to eventually obtain a proper theorem from this vision:
            \begin{enumerate}
                \item The very first thing to do is to understand the construction of D-modules on (pre)stacks locally of finite type, and we can do this by learning about crystals (in the sense of Grothendieck) and their infinitesimal/crystalline cohomology over base fields of characteristic $0$ (crystalline cohomology over base fields of positive characteristics and the accompanying theory of arithmetic D-modules is significantly more complicated than their characteristic $0$ counterparts, which incidentally is why we have required that $\chara k = 0$).
                \item Then, we must know what $\check{G}$ actually is, i.e. we must understand Langlands duals. There is a tool for this, which is the Geometric Satake Equivalence. However, we are going to have to go through two substeps:
                    \begin{enumerate}
                        \item To begin, we shall need to understand what the affine Grassmannian is and its roles in the representation theory of algebraic groups.
                        \item We shall also have to know what it means to have a group act upon a (nice enough) category so as to be able to define the category of so-called \textbf{spherical D-modules}, which are certain kinds of equivariant D-modules.
                        \item We shall then establish the Geometric Satake Correspondence to be a Tannakian equivalence:
                            $$\Rep^{\heart}_F(\check{G}_{K_{X, x}}) \cong \Sph^{\heart}_{G, X, x}$$
                        between the hearts of the t-structures of the rigid monoidal derived categories of $F$-linear representations of the $K_{X, x}$-points of the Langlands dual group $\check{G}$ and of $G(\scrO_{X, x})$-equivariant/spherical D-modules over the local affine Grassmannian $\Gr_{G, X, x}$.
                    \end{enumerate}
                \item Lastly, we shall seek to understand the subtle technical differences between quasi-coherent sheaves and ind-coherent sheaves, and why restricting ourselves to the case of tori allows us to forego the ind-coherent sheaf machinery. 
            \end{enumerate}
        Of course, before embarking on this journey, we might also want to learn some (derived) algebraic geometry, which will help us understand $\Bun_G(X)$ and $\LocSys_F(X)^{\check{G}}$, what these categories have to do with the theory of Galois representations (because at the end of the day, the Langlands Programme is all about understanding higher reciprocity laws), or even simply why we have required that our curve $X$ is smooth (spoiler: smoothness helps us identify $\QCoh(X)$ with the category $\QCoh(X)^{\perf}$ of perfect complexes on $X$), proper, and geometrically connected, beyond wanting our machineries to be applicable to important classes of examples such as elliptic curves and abelian varieties. For details, see chapters \ref{chapter: schemes} and \ref{chapter: cohomology_and_derived_schemes}.
        
        We should also make some remarks about the above equivalence of categories as well. Thanks to Grothendieck's Galois theory, the left-hand side can be thought of as the \say{\textbf{Automorphic Side}} of the Langlands Correspondence, which holds information about Galois representations. Drawing inspiration from another one of Grothendieck's major contributions, $\ell$-adic \'etale cohomology, the right-hand side in turn can be thought of as the \say{\textbf{Spectral Side}}, which tells interesting stories\footnote{Fairy tales, really...} through harmonic analysis.
    
        \subsection{The Categorical-Geometric Langlands Correspondence for algebraic tori}
            This section, as the title suggests, shall be dedicated to outlining our hopes and dreams (or the lack thereof) for a Categorical-Geometric Langlands Correspondence for algebraic tori; specifically, we would like to present of a list of key results known to be involved in a proof of the Correspondence. We will also give run-down of the various technical tools used for establishing said key results. 
            
            \subsubsection{Equivariant local systems}
        
            \subsubsection{The Fourier-Muka\"i-Laumon Transform}
            
            \subsubsection{Factor-wise Langlands duality}
            
        \subsection{The Conjecture for non-abelian groups}
        
        \subsection{Outline of the proof for the case of \texorpdfstring{$G = \GL_2$}{}}
        
    \section{The Local Correspondence for complex loop groups}
        \subsection{The appearance of Langlands parameters}
            Consider the formal loop group $G(\!(t)\!)$ associated to some chosen connected complex reductive group $G$. 
            
            Let us start by describing the absolute Galois group of the field $\bbC(\!(t)\!)$. First of all, notice that:
                $$\Gal(\bbC(\!(t^{\frac1n})\!)/\bbC(\!(t)\!)) \cong \Z/n\Z$$
            and so:
                $$\Gal(\overline{\bbC(\!(t)\!)}/\bbC(\!(t)\!)) \cong \hat{\Z}$$
            which is a canonical homeomorphism of topological groups obtained via the Fundamental Theorem of Galois Theory. Now, one thing to note is that for some fixed power $q$ of a prime $p$, one also has:
                $$\Gal(\overline{\F_q}/\F_q) \cong \hat{\Z}$$
            but unlike the complex case, the group $\Gal(\overline{\F_q(\!(t)\!)}/\F_q(\!(t)\!))$ surjects (continuously) onto the non-trivial group $\Gal(\overline{\F_q}/\F_q)$ ($\bbC$ is algebraically closed so $\Gal(\bar{\bbC}/\bbC)$ is trivial), a fact known through local class field theory. As a consequence, describing the Weil group (and by extension, Weil-Deligne representations thereof) attached to $\bbC(\!(t)\!)$ will - hopefully - be somewhat simpler than that of $\F_q(\!(t)\!)$ and might therefore help us gain insight into the nature of the Langlands Correspondence. Better yet, we have via Grothendieck's Galois Theory, that:
                $$\Gal(\overline{\bbC(\!(t)\!)}/\bbC(\!(t)\!)) \cong \hat{\Z} \cong \pi_1^{\et}(\bbD^{\x}_{\bbC})$$
            wherein $\bbD^{\x}_{\bbC} \cong \Spec \bbC(\!(t)\!)$; through the discussion above, one sees that this is not the case for $\bbD^{\x}_{\F_q}$, i.e.:
                $$\Gal(\overline{\F_q(\!(t)\!)}/\F_q(\!(t)\!)) \not \cong \pi_1^{\et}(\bbD^{\x}_{\F_q})$$
            Since representations of the (\'etale) fundamental group correspond to certain D-modules, we essentially have access to the theory of D-modules in studying the Langlands Correspondence for the case of $G\!(t)\!)$, which roughly postulates a bijective relationship between homomorphisms $W_{\bbC(\!(t)\!)} \to \check{G}$ and certain representations of $G(\!(t)\!)$.
        
        \subsection{Representations of loop groups; Kac-Moody algebras}
    
    \section{Deformation quantisation of the Local Correspondence}
        
        \chapter{Algebraic groups}
    \begin{abstract}
        
    \end{abstract}
    
    \minitoc
    
    \section{The geometry of algebraic groups}
        \subsection{Introducing algebraic groups} \label{subsubsection: algebraic_groups}
            \subsubsection{General properties of group schemes}
                \begin{definition}[Algebraic groups] \label{def: algebraic_groups} \index{Algebraic groups}
                    Let $S$ be a base scheme.
                    \begin{enumerate}
                        \item A \textbf{group scheme over $S$} is a group object in the cartesian closed category $\Sch_{/S}$ of schemes over $S$. 
                        \item In the event that $S$ is isomorphic to the spectrum of some field $k$, a group scheme $G$ over $\Spec k$ will be called a \textbf{(locally) algebraic group scheme} if and only if it is (locally) of finite type over $\Spec k$ (i.e. if and only if it is (locally) algebraic as a scheme over $\Spec k$).
                        \item An abstract group $S$-scheme:
                            $$\pi: G \to S$$
                        is called \textbf{geometrically (locally) algebraic} if and only if over each point:
                            $$s: \Spec \kappa_s \to S$$
                        with corresponding residue field $\kappa_s$, and for each field extension $K_s/\kappa_s$ corrsponding to a morphism:
                            $$i: \Spec K_s \to \Spec \kappa_s$$
                        of affine schemes, the canonical projection:
                            $$\pr_2: \left(G \x_{\pi, S, s} \Spec \kappa_s\right) \x_{\pr_2, \Spec \kappa_s, i} \Spec K_s \to \Spec K_s$$
                        is (locally) of finite type.
                    \end{enumerate}
                \end{definition}
                \begin{example}
                    \noindent
                    \begin{enumerate}
                        \item \textbf{(General linear groups)} Traditionally, one thinks of general linear groups as groups of automorphisms of vector spaces of a given (not necessarily finite) dimension; so for instance, $\GL_n(\R)$ is the group of all automorphisms of $n$-dimensional real vector spaces, i.e. its elements are $n \x n$ invertible matrices with real coefficients. However, there is another way. By squinting a bit, we can see that for each natural number $n$, the presheaf:
                            $$\GL_n: (\Cring^{\op})^{\op} \to \Sets$$
                        is represented by the affine scheme $\Spec \Z\left[\{x_{ij}\}_{1 \leq i, j \leq n}, \frac{1}{\det}\right]$, since an $n \x n$ matrix has $n^2$ free entries, which are only subjected to the requirement that the determinant (which can be viewed as a polynomial in the $n^2$ variables $x_{11}, ..., x_{nn}$) is invertible (hence the localisation at $\det$); incidentally, this also ensures that $\GL_n$ is a Zariski sheaf (cf. remark \ref{remark: affine_schemes_are_zariski_sheaves}), and hence an affine scheme (cf. definition \ref{def: zariski_topoi}). It remains to check whether or not $\GL_n$ is a group object. We claim that $\GL_n$, as it ought to be, is a group, and that its structure is determined by the following multiplication:
                            $$
                                \begin{aligned}
                                    \GL_n & \leftarrow \GL_n \x \GL_n
                                    \\
                                    \Spec \Z\left[\{x_{ij}\}_{1 \leq i, j \leq n}, \frac{1}{\det}\right] & \leftarrow \Spec \Z\left[\{x_{ij}\}_{1 \leq i, j \leq n}, \frac{1}{\det}\right] \x \Spec \Z\left[\{x_{ij}\}_{1 \leq i, j \leq n}, \frac{1}{\det}\right]
                                    \\
                                    \Z\left[\{x_{ij}\}_{1 \leq i, j \leq n}, \frac{1}{\det}\right] & \to \Z\left[\{x_{ij}\}_{1 \leq i, j \leq n}, \frac{1}{\det}\right] \tensor_{\Z} \Z\left[\{x_{ij}\}_{1 \leq i, j \leq n}, \frac{1}{\det}\right]
                                    \\
                                    x_{ij} & \mapsto \sum_{k=1}^n x_{ik} \tensor x_{kj}
                                \end{aligned}
                            $$
                        It is not hard to see that $\GL_n$ is a scheme of finite type, and hence it is an algebraic group, per definition \ref{def: algebraic_groups}.
                        
                        When $n = 1$, we recover the multiplicative group scheme $\G_m$.
                        \item \textbf{(Special linear groups):} Via the above discussion surrounding $\GL_n$, let us define $\SL_n$ to be the Zariski-closed subgroup scheme represented by the affine scheme $\Spec \frac{\Z\left[\{x_{ij}\}_{1 \leq i,j \leq n}, \frac{1}{\det}\right]}{(\det - 1)}$: indeed, special linear groups are, by definition, subgroups of generali linear groups consisting of matrices of determinant $1$. As we shall see, this means two things: that $\SL_n$ is an algebraic group, and as a consequence of this, that $\SL_n$ is smooth over fields. 
                        \item \textbf{(Roots of unity):} Let:
                            $$\mu_n: (\Cring^{\op})^{\op} \to \Sets$$
                        be the presheaf on $\Cring^{\op}$ that is represented by the affine scheme $\Spec \Z[z]/(z^n - 1)$. Its group structure is given by:
                            $$
                                \begin{aligned}
                                    \mu_n & \leftarrow \mu_n \x \mu_n
                                    \\
                                    \Spec \Spec \Z[z]/(z^n - 1) & \leftarrow \Spec \Z[z]/(z^n - 1) \x \Spec \Z[z]/(z^n - 1)
                                    \\
                                    \Z[z]/(z^n - 1) & \to \Z[z]/(z^n - 1) \tensor_{\Z} \Z[z]/(z^n - 1)
                                    \\
                                    \zeta & \mapsto \zeta \tensor \zeta
                                \end{aligned}
                            $$
                        and because it is of finite type, it is also algebraic, just like $\GL_n$ and $\SL_n$. When $n = p^r$, one gets the group $\mu_{p^r}$ of $p^r$-th roots. 
                    \end{enumerate}
                \end{example}
                
                \begin{proposition}[The conormal sheaf associated to the identity] \label{prop: conormal_sheaf_of_the_identity}
                    \noindent
                    \begin{enumerate}
                        \item \textbf{(The identity is a closed immersion):} Let $S$ be a base scheme and let $G$ be a group scheme over $S$. Then, $G$ is separated (respectively quasi-separated) if and only if the identity:
                            $$e: S \to G$$
                        is a closed immersion (respectively quasi-compact). 
                        \item \textbf{(The conormal sheaf associated to the identity):} Since this is good a place as any to state the definition of the conormal sheaf associated to a closed immersion of schemes, let us write it out first.
                        
                        Let $X$ be a scheme and let $\calI$ be a $\calO_X$-ideal. Then, the \textbf{conormal sheaf} or \textbf{conormal module} associated to the canonical closed immersion:
                            $$\Spec \calO_X/\calI \hookrightarrow X$$
                        is just the quasi-coherent $\calO_X$-module $\calI/\calI^2$, which we note to actually be a quasi-coherent $\calO_Z$-module (see proposition \ref{prop: Zariski_tangent_spaces_are_vector_spaces} for a proof instruction). We shall denote this by $\calN^{\vee}_{Z/X}$.
                        
                        As the identity:
                            $$e: S \to G$$
                        of a separated group scheme:
                            $$\pi: G \to S$$
                        is a closed immersion, its conormal module $\calN^{\vee}_{S/G}$ is well-defined. Furthermore, one has:
                            $$\Omega^1_{G/S} \cong \pi^* \calN^{\vee}_{S/G} \cong \pi^* e^* \Omega^1_{G/S}$$
                        In particular, should $S$ be the spectrum of a field (i.e. if one can recognise the identity $e$ has a rational point of $G$), then $\Omega^1_{G/S}$ is free as an $\calO_G$-module.   
                    \end{enumerate}
                \end{proposition}
                    \begin{proof}
                        \noindent
                        \begin{enumerate}
                            \item \textbf{(The identity is a closed immersion):}
                            \item \textbf{(The conormal sheaf associated to the identity):}
                        \end{enumerate}
                    \end{proof}
                    
            \subsubsection{Properties of group schemes over a field}
                
            \subsubsection{Properties of algebraic groups}
                \begin{theorem}[Cartier's Smoothness Theorem] \label{theorem: algebraic_groups_over_characteristic_0_are_smooth} \index{Algebraic groups! are smooth over characteristic $0$}
                    Let $k$ be a ground field of characteristic $0$. Then, any locally algebraic group over $\Spec k$ is smooth.
                \end{theorem}
                    \begin{proof}
                        Firstly, because smoothness is a local property (cf. definition \ref{def: standard_smoothness}), we can assume that $G$ is affine. This, in particular, implies that $G$ is of finite type instead of being merely \textit{locally} of finite type. Now, we will attempt to show that $G$ is of finite presentation, instead of being merely of finite type, and formally smooth. 
                    \end{proof}
                    
                Algebraic groups over positive characteristics $p$ are, however, not so nice. Consider, for instance, the algebraic group $(\mu_p)_{/k}$ of $p^{th}$ roots of unity over some field $k$ of characteristic $p$ that is \textit{not} perfect. This algebraic group is represented by the affine scheme $\Spec \frac{k[x]}{(x^p - 1)}$; the associated Jacobian is thus visibly singular, and $(\mu_p)_{/k}$ is therefore not smooth. Due to this pathology, we will need to figure out the necessary and sufficient conditions for algebraic groups to be smooth over positive characteristics.
                \begin{theorem}[Cartier's Smoothness Theorem over positive characteristics] \label{theorem: smoothness_of_algebraic_groups_over_characteristic_p} \index{Algebraic groups! are smooth over characteristic $p$}
                    Let $k$ be a \textit{perfect} ground field of some prime characteristic $p$ and let $G$ be a \textit{reduced} locally algebraic group scheme over $\Spec k$. Then, $G$ is smooth over $\Spec k$.
                \end{theorem}
                    \begin{proof}
                        
                    \end{proof}
                \begin{corollary}[Algebraic groups over algebraically closed fields are smooth]
                    In particular, reduced group schemes over algebraically closed ground fields are always smooth (note how schemes are \textit{a priori} reduced in characteristic $0$).
                \end{corollary}
                \begin{remark}
                    Every assumption in theorem \ref{theorem: smoothness_of_algebraic_groups_over_characteristic_p} is crucial. If $k$ is a \textit{non-perfect} field of characteristic $p$ and $a \in k$ is not a $p^{th}$ power, then for instance, the closed subgroup scheme $\Spec \frac{k[x, y]}{(x^p + ay^p)}$ of the additive group scheme $\Spec k[x, y]$ is reduced but not smooth.
                \end{remark}
                
                Algebraic groups over positive characteristics are not wholefully pathological, though. Their geometries are somewhat intuitive.
                \begin{proposition}[Algebraic groups are equidimensional] \label{prop: algebraic_groups_are_locally_equidimensional} \index{Algebraic groups! are equidimensional}
                    Let $k$ be an arbitrary field and let $G$ be a locally algebraic group over $\Spec k$. Then, $G$ is locally equidimensional, in the sense that:
                        $$\forall g \in G: \dim G = \dim_g G$$
                    where $\dim_g G := \inf\{\dim U \in \Ouv_{\Zar}(G) \mid U \ni g\}$; in particular, if $g \in |G|$ is a closed point, then:
                        $$\dim G = \dim \calO_{G, g}$$
                \end{proposition}
                    \begin{proof}
                        
                    \end{proof}
                \begin{corollary}
                    Integral algebraic groups (such as group varieties) are equidimensional, in the sense that even over the their generic fibre, say $\eta$, one has:
                        $$\dim G = \dim \calO_{G, \eta}$$
                \end{corollary}
                
                \begin{proposition}[Algebraic groups are quasi-projective] \label{prop: algebraic_groups_are_quasi_projective} \index{Algebraic groups! are quasi-projective}
                    Let $k$ be an \textit{arbitrary} ground field and let $G$ be an algebraic group over $\Spec k$. Then, $G$ is quasi-projective as a scheme over $\Spec k$.
                \end{proposition}
                    \begin{proof}
                        
                    \end{proof}
                
                \begin{proposition}
                    If $k$ is an algebraically closed ground field (which we shall allow to be of arbitrary characteristic), then for $G$ a locally algebraic group over $\Spec k$ and $g_1, ..., g_n \in G(k)$ a finite number of $k$-rational points thereof, there exists an affine open subscheme $U$ of $G$ that contains the points $g_1, ..., g_n$; this is to say, there exist an affine open subscheme $U$ of $G$ such that all the following liftings exist:
                        $$
                            \begin{tikzcd}
                            	& U \\
                            	{\Spec k} & G
                            	\arrow[hook, from=1-2, to=2-2]
                            	\arrow["{g_i}", from=2-1, to=2-2]
                            	\arrow["{g_i|_U}", dashed, from=2-1, to=1-2]
                            \end{tikzcd}
                        $$
                \end{proposition}
                    \begin{proof}
                        Without loss of generality, we may assume that $G$ is connected, or atleast, that the points $g_1, ..., g_n$ are all on the same connected component of $G$.   
                    \end{proof}
                    
        \subsection{Some useful results}
            \subsubsection{Actions, quotients, and Chevalley's Theorem}
                \begin{proposition}[Quotients by algebraic groups] \label{prop: quotients_by_algebraic_groups}
                    Let $k$ be a field and let $G$ be an algebraic group over $\Spec k$. Additionally, fix a Zariski-closed subgroup $H \leq G$ (which we note to also be an algebraic group over $\Spec k$). Then:
                        \begin{enumerate}
                            \item the quotient fppf-sheaf $G/H$ carries the structure of a scheme over $\Spec k$,
                            \item $G/H$ is a locally algebraic group scheme if and only if $H$ is a normal subgroup of $G$,
                            \item the quotient map $G \to G/H$ is a $H$-invariant faithfully flat morphism of schemes over $\Spec k$.
                        \end{enumerate}
                \end{proposition}
                    \begin{proof}
                        \noindent
                        \begin{enumerate}
                            \item First of all, since the fppf topos $(\Spec k)_{\fppf}$ is cocomplete, the quotient sheaf $G/H$ must exist. Second of all, because group scheme actions define fppf-equivalence relations, it is not hard to see that $G/H$ is an fppf-algebraic space. 
                            \item 
                            \item 
                        \end{enumerate}
                    \end{proof}
                
                \begin{theorem}[Chevalley's Decomposition Theorem] \label{theorem: chevalley_decomposition_theorem}
                    This is also known as \textbf{Chevalley's Structure Theorem} and \textbf{Chevelley's theorem on algebraic groups}.
                
                    Let $k$ be a perfect field and let $G$ be an algebraic group over $\Spec k$. Then, $G$ decomposes into a \textit{normal} (and necessarily closed) linear algebraic subgroup $H$ and an abelian variety $A$ via the following short exact sequence of algebraic groups over $\Spec k$:
                        $$1 \to H \to G \to A \to 1$$
                \end{theorem}
                \begin{remark}[Short exact sequences of algebraic groups ?] \label{remark: short_exact_sequences_algebraic_groups}
                    Thanks to proposition \ref{prop: quotients_by_algebraic_groups}, the notion of short exact sequences makes sense for algebraic groups. In particular, for any algebraic group $G$, one defines the cokernel of a closed immersion $H \trianglelefteq G$ of a normal algebraic subgroup $H$ to precisely be the canonical quotient map $G \to G/H$. Then, one can define the kernel of this quotient map simply as the canonical pullback $G \x_{G/H} 1$, which is possible thanks to the left-exactness of the Yoneda embedding. 
                    
                    One thing to note is that these notions of (co)kernels do not differ from those within (pre-)abelian categories, at least from a categorical standpoint. This is because the category of (locally) algebraic groups over a fixed base scheme admits a zero object, namely that very base scheme. 
                \end{remark}
                    \begin{proof}[Proof of theorem \ref{theorem: chevalley_decomposition_theorem}]
                        
                    \end{proof}
            
            \subsubsection{Grothendieck's theorem on tori}
            
        \subsection{The geometry of torsors}
            This subsection shall be dedicated to the study of $\Bun_G(X)$, the algebraic stack of principal $G$-bundles (with $G$ some affine algebraic group) over a smooth, projective, and geometrically connected curve $X$ over some field $k$ of charactersitic $0$. 
            
            \subsubsection{The geometry of \texorpdfstring{$\Bun_G(X)$}{}}
                \begin{definition}[$\Bun_G$] \label{def: Bun_G}
                    By $\Bun_G(X)$, we will mean the moduli space of $G$ bundles (cf. definition \ref{def: moduli_space}), i.e. the $(2, 1)$-sheaf on $(\Sch_{/\Spec k})_{\fppf}$ given - for each $S \in \Sch_{/\Spec k}$ - via $\Bun_G(X)(S)$ being the core of the category of $G$-bundles on $X \x_{/\Spec k} S$.
                \end{definition}
                
                Since every affine algebraic groups are closed subschemes of $\GL_n$, let us first try to understand $\Bun_{\GL_n}(X)$.
                \begin{proposition}[The geometry of $\Bun_{\GL_n}(X)$] \label{prop: geometry_of_Bun_GLn}
                    $\Bun_{\GL_n}(X)$ is a smooth algebraic stack that is locally of finite type and with diagonal representable (cf. definition \ref{def: affine_schematic}) by an affine scheme.
                \end{proposition}
                    \begin{proof}
                        
                    \end{proof}
                \begin{proposition}[The geometry of $\Bun_G(X)$] \label{prop: geometry_of_Bun_G}
                    $\Bun_G(X)$ is a smooth algebraic stack that is locally of finite type and with diagonal representable (cf. definition \ref{def: affine_schematic}) by an affine scheme.
                \end{proposition}
                    \begin{proof}
                        
                    \end{proof}
                    
                \begin{proposition}[Dimension of $\Bun_G(X)$] \label{prop: dimension_of_Bun_G}
                    If $G$ is reductive (cf. definition \ref{def: tori_and_reductive_groups}) and $X$ has genus $g$ then:
                        $$\dim \Bun_G(X) = (g - 1)\dim G$$
                \end{proposition}
                    \begin{proof}
                                
                    \end{proof}
            
            \subsubsection{Weil Uniformisation}
                \begin{convention}
                    Denote the function field of $X$ by $K_X$, its ring of integers by $\scrO_X$, and the corresponding ring of ad\`eles by $\scrA_X$. 
                \end{convention}
                
                \begin{theorem}[Weil Uniformisation] \label{theorem: weil_uniformisation}
                    There is a canonical equivalence of stacks:
                        $$G(K_X) \backslash G(\scrA_X) / G(\scrO_X) \cong \Bun_G(X)$$
                \end{theorem}
                    \begin{proof}
                        
                    \end{proof}
    
    \section{Reductive groups}
        \subsection{Generalities}
            \begin{definition}[Unipotent groups] \label{def: unipotent_groups}
                Let $k$ be an algebraically closed field and let $G$ be a linear algebraic group over $\Spec k$. Also, fix a natural number $n$. $G$ is said to be \textbf{unipotent} if and only if it is a Zariski-closed subgroup of the unipotent group:
                    $$(\opU_n)_{/k} \cong \Spec \frac{k[x_{11}, x_{12}, ..., x_{nn}]}{(\forall 1 \leq j \leq i \leq n: x_{i = j} = 1, x_{i > j} = 0)}$$
                of upper-triangular $n \x n$ matrices with entries in $k$.
            \end{definition}
            \begin{remark}[Unipotent groups are linear algebraic groups] \label{remark: unipotent_groups_are_linear_algebraic_groups}
                \noindent
                \begin{enumerate}
                    \item \textbf{(Embedding into $\GL_n$):} Since the determinants of upper triangular matrices are just the products of the diagonal entries, and since $1$ is always a unit (which implies that $(\opU_n)_{/k} \cong (\opU_n)_{/k}\left[\frac{1}{\det}\right]$), $(\opU_n){/k}$ is indeed a Zariski-closed subgroup of $(\GL_n)_{/k} \cong \Spec k[x_{11}, x_{12}, ..., x_{nn}]\left[\frac{1}{\det}\right]$. 
                    \item \textbf{(Elements of unipotent groups are unipotent):} One can show that the eigenvalues of any upper-triangular matrix are all equal to $1$, and hence these matrices are unipotent by definition (recall that an element $r \in R$ of a ring $R$ - which need not be commtuative - is unipotent if and only if $r - 1$ is nilpotent); to show why the eigenvalues all being $1$ is a sufficient and necessary condition, simply apply the Jordan Canonical Decomposition. Elements of unipotent groups are thus indeed unipotent.
                \end{enumerate}
            \end{remark}
            \begin{definition}[Radicals] \label{def: radicals_of_algebraic_groups}
                Let $k$ be an algebraically closed field and let $G$ be a connected smooth algebraic group over $\Spec k$. 
                    \begin{enumerate}
                        \item \textbf{(Solvability):} $G$ is said to be \textbf{solvable} if its group of $k$-rational points $G(k)$ is solvable, i.e. if the \href{https://ncatlab.org/nlab/show/derived+series}{\underline{derived series}} of $G(k)$:
                            $$G(k) \geq G(k)^{(1)} \geq ... $$
                        terminates at the trivial group after \textit{finitely} many steps. 
                        \item \textbf{(Radicals):} The \textbf{radical} of $G$, denoted by $\scrR_G$, is the maximal normal solvable algebraic subgroup of $G$. 
                    \end{enumerate}
            \end{definition}
            \begin{remark}[Unipotent radicals] \label{remark: unipotent_radicals}
                Let $k$ be an algebraically closed field and let $G$ be a linear algebraic group over $\Spec k$, and note that within such a setting, the radical $\scrR_G$ is necessarily affine too. Then, the \textbf{unipotent radical} of $G$ (denoted by $\U_G$) is maximal unipotent subgroup of $\scrR_G$; at the level of $k$-rational points, one also has:
                    $$\U_G(k) = \scrR_G(k) \cap U_G(k)$$
                where $U_G \trianglelefteq G$ is the maximal unipotent \textit{normal} subgroup of $G$.
            \end{remark}
            
            \begin{definition}[Tori and reductive groups] \label{def: tori_and_reductive_groups}
                Fix a base scheme $S$. We then have the following hierachy of (affine) algebraic groups:
                \begin{enumerate}
                    \item \textbf{(Algebraic tori):} An \textbf{algebraic $S$-torus} is a \textit{finite} product of $(\G_m)_{/S}$, the base change of $\G_m$ (understood to be over $\Spec \Z$) to $S$.
                    \item \textbf{(Reductive and semi-simple algebraic groups):} 
                        \begin{enumerate}
                            \item \textbf{(Semi-simple groups):} A linear algebraic $S$-group $G$ is \textbf{semi-simple} if and only if its geometric fibres $G_{\bar{x}}$ are semi-simple, which is to say, their radicals $\scrR_{G_{\bar{x}}}$ are all trivial. 
                            \item \textbf{(Reductive groups):} A linear algebraic $S$-group $G$ is \textbf{reductive} if and only if its geometric fibres $G_{\bar{x}}$ are reductive, which is to say, their unipotent radicals $\U_{G_{\bar{x}}}$ are all trivial.
                        \end{enumerate}
                \end{enumerate}
            \end{definition}
            \begin{remark} \label{remark: tori_and_reductive_groups}
                Let $S$ be a base scheme and let $G$ be a linear algebraic group over $S$.
                \begin{enumerate}
                    \item \textbf{(Tori are reductive):} First of all, since tori are abelian, their derived series are trivial and they are therefore automatically solvable. As a consequence, the radical of a torus is just the whole group, and thus any torus is semi-simple. This also implies that tori are reductive, as the unipotent radical - being the intersection of the radical and the maximal unipotent subgroup - must be trivial if the radical is trivial.
                    
                    Non-trivial tori, however, are not unipotent, since diagonal matrices are not necessarily the identity.
                    \item \textbf{(Semi-simple groups are reductive):} This is easy, as the radical being trivial obviously implies that the unipotent radical is trivial too (this is the other way to show that tori are reductive). 
                    
                    The converse statement is not true in general. For instance, unipotent groups are not reductive, as the unipotent radical of one such group is just the entire group.
                \end{enumerate}
            \end{remark}
                    
        \subsection{Special subgroups of reductive groups; flag varieties}
            
        \subsection{Kempf's Vanishing Theorem over \texorpdfstring{$\Z$}{}}
            \begin{convention}
                Throughout this subsection, we work with the following objects over a fixed perfect base field $k$:
                    \begin{itemize}
                        \item a semi-simple and simply connected algebraic group $G$, a maximal torus $T$ contained inside some Borel subgroup $B \leq G$,
                        \item the flag variety $\pi: G/B \to \Spec k$
                        \item $\bbX(T)$ the weight lattice/character group,
                        \item the root lattice $\Phi$ coroot lattice $\check{\Phi}$,
                        \item if $S_B$ denotes the set of simple roots relative to a choice of Borel subgroup $B$ which contains the fixed maximal torus $T$, then we shall write $P_{\alpha}$ for the minimal parabolic subgroup of $G$ attached to $\alpha$.
                    \end{itemize}
            \end{convention}
        
            \subsubsection{Semi-orthogonal decomposition for flag varieties}
                \begin{proposition}
                    
                \end{proposition}
                    \begin{proof}
                        
                    \end{proof}
            
            \subsubsection{The Steinberg line bundle}
        
        \subsection{Split reductive groups}
        
    \section{The Geometric Satake Equivalence and Langlands Duality}
        \subsection{The abelian version}
            \subsubsection{Beauville-Laszlo Uniformisation and the affine Grassmannian}
                \paragraph{Vector bundles on the formal disc and Beauville-Laszlo Uniformisation}
                    \begin{convention}
                        For now, by \say{vector bundles} on a locally Noetherian scheme $X$, we shall mean \say{locally free coherent sheaves}, which \textit{a priori} form a category - denoted by $\Vect(X)$ - equivalent to the full subcategory of $\Coh(X)$ spanned by projective objects. 
                    \end{convention}
                    \begin{remark}
                        Over every locally Noetherian scheme $X$, the category of vector bundles $\Vect(X)$ is not abelian. However, it is \href{https://ncatlab.org/nlab/show/exact+category}{\underline{exact}}.
                    \end{remark}
                    
                    \begin{convention}[Formal discs]
                        For any commutative ring $R$, we denote by $\bbD_R$ the formal disc $\Spec R[\![t]\!]$ and by $\bbD_R^{\circ}$ the punctured formal disc $\Spec R(\!(t)\!)$.
                    \end{convention}
                    
                    \begin{lemma}[Vector bundles on the formal disc] \label{lemma: vector_bundles_on_the_formal_disc}
                        Let $R$ be a Noetherian commutative ring. Then, one has the following $1$-limit of locally small categories:
                            $$\Vect(\bbD_R) \cong \underset{n \in \N}{\lim} \Vect(\Spec R[t]/t^{n + 1})$$
                    \end{lemma}
                        \begin{proof}
                            
                        \end{proof}
                
                \paragraph{The affine Grassmannian}
                    \begin{definition}[Local affine Grassmannians] \label{def: local_affine_grassmannians}
                        For any algebraic group $G$ over some base commutative ring $k$, there is a canonically defined moduli space, denoted by $\Gr_G^{\loc}$ and called the \textbf{\textit{local} affine Grassmannian} attached to $G$. It is the prestack which assigns to each $\Spec R \in \Sch_{/\Spec k}^{\aff}$ the groupoid that is the core of the category of $G$-torsors on $\bbD_R$ that trivialise over $\bbD_R^{\circ}$.  
                    \end{definition}
                    \begin{remark}
                        \noindent
                        \begin{itemize}
                            \item It is rather easy to see that $\Gr_G^{\loc}$ satisfies fppf descent and hence a stack in the fppf topology.
                            \item Furthermore, by introducing the so-called loop and arc groups $G[\![t]\!]$ and $G(\!(t)\!)$, defined by $G[\![t]\!](R) \cong G(R[\![t]\!])$ and $G(\!(t)\!)(R) \cong G(R(\!(t)\!))$ respectively, one can show that:
                                $$\Gr_G^{\loc} \cong G(\!(t)\!)/G[\![t]\!]$$
                            using the fact that $G(\!(t)\!)$ acts on $\Gr_G^{\loc}$ by changing the trivialisation. This is an important description of the affine Grassmannian, so let us state and prove it properly (cf. proposition \ref{prop: affine_grassmannians_as_coset_spaces}).
                        \end{itemize}
                    \end{remark}
                    
                    \begin{convention}
                        Let $k$ be a commutative ring and let $R$ be a commutative $k$-algebra. Then, for any scheme $X \in \Sch_{/\Spec k}$, we shall write $X_R$ for the base-change $X \x_{\Spec k} \Spec R$.
                    \end{convention}
                    \begin{definition}[Global affine Grassmannians] \label{def: global_affine_grassmannians}
                        Let $k$ be a field, $X$ be an algebraic curve over $\Spec k$, and $G$ be an algebraic group over $\Spec k$. Additionally, fix a $k$-rational point $x \in |X|$. Then, we define the \textbf{global affine Grassmannian} or \textbf{Hecke stack}\footnote{A terminology we shall reserve for later.} over $X$ attached to $G$, denoted by $\Gr_{G, X}^{\glob}$, to be the prestack that assigns to each $\Spec R \in \Sch_{/\Spec k}^{\aff}$ the core of the category of $G$-torsors on $X_R$ which trivialise over $(X \setminus \{x\})_R$.
                    \end{definition}
                    \begin{remark}[Local-to-global compatibility] \label{remark: local_to_global_for_affine_grassmannians}
                        Let $k$ be a field, $X$ be an algebraic curve over $\Spec k$, and $G$ be an algebraic group over $\Spec k$. Also, fix a $k$-rational point $x \in |X|$ and denote by $\Gr_{G, X, x}^{\loc}$ the local affine $G$-Grassmannian over the formal disc $\bbD$ around $x$ (cf. definition \ref{def: local_affine_grassmannians}). Then, via Beauville-Laszlo Uniformisation, one can easily show that:
                            $$\Gr_{G, X}^{\glob} \cong \Gr_{G, X, x}^{\loc}$$
                    \end{remark}
                    \begin{convention}
                        From now on, we shall write $\Gr_G$ instead of $\Gr_G^{\loc}$.
                    \end{convention}
                    
                    \begin{proposition}[Structure of affine Grassmannians] \label{prop: structure_of_affine_grassmannian}
                        Let $G$ be an algebraic group. Then:
                            \begin{enumerate}
                                \item $\Gr_G$ is an ind-scheme of ind-finite type, and
                                \item when $G$ is reductive over some field $k$, $\Gr_G$ will be ind-projective.
                            \end{enumerate}
                    \end{proposition}
                        \begin{proof}
                            \noindent
                            \begin{enumerate}
                                \item 
                                \item 
                            \end{enumerate}
                        \end{proof}
                        
                    \begin{proposition}[Affine Grassmannians as coset spaces] \label{prop: affine_grassmannians_as_coset_spaces}
                        Let $G$ be an algebraic group. Then, $\Gr_G \cong G(\!(t)\!)/G[\![t]\!]$.
                    \end{proposition}
                        \begin{proof}
                            
                        \end{proof}
                    \begin{corollary}
                        Let $k$ be a field, $X$ be an algebraic curve over $\Spec k$, and $G$ be an algebraic group over $\Spec k$. Also, fix a $k$-rational point $x \in |X|$. Then $\Gr_{G, X}^{\glob} \cong G(\!(t)\!)/G[\![t]\!]$.
                    \end{corollary}
                    \begin{corollary}[Loop group action on Grassmannians] \label{coro: loop_group_action_on_grassmannians}
                        There is a $G[\![t]\!]$-action on $\Gr_G$, which means that one can now meaningfully discuss equivariance of sheaves on the affine Grassmannian $\Gr_G$. In particular, we shall be interested in $G[\![t]\!]$-equivariant (i.e. \say{spherical}) D-modules on $\Gr_G$.
                    \end{corollary}
                    
                \paragraph{The Hecke stack and the factorisable Grassmannian}
                    \begin{definition}[The Hecke (pre)stack] \label{def: hecke_stack}
                        Let $k$ be a field, $X$ be an algebraic curve over $\Spec k$, and $G$ be an algebraic group over $\Spec k$. Additionally, fix a finite set of $k$-rational points $x_1, ..., x_n \in |X|$, which we shall view as a $k$-rational point $\vec{x} \in |X^n|$. Then, we define the \textbf{Hecke stack} over $X$ attached to $G$, denoted by $\Hecke_{G, X^n}$, to be the prestack that assigns to each $\Spec R \in \Sch_{/\Spec k}^{\aff}$ the core of the category of $G$-torsors on $X_R$ which trivialise over $(X \setminus \{x_1, ..., x_n\})_R$. 
                        
                        Alternatively, one defines $\Hecke_{G, X^n}$ as the prestack that assigns to each $\Spec R \in \Sch_{/\Spec k}^{\aff}$ the core of the category of $G$-torsors (\textit{not $G^n$-torsors!}) on $X^n_R$ which trivialise over $(X^n \setminus \{\vec{x}\})_R$. 
                    \end{definition}
                    \begin{remark}
                        It is an immediate consequence of the definitions that:
                            $$\Hecke_{G, X^1} \cong \Gr_{G, X}^{\glob}$$
                        and via remark \ref{remark: local_to_global_for_affine_grassmannians}, one also has the following equivalence of fppf-stacks:
                            $$\Hecke_{G, X^1} \cong \Gr_G$$
                    \end{remark}
            
            \subsubsection{Convolutions of spherical D-modules, the fusion product, and the rigid-monoidal structure on spherical D-modules}
                \begin{convention}
                    From now on, our ground field $k$ shall be of characteristic $0$.
                \end{convention}
                
                \paragraph{Convolution of equivariant sheaves on the affine Grassmannian and properties thereof}
                    \begin{definition}[Spherical D-modules] \label{def: spherical_D_modules}
                        Let $k$ be a field, $X$ be an algebraic curve over $\Spec k$, and $G$ be an algebraic group over $\Spec k$. Additionally, fix a finite set of $k$-rational points $x_1, ..., x_n \in |X|$. Then, the category of $G[\![t]\!]$-equivariant D-modules on the Hecke stack $\Hecke_{G, X^n}$, which we shall denote by $\Sph_{G, X^n}$, shall be called the $n^{th}$ category of \textbf{$G$-spherical D-modules} on $X$.
                    \end{definition}
                    \begin{remark}
                        Since the ground field $k$ is of characteristic $0$ and because $\Gr_G \cong \Hecke_{G, X^1}$ is an ind-scheme of ind-finite type, the category $\Dmod(\Gr_G)$ is well-defined. The category $\Sph_{G, X^1} \cong \Dmod(\Gr_G)^{G[\![t]\!]}$ can then be obtained via consideration of the natural $G[\![t]\!]$-action on $\Gr_G$. 
                                
                        We shall discuss the well-definiteness of $\Sph_{G, X^n}$ for $n > 1$ later.
                    \end{remark}
                    \begin{convention}
                        Thanks to the local-to-global compatibility of the affine Grassmannian established in remark \ref{remark: local_to_global_for_affine_grassmannians}, we shall henceforth be abbreviating $\Sph_{G, X^1}$ down to $\Sph_G$.
                    \end{convention}
                    
                \paragraph{Commutativity}
            
            \subsubsection{Hecke eigensheaves}
            
            \subsubsection{Fibre functors and weights}
            
            \subsubsection{Langlands Duality}
            
        \subsection{The quantised derived version}
        
        \chapter{Lie theory and formal geometry} \label{chapter: formal_geometry}
    \begin{abstract}
        So, what is \say{formal geometry} within the derived context ? Well, within the context of this book, the term shall mean the study of objects of the category $[\Spec k]^{\laft, \defm}$ of prestacks on ${}^{k/}\Comm\Alg^{\op}$ that are locally almost of finite type and admit deformations - which we note to be a full subcategory of $[\Spec k]^{\laft}$ of prestacks that are locally of finite type on ${}^{k/}\Comm\Alg^{\op}$ - along with nil-isomorphisms between them. As we shall see, this category is a rather natural seting in which one may consider the following geometrically-motivated phenomena:
            \begin{enumerate}
                \item Quotients by equivalence relations induced by actions of internal groupoids.
                \item An algebro-geometric analogue of the Lie Group-Lie Algebra Correspondence, which particularly, is a correspondence between groups over a give base prestack $\calX$ and Lie algebras internal to the symmetric monoidal category $\Ind\Coh(\calX)$ of ind-coherent sheaves on $\calX$. 
                \item Universal enveloping algebras of Lie algebras and Lie algebroids, and other differential-geometric entities such as Hodge filtrations/de Rham resolutions.
                \item D-modules and the Riemann-Hilbert Correspondence. 
            \end{enumerate}
        We realise that this chapter leans more towards geometric representation theory rather than commutative algebra, but a glimpse of the true strength of commutative algebraic geometry should not be to anyone's detriment.
    \end{abstract}
    
    \minitoc
    
    \begin{convention}[Everything is derived!] \label{conv: formal_geometry_everything_is_derived}
        \noindent
        \begin{itemize}
            \item From now on until the end of the chapter, everything will be assumed to be derived. 
            \item By $1\-\Cat_1$, or simply $1\-\Cat$, we shall actually mean $(\infty, 1)\-\Cat_1$, i.e. the $(\infty, 1)$-category of $(\infty, 1)$-categories and functors between them, and by $1\-\Cat_2$ we will be referring to the $(\infty, 2)$-category of $(\infty, 1)$-categories, functors between them, and natural transformations between these functors. 
            
            Similarly, by $\Grpd^1$, or simply $\Grpd$, we will actually mean the $(\infty, 1)$-category of $\infty$-groupoids and functors between them, and by $\Grpd^2$, we shall mean the $(\infty, 2)$-category of $\infty$-groupoids, functors between them, and natural transformations between these functors.
            \item A subcategory of $1\-\Cat$ this is of particular interest is $\dg\Cat^{\cont}_2$ (or simply $\dg\Cat^{\cont}$), the $(\infty, 2)$-category of stable linear (i.e. differential-graded) $(\infty, 1)$-categories (see section \ref{section: homological_algebra} for the notion of stable $(\infty, 1)$-categories). Of course, we can also view $\dg\Cat^{\cont}$ as a mere $(\infty, 1)$-category; when necessary, we shall write $\dg\Cat^{\cont}_1$ to put emphasis on the disregard of $2$-morphisms.
        \end{itemize} 
    \end{convention}
    
    \section{Cohomology of Lie algebras}
        \begin{convention}
            From now until the end of the section let us fix a base field $k$ of characteristic $0$.
        \end{convention}
                
        \subsection{Lie operads and Koszul duality}
            \subsubsection{A bit about (co)algebras over (co)operads}
                \begin{definition}[Symmetric sequences] \label{def: symmetric_sequences}
                    Let $(\O, \tensor, \1)$ be a \textit{symmetric} monoidal category (typically taken to be $\Grp$) and let $\Sigma$ be an $\N$-graded monoid internal to $\O$ (i.e. a functor $\N \to \Mon(\O)$ or equivalently, a monoid internal to the functor category $[\N, \O]$, which inherits the symmetric monoidal structure of $\O$); $\Sigma$ is typically taken to be the graded monoid of symmetric groups, i.e. $\Sigma = \{S_n\}_{n \in \N}$. The naturally symmetric monoidal category of \textbf{$\Sigma$-symmetric $\O$-sequence} is thus simply the functor category $\O^{\Sigma} := [\bfB \Sigma, \O]$.
                \end{definition}
            
                \begin{convention}[Operads ?] \label{conv: symmetric_sequences_as_operads}
                    \noindent
                    \begin{enumerate}
                        \item \textbf{(Operads):} For the purposes of this subsection and section \ref{section: lie_algebras_and_formal_groups}, we shall think of \textbf{$\Sigma$-symmetric operads} in a given symmetric monoidal category $\O$ (for some $\N$-graded monoid $\Sigma$ internal to $\O$) as monoids in the symmetric monoidal category $\O^{\Sigma}$ of $\Sigma$-symmetric $\O$-sequences. For a more careful treatment, we refer the reader to section \ref{section: algebras_and_modules_over_operads} and particularly, subsection \ref{subsection: operads} therein. The category of $\Sigma$-symmetric operads shall be denoted by $\Opd(\O^{\Sigma})$ (it is actually just $\Mon(\O^{\Sigma})$, but we thought the notation could be a bit suggestive).
                        \item \textbf{(Regarding unitality):} Technically speaking, operads need not be unital, which is to say, for each given symmetric monoidal category $\O$ and a fixed $\N$-graded monoid $\Sigma$ therein, one can define $\Opd(\O^{\Sigma})$ to simply be $\Assoc\Alg(\O^{\Sigma})$. We will, however, only work with unital associative operads.
                    \end{enumerate}
                \end{convention}
                \begin{convention}
                    From now on, let us always take $\Sigma$ as the graded monoid of symmetric groups. Due to this specification of $\Sigma$, we shall only write $\Opd(\O)$ for the category of $\Sigma$-symmetric $\O$-operads (where, of course, $\O$ is a symmetric monoidal category).
                \end{convention}
                
                \begin{proposition}[Universal property of symmetric operads] \label{prop: symmetirc_operads_universal_property}
                    Let $(\O, \tensor, \1)$ be a symmetric monoidal category. Then, 
                \end{proposition}
                    \begin{proof}
                        
                    \end{proof}
            
            \subsubsection{Koszul Duality}
            
        \subsection{Lie algebrs and cocommutative coalgebras}
        
    \section{Lie algebroids}
        
    \section{Formal groups and their Lie algebras} \label{section: lie_algebras_and_formal_groups}
    
        \chapter{D-modules in geometric representation theory}
    \begin{abstract}
        
    \end{abstract}
    
    \minitoc
    
    \begin{convention}[Everything is derived!] \label{conv: D_modules_everything_is_derived}
        \noindent
        \begin{itemize}
            \item From now on until the end of the chapter, everything will be assumed to be derived. 
            \item By $1\-\Cat_1$, or simply $1\-\Cat$, we shall actually mean $(\infty, 1)\-\Cat_1$, i.e. the $(\infty, 1)$-category of $(\infty, 1)$-categories and functors between them, and by $1\-\Cat_2$ we will be referring to the $(\infty, 2)$-category of $(\infty, 1)$-categories, functors between them, and natural transformations between these functors. 
            
            Similarly, by $\Grpd^1$, or simply $\Grpd$, we will actually mean the $(\infty, 1)$-category of $\infty$-groupoids and functors between them, and by $\Grpd^2$, we shall mean the $(\infty, 2)$-category of $\infty$-groupoids, functors between them, and natural transformations between these functors.
            \item A subcategory of $1\-\Cat$ this is of particular interest is $\dg\Cat^{\cont}_2$ (or simply $\dg\Cat^{\cont}$), the $(\infty, 2)$-category of stable linear (i.e. differential-graded) $(\infty, 1)$-categories (see section \ref{section: homological_algebra} for the notion of stable $(\infty, 1)$-categories). Of course, we can also view $\dg\Cat^{\cont}$ as a mere $(\infty, 1)$-category; when necessary, we shall write $\dg\Cat^{\cont}_1$ to put emphasis on the disregard of $2$-morphisms.
        \end{itemize} 
    \end{convention}
    
    \section{D-modules over characteristic zero} \label{section: D_modules_over_characteristic_0}
        In this section, we will be giving a demonstration of how the theory of ind-coherent sheaves on inf-schemes (cf. section \ref{section: formal_schemes_and_inf_schemes}) helps us define and describe D-modules as ind-coherent sheaves over a certain kind of prestacks, thereby naturally establishing the six-functor formalism \textit{\`a la} Grothendieck in a natural manner for D-modules, in the sense that, should a prestack $\calZ$ be sufficiently nice, one would even be able to figure out the behaviours of D-modules on $\calZ$ only from what one would expect from ind-coherent sheaves on $\calZ$. 
        
        Before submerging ourselves in the tecnnical details, let us first recall some notable features of the classical theory of D-modules (in the sense of say, Kashiwara), both to equip ourselves with an intuitive sense of purpose for our endeavour, as well as to justify this highly abstract alternative approach to D-modules. 
            \begin{enumerate}
                \item \textbf{(The classical theory):} Let $k$ be a field of characteristic $0$, and let $X$ be a scheme over $\Spec k$, which we will assume to be flat, proper, and of finite presentation for simplicity. If $X$ is smooth, then the relative cotangent complex $\bfL^*_{X/k}$ of $X/k$ shall be quasi-isomorphic to the zero complex (cf. definition \ref{def: cohomological_smoothness}), which means that the de Rham complex $\Omega^*_{X/k}$ attached to $X/k$ can only have non-zero terms in non-negative degrees; this, in turn, implies that the de Rham complex:
                    $$
                        \begin{tikzcd}
                        	0 & \E & {\Omega^1_{X/k} \tensor_{\calO_{X/k}} \E} & {\Omega^2_{X/k} \tensor_{\calO_{X/k}} \E} & {...}
                        	\arrow["\nabla", from=1-4, to=1-5]
                        	\arrow["\nabla", from=1-3, to=1-4]
                        	\arrow["\nabla", from=1-2, to=1-3]
                        	\arrow[from=1-1, to=1-2]
                        \end{tikzcd}
                    $$
                (henceforth abbreviated by $\E \tensor_{\calO_{X/k}} \Omega^*_{X/k}$) corresponding to any given \textit{flat} $\calO_{X/k}$-linear connection $\nabla: \E \to \Omega^1_{X/k} \tensor_{\calO_{X/k}} \E$ will also be concentrated in non-negative degrees. Recall also, that to specify a \textit{flat} $\calO_{X/k}$-linear connection $\nabla: \E \to \Omega^1_{X/k} \tensor_{\calO_{X/k}} \E$ on a $\calO_{X/k}$-vector bundle $\E$ over a smooth scheme $X/k$ is the same as specifying the structure of a left-$\D_{X/k}$-module on $\E$. Together, these two facts imply that the dg-category ${}^l\Dmod(X/k)$ of chain complexes of left-$\D_{X/k}$-modules is equivalent to the category whose objects are the de Rham complexes corresponding to flat connections on vector bundles over $X/k$, which we will denote by ${}^{\geq 0}\Vect(X/k)^{\nabla}_{\dR}$. As a consequence, given any smooth morphism:
                    $$f: X \to Y$$
                of schemes over $\Spec k$, one obtains a six-functor formalism between ${}^l\Dmod(X/k)$ and ${}^l\Dmod(Y/k)$ simply as that between ${}^{\geq 0}\Vect(X/k)^{\nabla}_{\dR}$ and ${}^{\geq 0}\Vect(Y/k)^{\nabla}_{\dR}$. While this is all well and good, categories of vector bundles equipped with flat connections (over smooth schemes) are rather unnatural as far as functorial constructions are concerned. 
                
                But what about cases wherein $X$ is not smooth ? In such a situation, one would embed $X$ into some smooth ambient scheme, say $Y$ (which can always be done, as one can take $X$ to be the affine singular locus and then embed that locus into a smooth deformation of its ambient scheme). Then, we can make use of what Kashiwara taught us, which is that regardless of our choice of embedding $X \hookrightarrow Y$, the restriction ${}^l\Dmod(Y/k)|_X$ of the category of left-$\D_{Y/k}$-modules onto objects (set-theoretically) supported on $X$ will be the same. Choosing such an embedding into a smooth scheme, however, would still involve appealing to resolutions, which is a rather unnatural procedure (at least from a $\infty$-categorical standpoint).
                
                Additionally, the relationship between left-D-module and right-D-modules is rather unclear within this classical framework (it turns out that via the dualising complex, left and right-D-modules are equivalent, but this does not excuse the fact that the correspondence seems rather unnatural).
                \item \textbf{(The new perspective):} Again, suppose that $k$ is a field (perhaps of characteristic $0$) and that $X$ is a scheme that is flat, proper, and of finite presentation over $\Spec k$.
                
                Let us firstly that a crystal in quasi-coherent modules over $X$ is one which is isomorphic to its restriction to the maximal reduced closed subscheme (note that $|X| \cong |{}^{\red}X|$; cf. example \ref{example: reducedeness_and_nilpotency}), which is to say that:
                    $$\E \cong \E|_{X_{\dR}}$$
                (here $X_{\dR}$ denotes the so-called \textbf{de Rham space} attached to $X$, which as a presheaf is given by $X_{\dR}(R) \cong X({}^{\red}R)$ for all commutative $k$-algebras $R$). Additionally, recall that to any crystal in quasi-coherent modules $\E \in \QCoh(X_{\dR}/k)$, one can canonically associate an $\calO_{X/k}$-linear flat connection:
                    $$\nabla: \E \to \Omega^1_{X/k} \tensor_{\calO_{X/k}} \E$$
                As stated above, when $X$ is smooth over $\Spec k$, this is the same as giving a left-$\D_{X/k}$-module, so for smooth schemes, one might as well define the dg-category of left-$\D_{X/k}$-modules to literally be the dg-category $\QCoh(X_{\dR}/k)$ of crystals in quasi-coherent modules over $X$; in fact, as schemes smooth over fields are \textit{a priori} reduced, ${}^l\Dmod(X/k)$ is nothing but $\QCoh(X/k)$.
                
                The nice thing about this approach is that we understand quasi-coherent sheaves very well, and better yet, the theory of quasi-coherent sheaves over schemes is essentially that of modules over commutative rings (cf. definition \ref{def: qcoh_def}). Moreover, it now makes sense to speak of D-modules over smooth algebraic stacks and so on (perhaps with some sort of properness or separatedness assumption imposed), as there is nothing preventing us from considering de Rham spaces attached to \textit{any presheaf} on $\Comm\Alg^{\op}$, as the definition is completely functorial. This is very useful, as there are many geometric objects which appear naturally in geometric representation theory yet are not schemes, such as the stack $\Bun_G(X)$ of principal $G$-bundles over a smooth curve $X$, for $G$ a reductive group; in fact, the so-called \say{Automorphic Side} of the Geometric Global Langlands Correspondence is the category of D-modules on this stack.
                \item \textbf{(What happens in positive characteristics ?):} The new perspective that we have just described is not without flaws, however. One of its most prominent shortcoming is that it fails completely when $k$ is instead of some positive characteristic $p$. In that situation, one can still attach to each crystal in quasi-coherent sheaves $\E$ - in a canonical manner - a flat connection $\nabla: \E \to \Omega^1_{X/k} \tensor_{\calO_{X/k}} \E$ that would generate a de Rham complex:
                    $$
                        \begin{tikzcd}
                        	0 & \E & {\Omega^1_{X/k} \tensor_{\calO_{X/k}} \E} & {\Omega^2_{X/k} \tensor_{\calO_{X/k}} \E} & {...}
                        	\arrow["\nabla", from=1-4, to=1-5]
                        	\arrow["\nabla", from=1-3, to=1-4]
                        	\arrow["\nabla", from=1-2, to=1-3]
                        	\arrow[from=1-1, to=1-2]
                        \end{tikzcd}
                    $$
                (see \cite[\href{https://stacks.math.columbia.edu/tag/07J5}{Tag 07J5}]{stacks} for details). However, a crystal in quasi-coherent sheaves is now no longer simply an object of $\QCoh(X_{\dR}/k)$, even when $X$ is smooth. \todo{Continue this}
            \end{enumerate}
                    
            \begin{convention}
                Until the end of this section, we shall be working over a field of characteristic $0$ ($k = \Q$ and $k = \bbC$ are cases of particular interest). Additionally, by \say{prestacks}, we shall always mean \say{prestacks fibred in $\infty$-groupoids}. Due to these reasons, the $\infty$-presheaf $\infty$-topos $\Spec k$ shall be denoted somewhat ambiguously by $\Pre\Stk$; likewise, by $\Comm\Alg$ we shall mean ${}^{k/}\Comm\Alg$ and by $\Sch$ (respectively $\Sch^{\aff}$), we shall mean $\Sch_{/\Spec k}$ (respectively $\Sch^{\aff}_{/\Spec k}$).
            \end{convention}
    
        \subsection{Crystals as sheaves}
            \subsubsection{The de Rham prestack}
                As eluded to in the preceding introduction, crystals (and hence D-modules) are sheaves of modules over so-called \textbf{de Rham spaces}. Therefore, before actually diving in, we shall need to study the geometry of these de Rham spaces.
            
                Before we introduce the notion of de Rham spaces attached to prestacks, however, let us make some remark regarding reduced commutative $k$-algebras. 
                \begin{remark}[Reduced affine schemes] \label{remark: reduced_affine_schemes}
                    \noindent
                    \begin{itemize}
                        \item \textbf{(Reduced rings):} Reduced objects of $\Comm\Alg$ are nothing but $0$-connective commutative $k$-algebras with vanishing nilradicals, i.e. they have no non-zero nilpotent elements, and because ring homomorphisms preserve $0$, a homomorphism:
                            $$\varphi: R \to S$$
                        between two reduced commutative $k$-algebras $R$ and $S$ will be nothing more than an ordinary algebra homomorphism. As a consequence, reduced $k$-algebras form a full subcategory of $\Comm\Alg$ (or for that matter, of ${}^{\leq 0}\Comm\Alg$); we shall denote it by ${}^{\leq 0}\Comm\Alg^{\red}$; note that $k$ itself, by virtue of being a field of characteristic $0$, is trivially reduced as an algebra over itself, and hence is still the initial object of ${}^{\leq 0}\Comm\Alg^{\red}$. 
                        \item \textbf{(Prestacks over reduced affine schemes):} The category of prestacks over ${}^{\leq 0}\Comm\Alg^{\red}$ shall be denoted by $\Pre\Stk|_{{}^{\leq 0}\Sch^{\aff, \red}}$. This is nothing but the domain restriction of the presheaf $\infty$-topos $\Pre\Stk$ (if we would view the assignment of $\infty$-presheaves $\infty$-topoi to $\infty$-categories as fibration $\Pre\Stk \to 1\-\Cat_1$) down from $\Comm\Alg^{\op}$ onto $({}^{\leq 0}\Comm\Alg^{\red})^{\op}$. 
                    \end{itemize}
                \end{remark}
                
                And now, the definition of de Rham spaces:
                \begin{definition}[de Rham prestacks] \label{def: de_rham_prestacks}
                    We can associate to any prestack $\calZ \in \Pre\Stk$ a prestack $\calZ_{\dR} \in \Pre\Stk$, called \textbf{the de Rham prestack attached to $\calZ$}, that is defined object-wise by the following formula:
                        $$\calZ_{\dR}(R) \cong \calZ({}^{\red}R)$$
                    where ${}^{\red}R$ is the classical commutative ring isomorphic to the quotient of the underlying classical commutative ring of $R$ by its nilradical. Note that this formula gives us a canonical arrow:
                        $$\calZ \to \calZ_{\dR}$$
                    coming from the canonical quotient map $R \to {}^{\red}R$. Sometimes de Rham prestacks might also go by the name \say{\textbf{de Rham spaces}}.
                \end{definition}
                \begin{remark}[Functoriality of de Rham prestacks] \label{remark: de_rham_prestacks_functoriality}
                    \noindent
                    \begin{enumerate}
                        \item Consider a morphism:
                            $$\calX \to \calY$$
                        between prestacks on $\Comm\Alg^{\op}$. It is not hard to show, via evaluation at quotients by nilradicals, that such a morphism induced a morphism of de Rham spaces:
                            $$\calX_{\dR} \to \calY_{\dR}$$
                        which tells us that there exists a so-called \textbf{de Rham space functor}:
                            $$\dR: \Pre\Stk|_{{}^{\leq 0}\Sch^{\aff, \red}} \to \Pre\Stk$$
                        that assigns de Rham spaces $\calZ_{\dR}$ to prestacks $\calZ \in \Pre\Stk|_{{}^{\leq 0}\Sch^{\aff, \red}}$. 
                        \item Due to the fact that limits and colimits of prestacks are computed object-wise, the functor $\dR$ commutes with all limits and colimits in $\Pre\Stk|_{{}^{\leq 0}\Sch^{\aff, \red}}$.
                    \end{enumerate}
                \end{remark}
                
                \begin{proposition}[The locally almost of finite type case] \label{prop: laft_de_rham_spaces}
                    If $\calZ$ is a prestack that is locally almost of finite type then so is its associated de Rham prestack $\calZ_{\dR}$. This is to say, the essential image of the functor:
                        $$\dR|_{\Pre\Stk^{\laft}}: \Pre\Stk^{\laft} \to \Pre\Stk$$
                    is a subcategory of $\Pre\Stk^{\laft}$.
                \end{proposition}
                    \begin{proof}
                        We will need to show that for all $\calZ \in \Pre\Stk^{\laft}$, the corresponding de Rham space $\calZ_{\dR}$ is convergent, and that for all $n \in \N$, the de Rham space $({}^{\leq n}\calZ)_{\dR}$ attached to the $n$-coconnective prestack ${}^{\leq n}\calZ \in {}^{\leq n}\Pre\Stk^{\laft}$ is also $n$-coconnective.
                            \begin{enumerate}
                                \item 
                                \item 
                            \end{enumerate}
                    \end{proof}
                
                \begin{remark}[Universal property of the de Rham space functor] \label{remark: universal_property_of_de_rham_spaces}
                    
                \end{remark}
                    
            \subsubsection{Left and right-crystals}
                \begin{definition}[Right-crystals] \label{def: right-crystals}
                    By composing $\dR: \Pre\Stk^{\laft} \to \Pre\Stk^{\laft}$ with the functor:
                        $$\Ind\Coh^!: (\Pre\Stk^{\laft})^{\op} \to \dg\Cat^{\cont}_1$$
                    one obtains a so-called functor of \textbf{right-crystals}:
                        $$\Crys^! \cong \Ind\Coh^! \circ \dR$$
                \end{definition}
        
        \subsection{Crystals as functors on the category of correspondences}
        
        \subsection{D-modules}
        
        \subsection{Twistings}
            \subsubsection{The notion of twists}
            
            \subsubsection{Twisted crystals}
    
    \section{Localisations of Lie algebras representations}
        \subsection{Some supplementary Lie theory}
            \subsubsection{Lie algebras in symmetric monoidal categories and their enveloping algebras}
                \begin{definition}[Algebras, coalgebras, and bialgebras] \label{def: algebras_and_coalgebras}
                    Let $(\O, \tensor, 1)$ be a monoidal category. 
                        \begin{enumerate}
                            \item A(n) (associative and unital) algebra $A$ internal to $\O$ is a monoid object of $\O$, i.e. one equipped with a so-called multiplication:
                                $$\nabla: A \tensor A \to A$$
                            and unit:
                                $$\eta: 1 \to A$$
                            satisfying the following commutative diagrams:
                                $$
                                    \begin{tikzcd}
                                    	{A \tensor A \tensor A} & {A \tensor A} \\
                                    	{A \tensor A} & {A}
                                    	\arrow["{\id_A \tensor \nabla}"', from=1-1, to=2-1]
                                    	\arrow["{\nabla}", from=2-1, to=2-2]
                                    	\arrow["{\nabla \tensor \id_A}", from=1-1, to=1-2]
                                    	\arrow["{\nabla}", from=1-2, to=2-2]
                                    \end{tikzcd}
                                $$
                                $$
                                    \begin{tikzcd}
                                    	{1 \tensor A} & {A \tensor A} \\
                                    	{A \tensor A} & {A}
                                    	\arrow["{\nabla}", from=1-2, to=2-2]
                                    	\arrow["{\nabla}"', from=2-1, to=2-2]
                                    	\arrow["{\id_A \tensor \eta}"', from=1-1, to=2-1]
                                    	\arrow["{\eta \tensor \id_A}", from=1-1, to=1-2]
                                    \end{tikzcd}
                                $$
                            \item A (coassociative and counital) coalgebra internal to $\O$ is then a comonoid object of $\O$, or in other words, an monoid object in $\O^{\op}$. Typically, the so-called multiplication and counit maps on a coalgebra will be denoted by $\Delta$ and $\e$.
                        \end{enumerate}
                    Algebras, and coalgebras internal to a monoidal category $\O$ form full subcategories, which we will denote, respectively, by $\Mon(\O)$ and $\co\Mon(\O)$.
                \end{definition}
                \begin{example}
                    \noindent
                    \begin{enumerate}
                        \item \textbf{(Rings and algebras over them)} All rings are associative and unital algebra in the (symmetric monoidal) category of abelian groups. More generally, for any given base ring $R$ (not necessarily commutative), associative and unital left/right/two-sided $R$-algebras are monoids in the (monoidal) category of left/right/two-sided $R$-modules.
                        \item \textbf{(Lie algebras and their universal enveloping algebras)} Lie algebras, despite their names, are not algebras in the sense of definition \ref{def: algebras_and_coalgebras}, as their Lie brackets are not associative (see definition \ref{def: lie_algebras} for more details). Their univeral enveloping algebras (cf. definition \ref{def: enveloping_algebras} and theorem \ref{theorem: universal_enveloping_algebras_universal_property}), on the other hand, are associative and unital algebras. In fact, they are coassociative and counital coalgebras too (cf. remark \ref{remark: universal_enveloping_algebras_are_bialgebras}). 
                    \end{enumerate}
                \end{example}
            
                Let us now take a closer look at how universal enveloping algebras are bialgebras. To do so, however, we will need to have a good understanding of these algebras behave. 
                \begin{definition}[Lie algebras] \label{def: lie_algebras}
                    Let $k$ be a ring (not necessarily commutative) and let $(\O, \tensor, 1, \tau)$ be a symmetric monoidal $k$-linear category with braiding isomorphisms:
                        $$\tau_{x, y}: x \tensor y \cong y \tensor x$$
                    A Lie algebra internal to is then an object $\g \in \O$ equipped with a so-called Lie bracket:
                        $$[-,-]: \g \tensor \g \to \g$$
                    subject to two requirements:
                        \begin{enumerate}
                            \item \textbf{(Skew-symmetry)}
                                $$[-,-] + [-,-] \circ \tau_{\g, \g} = 0$$
                            \item \textbf{(The Jacobi identity)}
                                $$
                                    \begin{aligned}
                                        & \left[-, [-,-]\right]
                                        \\
                                        + & \left[-, [-,-]\right] \circ \left(\id_{\g} \tensor \tau_{\g, \g}\right) \circ \left(\tau_{\g, \g} \tensor \id_{\g}\right)
                                        \\
                                        + & \left[-, [-,-]\right] \circ \left(\tau_{\g, \g} \tensor \id_{\g}\right) \circ \left(\id_{\g} \tensor \tau_{\g, \g}\right)
                                        \\
                                        = & \: 0
                                    \end{aligned}
                                $$
                        \end{enumerate}
                    Lie algebras internal to a symmetric monoidal $k$-linear category $\O$ form a full subcategory which we shall denote by $\Lie\Alg(\O)$. Its objects are the Lie algebra objects of $\g$, and its morphisms are arrows in $\O$ that intertwine with Lie brackets, i.e. they are arrows $\phi: \g \to \h$ such that:
                        $$[-,-]_{\h} \circ (\phi \tensor \phi) = \phi \circ [-,-]_{\g}$$
                    or equivalently, such that diagrams of the following form commute in $\O$:
                        $$
                            \begin{tikzcd}
                            	{\g \tensor \g} & {\g} \\
                            	{\h \tensor \h} & {\h}
                            	\arrow["{\phi}", from=1-2, to=2-2]
                            	\arrow["{\phi \tensor \phi}"', from=1-1, to=2-1]
                            	\arrow["{[-,-]_{\h}}", from=2-1, to=2-2]
                            	\arrow["{[-,-]_{\g}}", from=1-1, to=1-2]
                            \end{tikzcd}
                        $$
                \end{definition}
                
                \begin{definition}[Enveloping algebras] \label{def: enveloping_algebras}
                    Let $k$ be a ring (not necessarily commutative) and let $(\O, \tensor, 1, \tau)$ be a symmetric monoidal $k$-linear category.
                        \begin{enumerate}
                            \item \textbf{(The Lie functor)} The Lie functor is the one that assigns to each associative and unital algebra $A$ internal to $\O$ a Lie algebra $\frakLie(A)$ whose underlying object is just $A$, and whose Lie bracket is given by:
                                $$[-,-]_{\frakLie(A)} := \nabla_A - \nabla_A \circ \tau_{A,A}$$
                            \item \textbf{(Enveloping algebras)} Fix a Lie algebra object $\g$ of $\O$. Then, an enveloping algebra of a Lie algebra $\g$ internal to $\O$ is just a Lie algebra homomorphism:
                                $$e: \g \to \frakLie(A)$$
                            for some $A \in \Mon(\O)$. These enveloping algebras form a category, whose objects are Lie algebra homomorphisms as described above, and whose morphisms are commutative triangles in $\Lie\Alg(\O)$ as follows:
                                $$
                                    \begin{tikzcd}
                                    	& {\g} \\
                                    	{\frakLie(A)} && {\frakLie(A')}
                                    	\arrow[from=2-1, to=2-3]
                                    	\arrow["{e}"', from=1-2, to=2-1]
                                    	\arrow["{e'}", from=1-2, to=2-3]
                                    \end{tikzcd}
                                $$
                            which we note to be induced by algebra homomorphisms $A \to A'$. We will denote the category of enveloping algebras of $\g$ by $\Env(\g)$. 
                            \\
                            The universal enveloping algebra of $\g$ (denoted by $\U(\g)$), if it exists, then its Lie algebra will be the initial object of $\Env(\g)$. 
                        \end{enumerate}
                \end{definition}
                
                \begin{theorem}[Existence and uniqueness of universal enveloping algebras] \label{theorem: universal_enveloping_algebras_universal_property}
                     Let $k$ be a ring (not necessarily commutative) and let $(\O, \tensor, 1, \tau)$ be a symmetric monoidal $k$-linear category. Then:
                        \begin{enumerate}
                            \item \textbf{(Existence and uniqueness)} There is the following ($\O$-enriched) adjunction if $\O$ has all countable coproducts:
                                $$
                                    \begin{tikzcd}
                                    	{(\U \ladjoint \frakLie): \Mon(\O)} & {\Lie\Alg(\O)}
                                    	\arrow["{\frakLie}"{name=0, swap}, from=1-1, to=1-2, shift right=2]
                                    	\arrow["{\U}"{name=1, swap}, from=1-2, to=1-1, shift right=2]
                                    	\arrow["\dashv"{rotate=-90}, from=1, to=0, phantom]
                                    \end{tikzcd}
                                $$
                            wherein $\U$ is the functor sending each Lie algebra in $\O$ to its universal enveloping algebra.
                            \item \textbf{(Explicit construction)} If $\O$ also has all cokernels then we can explicitly characterise the universal enveloping algebra of a Lie algebra $\left(\g, [-,-]_{\g}\right)$ as the following quotient of the (Lie algebra canonically associated to) the tensor algebra $T(\g)$:
                                $$\frakLie\left(\U(\g)\right) \cong \coker \bigg(\left([-,-]_{\frakLie T(\g)} - \left(\nabla_{T(\g)} - \nabla_{T(\g)} \circ \tau_{T(\g), T(\g)}\right)\right): T(\g) \tensor T(\g) \to T(\g)\bigg)$$
                        \end{enumerate}
                \end{theorem}
                    \begin{proof}
                        \noindent
                        \begin{enumerate}
                            \item We will be using Freyd's Adjoint Functor Theorem \cite[Theorem V.6.2]{maclane} to prove this assertion, and to that end, let us first note that the category $\Mon(\O)$ is complete and locally small (this can be proven in the exact same way that one might prove that algebraic categories such as $\Ab$ or $\Ring$ are complete and locally small). Next, we will try to show that the functor $\frakLie$ satisfies the \href{https://ncatlab.org/nlab/show/solution+set+condition}{\underline{solution set condition}}, which we can do by finding a small indexing set $I$ such that for all enveloping algebras $\g \to \frakLie(A)$ of $\g$, there exists a family of algebras $A_i$ indexed by $i \in I$ and factorisations:
                                $$
                                    \begin{tikzcd}
                                    	& {\g} \\
                                    	{\frakLie(A_i)} && {\frakLie(A)}
                                    	\arrow[from=2-1, to=2-3, dashed]
                                    	\arrow["{}"', from=1-2, to=2-1]
                                    	\arrow["{}", from=1-2, to=2-3]
                                    \end{tikzcd}
                                $$
                            (we actually do not need to worry about the size of $I$, since we have already fixed a sufficiently large Grothendieck universe). To that end, note that because $\O$ has all countable coproducts, one can always construct tensor algebras, whose universal property implies that there is the following adjunction:
                                $$
                                    \begin{tikzcd}
                                    	{(T \ladjoint \oblv): \Mon(\O)} & {\O}
                                    	\arrow["{\oblv}"{name=0, swap}, from=1-1, to=1-2, shift right=2]
                                    	\arrow["{T}"{name=1, swap}, from=1-2, to=1-1, shift right=2]
                                    	\arrow["\dashv"{rotate=-90}, from=1, to=0, phantom]
                                    \end{tikzcd}
                                $$
                            In particular, this means that for each $A \in \Mon(\O)$ and each Lie algebra $\g$, there is a canonical morphism from $T(\g)$ into $A$. At the same time, note that because tensor algebras are constructed as coproducts of tensor powers, one has canonical inclusions of the tensor powers $\g^{\tensor n}$ into $T(\g)$. Thus, there are commutative diagrams in $\O$ as follows for each $A \in \Mon(\O)$ and each $n \in \N$:
                                $$
                                    \begin{tikzcd}
                                    	{\g^{\tensor n}} \\
                                    	& {T(\g)} & {A}
                                    	\arrow[from=1-1, to=2-2]
                                    	\arrow[from=2-2, to=2-3]
                                    	\arrow[from=1-1, to=2-3]
                                    \end{tikzcd}
                                $$
                            Thus, for each associative and unital algebra $A$, there is a universal morphism in $\Env(\g)$ as follows:
                                $$
                                    \begin{tikzcd}
                                    	& {\g} \\
                                    	{\frakLie\left(T(\g)\right)} && {\frakLie(A)}
                                    	\arrow[from=2-1, to=2-3]
                                    	\arrow[from=1-2, to=2-1]
                                    	\arrow[from=1-2, to=2-3]
                                    \end{tikzcd}
                                $$
                            proving the existence of a small index set $I$ (the singleton in this instance) and a family of objects $\{A_i\}_{i \in I}$ of $\Mon(\O)$ such that there are factorisations as below for all $i \in I$:
                                $$
                                    \begin{tikzcd}
                                    	& {\g} \\
                                    	{\frakLie(A_i)} && {\frakLie(A)}
                                    	\arrow[from=2-1, to=2-3, dashed]
                                    	\arrow["{}"', from=1-2, to=2-1]
                                    	\arrow["{}", from=1-2, to=2-3]
                                    \end{tikzcd}
                                $$
                            (for the sake of clarity, let us note that here, we take $A_i = T(\g)$ for all $i \in I$). Lastly, note that the category $\Mon(\O)$ is complete and locally small. Thus, all the conditions in Freyd's Adjoint Functor Theorem are satisfied, and therefore, $\frakLie$ is a right-adjoint. This of course means that we can construct a functor:
                                $$\U: \Lie\Alg(\O) \to \Mon(\O)$$
                            to be left-adjoint to $\frakLie$. Then, as a consequence of the universal property of adjoint pairs, the category of enveloping algebras of $\g$ must have $\frakLie\left(\U(\g)\right)$ as an initial object, which by definition is the so-called universal enveloping algebra of $\g$.
                            \item This is trivial if we note that taking the quotient of $T(\g)$ by the equivalence relation generated by the image of $[-,-]_{\frakLie T(\g)} - \left(\nabla_{T(\g)} - \nabla_{T(\g)} \circ \tau_{T(\g), T(\g)}\right)$ (which incidentally, also canonically endows $T(\g)$ with a Lie bracket) is just to ensure that the canonical map:
                                $$\g \to (\frakLie \circ \U \circ \frakLie \circ T)(\g)$$
                            is a Lie algebra homomorphism for every $\g$.
                        \end{enumerate}
                    \end{proof}
                \begin{corollary}
                    $\frakLie$ is left-exact and $\U$ is right-exact. In particular, we have:
                        $$\U\left(\g \oplus \g'\right) \cong \U(\g) \tensor \U(\g')$$
                    for any pair $\g, \g'$ of Lie algebras. 
                \end{corollary}
                    \begin{proof}
                        These are general properties of adjoint functors.
                    \end{proof}
                \begin{remark}
                    The adjunction as presented in the preceding theorem can be understood as fitting into the following (non-commutative) diagram:
                        $$
                            \begin{tikzcd}
                            	{\Mon(\O)} & {} & {\O} \\
                            	& {\Lie\Alg(\O)}
                            	\arrow["{\oblv}"{name=0, swap}, from=1-1, to=1-3, shift right=2]
                            	\arrow["{T}"{name=1, swap}, from=1-3, to=1-1, shift right=2]
                            	\arrow["{L}"{name=2, swap}, from=2-2, to=1-3, shift right=5]
                            	\arrow[""{name=3, inner sep=0}, from=1-3, to=2-2, shift left=1]
                            	\arrow["{\U}"{name=4, swap}, from=2-2, to=1-1, shift left=1]
                            	\arrow["{\frakLie}"{name=5, swap}, from=1-1, to=2-2, shift right=5]
                            	\arrow["\dashv"{rotate=-90}, from=1, to=0, phantom]
                            	\arrow["\dashv"{rotate=61}, from=5, to=4, phantom]
                            	\arrow["\dashv"{rotate=129}, from=2, to=3, phantom]
                            \end{tikzcd}
                        $$
                \end{remark}
                \begin{remark}[Universal enveloping algebras are bialgebras] \label{remark: universal_enveloping_algebras_are_bialgebras}
                    Let $k$ be a ring and let $\g$ be a Lie algebra internal to some $k$-linear symmetric monoidal category $(\V, \tensor, 1, \tau)$. Then, its universal enveloping algebra $\U(\g)$ is a cocommutative bialgebra internal to $\V$, which is commutative if and only if $\g$ is abelian. 
                    
                    This becomes trivial if we let the comultiplication be given by:
                        $$\Delta_{\U(\g)} := \id_{\U(\g)} \tensor \e + \e \tensor \id_{\U(\g)}$$
                    and the counit be:
                        $$\e_{\U(\g)} := 0$$
                \end{remark}
                    
                \begin{lemma}[Embedding of Lie algebras into their universal enveloping algebras]
                    Let $k$ be a ring (not necessarily commutative) and let $(\O, \tensor, 1, \tau)$ be a symmetric monoidal $k$-linear \textbf{abelian} category. Also, suppose that $\g$ is a (faithfully ?) flat Lie algebra object internal to $\O$, i.e. that the functor $\g \tensor -$ is (faithfully ?) flat. Then, there is a canonical monomorphism embedding $\g$ itno $\frakLie \U(\g)$
                \end{lemma}
                    \begin{proof}
                        \todo{Currently I'm not certain that this statement is entirely correct.}
                    \end{proof}
                
                \begin{example}[The abelian case]
                    If $\g$ is an abelian Lie algebra (i.e. one whose Lie bracket is just the zero morphism), then as a direct consequence of the definition of symmetric algebras and of theorem \ref{theorem: universal_enveloping_algebras_universal_property}, we have the following isomorphism of associative and unital algebras:
                        $$\U(\g) \cong \Sym(\g)$$
                \end{example}
                \begin{example}[The Poincar\'e-Birkhoff-Witt Theorem]
                    Let $k$ be a commutative and unital $\Q$-algebra, let $(\O, \tensor, 1, \tau)$ be a symmetric monoidal $k$-linear category with all countable coproducts and cokernels, and let $\g$ be a Lie algebra object of $\O$ that is (faithfully ?) flat (i.e. one such that the functor $\g \tensor -$ is left-exact). Then, there is the following isomorphism of cocommutative $\N$-graded $k$-bialgebras:
                        $$\U(\g) \cong \Sym(\g)$$
                    In other words, one can understand representations of such a Lie algebra $\g$ via representations of the symmetric algebra $\Sym(\g)$, which in turn are just representations of symmetric groups.
                \end{example}
                \begin{example}[A counter-example]
                    Let $k$ be a field of characteristic $0$ and consider the symmetric monoidal category of finite-dimensional $k$-vector spaces. Then clearly, $\O$ does not have all countable coproducts (for example, the vector space $k^{\oplus \aleph_0}$, which is a coproduct indexed by the countable infinite cardinal $\aleph_0$, is not an object as it is infinite-dimensional), and thus not all Lie $k$-algebras have a universal enveloping algebra.
                \end{example}
                \begin{example}[Open problem]
                    It is not known if the functor:
                        $$\U: \frakLie(\O) \to \Lie\Alg(\O)$$
                    is faithful, and even if that is not the case in general, we also do not have a good understanding of which extra assumptions to impose on our ambient symmetric monoidal linear category $\O$ so that in such a setting, $\U$ might be faithful. 
                \end{example}
                
                \begin{lemma}[Tannaka duality] \label{lemma: tannaka_duality}
                    Let $\O$ be a closed symmetric monoidal category that is locally small (such as the symmetric monoidal category of vector spaces over a field), so that it may be enriched over itself via its internal homs (which exist thanks to the monoidal closure assumption), and let $A$ be a monoid object of $\O$. If we denote the category of $\O$-representations on $A$ by:
                        $$\Rep_{\O}(A) := \O\-\Cat(\bfB A^{\op}, \O)$$
                    then the algebra $\End_{\O\-\Cat(\Rep_{\O}(A), \O)}(F)$ of $\O$-natural endomorphisms on the canonical forgetful functor $F: \Rep_{\O}(A) \to \O$ is isomorphic to $A$.
                \end{lemma}
                    \begin{proof}
                        Apply the $\O$-enriched Yoneda's lemma:
                            $$
                                \begin{aligned}
                                    \End_{\O\-\Cat(\Rep_{\O}(A), \O)}(F) & \cong \O\-\Cat(\Rep_{\O}(A), \O)(F, F)
                                    \\
                                    & \cong \Psh_{\O}\left(\Psh_{\O}(\bfB A)\right)\bigg(\Psh_{\O}(\bfB A)(*, -), \Psh_{\O}(\bfB A)(*, -)\bigg)
                                    \\
                                    & \cong \Psh_{\O}(\bfB A)(*, *)
                                    \\
                                    & \cong \Rep_{\O}(A)(*, *)
                                    \\
                                    & \cong \End_{\O}(A)
                                    \\
                                    & \cong A
                                \end{aligned}
                            $$
                    \end{proof}
                \begin{theorem}
                    If $k$ is a ring, $\O$ is a \textit{locally small} \textit{closed} symmetric monoidal $k$-linear category with all coproducts and cokernels, and $\g$ is a (faithfully ?) flat Lie algebra object of $\O$, then there is an equivalence of abelian monoidal $k$-linear categories as follows:
                        $$\Rep_{\O}(\g) \cong {\U(\g)}\mod$$
                    wherein $\Rep_{\O}(\g)$, the category of representations of $\g$ on $\O$, is the category whose objects are Lie algebra homomorphisms from $\g$ to $\frakgl(V)$ (with $\frakgl(V)$ the canonical Lie algebra associated to the associative and unital endomorphism algebra $\End_{\O}(V)$), and morphisms are commutative triangles in $\Lie\Alg(\O)$ of the form:
                        $$
                            \begin{tikzcd}
                            	& {\g} \\
                            	{\frakgl(V)} && {\frakgl(V')}
                            	\arrow[from=2-1, to=2-3]
                            	\arrow[from=1-2, to=2-1]
                            	\arrow[from=1-2, to=2-3]
                            \end{tikzcd}
                        $$
                    Also, note that ${\U(\g)}\mod$ is \href{https://ncatlab.org/nlab/show/module+over+a+monoid}{\underline{well-defined}} as the category of $\U(\g)$-left-equivariant objects of $\O$, and it exhibits all properties that one might expect of a module category.
                \end{theorem}
                    \begin{proof}
                        Before we prove this claim, let us first note that because $\O$ has all countable coproducts and cokernels, all Lie algebras possess universal enveloping algebras. With that out of the way, let us note that by Tannaka duality (cf. lemma \ref{lemma: tannaka_duality}), each left-$\U(\g)$-module is actually just a $\U(\g)$-representation. Thus, to show that:
                            $$\Rep_{\O}(\g) \cong {\U(\g)}\mod$$
                        it will suffice to show that:
                            $$\Rep_{\O}(\g) \cong \Rep_{\O}\left(\U(\g)\right)$$
                        (this might seem like a round-about method, but without Tannaka duality, guaranteeing that the equivalence is actually between abelian monoidal $k$-linear categories instead of just between ordinary categories will be difficult). In turn, one can do this via showing that there is the following equivalence of categories of Lie algebra representations:
                            $$\Rep_{\O}(\g) \cong \Rep_{\O}\left(\frakLie \U(\g)\right)$$
                        thanks to the uniqueness of the universal enveloping algebra. Then, we can simply apply lemma 1.1.1, which states that $\g$ is a subobject of $\frakLie \U(\g)$, and consider commutative diagrams as follows in $\Lie\Alg(\O)$:
                            $$
                                \begin{tikzcd}
                                	{\g} \\
                                	& {\frakgl(V)} \\
                                	{\frakLie \U(\g)}
                                	\arrow[from=1-1, to=2-2]
                                	\arrow[from=3-1, to=2-2]
                                	\arrow[from=1-1, to=3-1, tail]
                                \end{tikzcd}
                            $$
                        to show that for each representation of $\g$ on $V \in \O$, there is a representation of $\frakLie \U(\g)$ on $V$ as well, and vice versa.
                    \end{proof}
                    
            \subsubsection{Cohomology of Lie algebras}
                \paragraph{Cohomologies and homologies of Lie algebra}
                    \begin{definition}[Lie algebra cohomologies] \label{def: lie_algebra_cohomologies}
                        Let $\g$ be a Lie algebra over a commutative ring $k$ and let $A$ be a $\g$-module. Then, we define the $n^{th}$ cohomology group of $\g$ with coefficient in $A$ as:
                            $$H^n(\g, A) \cong \Ext^n_{\U(\g)}(k, A)$$
                        where we view $k$ as the $1$-dimensional $\g$-representation on the right-hand side. 
                    \end{definition}
                    \begin{remark}
                        It is not hard to see via induction on the cohomological dimension $n \in \N$ that $H^n(\g, A)$ (as in definition \ref{def: lie_algebra_cohomologies}) actually has a $k$-module structure.
                    \end{remark}
                    
                    Let us now attempt to compute cohomologies of Lie algebras in low dimensions (namely, $n = 0, 1, 2, 3$) as well as give meaning to these spaces. We shall require, first of all, the following lemma:
                    \begin{lemma}[de Rham resolutions of Lie algebras] \label{lemma: de_rham_resolutions_of_lie_algebras}
                        For $\g$ a Lie algebra over a commutative ring $k$ and $A$ any $\g$-module, one has the following isomorphism of cohomology groups:
                            $$H^n(\g, A) \cong \Ext^n_k(\Lambda^{\bullet}(\g), A)$$
                    \end{lemma}
                        \begin{proof}
                            By definition, we have:
                                $$H^n(\g, A) \cong \Ext^n_{\U(\g)}(k, A)$$
                            Through noting that $\Lambda^{\bullet}(\g) \cong \{k \to \g \to \g \wedge \g \to \cdots\}$ is an injective resolution of $k \in \U(\g)\mod$, one sees that these isomorphisms of cohomologies \say{lift} to the following quasi-isomorphism of cochain complexes:
                                $$H^{\bullet}(\g, A) \cong_{\qis} \R\Hom_{\U(\g)}\left(\U(\g) \tensor_k^{\L} \Lambda^{\bullet}(\g), A\right)$$
                            An application of the tensor-hom adjunction then gives:
                                $$H^{\bullet}(\g, A) \cong_{\qis} \R\Hom_k(\Lambda^{\bullet}(\g), A)$$
                            which implies:
                                $$H^n(\g, A) \cong \Ext^n_k(\Lambda^{\bullet}(\g), A)$$
                        \end{proof}
                    We can now perform the aforementioned computations:
                    \begin{proposition}[Low-dimensional cohomologies of Lie algebras]
                        For $\g$ a Lie algebra over a commutative ring $k$ and $A$ any $\g$-module, one has the following interpretations of the low-dimensional cohomology spaces of $\g$ with coeffcients in $A$:
                        \begin{enumerate}
                            \item \textbf{($n = 0$: Invariants):} $H^0(\g, A) \cong \{a \in A \mid \forall x \in \g: x \cdot a = 0\}$.
                            \item \textbf{($n = 1$: Derivations):}
                            \item \textbf{($n = 2$: Extensions):}
                            \item \textbf{($n = 3$: Obstructions):}
                        \end{enumerate}
                    \end{proposition}
                        \begin{proof}
                            \noindent
                            \begin{enumerate}
                                \item \textbf{($n = 0$: Invariants):}
                                \item \textbf{($n = 1$: Derivations):}
                                \item \textbf{($n = 2$: Extensions):}
                                \item \textbf{($n = 3$: Obstructions):}
                            \end{enumerate}
                        \end{proof}
                    
                    Lemma \ref{lemma: de_rham_resolutions_of_lie_algebras} also help us makes sense of homologies of Lie algebras, defined as follows:
                    \begin{definition}[Lie algebra homologies] \label{def: lie_algebra_homologies}
                        Let $\g$ be a Lie algebra over a commutative ring $k$ and let $A$ be a $\g$-module. Then, we define the $n^{th}$ cohomology group of $\g$ with coefficient in $A$ as:
                            $$H_n(\g, A) \cong \Tor_n^{\U(\g)}(k, A)$$
                        where we view $k$ as the $1$-dimensional $\g$-representation on the right-hand side. 
                    \end{definition}
                    \begin{remark}
                        It is not hard to show, using the associativity of the derived tensor product, that one has the following quasi-isomorphism:
                            $$\Tor_{\bullet}^{\U(\g)}(k, A) \cong_{\qis} \Lambda^{\bullet}(\g) \tensor_k^{\L} A$$
                    \end{remark}
                
                \paragraph{Cohomology of semi-simple Lie algebras over characteristic \texorpdfstring{$0$}{} and Whitehead's two lemmas}
                    We begin with some algebraic prelimiaries. Specifically, we want to understand the notion of (semi)simplicity within the context of Lie theory, due to simple Lie algebras not quite being simple objects in categories of Lie algebras.
                    \begin{definition}[Subalgebras and ideals] \label{def: lie_subalgebras_and_lie_ideals}
                        
                    \end{definition}
                    
                    \begin{definition}[Simple Lie algebras] \label{def: simple_lie_algebras}
                        A Lie algebra internal to an appropriate tensor category $\O$ (cf. definition \ref{def: lie_algebras}) is \textbf{simple} if and only if it is a \textit{non-abelian} (i.e. the Lie bracket in question is not the zero morphism) simple object of $\Lie\Alg(\O)$.
                        
                        More algebraically, one might say that a simple Lie algebra is a non-abelian Lie algebra with no non-trivial proper ideal.
                    \end{definition}
                
                    Let us now move on to two important results concerning semi-simple Lie algebras, namely:
                        \begin{itemize}
                            \item the fact that finite-dimensional representations of semi-simple Lie algebras are completely reducible, and
                            \item that every Lie algebra which is finite-dimensional splits into the direct sum of its so-called \say{radical} and some semi-simple Lie algebra. 
                        \end{itemize}
                    During the way, we shall prove two technical lemmas, commonly known as Whitehead's First and Second Lemmas.
                    
                    \begin{convention} \label{conv: cohomology_of_semi_simple_lie_algebras_conventions}
                        Throughout this paragraph, $\g$ shall denote a finite-dimensional semi-simple Lie algebra over a field $k$ of characteristic $0$, and $A$ shall denote a $\g$-module that is of finite dimension as a $k$-vector space (i.e. a finite-dimensional $\g$-representation) with structural homomorphism:
                            $$\rho: \g \to \End_k(A)$$
                    \end{convention}
                    
                    \begin{proposition}[Associated invariant bilinear forms] \label{prop: associated_invariant_bilinear_forms_of_lie_algebras}
                        To $\g$, $A$, and $\rho$ as in convention \ref{conv: cohomology_of_semi_simple_lie_algebras_conventions}, there exists a symmetric bilinear form:
                            $$\beta: \g \tensor \g \to k$$
                        given by\footnote{Note that $\beta(x, y)$ should technically be written as $\beta(x \tensor y)$.}:
                            $$x \tensor y \mapsto \beta(x, y) := \trace(\rho(x) \rho(y))$$
                        Furthermore, such a bilinear form is $\ad_{\g}$-invariant, i.e.:
                            $$\beta([x, y], z) = \beta(x, [y, z])$$
                        for all $x, y, z \in \g$.
                    \end{proposition}
                        \begin{proof}
                            Bilinearity is a straightforward consequence of the linearity of $\rho$ (in the two factors $x$ and $y$) and the linearity of the trace map $\trace: \End_k(A) \to k$. As for symmetry, it is a consequence of the well-known fact that traces of endomorphisms on finite-dimensional vector spaces (whihc $\End_k(A)$ is, since $A$ is a finite-dimensional $k$-vector space) are invariant under cyclic permutations.
                            
                            Now, to prove that $\beta$ is $\ad_{\g}$-invariant, consider the following:
                                $$
                                    \begin{aligned}
                                        \beta([x, y], z) & = \trace(\rho([x, y]) \rho(z))
                                        \\
                                        & = \trace([\rho(x), \rho(y)] \rho(z))
                                        \\
                                        & = \trace((\rho(x)\rho(y) - \rho(y)\rho(x)) \rho(z))
                                        \\
                                        & = \trace(\rho(x) \rho(y)\rho(z)) - \trace(\rho(y)\rho(x)\rho(z))
                                        \\
                                        & = \trace(\rho(x) \rho(y)\rho(z)) - \trace(\rho(x)\rho(z)\rho(y))
                                        \\
                                        & = \trace(\rho(x)[\rho(y), \rho(z)])
                                        \\
                                        & = \beta(x, [y, z])
                                    \end{aligned}
                                $$
                        \end{proof}
                    \begin{example}[The Killing Form] \label{example: the_killing_form}
                        Since $\g$ is assumed to be finite-dimensional over $k$, one can meaningfully construct ymmetric bilinear forms in the fashion of proposition \ref{prop: associated_invariant_bilinear_forms_of_lie_algebras} for the case $A \cong \g$. In particular, when $\rho$ is the adjoint representation:
                            $$\ad_{\g}: \g \to \End_k(\g): x \mapsto \left(x \mapsto [x, -]: \g \to \g\right)$$
                        the associated symmetric bilinear form is the \textbf{Killing Form} $\kappa: \g \tensor \g \to k$, which is defined via:
                            $$\kappa(x, y) := \trace(\ad_{\g}(x) \ad_{\g}(y))$$
                    \end{example}
                    
                    \begin{theorem}[Non-degeneracy of associated invariant bilinear forms] \label{theorem: nondegeneracy_of_associated_invariant_bilinear_forms_of_lie_algebras}
                        The associated invariant bilinear form is non-degenerate when the representation $\rho: \g \to \End_k(A)$ is faithful.
                    \end{theorem}
                        \begin{proof}
                            
                        \end{proof}
                    \begin{corollary}
                        The adjoint representation is faithful, so the Killing Form is non-degenerate. 
                    \end{corollary}
                    
            \subsubsection{Structure and classification of compact Lie algebras}
    
        \subsection{Localisation of \texorpdfstring{$\g$}{}-modules; the Beilinson-Bernstein Equivalence}
            \begin{convention} \label{conv: beilinson_bernstein_localisation_conventions}
                \noindent
                \begin{itemize}
                    \item We work with a complex algebraic group $G$ of adjoint type with \textit{a priori} simple Lie algebra $\g$, along with universal enveloping algebra $\U(\g)$ and centre $\rmZ(\g)$ thereof. $B$ shall be a Borel subgroup thereof. 
                    \item $\Gr_G$ shall denote the loop affine Grassmannian $G(\!(t)\!)/G[\![t]\!]$ attached to $G$.
                    \item If $T$ is a maximal torus inside $G$ and $\lambda \in \bbX(T)$ is a weight then we shall denote by $\Dmod(G/B)^{(\lambda)}$ the category of $\lambda$-twisted D-modules on $G/B$ (i.e. D-modules on $G/B$ which act on the line bundle $G \x_B \lambda$), where $B \leq G$ is a choice of Borel subgroup that contains the fixed torus $T$.
                    \item The Lie algebras of $G, B$, and $T$ shall be denoted - respectively - by $\g, \b$, and $\t$.
                \end{itemize}
            \end{convention}
            
            In this subsection, we shall be presenting a proof by Gaitsgory and Frenkel of the Beilinson Bernstein Localisation Theorem, which shall then (in subsection \ref{subsection: localisation_of_affine_lie_algebras}) be generalised to obtain an analogous statement regarding localisations of modules over affine Lie algebras. Let us first jot down some initial observations and then give an outline of the idea behind the proof.
            
            Let $\pi: G \to G/B$ denote the canonical projection and observe that there exists a canonically determined sheaf pullback functor:
                $$\pi^*: \Dmod(G/B) \to \Dmod(G)$$
            which in turn induces a \say{twisted} pullback functor, compatible with the aforementioned pullback $\pi^*$ in some appropriate and natural sense:
                $$(\pi^{(\lambda)})^*: \Dmod(G/B)^{(\lambda)} \to \Dmod(G)$$
            Also, note the fact that for all $\calM \in \Dmod(G)$, the global section $\Gamma(G, \calM)$ has a natural $\U(\g)$-bimodule structure thanks to the self-actions of $G$ via left and right-translations. Among other things, this implies that there exists the following natural composition of functors:
                $$
                    \begin{tikzcd}
                    	{\Dmod(G/B}) & {\Dmod(G)} & {\U(\g)\bimod}
                    	\arrow["{\pi^*}", from=1-1, to=1-2]
                    	\arrow["{\Gamma(G, -)}", from=1-2, to=1-3]
                    \end{tikzcd}
                $$
            which induces the following composition wherein the first functor is now \say{twisted} by the character $\lambda$:
                $$
                    \begin{tikzcd}
                    	{\Dmod(G/B)^{(\lambda)}} & {\Dmod(G)} & {\U(\g)\bimod}
                    	\arrow["{(\pi^{(\lambda)})^*}", from=1-1, to=1-2]
                    	\arrow["{\Gamma(G, -)}", from=1-2, to=1-3]
                    \end{tikzcd}
                $$
            Set:
                $$\Gamma^{(\lambda)} := \Gamma(G, -) \circ (\pi^{(\lambda)})^*$$
            
            Now, let $\calV^{(\lambda)}$ denote the Verma module attached to a given weight $\lambda$ and set:
                $$\bfGamma^{(\lambda)} := \Hom_{\U(\g)}(\calV^{(\lambda)}, -) \circ \Gamma^{(\lambda)}$$
            Eventually, we will be establishing an adjunction:
                $$
                    \begin{tikzcd}
                    	{\Dmod(G/B)^{(\lambda)}} & \O
                    	\arrow[""{name=0, anchor=center, inner sep=0}, "{\bfGamma^{(\lambda)}}"', shift right=2, from=1-1, to=1-2]
                    	\arrow[""{name=1, anchor=center, inner sep=0}, "{\Delta^{(\lambda)}}"', shift right=2, from=1-2, to=1-1]
                    	\arrow["\dashv"{anchor=center, rotate=-90}, draw=none, from=1, to=0]
                    \end{tikzcd}
                $$
            wherein $\O$ is the so-called \say{Category $\O$} whose construction we shall get to\footnote{For now, think of $\O$ as a suitable subcategory of the category of right-$\U(\g)$-modules.}; here $\Delta^{(\lambda)}$ is given by:
                $$M \mapsto \D_{G/B} \tensor^{\L}_{\U(\g)} M$$
                
            All of this is simply to say that, the global section of the pullback along $\pi: G \to G/B$ of any D-modules on $G/B$ twisted by some character $\lambda$, remains unchanged under \say{twisting} by the Verma module $\calV^{(\lambda)}$. The adjunction $\Delta^{(\lambda)} \ladjoint \bfGamma^{(\lambda)}$ - as complicated as it may seem - is actually just an enhanced version of the usual tensor-hom adjunction.
                    
            \subsubsection{Kazhdan-Lusztig modules, the category \texorpdfstring{$\O$}{}, and Verma modules}
                    
            \subsubsection{Proving the theorem}
                The following geometric result shall help us make better sense of the Beilinson-Bernstein Localisation Theorem (cf. theorem \ref{theorem: beilinson_bernstein_localisation}). In fact, it is the true \say{localisation theorem}: the Beilinson-Bernstein Theorem simply refines the adjunction established by this results down to an adjoint equivalence (which, of course, is of great representation-theoretic significance)\footnote{Additionally, note how the notion of weight does not appear at all in theorem \ref{theorem: localisation_adjunction_for_D_modules}}.
                
                \begin{lemma}[The localisation adjunction for quasi-coherent associative algebras] \label{lemma: localisation_adjunction_for_quasi_coherent_associative_algebras}
                    Let $X$ be a scheme and consider $\calB \in \Assoc\Alg(\QCoh(X))$ with global section $B \cong \Gamma(X, \calB)$. Then, one has the following derived adjunction:
                        $$
                            \begin{tikzcd}
                            	{{}^l\calB\mod} & {B\bimod}
                            	\arrow[""{name=0, anchor=center, inner sep=0}, "{\R\Gamma(X, -)}"', shift right=2, from=1-1, to=1-2]
                            	\arrow[""{name=1, anchor=center, inner sep=0}, "{\calB \tensor_B^{\L} -}"', shift right=2, from=1-2, to=1-1]
                            	\arrow["\dashv"{anchor=center, rotate=-90}, draw=none, from=1, to=0]
                            \end{tikzcd}
                        $$
                \end{lemma}
                    \begin{proof}
                        By definition, $\R\Gamma(X, -) \cong \R\Hom_{{}^l\calB\mod}(\calB, -)$, so this is just the hom-tensor adjunction.
                    \end{proof}
                \begin{corollary} \label{coro: localisation_adjunction_for_quasi_coherent_associative_algebras}
                    If $B$ (as in lemma \ref{lemma: localisation_adjunction_for_quasi_coherent_associative_algebras}) is a bialgebra over some \say{deeper} base associative ring $B_0$ then the adjunction of lemma \ref{lemma: localisation_adjunction_for_quasi_coherent_associative_algebras} will extend down to:
                        $$
                            \begin{tikzcd}
                            	{{}^l\calB\mod} & {B_0\bimod}
                            	\arrow[""{name=0, anchor=center, inner sep=0}, "{\R\Gamma(X, -)}"', shift right=2, from=1-1, to=1-2]
                            	\arrow[""{name=1, anchor=center, inner sep=0}, "{\calB \tensor_{B_0}^{\L} -}"', shift right=2, from=1-2, to=1-1]
                            	\arrow["\dashv"{anchor=center, rotate=-90}, draw=none, from=1, to=0]
                            \end{tikzcd}
                        $$
                \end{corollary}
                \begin{theorem}[The localisation adjunction for D-modules] \label{theorem: localisation_adjunction_for_D_modules}
                    Let $X$ be a smooth algebraic variety with a $G$-action\footnote{$G$ still as in convention \ref{conv: beilinson_bernstein_localisation_conventions}} (which means that $G$ \textit{a priori} has the structure of an $X$-scheme through a structural morphism $\pi: G \to X$). Then, there exists the following derived adjunction:
                        $$
                            \begin{tikzcd}
                            	{\Dmod(X)} & {\U(\g)\bimod}
                            	\arrow[""{name=0, anchor=center, inner sep=0}, "{\bfGamma}"', shift right=2, from=1-1, to=1-2]
                            	\arrow[""{name=1, anchor=center, inner sep=0}, "{\D_X \tensor^{\L}_{\U(\g)} -}"', shift right=2, from=1-2, to=1-1]
                            	\arrow["\dashv"{anchor=center, rotate=-90}, draw=none, from=1, to=0]
                            \end{tikzcd}
                        $$
                    wherein the (derived) functor $\bfGamma: \Dmod(X) \to \U(\g)\bimod$ returns global sections of pullbacks of D-modules on $X$ to $G$, i.e. for all $\calM \in \Dmod(X)$, one has:
                        $$\bfGamma(\calM) \cong \Gamma(G, \pi^*\calM)$$
                \end{theorem}
                    \begin{proof}
                        Corollary \ref{coro: localisation_adjunction_for_quasi_coherent_associative_algebras} tells us that there exists the following adjunction:
                            $$
                                \begin{tikzcd}
                                	{\Dmod(G)} & {\U(\g)\bimod}
                                	\arrow[""{name=0, anchor=center, inner sep=0}, "{\R\Gamma(G, -)}"', shift right=2, from=1-1, to=1-2]
                                	\arrow[""{name=1, anchor=center, inner sep=0}, "{\D_G \tensor^{\L}_{\U(\g)} -}"', shift right=2, from=1-2, to=1-1]
                                	\arrow["\dashv"{anchor=center, rotate=-90}, draw=none, from=1, to=0]
                                \end{tikzcd}
                            $$
                        Next, consider the following pair of derived adjunctions:
                            $$
                                \begin{tikzcd}
                                	{\Dmod(X)} & {\Dmod(G)} & {\U(\g)\bimod}
                                	\arrow[""{name=0, anchor=center, inner sep=0}, "{\R\pi^*}"', shift right=2, from=1-1, to=1-2]
                                	\arrow[""{name=1, anchor=center, inner sep=0}, "{\D_X \tensor^{\L}_{\D_G} -}"', shift right=2, from=1-2, to=1-1]
                                	\arrow[""{name=2, anchor=center, inner sep=0}, "{\R\Gamma(G, -)}"', shift right=2, from=1-2, to=1-3]
                                	\arrow[""{name=3, anchor=center, inner sep=0}, "{\D_G \tensor^{\L}_{\U(\g)} -}"', shift right=2, from=1-3, to=1-2]
                                	\arrow["\dashv"{anchor=center, rotate=-90}, draw=none, from=1, to=0]
                                	\arrow["\dashv"{anchor=center, rotate=-90}, draw=none, from=3, to=2]
                                \end{tikzcd}
                            $$
                        They are trivially composable, and the resulting pair of functors is \textit{a priori} the sought-for derived adjunction:
                            $$
                                \begin{tikzcd}
                                	{\Dmod(X)} & {\U(\g)\bimod}
                                	\arrow[""{name=0, anchor=center, inner sep=0}, "{\bfGamma}"', shift right=2, from=1-1, to=1-2]
                                	\arrow[""{name=1, anchor=center, inner sep=0}, "{\D_X \tensor^{\L}_{\U(\g)} -}"', shift right=2, from=1-2, to=1-1]
                                	\arrow["\dashv"{anchor=center, rotate=-90}, draw=none, from=1, to=0]
                                \end{tikzcd}
                            $$   
                        thanks to the right/left-exactness (with respect to the canonical t-structures) of the composite functors (the composition of the two tensoring-up functors is trivially right-exact, and the composition $\R\Gamma(G, -) \circ \R\pi^*$ is left-exact since both factors are \textit{a priori} left-exact functors).  
                    \end{proof}
                \begin{corollary}[An application to flag varieties]
                    Since $G/B$ is a smooth variety, theorem \ref{theorem: localisation_adjunction_for_D_modules} specialises to the case of $X \cong G/B$.
                \end{corollary}
                
                It now remains to establish a functor from the category of $\U(\g)$-bimodules to the category $\O$, which shall be $\Hom_{\U(\g)}(\calV^{(\lambda)}, -)$ where $\calV^{(\lambda)}$ is the Verma module attached to a weight $\lambda$. 
                \begin{theorem}[The Beilinson-Bernstein Localisation Theorem] \label{theorem: beilinson_bernstein_localisation}
                    
                \end{theorem}
                    \begin{proof}
                        
                    \end{proof}
                \begin{corollary}[The Borel-Weil-Bott Theorem] \label{coro: borel_weil_bott_theorem}
                    
                \end{corollary}
        
        \subsection{Chiral D-modules on the affine Grassmannians and representations of Kac-Moody algebras} \label{subsection: localisation_of_affine_lie_algebras}
            \begin{convention} \label{conv: chiral_beilinson_bernstein_localisation_conventions}
                \noindent
                \begin{itemize}
                    \item We work with a complex algebraic group $G$ of adjoint type with \textit{a priori} simple Lie algebra $\g$, along with universal enveloping algebra $\U(\g)$ and centre $\rmZ(\g)$ thereof. $B$ shall be a Borel subgroup thereof. 
                    \item $\Gr_G$ shall denote the loop affine Grassmannian $G(\!(t)\!)/G[\![t]\!]$ attached to $G$.
                    \item If $T$ is a maximal torus inside $G$ and $\lambda \in \bbX(T)$ is a weight then we shall denote by $\Dmod(G/B)^{(\lambda)}$ the category of $\lambda$-twisted D-modules on $G/B$ (i.e. D-modules on $G/B$ which act on the line bundle $G \x_B \lambda$), where $B \leq G$ is a choice of Borel subgroup that contains the fixed torus $T$.
                    \item The Lie algebras of $G, B$, and $T$ shall be denoted - respectively - by $\g, \b$, and $\t$.
                \end{itemize}
            \end{convention}
            
            \begin{convention} \label{conv: loop_algebras_1}
                If $\g$ is a simple complex Lie algebra then we shall denote by $\g(\!(t)\!)$ the corresponding loop algebra, which is defined to be isomorphic to $\g \tensor_{\bbC} \bbC(\!(t)\!)$.
            \end{convention}
            
            \subsubsection{\textit{Pr\'elude}: Affine Kac-Moody algebras}
                We begin by recalling the following result, which classifies $\ad_{\g}$-invariant inner products on our given simple Lie algebra $\g$:
                \begin{lemma}[Invariant inner products on simple complex Lie algebras] \label{lemma: invariant_inner_products_on_simple_complex_lie_algebras}
                    On any given simple complex Lie algebra $\g$, there exists (up to non-zero scalar multiples) only one $\ad_{\g}$-invariant $\kappa$, namely the Killing Form, which is defined to be a linear map:
                        $$\kappa: \g \tensor \g \to \bbC$$
                    given by:
                        $$x \tensor y \mapsto \trace(\ad_{\g}(x) \ad_{\g}(y))$$
                    Furthermore, this space of invariant inner products is isomorphic (as a $\bbC$-vector space) to $H^2(\g(\!(t)\!), \bbC)$.
                \end{lemma}
                    \begin{proof}
                        
                    \end{proof}
                We shall also need to recall the general fact that extensions of a given Lie algebra $\g$ (for now, not necessarily simple), i.e. short exact sequences of $R$-modules of the form $0 \to M \to \hat{\g}_M \to \g \to 0$ by some $\g$-module $M$\footnote{Note that $M$ is necessarily a Lie algebra ideal of $\hat{g}_M$ and hence a Lie subalgebra of $\hat{g}$, due to Lie algebras being non-unital.}, are parametrised by $2$-cocyles $\kappa \in H^2(\g, M)$:
                \begin{lemma}[Extensions of Lie algebras are $2$-cocycles] \label{lemma: lie_algebra_extensions_are_2_cocycles}
                    Let $\g$ be an arbitrary Lie algebra over any commutative ring (cf. definition \ref{def: lie_algebras}) and let $M$ be a $\g$-module. Then, extensions $\hat{\g}_M$ of $\g$ by $M$ are in bijective correspondence with elements $\kappa \in H^2(\g, M)$.
                \end{lemma}
                    \begin{proof}
                        
                    \end{proof}
                \begin{corollary}
                    Central extensions of any Lie algebra $\g$ over a commutative ring $k$ are in bijection with elements of $H^2(\g, R)$.
                \end{corollary}
                    
                Now, by putting lemma \ref{lemma: invariant_inner_products_on_simple_complex_lie_algebras} and lemma \ref{lemma: lie_algebra_extensions_are_2_cocycles} together, one can meaningfully speak of central extensions of \say{central extensions of the loop algebra $\g(\!(t)\!)$ with respect to a given invariant inner product $\kappa$ on $\g$} in the following manner:
                \begin{definition}[Central extension with respect to invariant inner products] \label{def: central_extensions_with_respect_to_invariant_inner_products}
                    Let $\g$ be a simple complex Lie algebra and let $\kappa$ be an $\ad_{\g}$-invariant inner product thereon. Then, an \textbf{extension of $\g(\!(t)\!)$ with respect to $\kappa$} is nothing but a (central) extension $\hat{\g}_{\kappa}$ of $\g$ with respect to $\bbC$.
                    
                    Due to $H^2(\g(\!(t)\!), \bbC)$ being of dimension $1$ over $\bbC$, the extension $\hat{\g}_{\kappa}$ is actually unique up to scalar multiples, and thus is usually known as \textbf{\textit{the} affine Kac-Moody algebra associated to $\g$ of level $\kappa$}.
                \end{definition}
                
                Let us also state the following accompanying definition:
                \begin{definition}[Representations of the Kac-Moody algebra] \label{def: kac_moody_algebra_representations}
                    \noindent
                    \begin{enumerate}
                        \item 
                        \item 
                    \end{enumerate}
                \end{definition}
                
            \subsubsection{Towards a chiral version of the Beilinson-Bernstein Localisation Theorem}
            
            \subsubsection{The case of the Kac-Moody algebra at negative and at irrational levels}
            
            \subsubsection{The case of the Kac-Moody algebra at the critical level}
            
            \subsubsection{Global sections of chiral D-modules on the affine Grassmannian}
            
        \subsection{Localisation of \texorpdfstring{$\hat{\g}$}{}-modules}
            \begin{convention} \label{conv: quantum_beilinson_bernstein_localisation_conventions}
                \noindent
                \begin{itemize}
                    \item We work with a complex algebraic group $G$ of adjoint type with \textit{a priori} simple Lie algebra $\g$, along with universal enveloping algebra $\U(\g)$ and centre $\rmZ(\g)$ thereof. $B$ shall be a Borel subgroup thereof. 
                    \item $\Gr_G$ shall denote the loop affine Grassmannian $G(\!(t)\!)/G[\![t]\!]$ attached to $G$.
                    \item If $T$ is a maximal torus inside $G$ and $\lambda \in \bbX(T)$ is a weight then we shall denote by $\Dmod(G/B)^{(\lambda)}$ the category of $\lambda$-twisted D-modules on $G/B$ (i.e. D-modules on $G/B$ which act on the line bundle $G \x_B \lambda$), where $B \leq G$ is a choice of Borel subgroup that contains the fixed torus $T$.
                    \item The Lie algebras of $G, B$, and $T$ shall be denoted - respectively - by $\g, \b$, and $\t$.
                \end{itemize}
            \end{convention}
            
            \subsubsection{The Hecke Category}
            
            \subsubsection{The Main Theorem}
            
    \section{The Riemann-Hilbert Correspondence}
        
        \chapter{Chiral algebras; Opers}
    \begin{abstract}
        
    \end{abstract}
    
    \section{D-schemes}
        \subsection{Pseudo-tensor categories; compound tensor categories} \label{subsection: pseudo_and_compound_tensor_categories}
            We introduce the following terminology in keeping with \cite[Chapter 1]{beilinson2004chiral}
            \begin{definition}[Pseudo-tensor categories] \label{def: }
                A \textbf{pseudo-tensor category} is a coloured operad enriched in some (symmetric monoidal) category of vector spaces. 
            \end{definition}
    
        \subsection{The compound tensor structure on D-modules}
            In this subsection we establish how the category of D-modules on a smooth scheme possesses the structure of a compound tensor category. For details on pseudo-tensor categories and compound tensor structures, see subsection \ref{subsection: pseudo_and_compound_tensor_categories}.
    
        \subsection{The geometry of D-schemes}
    
    \section{Chiral algebras}
    
    \section{Opers and Miura Opers}
        \subsection{Opers}
            \subsubsection{The definition of opers}
                \begin{definition}[Opers] \label{def: opers}
                    
                \end{definition}
        
        \subsection{Miura Opers}
    
    \section{Wakimoto Modules}
    
    \section{\texorpdfstring{$q$}{}-opers}
        
        \chapter{Categorical harmonic analysis}
    \begin{abstract}
        
    \end{abstract}
    
    \minitoc
    
    \section{Some classical analysis}
        \subsection{The theory of topological vector spaces and functional analysis}
            In functional analysis, the concept of distributional density, usually just called \textbf{distribution} for short, is something that may be integrated against a bump function to produce a number (and hence a \say{mathematically analytic} version of the notion of physical density\footnote{Probability density functions, in fact, are special kinds of distribution.}). If a non-degenerate background density/volume form:
                $$d\vol$$
            is fixed, then each other density is a function relative to $d\vol$, and hence with such an identification understood distributional densities are generalized functions, namely objects that may arise as potentially singular limits of sequences of smooth functions (i.e. of non-singular distributions). Famous examples of such are the Dirac delta-\say{function} and the Heaviside distribution which behave like constant functions with an infinitely sharp \say{spike} or \say{kink}, respectively.

            Distributional densities appear notably as fundamental solutions to linear partial differential equations (such as for the wave equation and/or Klein-Gordon equation, whose fundamental solutions are the propagators of free quantum fields), which is the context in which the concept was originally introduced. The study of their singularity structure (encoded by their singular support and their wave front set) is a fundamental tool in PDE theory (for instance in the propagation of singularities theorem), known as microlocal analysis. Distributions are also fundamental in the rigorous construction of perturbative quantum field theory, where they appear in the variant as operator-valued distributions.
            
            Often distributions are considered by default just on open subsets of Euclidean space with its canonical volume form tacitly understood. But the concept of distributions makes sense more generally on general smooth manifolds (at least). If these are equipped with the structure of a (pseudo-)Riemannian manifold then the induced volume form again identifies distributions with generalised functions.
            
            \subsubsection{Generalities on topological vector spaces}
                We start with a reminder of the notion of (locally) convex topological vector spaces, that of Fr\'echet spaces, and then the notion of \textbf{LF-spaces}.
                
                The first technical notion is that of semi-norms and gauges.
                \begin{definition}[Semi-norms and gauge spaces] \label{def: seminorms_and_gauges}
                    \noindent
                    \begin{enumerate}
                        \item \textbf{(Pseudo-norms):} A \textbf{pseudo-metric space} is a pair $(X, d)$ is a pair consisting of:
                            \begin{itemize}
                                \item a set $X$,
                                \item a function $d: X \x X \to \R_{\geq 0}$ subjected to the following conditions:
                                    \begin{itemize}
                                        \item $\forall x \in X: d(x, x) = 0$,
                                        \item $\forall x, y \in X: d(x, y) = d(y, x)$,
                                        \item $\forall x, y, z \in X: d(x, z) \leq d(x, y) + d(y, z)$.
                                    \end{itemize}
                            \end{itemize}
                        By requiring furthermore that $d(x, y) > 0$ for all pairs of distinct points $x \not y$, one gets the familiar notion of \textbf{metric spaces}.
                        
                        If $X$ is a vector space, then the $d: X \x X \to \R_{\geq 0}$ shall be known as a \textbf{semi-norm}.
                        \item \textbf{(Gauge spaces):} A \textbf{gauge} is a \textit{countable} direct set $\{d_i\}_{i \in I}$ of pseudo-metrics on a given set $X$. 
                        
                        A \textbf{gauge space} $(X, \del)$ is a pair consisting of a set $X$ (typically, a vector space) and a gauge $\del$.
                    \end{enumerate}
                \end{definition}
                \begin{example}
                    Consider the countable directed set $\{(\R^n, d_n)\}_{n \in \N}$ of finite-dimensional real vector spaces equipped with their canonical Euclidean norms. Its filtered limit taken in the category of topological spaces:
                        $$\R^{\infty} \cong \underset{n \in \N}{\lim} \R^n$$
                    then carries a natural gauge:
                        $$d_{\infty}: \R^{\infty} \x \R^{\infty} \to \R_{\geq 0}$$
                    which is naturally the limit of norms thanks to some abstract nonsense in the category of topological spaces, i.e.:
                        $$d_{\infty} \cong \underset{n \in \N}{\lim} d_n$$
                    More succinctly, one can define the gauge space $(\R^{\infty}, d_{\infty})$ as:
                        $$(\R^{\infty}, d_{\infty}) \cong \underset{n \in \N}{\lim} (\R^n, d_n)$$
                    where we take the limit in the category of metric spaces (recall that metrics are \textit{a priori} continuous functions).
                \end{example}
                
                \begin{proposition}[Topologies induced by pseudo-metrics and gauges] \label{prop: topologies_induced_by_pseudo_metrics_and_gauges}
                    Every pseudo-metric induces a topology, and so does every gauge.
                \end{proposition}
                    \begin{proof}
                       \noindent
                       \begin{enumerate}
                           \item 
                           \item 
                       \end{enumerate}
                    \end{proof}
                
                Next, we introduce the notion of locally convex topological vector spaces.
                \begin{remark}
                    Let us first remind ourselves with the elementary fact that the category of \textbf{topological vector spaces} (over some fixed topological field $K$) and continuous linear maps between them is the same as the category of internal $K$ vector spaces in $\Top$. When $K$ is understood (as is the case for us), we shall write $\TVS$ for this category. 
                    
                    One crucial thing to note is that this category does not have a well-behaved monoidal structure. For instance, it is not clear how one would topologise the $\Q$-vector space\footnote{Spoilers: the answer is the condesned formalism; cf. section \ref{section: condensed_mathematics}.}:
                        $$\R \tensor_{\Q} \Q_p$$
                    wherein $\Q$ is endowed with the usual Archimedean norm. 
                \end{remark}
                
                \begin{definition}[Locally convex topological vector space] \label{def: locally_convex_topological_vector_spaces}
                    A \textbf{locally convex topological vector space} is a Hausdorff gauge space. 
                \end{definition}
                
                \begin{definition}[Fr\'echet spaces] \label{def: frechet_spaces}
                    A \textbf{Fr\'echet space} is a complete Gauge space.
                \end{definition}
                
                \begin{definition}[LF-spaces] \label{def: LF_spaces}
                    An \textbf{LF-space} is a \textit{countable} directed colimit of \textit{locally convex} topological vector spaces.
                \end{definition}
                
            \subsubsection{Test functions and distributions}
                Let us first write down the definition of the space of distribution on a pre-determined open subset $X \subseteq \R^n$, which shall be a filtered colimit of Fr\'echet spaces endowed with a natural final topology.
                \begin{definition}[Distributions] \label{def: distributions}
                    
                \end{definition}
            
            \subsubsection{Spaces of distributions}
        
            \subsubsection{Sobolev spaces}
            
        \subsection{Abelian harmonic analysis}
    
    \section{The Fourier-Mukai-Laumon Transform}
    
    \section{The Global Correspondence for tori}
        
        \chapter{Categorical traces, shtukas, and automorphic sheaves} \label{chapter: automorphic_forms}
    \begin{abstract}
        
    \end{abstract}
    
    \section{How to invent shtukas}
        \subsection{Traces of Frobenii}
            \subsubsection{Traces of endomorphisms in symmetric monoidal categories}
                Let us start by reviewing the notion of the trace of an endomorphism in a symmetric monoidal category. For this, let us first fix a symmetric monoidal category $(\O, \tensor, \1)$ along with a dualisable object $o \in \O$ therein. Additionally, consider an endormophism:
                    $$T: o \to o$$
                Then, the trace of $T$ on $o$, which is an element of $\End_{\O}(\1)$ and denoted by $\trace(T, o)$, is defined to be the following composition:
                    $$
                        \begin{tikzcd}
                        	\1 & {o \tensor o^{\vee}} & {o \tensor o^{\vee}} & \1
                        	\arrow["{T \tensor \id_{o^{\vee}}}", from=1-2, to=1-3]
                        	\arrow["\ev_o", from=1-3, to=1-4]
                        	\arrow["\co\ev_o", from=1-1, to=1-2]
                        \end{tikzcd}
                    $$
                If we were to now consider $\O \cong \dg\Cat_1^{\cont}$, the $1$-category of dg-categories and continuous functors between them, equipped with the natural monoidal structure given via the Lurie tensor product along with the dualisable dg-category $\Dmod(\calY)$ of D-modules on some quasi-compact algebraic stack $\calY$ with affine-schematic diagonal, then one can consider traces of endofunctors:
                    $$F^!: \Dmod(\calY) \to \Dmod(\calY)$$
                (which shall come from endomorphisms $F: \calY \to \calY$, but more on that later) as compositions of functors as follows:
                    $$
                        \begin{tikzcd}
                        	\Vect & {\Dmod(\calY) \tensor \Dmod(\calY)^{\vee}} & {\Dmod(\calY) \tensor \Dmod(\calY)^{\vee}} & \Vect
                        	\arrow["{F^! \tensor \id}", from=1-2, to=1-3]
                        	\arrow["\ev", from=1-3, to=1-4]
                        	\arrow["\co\ev", from=1-1, to=1-2]
                        \end{tikzcd}
                    $$
                Here, the evaluation functor $\ev = \ev_{\Dmod(\calY)}$ is given by the composition:
                    $$
                        \begin{tikzcd}
                        	{\Dmod(\calY) \tensor \Dmod(\calY)} & {\Dmod(\calY \x \calY)} & {\Dmod(\calY)} & \Vect
                        	\arrow["\boxtimes", from=1-1, to=1-2]
                        	\arrow["{\Delta_{\calY}^!}", from=1-2, to=1-3]
                        	\arrow["{H^*(\calY, \omega_{\calY})}", from=1-3, to=1-4]
                        \end{tikzcd}
                    $$
            
            \subsubsection{Action of local systems}
            
            \subsubsection{Hecke actions}
            
        \subsection{Traces, but better}
        
    \section{Automorphic forms as traces of Frobenii}
        
    \part{The Arithmetic Langlands Correspondence}
        \chapter{Introduction}
    \begin{abstract}
        
    \end{abstract}
    
    \minitoc
        
        \chapter{Class field theory}
    \begin{abstract}
        
    \end{abstract}
    
    \minitoc
    
    \section{Local class field theory}
        \subsection{Local class field theory via Galois cohomology}
            \subsubsection{Cohomological duality for profinite groups}
            
            \subsubsection{Abstract class field theory}
            
            \subsubsection{Iwasawa theory}
    
        \subsection{The \texorpdfstring{$p$}{}-adic Local Langlands Correspondence for \texorpdfstring{$\GL_1$}{}}
            \subsubsection{Pro-algebraic groups and their homotopy groups}
                Local class field theory in the sense of Serre-Hazewinkel-Suzuki-Yoshida is the realisation that certain important constructions surrounding Galois groups of local fields (namely, the Galois group of the maximal abelian extension of a given non-archimedean discrete valuation field), such as the inertia group and so on, are actually just fundamental groups of particular infinite-dimensional group schemes, called \textbf{pro-algebraic groups} (and as the name suggests, are objects of the pro-completion of the category of algebraic groups). The issue, however, is that algebraic fundamental group (cf. definition \ref{def: etale_fundamental_groups}) are only commonly within the finite-dimensional context. Thus, we shall need some sort of profinite analogue of the usual \'etale fundamental group in order to meaningfully our desired results.
                
                \begin{definition}[Pro-algebraic groups] \label{def: pro_algebraic_groups}
                    The category of \textbf{pro-algebraic groups} (over some fixed base field $k$) is nothing but the pro-completion of the category of algebraic groups over $\Spec k$ (cf. definition \ref{def: algebraic_groups}). We shall suggestively denote it by $\Pro\Alg\Grp_{/\Spec k}$.
                \end{definition}
                \begin{remark}[Quasi-algebraic groups ?]
                    Since pro-algebraic groups arise most commonly over fields of characteristic $p > 0$, many authors use the term \textbf{pro-algebraic group} to mean objects of the category:
                        $$\Pro(\Alg\Grp_{/\Spec k^{\flat}}^{\perf})$$
                    which arises most naturally as the essential image of the schematic tilting functor:
                        $$(-)^{\flat}: \Pro(\Alg\Grp_{/\Spec k}) \to \Pro(\Alg\Grp_{/\Spec k^{\flat}})$$
                    defined in definition \ref{def: tilting_schemes}. 
                \end{remark}
                
                Nevertheless, one should verify that tilts of pro-algebraic groups are still pro-algebraic groups.
                \begin{lemma}[Tilts of pro-algebraic groups] \label{lemma: tilts_of_pro_algebraic_groups}
                    Let $k$ be a field of characteristic $p > 0$ on which the Frobenius is surjective and let $\calG$ be a pro-algebraic group over $\Spec k$. Then, $\calG^{\flat}$ is also a pro-algebraic group (but of course, over $\Spec k^{\flat}$). Furthermore, it is perfect, in the sense that the (geometric) Frobenius $\Frob_{\calG/k}$ is invertible.
                \end{lemma}
                    \begin{proof}
                        The first assertion is entirely formal. The second assertion comes from lemma \ref{lemma: tilts_are_perfect}.
                    \end{proof}
                \begin{remark}
                    Thanks to lemma \ref{lemma: tilts_of_pro_algebraic_groups}, we shall assume most of the time that the base field $k$ is perfect (as a field of characterisitic $p > 0$).
                \end{remark}
                
                \begin{convention}
                    From now on, denote by $\Pro\Alg\Ab_{/\Spec k}$ the category of pro-algebraic \textit{commutative} groups over $\Spec k$, where $k$ will commonly be some base field of characteristic $p > 0$ on which the Frobenius is surjective.
                \end{convention}
                \begin{proposition}[The coGrothendieck category of pro-algebraic commutative groups] \label{prop: the_cogrothendieck_category_of_pro_algebraic_commutative_groups}
                    Fix a perfect base field $k$ of characteristic $p > 0$. Then:
                        \begin{enumerate}
                            \item $\Pro\Alg\Ab_{/\Spec k}$ is an abelian category,
                            \item $\Pro\Alg\Ab_{/\Spec k}$ is an $AB5$-category, i.e. it is cocomplete and filtered colimits of exact sequences are once more exact,
                            \item and $\Pro\Alg\Ab_{/\Spec k}$ has a \textit{projective} cogenerator.
                        \end{enumerate}
                \end{proposition}
                    \begin{proof}
                        \noindent
                        \begin{enumerate}
                            \item It is known that for any category $\C$ with enough limits then $\Ab(\C)$ is abelian if and only if $\C$ is exact, in the sense that every equivalence is a kernel pair. This is certainly the case for $\C \cong \Pro(\Sch^{\ft}_{/\Spec k})$, and because $\Pro\Alg\Ab_{/\Spec k} \cong \Ab(\Pro(\Sch^{\ft}_{/\Spec k}))$, we have shown that $\Pro\Alg\Ab_{/\Spec k}$ is an abelian category.
                            \item 
                            \item 
                        \end{enumerate}
                    \end{proof}
                
                \begin{convention}
                    From now on, we work with a complete non-archimedean field $K$ equipped with a discrete rank-$1$ valuation $v$. The subring of bounded elements shall be denoted by $\scrO_K$, and $\m_K$ shall be the maximal ideal of topologically nilpotent elements. Finally, denote the residue field $\scrO_K/\m_K$ by $k$; we shall work under the assumption that $k$ is known to be perfect as a field of characteristic $p > 0$.
                \end{convention}
                \begin{convention}
                    We fix the following notation:
                        $$\U_K \cong \underset{n \in \N}{\lim} (\G_m)_{/\scrO_K/\m_K^{n + 1}}$$
                    and recall that the groups of units:
                        $$(\G_m)_{/\scrO_K/\m_K^{n + 1}}$$
                    are represented by the affine schemes:
                        $$\Spec \scrO_K/\m_K^{n + 1}[t, 1/t]$$
                \end{convention}
                \begin{proposition}[Properties of $\U_K$] \label{prop: properties_of_the_pro_algebraic_group_of_units}
                    $\U_K$ is a commutative pro-algebraic group over $\Spec k$ such that:
                        $$\U_K(k) \cong K^{\x}$$
                \end{proposition}
                    \begin{proof}
                        $\U_K$ being a commutative pro-algbebraic group over $\Spec k$ is an obvious consequence of its construction (note that the component groups:
                            $$(\G_m)_{/\scrO_K/\m_K^{n + 1}}$$
                        are all commutative), so the only thing to show is that $\U_K(k) \cong K^{\x}$. For this, we can use the fact that the component groups:
                            $$(\G_m)_{/\scrO_K/\m_K^{n + 1}}$$
                        are represented by the affine schemes:
                            $$\Spec \scrO_K/\m_K^{n + 1}[t, 1/t]$$
                        and the fact that the Yoneda embedding preserves limits to see that:
                            $$\U_K(k) \cong \underset{n \in \N}{\lim} (\G_m)_{/\scrO_K/\m_K^{n + 1}}(k) \cong \underset{n \in \N}{\lim} (\scrO_K/\m_K^{n + 1})^{\x}$$
                    \end{proof}
            
            \subsubsection{Local class field theory \textit{\`a la} Serre-Hazelwinkel}
                \begin{convention}
                    From now on, we work with a complete non-archimedean field $K$ equipped with a discrete rank-$1$ valuation $v$. The subring of bounded elements shall be denoted by $\scrO_K$, and $\m_K$ shall be the maximal ideal of topologically nilpotent elements. Finally, denote the residue field $\scrO_K/\m_K$ by $k$; we shall work under the assumption that $k$ is known to be perfect as a field of characteristic $p > 0$.
                \end{convention}
                
                The story of classical local class field theory (which started in the 19th century and culminated in the 1930s with a theorem of Artin) is that should $k$ be a finite extension of $\F_p$ (i.e. $k \cong \F_q$ for $q$ some power of $p$), then one has a canonical group homomorphism:
                    $$K^{\x} \to \Gal(K^{\ab}/K)$$
                (where $K^{\ab}/K$ is the maximal abelian extension of $K$) inducing the following commutative diagram in $\Ab$ with exact rows:
                    $$
                        \begin{tikzcd}
                        	0 & {k^{\x}} & {K^{\x}} & \Z & 0 \\
                        	0 & {\rmI(K)} & {\Gal(K^{\ab}/K)} & {\Gal(\bar{k}/k)} & 0
                        	\arrow[from=1-1, to=1-2]
                        	\arrow[tail, from=1-2, to=1-3]
                        	\arrow[two heads, from=1-3, to=1-4]
                        	\arrow[from=1-4, to=1-5]
                        	\arrow[from=2-1, to=2-2]
                        	\arrow[tail, from=2-2, to=2-3]
                        	\arrow[two heads, from=2-3, to=2-4]
                        	\arrow[from=2-4, to=2-5]
                        	\arrow[from=1-2, to=2-2]
                        	\arrow[from=1-3, to=2-3]
                        	\arrow[from=1-4, to=2-4]
                        \end{tikzcd}
                    $$
                Serre then went on to generalise this programme to cases where $k$ need not be a finite extension of $\F_p$, but he had to make the assumption that $k$ was algebraically closed (so for instance, $k$ could be $\overline{\F_p(t)}$ but not $\F_p(\!(t)\!)$ or even $\F_p(\!(t^{1/p^{\infty}})\!)$). Hazelwinkel and Suzuki-Yoshida (cf. \cite{suzuki_yoshida_lcft_refinement}) then managed to generalised this to the case where $k$ could be any perfect field (so now, for instance, one can consider $k \cong \F_p(\!(t^{1/p^{\infty}})\!)$).
            
            \subsubsection{The \texorpdfstring{$p$}{}-adic Local Langlands Correspondence for \texorpdfstring{$\GL_1$}{}}
        
        \subsection{The Fargues-Fontaine Curve} \label{section: the_fargues_fontaine_curve}
            In this section, we will motivate and introduce the construction of the Fargues-Fontaine Curve. We will also touch on its applications, not just to $p$-adic Hodge theory, but also to the larger $p$-adic Local Langlands Correspondence (in the sense of Fargues-Scholze; cf. \cite{fargues_scholze_geometrization_of_local_langlands}).
        
            \subsubsection{Construction of The Fargues-Fontaine Curve}
                The following description of the Fargues-Fontaine Curve, aside from detailing its applications to Local Class Field Theory, will also serve as a continuation of section \ref{section: perfectoid_spaces}, for The Curve can be viewed as the moduli space of untilts a given algebraically closed perfectoid field of positive characteristic. Specifically, what we want to know are perfectoid fields $E$ of mixed characteristics $(0, p)$ such that $E^{\flat} \cong F$ for some prescribed perfectoid field $F$ of characteristic $p$. Also, let us note that this is in no way an ill-posed question, as for instance, we know that the tilts of both $\Q_p(p^{\frac{1}{p^{\infty}}})^{\wedge}$ and $\Q_p(\mu_{p^{\infty}})^{\wedge}$ are isomorphic to $\F_p(\!(t^{\frac{1}{p^{\infty}}})\!)$. The Fargues-Fontaine Curve, which we shall denote by $X_{\FF}$, should thus be some sheaf of sets on $\Perfd_{/\Spec \F_p}$ whose fibres over adic spectra of algebraic closures of adic residue fields (understood as \textit{geometric points}) shall be sets of untilts of that very algebraically closed field.
                
                We would like our to-be adic Fargues-Fontaine Curve to be a curve over a base field that is:
                    \begin{itemize}
                        \item projective, so it would have a chance of admitting a GAGA-esque functor,
                        \item smooth, so that its topology would behave nicely (read: so that we would be able to apply \'etale cohomology) and so important cases such as elliptic curves would be covered, 
                        \item proper, so the Proper Base Change Theorem from the theory of \'etale cohomology would apply 
                    \end{itemize}
                and most importantly, such that it would have a very straight forward connection to the Weil group of a $p$-adic number field because then, certain local systems on The Curve would correspond with the Weil-Deligne representations of the aforementioned Weil group, which is something that we would want for the establishment of the Galois/spectral side of the Local Langlands Correspondence. Also, due to this last point, one should imagine the Fargues-Fontaine Curve as a sort of $p$-adic Riemann surface; in fact, we will see (eventually) that it does behave similarly to Riemann surfaces in many ways, particularly via its connection to $p$-adic Shimura varieties.
            
            \subsubsection{A GAGA theorem for vector bundles on The Curve}
        
    \section{Global class field theory for function fields}
        \subsection{Grothendieck's Galois Theory} \label{subsection: grothendieck_galois_theory}
            \subsubsection{\'Etale fundamental groups of schemes}
                \begin{definition}[Noohi groups] \label{def: fintie_galois_categories}
                    \noindent
                    \begin{enumerate}
                        \item \textbf{(Finite Galois categories \cite[\href{https://stacks.math.columbia.edu/tag/0BMY}{Tag 0BMY}]{stacks}):} A \textbf{finite Galois category} is defined via the data contained in a pair $(\calG, F)$ consisting of an \textit{exact} functor $F: \calG^{\op} \to \Sets^{\fin}$ and a category $\calG$ such that:
                            \begin{itemize}
                                \item $\calG$ is finitely complete and finitely cocomplete.
                                \item Objects in $\calG$ can all be written as a (possibly empty but necessarily finite) coproduct of connected objects (objects $X \in \calG$ such that the functor $\calG(X, -)$ preserves all coproducts). 
                            \end{itemize}
                        Functors such as the functor $F$ above are commonly called \textbf{fibre functors}. 
                        \item \textbf{(Noohi groups):} In the sense of \cite[Theorem 2.16]{noohi_fundamental_group}, a so-called \textbf{Noohi group} is the group of natural automorphisms on the $\Sets^{\fin}$-valued functor defining a Galois finite category; that is to say, given a Galois finite category $(\calG, F)$, its Noohi group is $\Aut(F)$.  
                    \end{enumerate}
                \end{definition}
                
                \begin{lemma}[Profiniteness of Noohi groups] \label{lemma: profiniteness_of_noohi_groups}
                    Let $(\calG, F)$ be a finite Galois category. Then:
                        \begin{enumerate}
                            \item The associated Noohi group $\Aut(F)$ is profinite.
                            \item $\calG$ is equivalent to the category $[\bfB\Aut(F)^{\op}, \Sets^{\fin}]$ of $\Aut(F)$-equivariant finite sets.
                        \end{enumerate}
                \end{lemma}
                    \begin{proof}
                        \cite[Theorem 2.16]{noohi_fundamental_group}
                    \end{proof}
                    
                \begin{definition}[\'Etale fundamental group] \label{def: etale_fundamental_groups}
                    Recall first of all that for any given base scheme $X$, the category $\Sch_{/X, \fet}$ of schemes finite and \'etale over $X$ is a category wherein:
                        \begin{itemize}
                            \item all finite limits and all finite colimits exist, and
                            \item all objects can be written as a (possibly empty) finite coproduct of connected objects, which happen to be schemes that are \'etale over $X$.  
                        \end{itemize}
                    In other words, the category spanned by (possibly empty) finite coproducts of schemes \'etale over $X$ can serve as the underlying category of a finite Galois category. Let us then fix a geometric point:
                        $$\overline{x}: \Spec \overline{\kappa_x} \to X$$
                    (where $\overline{\kappa_x}$ denotes an algebraic closure of the residue field $\kappa_x$ at some point $x \in |X|$) of $X$ and define the following fibre functor:
                        $$F_{\overline{x}}: \Sch_{/X, \fet} \to \Sets^{\fin}$$
                    by the rule:
                        $$F_{\overline{x}}(f: Y \to X) \cong |Y \x_{f, X, \overline{x}} \Spec \overline{\kappa_x}|$$
                    The pair $(\Sch_{/X, \fet}, F_{\overline{x}})$ as above thus define a finite Galois category. Its Noohi group $\Aut(F_{\overline{x}})$ is commonly denoted by $\pi_1^{\fet}(X, \overline{x})$.
                \end{definition}
                \begin{remark}
                    Definition \ref{def: etale_fundamental_groups} is actually a bit subtle and honestly, somewhat ill-founded, as did not actually prove that $F_{\overline{x}}$ was an honest-to-Grothendieck fibre functor. It is certainly left-exact, by virtue of being defined via pullbacks, and it is right-exact because any \'etale algebra over a field can be written as a finite direct sum of finite extensions of that field \cite[\href{https://stacks.math.columbia.edu/tag/00U3}{Tag 00U3}]{stacks}, and direct sums are biproducts of vector spaces. However, the fact that the sets $|Y \x_{f, X, \overline{x}} \Spec \overline{\kappa_x}|$ are finite is not really trivial, although it is not too hard to prove either. Basically, this fact is also consequence \'etale algebras being isomorphic to finite direct sums of finite extensions: in our case, since $\overline{\kappa_x}$ is algebraically closed, the underlying vector space of \'etale $\overline{\kappa_x}$-algebras must be isomorphic to a finite direct sum of $\overline{\kappa_x}$ itself. In terms of schemes, this means that when both $Y$ and $X$ are affine, the pullback $Y \x_{f, X, \overline{x}} \Spec \overline{\kappa_x}$ would be nothing but a coproduct of finitely many copies of $\Spec \overline{\kappa_x}$, and hence the set $|Y \x_{f, X, \overline{x}} \Spec \overline{\kappa_x}|$ would have to be finite. Then, by using the fact the \'etale-ness is a local property, we can deduce that the set $|Y \x_{f, X, \overline{x}} \Spec \overline{\kappa_x}|$ must be finite regardless of whether $Y$ and $X$ are finite or not. The functor:
                        $$F_{\overline{x}}: \Sch_{/X, \fet} \to \Sets^{\fin}: (f: Y \to X) \mapsto |Y \x_{f, X, \overline{x}} \Spec \overline{\kappa_x}|$$
                    is therefore indeed a fibre functor.
                \end{remark}
                
                \begin{theorem}[Grothendieck's Galois theory] \label{theorem: grothendieck's_galois_theorem}
                    Let $X$ be a \textit{connected} base scheme and let:
                        $$\overline{x}: \Spec \overline{\kappa_x} \to X$$
                    be a geometric point therein. By lemma \ref{lemma: profiniteness_of_noohi_groups}, we have the following equivalence of categories:
                        $$\Sch_{/X, \fet} \cong [\bfB\pi_1^{\fet}(X, \overline{x})^{\op}, \Sets^{\fin}]$$
                    But this purely topological equivalence can be upgraded to an algebraic one via the the following canonical isomorphism of groups:
                        $$\pi_1^{\fet}(X, \overline{x}) \cong \Gal(k^{\sep}/k)$$
                    wherein $k^{\sep}$ is the unique separable closure inside $\overline{k}$, which holds if and only if $X$ is the spectrum of some field $k$ (note that in such a situation, the geometric point $\overline{x}$ is nothing but the canonical morphism $\Spec \overline{k} \to \Spec k$).
                \end{theorem}
                    \begin{proof}
                        
                    \end{proof}
                    
                \begin{theorem}[\'Etale fundamental groups are unique up to universal homeomorphisms] \label{theorem: etale_fundamental_groups_are_unique_up_to_universal_homeomorphisms}
                    Let $f: Y \to X$ be a universal homeomorphism. Then, one has the following equivalence of categories:
                        $$\Sch_{/X}^{\fet} \cong \Sch_{/Y}^{\fet}: (j: U \to X) \mapsto U \x_{j, X, f} Y$$
                    which in particular, implies that for any geometric point $\overline{x}$ of $X$, there is an isomorphism of \'etale fundamental groups:
                        $$\pi_1^{\fet}(X, \overline{x}) \cong \pi_1^{\fet}(Y, \overline{y})$$
                    where $\overline{y}$ is the geometric point of $Y$ lying over $\overline{x}$ (it is uniquely determined as $f: Y \to X$ is a universal homemomorphism). 
                \end{theorem}
                    \begin{proof}
                        
                    \end{proof}
                \begin{corollary}
                    Let $X \to X'$ be a morphism of schemes which is a homeomorphism at the level of the underlying topological spaces and enjoys the universal property of a  filtered colimit or that of a limit. This is a special case of a universal homeomorphism, and one thus has:
                        $$\pi_1^{\fet}(X) \cong \pi_1^{\fet}(X')$$
                    Examples include but certainly not limited to the following:
                        \begin{itemize}
                            \item $X \to X'$ is a thickening.
                            \item 
                        \end{itemize}
                \end{corollary}
                
            \subsubsection{\texorpdfstring{$\ell$}{}-adic sheaves and Galois representations} \label{subsubsection: l_adic_sheaves}
                Let us start with the notion of $\ell$-adic representations. 
                \begin{definition}[$\ell$-adic representations] \label{def: l_adic_representations}
                    Let $K$ be a field and let $L/K$ be a Galois extension thereof. Additionally, let $F$ be a local field (we shall view finite fields as $0$-dimensional local fields) equipped with its natural topology (e.g. non-archimedean when $F$ is some sort of $\ell$-adic number field, archimedean when $F$ is $\R$ or $\bbC$, and discrete when $F$ is finite); also, we shall require that $\ell \not = \chara K$. An \textbf{$\ell$-adic representation of $\Gal(L/K)$} is thus a finite-dimensional \textit{continuous} $F$-linear representation of $\Gal(L/K)$, i.e. a continuous group homomorphism:
                        $$\rho: \Gal(L/K) \to \GL_n(F)$$
                    for some natural number $n$. $\ell$-adic representations of \textit{absolute} Galois groups are known as \textbf{$\ell$-adic Galois representations}, or just Galois representations for short.
                \end{definition}
                \begin{remark}[It's actually a bit simpler than we've been led to believe]
                    Definition \ref{def: l_adic_representations} can seem a bit complicated, but what it actually does is just giving names to certain continuous finite-dimensional $F$-linear representations of certain topological groups (recall how Galois groups naturally carry the profinite topology which reduces to the discrete topology in finite cases). The category of $\ell$-adic representations of a given Galois group is thus nothing special, from a categorical point of view, and a lot of the basic properties of $\ell$-adic representations are actually just abstract-nonsensical. 
                \end{remark}
                \begin{example} \label{example: l_adic_representations}
                    Let $K$ be a field and let $L/K$ be a Galois extension thereof. Additionally, let $F$ be a local field (we shall view finite fields as $0$-dimensional local fields) equipped with its natural topology (e.g. non-archimedean when $F$ is some sort of $\ell$-adic number field, archimedean when $F$ is $\R$ or $\bbC$, and discrete when $F$ is finite); also, we shall require that $\ell \not = \chara K$.
                    \begin{enumerate}
                        \item \textbf{($\ell$-adic representations that are not Galois):} 
                            \begin{itemize}
                                \item \textbf{(The trivial representation):} This is a bit of a silly example, but if $F$ were to be equipped with the discrete topology then any finite-dimensional $F$-linear representation of $\Gal(L/K)$ would be an $\ell$-adic representation for trivial reasons. Note that the trivial representation is a special case of this, since the trivial subgroup $1 \leq \GL_n(F)$ can not have any topology other than the discrete one. 
                                \item \textbf{(Finite Galois representations):} Any finite-dimensional $F$-linear represetation of a finite Galois group is trivially $\ell$-adic, due to the fact that every subset is defined to be open in the discrete topology. 
                            \end{itemize}
                        \item \textbf{(Galois representations):}
                            \begin{itemize}
                                \item \textbf{(Tate modules):} \index{Tate module} \index{Tate twist} Let $X$ be an abelian variety over $\Spec K$ and let us write $(\mu_{\ell^{\infty}})_{/X}$ for the base change:
                                    $$(\mu_{\ell^{\infty}})_{/\Spec K} \x_{\Spec K} X$$
                                of the $\Spec K$-algebraic group:
                                    $$(\mu_{\ell^{\infty}})_{/\Spec K} \cong \underset{n \in \N}{\lim} \Spec \frac{K[x]}{(x^{\ell^n} - 1)}$$
                                of all $\ell^{th}$-roots of unity to $X$, which is once more an commutative group scheme for trivial reasons. Then, the \textbf{$\ell$-adic Tate module} of $X$ is the abelian group of $\Spec K^{\sep}$-points of $\mu_{\ell^{\infty} /X}$; we shall denote it by $\T_{\ell}(X)$. Alternatively, one might define the $\ell$-adic Tate module of $X$ to be the filtered limit of all $\ell$-torsion subgroups of $X(\Spec K^{\sep})$, which form the following descending filtration: 
                                    $$X[\ell] \supset X[\ell^2] \supset ... \supset \T_{\ell}(X)$$
                                wherein $X[\ell^n] \cong X(\Spec K^{\sep}) \tensor_{\Z} \Z/\ell^n\Z$ is the subgroup with $\ell^n$-torsion.
                                    
                                Because $\ell$ is prime, $\T_{\ell}(X)$ is thus an abelian pro-$\ell$-group (i.e. a filtered limit of finite abelian $\ell$-groups), and hence isomorphic to a free $\Z_{\ell}$-module. 
                                    \begin{enumerate}
                                        \item If $X$ were to be isomorphic to the multiplicative group scheme $(\G_m)_{/\Spec K}$ (i.e. the unique abelian variety of dimension $0$, up to isomorphisms) then:
                                            $$X[\ell^n] \cong (\G_m)_{/\Spec K}[\ell^n] \cong \Z/\ell^n\Z$$
                                        for all $n$, which would imply that:
                                            $$\T_{\ell}( (\G_m)_{/\Spec K} ) \cong \Z_{\ell}$$
                                        \item When $X$ is an elliptic curve (i.e. an abelian variety of dimension $1$; cf. definitions \ref{def: moduli_of_elliptic_curves} and \ref{def: abelian_varieties}), we can apply \cite[Corollary 6.4]{silverman_elliptic_curves} to get:
                                            $$\T_{\ell}(X) \cong \Z_{\ell} \oplus \Z_{\ell}$$
                                        
                                        More generally, one can make use of some \'etale homotopy theory to show that for all integers $N$ coprime with $p$, there exists the following decomposition of the $N$-torsion subgroup of $X(K)$:
                                            $$X[N] \cong \Z/N\Z \oplus \Z/N\Z$$
                                        First of all, it will have to be shown that every elliptic curve admits a finite \'etale covering which is also an elliptic curve. 
                                    \end{enumerate}
                            \end{itemize}
                    \end{enumerate}
                \end{example}
            
                \begin{definition}[Lisse sheaves] \label{def: lisse_sheaves}
                    Let $F$ be a local field (we shall view finite fields as $0$-dimensional local fields) equipped with its natural topology (e.g. non-archimedean when $F$ is some sort of $\ell$-adic number field, archimedean when $F$ is $\R$ or $\bbC$, and discrete when $F$ is finite). We shall refer to functions into $F$ as being \say{$\ell$-adic} as typically, one takes $F$ to be $\Q_{\ell}$ or extensions thereof (the reason we are using $\ell$ instead of a simple \say{$p$} as our prime is historical: $\ell$-adic sheaves were first conceived for the purposes of the Riemann Hypothesis on varieties over characteristics $p$).
                    \begin{enumerate}
                        \item \textbf{(Lisse $\ell$-adic functions):} Let $X$ be a \textit{totally disconnected} topological space (typically just locally profinite, although there are interesting non-profinite examples such as $\Q$). An $\ell$-adic function $f: X \to F$ shall then be called \textbf{lisse} if and only if it is \textit{compactly supported} and \textit{locally constant} (this terminology is suppose to be a nod to the notion of smooth functions on locally profinite spaces). The space of lisse $\ell$-adic functions on any open subset $U \subseteq X$ is denoted by $C^{\infty}_c(U, F)$ or simply $C^{\infty}_c(U)$ when $F$ is understood.
                        
                        One thing to note is that when $F = \bbC$, this notion does \textit{not} coincide with that of smoothness, since totally disconnected spaces can not admit any sort of archimedean metric. This is another reason why we opted for \say{lisse functions} instead of \say{smooth functions} or \say{bump functions}.
                        \item \textbf{(Lisse $\ell$-adic sheaves):} In analogy with the above notion of lisse $\ell$-adic functions, let us define a \textbf{lisse $\ell$-adic sheaf} as a \textit{finite-dimensional} $F$-linear local system over some pro-\'etale site $X_{\proet}$ of a given base scheme $X$. It is not hard to see that lisse sheaves on $X$ form a category, which we shall denote by $\LocSys_F(X_{\proet})^{\fin}$.
                    \end{enumerate}
                \end{definition}
            
                \begin{theorem}[The $\ell$-adic Monodromy Correspondence] \label{theorem: l_adic_monodromy_correspondence}
                    Let $\ell$ be a prime, let $F$ be a local field (we shall view finite fields as $0$-dimensional local fields) equipped with its natural topology (e.g. non-archimedean when $F$ is some sort of $\ell$-adic number field, archimedean when $F$ is $\R$ or $\bbC$, and discrete when $F$ is finite). Also, let $X$ be a \textit{connected} base scheme. There is then the following equivalence of rigid symmetric monoidal categories:
                        $$\LocSys_F(X_{\proet})^{\fin} \cong \Rep_F^{\cont}(\pi_1^{\fet}(X))$$
                \end{theorem}
                    \begin{proof}
                        
                    \end{proof}
                \begin{corollary}[Continuous Galois representations as lisse sheaves] \label{coro: continuous_galois_representations_as_lisse_sheaves}
                    Let $K$ be a field whose characteristic is different from $\ell$. Then, we have the following equivalence of rigid symmetric monoidal categories:
                        $$\LocSys_F(*_{\proet})^{\fin} \cong \Rep_F^{\cont}(\bfG_K)$$
                \end{corollary}
                \begin{example}[\'Etale cohomologies as geometric Galois representations] \label{example: etale_cohomologies_as_galois_representations}
                    Let $K$ be a separably closed field and let $X$ be a smooth and proper scheme over $\Spec K$. Since $\ell$-adic cohomology is a Weil cohomology theory, the $\ell$-adic cohomologies $H^i_{\Q_{\ell}}(X)$ are, in particular, finite-dimensional. Since we wish to show that these cohomologies are naturall Galois representations, it then remains to verify that $\bfG_K$ indeed acts on them.
                                
                    For this, let us first use \cite[\href{https://stacks.math.columbia.edu/tag/0BUM}{Tag 0BUM}]{stacks} along with the assumption that $X$ is a scheme over a separably closed field, we get that:
                        $$\pi^{\fet}(X) \cong \bfG_k$$
                    Then, note that the chain complex $H^*_{\Q_{\ell}}(X)$ is actually the same as the \textit{finite-dimensional} chain complex of pro-\'etale cohomologies $H^*_{\proet}(X, \Q_{\ell})$. An appliction of theorem \ref{theorem: l_adic_monodromy_correspondence} then gives us the desired Galois action on the cohomologies $H^i_{\Q_{\ell}}(X)$.
                \end{example}
                
        \subsection{Classical abelian global class field theory for function fields}
            \begin{convention} \label{conv: classical_abelian_global_class_field_theory_conventions}
                \noindent
                \begin{itemize}
                    \item 
                        \begin{itemize}
                            \item Let $K$ be a global function field over $\F_q$, with $q$ some power of a prime $p$ (that is, let $K$ be a finite extension of $\F_q(t)$). 
                            \item The ring of integers of $K$ shall be denoted by $\scrO_K$.
                        \end{itemize}
                    \item For $\ell \not = p$ an auxiliary prime, $\overline{\Q_{\ell}}$ shall be our field of coefficients. Sometimes, we shall fix an implicit (algebraic) isomorphism $\overline{\Q_{\ell}} \cong \bbC$.
                    \item 
                        \begin{itemize}
                            \item The words \say{prime}, \say{valuation}, and \say{place} shall be used interchangeably. For an explanation/excuse, see convention \ref{conv: places_and_primes}. 
                            \item If $v$ is a place of $K$, then the completion of $K$ at $v$ shall be denoted by $K_v$, and the corresponding local ring of integers shall be denoted by $\scrO_v$. In particular, $K_{\infty}$ shall denote the completion of $K$ at the \say{place at infinity} $(0) \in \Spec \scrO_K$ (note that because $\scrO_K$ is an integral domain, this generic point is unique).
                            \item $K_{\R}$ shall denote the product of all the completions $K_v$ of $K$ along \textit{archimedean} places $v$, and $\hat{\scrO_K}$ the product of all the \textit{non-archimedean} local ring of integers $\scrO_{\p}$. 
                        \end{itemize}
                \end{itemize}
            \end{convention}
            
            \subsubsection{Ad\`eles}
                We would like to begin by introducing some fundamental constructions, most notable among which is the ring of ad\`eles of our fixed global function field $K$. Before we can, however, we must discuss some topological preliminaries.
                    \begin{definition}[Ad\`eles] \label{def: ring_of_adeles}
                        For a moment, suppose that $K$ is an arbitrary global field (not even necessarily of characteristic $p > 0$). 
                            \begin{enumerate}
                                \item \textbf{(Integral ad\`eles):} The ring of \textbf{integral ad\`eles} of $K$, denoted by $\A_{\scrO_K}$, shall then be:
                                    $$\A_{\scrO_K} := K_{\R} \x \hat{\scrO_K}$$
                                \item \textbf{(Rational ad\`eles):} The \say{rationalisation} of $\A_{\scrO_K}$, i.e. the ring:
                                    $$\A_K \cong \A_{\scrO_K} \tensor_{\scrO_K} K$$
                                is known as the ring of \textbf{rational ad\`eles} or simply the ring of \textbf{ad\`eles} of $K$.
                            \end{enumerate}
                    \end{definition}
                    \begin{example}[The ad\`eles of $\Q$] \label{example: adeles_of_Q}
                        The ring of integral ad\`eles of $\Q$ is:
                            $$\A_{\Z} := \R \x \prod_{p \in \Spec \Z \setminus \{(0)\}} \Z_p \cong \R \x \hat{\Z}$$
                        and of course, the ring of rational ad\`eles of $\Q$ is:
                            $$\A_{\Q} \cong (\R \x \hat{\Z}) \tensor_{\Z} \Q$$
                        Now, it is well-known that $\hat{\Z} \cong \underset{n \in \Z}{\lim} \Z/n$ and therefore, through some abstract nonsense, one can swap the product and tensor product to get:
                            $$\A_{\Q} \cong \R \x \left(\hat{\Z} \tensor_{\Z} \Q\right)$$
                    \end{example}
            
            \subsubsection{Unramified abelian global class field theory for function fields}
            
            \subsubsection{Ramified abelian global class field theory for function fields}
    
        \subsection{Unramified geometric abelian global class field theory}
        
        \subsection{Ramified geometric abelian global class field theory} 
        
        \chapter{The \texorpdfstring{$p$}{}-adic Geometric Satake Equivalence}
    \begin{abstract}
        
    \end{abstract}
    
    \minitoc
    
    \section{The Geometric Satake Equivalence in mixed characteristics}
        \begin{convention} \label{conv: p_adic_geometric_satake_conventions}
            \noindent
            \begin{itemize}
                \item Let us fix once and for all a non-archimedean local field $E$ with residue field $\F_q$, along with some choice of pseudo-uniformiser $\varpi \in E^{\circ \circ}$. 
                \item Additionally, $X_S$ shall denote the Fargues-Fontaine Curve over any given perfectoid space $S \in \Perfd_{/\Spa \F_q}$. We refer the reader to section \ref{section: the_fargues_fontaine_curve} for details on the construction and properties of The Curve. 
                \item Lastly, $G$ shall be a connected reductive group over $\Spd E$.
            \end{itemize}
        \end{convention}
        
        In this section we attempt to understand a version of the Geometric Satake Correspondence in mixed characteristics as written down in \cite{fargues_scholze_geometrization_of_local_langlands}, whose statement asserts an equivalence between certain kinds of sheaves on the affine Grassmannian attached to $G$ and the category of representations of the Langlands dual of $G$, thereby establishing a rudimentary version of Langlands Duality. 
    
        \subsection{The \texorpdfstring{$\B_{\dR}$}{}-affine Grassmannian}
            \subsubsection{The affine Grassmannian over a point}
                \begin{convention}[$\B_{\dR}$-discs]
                    For each $\Spa(R, R^+) \in \Perfd^{\affd}_{/\Spa E^{\flat}}$, let us write $\bbD_{\dR}^+(R)$ for the so-called \textbf{$\B_{\dR}$-disc} $\Spec \B^+_{\dR}(R)$, and $\bbD_{\dR}(R)$ for the \textbf{punctured $\B_{\dR}$-disc} $\Spec \B_{\dR}(R)$.
                \end{convention}
            
                \begin{definition}[Local $\B_{\dR}$-affine Grassmannians] \label{def: local_B_dR_affine_grassmannian}
                    There is a canonically defined moduli space, denoted by $\Gr_G^{\loc}$ and called the \textbf{\textit{local} $\B_{\dR}$-affine Grassmannian} attached to $G$. It is the prestack which assigns to each $\Spa R \in \Perfd_{/\Spd E}^{\affd}$ the groupoid that is the core of the category of \'etale $G$-torsors on $\bbD_{\dR}^+(R)$ that trivialise over $\bbD_{\dR}(R)$.  
                \end{definition}
                \begin{remark}
                    \noindent
                    \begin{itemize}
                        \item It is rather easy to see that $\Gr_G^{\loc}$ satisfies \'etale descent and hence tautologically a stack on $(\Perfd^{\affd}_{/\Spd E})_{\et}$.
                        \item Furthermore, by introducing the so-called $\B_{\dR}$-loop and $\B_{\dR}$-arc groups $G_{\dR}^+$ and $G_{\dR}$\footnote{Note that in this context $(-)_{\dR}$ is not the functor of de Rham spaces like in section \ref{section: D_modules_over_characteristic_0}.}, defined by $G_{\dR}^+(R) \cong G(\B_{\dR}^+(R))$ and $G_{\dR}(R) \cong G(\B_{\dR}(R))$ respectively, one can show that:
                            $$\Gr_G^{\loc} \cong G_{\dR}/G_{\dR}^+$$
                        using the fact that $G_{\dR}^+$ acts on $\Gr_G^{\loc}$ by changing the trivialisation. This is an important description of the affine Grassmannian, so let us state and prove it properly (cf. proposition \ref{prop: B_dR_affine_grassmannian_as_coset_spaces}).
                    \end{itemize}
                \end{remark}
                
                Let us now investigate the geometry of $\Gr_G^{\loc}$. In particular, our aim is to establish properties of $\Gr_G^{\loc}$ that would make the consideration of (equivariant) perverse sheaves thereon a meaningful process.
                \begin{lemma}[Affine Grassmannians are $v$-sheaves] \label{lemma: B_dR_affine_grassmannians_are_v_sheaves}
                    $\Gr_G^{\loc}$ satisfies $v$-descent, and hence pro-\'etale descent as well.
                \end{lemma}
                    \begin{proof}
                        
                    \end{proof}
                \begin{proposition}[Affine Grassmannians as coset spaces] \label{prop: B_dR_affine_grassmannian_as_coset_spaces}
                    Let $G$ be an algebraic group. Then, $\Gr_G^{\loc} \cong G_{\dR}/G_{\dR}^+$, with the right-hand side implicitly meaning the \'etale sheafification of quotient presheaf.
                \end{proposition}
                    \begin{proof}
                        
                    \end{proof}
                \begin{corollary}[Loop group action on Grassmannians] \label{coro: loop_group_action_on_B_dR_grassmannians}
                    There is a $G_{\dR}^+$-action on $\Gr_G^{\loc}$, which means that one can now meaningfully discuss equivariance of sheaves on the affine Grassmannian $\Gr_G^{\loc}$. In particular, we shall be interested in $G_{\dR}^+$-equivariant perverse sheaves on $\Gr_G^{\loc}$.
                \end{corollary}
                
                \begin{proposition}[Structure of affine Grassmannians] \label{prop: structure_of_B_dR_affine_grassmannian}
                    As a $v$-stack over $\Spd E$, the affine Grassmannian $\Gr_G^{\loc}$ is separated (hence quasi-separated) and proper.
                \end{proposition}
                    \begin{proof}
                        \noindent
                        \begin{enumerate}
                            \item \textbf{(Separatedness):}
                            \item \textbf{(Properness):}
                        \end{enumerate}
                    \end{proof}
                
                \begin{proposition}[Functoriality of the affine Grassmannian] \label{prop: B_dR_affine_grassmannian_functoriality}
                    Let $H$ be a closed reductive subgroup (also over $\Spd E$) of $G$. Then, the induced map $\Gr_H^{\loc} \to \Gr_G^{\loc}$ is a closed embedding as well.
                \end{proposition}
                    \begin{proof}
                        
                    \end{proof}
                \begin{remark}
                    Proposition \ref{prop: B_dR_affine_grassmannian_functoriality} is tremendously useful. This is because every reductive group is in particular a linear algebraic group, and the \say{maximal} reductive group - that being $\GL_n$ - is also the \say{maximal} linear algebraic group. One can thus prove assertions for $\Gr_{\GL_n}$ (which is relatively simple to understand) before extending these results to other reductive groups.  
                \end{remark}
                    
            \subsubsection{Bruhat Decomposition, Schubert cells, and ind-structures}
            
            \subsubsection{The Demazure Resolution for the affine Grassmannian}
            
            \subsubsection{Representability of the affine Grassmannian}
            
        \subsection{Beilinson-Drinfeld Grassmannians}
            \subsubsection{Definition}
            
            \subsubsection{Semi-infinite orbits}
            
            \subsubsection{Sheaves on Beilinson-Drinfeld Grassmannians}
            
        \subsection{The Satake Equivalence}
            \subsubsection{Convolution of equivariant perverse sheaves on the affine Grassmannian and properties thereof}
            
            \subsubsection{Hecke eigensheaves}
            
            \subsubsection{Fibre functors and weights}
            
            \subsubsection{Langlands Duality}
    
    \section{Relative perverse sheaves}
        
        \chapter{Geometrisation of the \texorpdfstring{$p$}{}-adic Local Langlands Programme}
    \begin{abstract}
        
    \end{abstract}
    
    \minitoc
    
    \section{Cohomology of the moduli space of shtukas}
    
    \section{The spectral action of the stack of L-parameters}
	
	\part{Technical appendices}
    \begin{appendices}
        \chapter*{Introduction}
    \begin{abstract}
        
    \end{abstract}
    
    \minitoc
    
    \section{The Global Correspondence}
        Let $X$ be a curve that is smooth, proper, and geometrically connected algebraic curve (for instance, we can take $X$ be an elliptic curve or $\P^1$) and suppose that $G$ is a reductive group (think $\GL_n$ or $\SL_n$, or more concretely, $\GL_1$, or groups of diagonal matrices); both shall be over a field $k$ of characteristic $0$. Additionally, denote the function field of our curve $X$ by $K_X$, the completions of said field at (closed) points $x \in |X|$ by $K_{X, x}$, and we shall write $\scrO_{X, x}$ for the associated rings of integers (note how they coincide with the adic completions $\calO_{X, x}^{\wedge}$).
            
        The end goal for us, shall be to construct some semblance of an equivalence of derived/abelian/stable $\infty$-categories:
            $$\Dmod\left(\Bun_G(X)\right) \cong \Ind\Coh\left(\LocSys_F(X)^{\check{G}}\right)$$
        between:
            \begin{itemize}
                \item the category $\Dmod\left(\Bun_G(X)\right)$ of D-modules on the moduli stack $\Bun_G(X)$ of $G$-bundles on $X$, and
                \item the category $\Ind\Coh\left(\LocSys_F(X)^{\check{G}}\right)$ of ind-coherent sheaves (cf. section \ref{section: indcoh}) on the moduli stack of $\check{G}$-equivariant local systems on $X$ with coefficients in some implicitly understood suitable field $F$. 
            \end{itemize}
        When $G$ is a torus - i.e. when it is abelian - the above correspondence is a bit simpler:
            $$\Dmod\left(\Bun_G(X)\right) \cong \QCoh\left(\LocSys_F(X)^{\check{G}}\right)$$
        (notice how now, we can work with the entire category of quasi-coherent sheaves instead of having to restrict ourselves to ind-coherent sheaves). One thing that needs to be made clear right away, however, is that aside from a few very special cases such as $G = \GL_1$ and $G = \SL_2$, this equivalence is \textit{entirely conjectural}. Nevertheless, we do have a rough idea of how to eventually obtain a proper theorem from this vision:
            \begin{enumerate}
                \item The very first thing to do is to understand the construction of D-modules on (pre)stacks locally of finite type, and we can do this by learning about crystals (in the sense of Grothendieck) and their infinitesimal/crystalline cohomology over base fields of characteristic $0$ (crystalline cohomology over base fields of positive characteristics and the accompanying theory of arithmetic D-modules is significantly more complicated than their characteristic $0$ counterparts, which incidentally is why we have required that $\chara k = 0$).
                \item Then, we must know what $\check{G}$ actually is, i.e. we must understand Langlands duals. There is a tool for this, which is the Geometric Satake Equivalence. However, we are going to have to go through two substeps:
                    \begin{enumerate}
                        \item To begin, we shall need to understand what the affine Grassmannian is and its roles in the representation theory of algebraic groups.
                        \item We shall also have to know what it means to have a group act upon a (nice enough) category so as to be able to define the category of so-called \textbf{spherical D-modules}, which are certain kinds of equivariant D-modules.
                        \item We shall then establish the Geometric Satake Correspondence to be a Tannakian equivalence:
                            $$\Rep^{\heart}_F(\check{G}_{K_{X, x}}) \cong \Sph^{\heart}_{G, X, x}$$
                        between the hearts of the t-structures of the rigid monoidal derived categories of $F$-linear representations of the $K_{X, x}$-points of the Langlands dual group $\check{G}$ and of $G(\scrO_{X, x})$-equivariant/spherical D-modules over the local affine Grassmannian $\Gr_{G, X, x}$.
                    \end{enumerate}
                \item Lastly, we shall seek to understand the subtle technical differences between quasi-coherent sheaves and ind-coherent sheaves, and why restricting ourselves to the case of tori allows us to forego the ind-coherent sheaf machinery. 
            \end{enumerate}
        Of course, before embarking on this journey, we might also want to learn some (derived) algebraic geometry, which will help us understand $\Bun_G(X)$ and $\LocSys_F(X)^{\check{G}}$, what these categories have to do with the theory of Galois representations (because at the end of the day, the Langlands Programme is all about understanding higher reciprocity laws), or even simply why we have required that our curve $X$ is smooth (spoiler: smoothness helps us identify $\QCoh(X)$ with the category $\QCoh(X)^{\perf}$ of perfect complexes on $X$), proper, and geometrically connected, beyond wanting our machineries to be applicable to important classes of examples such as elliptic curves and abelian varieties. For details, see chapters \ref{chapter: schemes} and \ref{chapter: cohomology_and_derived_schemes}.
        
        We should also make some remarks about the above equivalence of categories as well. Thanks to Grothendieck's Galois theory, the left-hand side can be thought of as the \say{\textbf{Automorphic Side}} of the Langlands Correspondence, which holds information about Galois representations. Drawing inspiration from another one of Grothendieck's major contributions, $\ell$-adic \'etale cohomology, the right-hand side in turn can be thought of as the \say{\textbf{Spectral Side}}, which tells interesting stories\footnote{Fairy tales, really...} through harmonic analysis.
    
        \subsection{The Categorical-Geometric Langlands Correspondence for algebraic tori}
            This section, as the title suggests, shall be dedicated to outlining our hopes and dreams (or the lack thereof) for a Categorical-Geometric Langlands Correspondence for algebraic tori; specifically, we would like to present of a list of key results known to be involved in a proof of the Correspondence. We will also give run-down of the various technical tools used for establishing said key results. 
            
            \subsubsection{Equivariant local systems}
        
            \subsubsection{The Fourier-Muka\"i-Laumon Transform}
            
            \subsubsection{Factor-wise Langlands duality}
            
        \subsection{The Conjecture for non-abelian groups}
        
        \subsection{Outline of the proof for the case of \texorpdfstring{$G = \GL_2$}{}}
        
    \section{The Local Correspondence for complex loop groups}
        \subsection{The appearance of Langlands parameters}
            Consider the formal loop group $G(\!(t)\!)$ associated to some chosen connected complex reductive group $G$. 
            
            Let us start by describing the absolute Galois group of the field $\bbC(\!(t)\!)$. First of all, notice that:
                $$\Gal(\bbC(\!(t^{\frac1n})\!)/\bbC(\!(t)\!)) \cong \Z/n\Z$$
            and so:
                $$\Gal(\overline{\bbC(\!(t)\!)}/\bbC(\!(t)\!)) \cong \hat{\Z}$$
            which is a canonical homeomorphism of topological groups obtained via the Fundamental Theorem of Galois Theory. Now, one thing to note is that for some fixed power $q$ of a prime $p$, one also has:
                $$\Gal(\overline{\F_q}/\F_q) \cong \hat{\Z}$$
            but unlike the complex case, the group $\Gal(\overline{\F_q(\!(t)\!)}/\F_q(\!(t)\!))$ surjects (continuously) onto the non-trivial group $\Gal(\overline{\F_q}/\F_q)$ ($\bbC$ is algebraically closed so $\Gal(\bar{\bbC}/\bbC)$ is trivial), a fact known through local class field theory. As a consequence, describing the Weil group (and by extension, Weil-Deligne representations thereof) attached to $\bbC(\!(t)\!)$ will - hopefully - be somewhat simpler than that of $\F_q(\!(t)\!)$ and might therefore help us gain insight into the nature of the Langlands Correspondence. Better yet, we have via Grothendieck's Galois Theory, that:
                $$\Gal(\overline{\bbC(\!(t)\!)}/\bbC(\!(t)\!)) \cong \hat{\Z} \cong \pi_1^{\et}(\bbD^{\x}_{\bbC})$$
            wherein $\bbD^{\x}_{\bbC} \cong \Spec \bbC(\!(t)\!)$; through the discussion above, one sees that this is not the case for $\bbD^{\x}_{\F_q}$, i.e.:
                $$\Gal(\overline{\F_q(\!(t)\!)}/\F_q(\!(t)\!)) \not \cong \pi_1^{\et}(\bbD^{\x}_{\F_q})$$
            Since representations of the (\'etale) fundamental group correspond to certain D-modules, we essentially have access to the theory of D-modules in studying the Langlands Correspondence for the case of $G\!(t)\!)$, which roughly postulates a bijective relationship between homomorphisms $W_{\bbC(\!(t)\!)} \to \check{G}$ and certain representations of $G(\!(t)\!)$.
        
        \subsection{Representations of loop groups; Kac-Moody algebras}
    
    \section{Deformation quantisation of the Local Correspondence}
    
        \chapter{Algebraic topology and higher category theory}
    \begin{abstract}
        
    \end{abstract}
    
    \minitoc
    
    \section{Higher operads and higher categories}
        \subsection{Monoidal categories; enrichments}
            \subsubsection{Classical monoidal categories}
                \begin{definition}[Monoidal categories] \label{def: monoidal_categories}
                    A monoidal category is a quintuple $(\calV, \tensor, 1, \alpha, (\lambda, \rho))$ of:
                        \begin{enumerate}
                            \item a category $\calV$;
                            \item a bifunctor $\tensor: \calV \x \calV \to \calV$;
                            \item a distinguished object $1 \in \calV$ - called the \textbf{unit} - which we shall view as a functor $\eta: \pt \to \calV$ from the terminal category;
                            \item a natural isomorphism of functors - called the \textbf{associator} - as follows:
                                $$
                                    \begin{tikzcd}
                                    	{(\calV \x \calV) \x \calV} && {\calV \x (\calV \x \calV)} \\
                                    	\\
                                    	{\calV \x \calV} && {\calV \x \calV} \\
                                    	& \calV
                                    	\arrow["{\tensor \x \id_{\calV}}"', from=1-1, to=3-1]
                                    	\arrow["\tensor"', from=3-1, to=4-2]
                                    	\arrow["\tensor", from=3-3, to=4-2]
                                    	\arrow["{\id_{\calV} \x \tensor }", from=1-3, to=3-3]
                                    	\arrow["\cong", from=1-1, to=1-3]
                                    	\arrow[""{name=0, anchor=center, inner sep=0}, "{(- \tensor -) \tensor -}"{description}, from=1-1, to=4-2]
                                    	\arrow[""{name=1, anchor=center, inner sep=0}, "{- \tensor (- \tensor -)}"{description}, from=1-3, to=4-2]
                                    	\arrow["\alpha", shorten <=26pt, shorten >=26pt, Rightarrow, from=0, to=1]
                                    \end{tikzcd}
                                $$
                            \item natural isomorphisms $\lambda$ and $\rho$ - known, respectively, as the \textbf{left and right-unitors} - as follows:
                                $$
                                    \begin{tikzcd}
                                    	{\calV \x \pt} & {\calV \x \calV} & {\pt \x \calV} \\
                                    	\calV & \calV & \calV
                                    	\arrow["{\tensor}"{description}, from=1-2, to=2-2]
                                    	\arrow["{\eta \x \id_{\calV}}"', from=1-3, to=1-2]
                                    	\arrow["{\id_{\calV} \x \eta}", from=1-1, to=1-2]
                                    	\arrow["{\id_{\calV}}"', from=2-1, to=2-2]
                                    	\arrow["{\pr_1}"', from=1-1, to=2-1]
                                    	\arrow["{\pr_2}", from=1-3, to=2-3]
                                    	\arrow["{\id_{\calV}}", from=2-3, to=2-2]
                                    	\arrow["\lambda"', shorten <=13pt, shorten >=13pt, Rightarrow, from=1-2, to=2-1]
                                    	\arrow["\rho", shorten <=13pt, shorten >=13pt, Rightarrow, from=1-2, to=2-3]
                                    \end{tikzcd}
                                $$
                            (note that the functors $\pr_1: \calV \x \pt \to \calV$ and $\pr_2: \pt \x \calV \to \calV$ are equivalences \textit{a priori}, thanks to the universal property of the terminal objects and that of products).
                        \end{enumerate}
                    
                \end{definition}
                \begin{definition}[Lax-monoidal categories] \label{def: lax_monoidal_categories}
                    Let $(\calV, \tensor, 1, \alpha, (\lambda, \rho))$ a quintuple as in definition \ref{def: monoidal_categories}, but now, suppose that $\alpha: (- \tensor -) \tensor - \to - \tensor (- \tensor -)$ is a non-invertible $2$-cell. Then, this quintuple will define a so-called \textbf{lax-monoidal category} if and only if 
                \end{definition}
                \begin{definition}[Non-unital monoidal categories]
                    
                \end{definition}
                
                \begin{definition}[Braidings and symmetries] \label{def: braided_and_symmetric_monoidal_categories}
                    
                \end{definition}
                
            \subsubsection{Categories enriched over monoidal categories; 2-categories}
            
            \subsubsection{Unbiased monoidal categories}
        
        \subsection{Operads and multicategories}
            \begin{definition}[Cartesian categories] \label{def: cartesian_categories}
                \noindent
                \begin{enumerate}
                    \item \textbf{(Cartesian categories):} A category is said to be \textbf{Cartesian} if it has all pullbacks. A \textbf{Cartesian} functor (not necessarily between Caterisan categories) is one that preserves pullbacks. 
                    \item \textbf{(Cartesian natural transformations):} A natural transformation:
                        $$
                            \begin{tikzcd}
                            	\C & \D
                            	\arrow[""{name=0, anchor=center, inner sep=0}, "G"', shift right=3, from=1-1, to=1-2]
                            	\arrow[""{name=1, anchor=center, inner sep=0}, "F", shift left=3, from=1-1, to=1-2]
                            	\arrow["\alpha", shorten <=2pt, shorten >=2pt, Rightarrow, from=1, to=0]
                            \end{tikzcd}
                        $$
                    between two functors $F, G: \C \to \D$ (where $\C, \D$ need not be Cartesian categories) if the naturality square induced by any arrow $f: x \to y$ in $\C$ is a pullback square in $\D$:
                        $$
                            \begin{tikzcd}
                            	Fx & Fy \\
                            	Gx & Gy
                            	\arrow["Ff", from=1-1, to=1-2]
                            	\arrow["Gf", from=2-1, to=2-2]
                            	\arrow["{\alpha_x}"', from=1-1, to=2-1]
                            	\arrow["{\alpha_y}", from=1-2, to=2-2]
                            	\arrow["\lrcorner"{anchor=center, pos=0.125}, draw=none, from=1-1, to=2-2]
                            \end{tikzcd}
                        $$
                    \item \textbf{(Cartesian monads):} Let $\E$ be a Cartesian category .A so-called \textbf{Cartesian monad} $(T: \E \to \E, \mu: T^2 \to T, \eta: \id \to T)$ internal to $\E$ is thus one wherein the defining endomofunctor $T: \E \to \E$ is Carterian, and so are the natural transformations $\mu: T^2 \to T$ and $\eta: \id \to T$.
                \end{enumerate}
            \end{definition}
            \begin{remark}[$2$-category of Catersian categories] \label{remark: 2_category_of_cartesian_categories}
                There exists a natural 
            \end{remark}
            \begin{example}[Cartesian categories] \label{example: cartesian_categories}
                    
            \end{example}
            \begin{example}[Cartesian monads] \label{example: cartesian_monads}
                    
            \end{example}
        
        \subsection{Weak \texorpdfstring{$n$}{}-categories}
            \subsubsection{Globular operads}
            
            \subsubsection{The many definitions of weak \texorpdfstring{$n$}{}-categories}
        
    \section{Algebraic K-theory}
    
    \section{\texorpdfstring{$(\infty, 1)$}{}-categories and \texorpdfstring{$\infty$}{}-topoi}
    
    \section{\texorpdfstring{$(\infty, 2)$}{}-categories}
    
        \input{Appendices/higher algebra}
        
        \chapter{The \texorpdfstring{$(\infty, 2)$}{}-category of correspondences}
    \begin{abstract}
        
    \end{abstract}
    
    \minitoc

    \section{The \texorpdfstring{$(\infty, 2)$}{}-category of correspondences}
        \subsection{What are correspondences ?}
        
        \subsection{The universal property of the category of correspondences}
        
        \subsection{Enlarging classes of 2-morphisms}
        
        \subsection{Factorisation}
        
    \section{Cohomological base-changing} \label{section: cohomological_base_change}
        
    \section{The symmetric monoidal structure on the category of correspondences}
        
        \chapter{Moduli problems}
    \begin{abstract}
        
    \end{abstract}
    
    \minitoc

    \section{Moduli problems}

    \section{Examples}
        \subsection{Hilbert schemes}
        
        \subsection{Picard schemes}
            \subsubsection{Line bundles and Picard stacks}
                \begin{definition}[Line bundles] \label{def: line_bundles}
                    \noindent
                    \begin{enumerate}
                        \item \textbf{(Line bundles):} Let $\calY$ be a prestack on $\Cring^{\op}$. Then, a \textbf{line bundle on $\calY$} is a line object (also referred to as an invertible object) in the \textit{symmetric} monoidal category $\QCoh(\calY)$ of quasi-coherent modules on $\calY$ (cf. definition \ref{def: qcoh_def}). For the definition of line objects, see \cite{nlab:line_object}.
                        \item \textbf{(Picard groupoids):} 
                            \begin{enumerate}
                                \item \textbf{(Picard groups ...):} It is not hard to see that given any symmetric monoidal category such as that of quasi-coherent modules over a prestack, isomorphism classes of its line objects (which are non-trivial in general, since monoidal structures are only defined up to coherence isomorphisms) form a group where the multiplication is the monoidal structure. This group is known as the \textbf{Picard group}; we shall denote the Picard group of a prestack $\calY$ on $\Cring^{\op}$ by $\Pic^0(\calY)$. Also, note that Picard groups are trivially abelian.
                                \item \textbf{(... and to categorify them):} However, one might not be content with line bundles forming just a puny group. If that is indeed the case, allow us to introduce the all-new \textbf{weak Picard $2$-group}. The weak Picard $2$-group associated to a given prestack $\calY$ on $\Cring^{\op}$ is first and foremost the category internal to the category $\Grp$ of groups (which we note to have finite pullbacks; see definitions \ref{def: internal_categories} and \ref{remark: internal_categories_alt_def} for an explanation of why this is necessary), whose object of objects is $\Pic^0(\calY)$ and whose object of arrows is the automorphism group $\Aut\left(\Pic^0(\calY)\right)$, which we shall suggestively abbreviate by $\Pic^1(\calY)$:
                                    $$
                                        \begin{tikzcd}
                                        	{\Pic^1(\calY)} & {\Pic^0(\calY)} \\
                                        	{\Pic^0(\calY)}
                                        	\arrow["t"', from=1-1, to=2-1]
                                        	\arrow["s", from=1-1, to=1-2]
                                        \end{tikzcd}
                                    $$
                                Next, observe that the above internal category possesses a natural groupoid structure, owing to the fact that the object of arrows $\Pic^1(\calY)$ is a group (by construction!), which in particular means that every arrow in the internal category $(s, t)$ is invertible, and thus there exists an inversion map:
                                    $$i: \Pic^1(\calY) \to \Pic^1(\calY): \sigma \mapsto \sigma^{-1}$$
                                which trivially fits into the following commutative diagram in $\Grp$ (we shall let the reader check the commutativity of this diagram as an exercise):
                                    $$
                                        \begin{tikzcd}
                                        	{\Pic^1(\calY)} \\
                                        	& {\Pic^1(\calY)} & {\Pic^0(\calY)} \\
                                        	& {\Pic^0(\calY)}
                                        	\arrow["t", from=2-2, to=3-2]
                                        	\arrow["s"', from=2-2, to=2-3]
                                        	\arrow["i"{description}, from=1-1, to=2-2]
                                        	\arrow["s"', from=1-1, to=3-2]
                                        	\arrow["t", from=1-1, to=2-3]
                                        \end{tikzcd}
                                    $$
                                    
                                One important thing to note is that the weak Picard $2$-group $(s, t): \Pic^1(\calY) \toto \Pic^0(\calY)$ is precisely the \href{https://ncatlab.org/nlab/show/core}{\underline{core}} of the full subcategory of $\QCoh(\calY)$ spanned by line bundles, as the automorphisms on $\Pic^0(\calY)$ are precisely the isomorphisms between line bundles on $\calY$, and thus it is also a groupoid in the usual sense (i.e. a groupoid internal to $\Cat$). In particular, this tells us that the weak Picard $2$-group of a prestack is actually strict, and it is for this reason that sometimes we might confuse the terms \say{Picard $2$-group} and \say{Picard groupoid}.
                                
                                From this point on, the Picard $2$-group/groupoid associated to a given prestack $\calY$ shall be denoted simply by $\Pic^1(\calY)$.
                            \end{enumerate}
                    \end{enumerate}
                \end{definition}
                \begin{remark}[A categorical comment] \label{remark: picard_2_groups_are_groups_in_Cat}
                    Because Picard $2$-groups associated to prestacks are strict $2$-groups, they also enjoy the property of being group objects in $\Cat$ and hence in the full subcategory $\Grpd$ thereof. See \cite{nlab:strict_2-group} for more details.
                \end{remark}
                \begin{remark}[Picard groups, Picard $2$-groups, and higher homotopical fairy tales]
                    Fix a prestack $\calY$ on $\Cring^{\op}$.
                    \begin{enumerate}
                        \item \textbf{(Decategorifying Picard $2$-groups):} The underlying set of the Picard group $\Pic^0(\calY)$ can be thought of as the set of connected components of the groupoid $\Pic^1(\calY)$, i.e.:
                            $$\Pic^0(\calY) \cong \pi_0\Pic^1(\calY)$$
                        Intuitively, this should make sense, since the data specifying $\Pic^1(\calY)$ differ from that specifying $\Pic^0(\calY)$ only by the \say{looping} isomorphisms on line bundles on $\calY$, which means that the set of connected components of $\Pic^1(\calY)$ is just the \say{discrete} part, namely the underlying set of $\Pic^0(\calY)$. 
                        \item \textbf{(Higher Picard groups):} For each natural number $n$, let us define the Picard $(n + 1)$-group $\Pic^n(\calY)$ to be the strict $(n + 1)$-group $\Aut^n\left(\Pic^0(\calY)\right)$, where:
                            $$\Aut^n\left(\Pic^0(\calY)\right) \cong \Aut\left( \Aut\left( \cdots \Aut\left(\Pic^0(\calY)\right) \right) \right)$$
                        Of course, for all $r \leq n$, we have:
                            $$\Pic^r(\calY) \cong \pi_r\Pic^n(\calY)$$
                    \end{enumerate}
                \end{remark}
                
                \begin{theorem}[\textit{Hilberts Satz 90}] \label{theorem: hilbert_90}
                    Let $X$ be an arbitrary scheme. Then, one has the following isomorphism of abelian groups:
                        $$\Pic^0(X) \cong H^1_{\tau}(X, \G_m)$$
                    where $\tau \in \{\text{Zariski, smooth, \'etale, fppf}\}$.
                \end{theorem}
                    \begin{proof}
                        
                    \end{proof}
                
                \begin{definition}[Picard prestacks] \label{def: picard_prestacks}
                    Let $k$ be a base commutative ring. Then, let us define the \textbf{Picard stack over $\Spec k$} to be the \href{https://ncatlab.org/nlab/show/core}{\underline{core}} of the full substack:
                        $$\Pic^1: [\Spec k]^{\op} \to \Grpd$$
                    spanned by line bundles of the stack $\QCoh: [\Spec k]^{\op} \to \Cat$ of quasi-coherent modules on ${}^{k/}\Comm\Alg^{\op}$. That is to say, the Picard prestack associates to each prestack $\calY \in [\Spec k]$, the category $\Pic^1(\calY)$ is the core of the full subcategory of $\QCoh(\calY)$ spanned by line bundles, i.e. nothing but the Picard $2$-group on $\calY$ (cf. definition \ref{def: line_bundles}). Naturally, the Picard prestack is fibred in groupoids, and thus has a shot at being geometric. Also, it is evident that the Picard prestack is a gerbe (cf. convention \ref{conv: prestacks}).
                \end{definition}
            
            \subsubsection{Picard stacks of curves}
        
        \subsection{Moduli stacks of elliptic curves and of abelian varieties}
            \subsubsection{Moduli of elliptic curves}
                \paragraph{Elliptic curves and their moduli}
                    \begin{definition}[Moduli spaces of elliptic curves] \label{def: moduli_of_elliptic_curves} \index{Moduli space! of elliptic curves}
                        \noindent
                        \begin{enumerate}
                            \item \textbf{(Elliptic curves):} An \textbf{elliptic curve} over a given base scheme $B$ is an $B$-scheme:
                                $$E \to B$$
                            that satisfies the following properties:
                                \begin{itemize}
                                    \item $E$ is smooth, proper, and of relative dimension $1$ over $B$.
                                    \item The fibre of $E$ over any point $x \in |B|$ is a smooth, proper, of pure dimension $1$, and geometric connected over the residue field $\kappa_x$.
                                    \item $E$ is a \textit{group $B$-scheme} and as such admits a \textbf{distinguished section} $o_E: B \to E$ as its unit.
                                \end{itemize}
                            In other words, an elliptic curve is an abelian scheme of relative dimension $1$.
                            \item \textbf{(Moduli spaces of elliptic curves):} Over a base scheme $S$, the \textbf{moduli space of elliptic curves} is a prestack:
                                $$(\calM_{1, 1})_{/S}: \Sch_{/S}^{\op} \to \Grpd$$
                            which associates to each $S$-scheme $B$ the \textit{core} (i.e. the maximal subgroupoid) of the category of elliptic curves over $B$ (in the above sense), which we will denote simply by $(\calM_{1, 1})_{/S}(B)$. One thing to note is that the pullback of fibres is given by fibred products, which is to say, given an underlying morphism:
                                $$f: B \to B'$$
                            the pullback along $f$ of an elliptic curve $E' \to B'$ (denoted by $f^*E'$) shall be the elliptic $B$-curve fitting into the following pullback square:
                                $$
                                    \begin{tikzcd}
                                    	{f^*E'} & {E'} \\
                                    	B & {B'}
                                    	\arrow[from=1-1, to=2-1]
                                    	\arrow["f", from=2-1, to=2-2]
                                    	\arrow[from=1-2, to=2-2]
                                    	\arrow[from=1-1, to=1-2]
                                    	\arrow["\lrcorner"{anchor=center, pos=0.125}, draw=none, from=1-1, to=2-2]
                                    \end{tikzcd}
                                $$
                        \end{enumerate}
                    \end{definition}
                    \begin{remark}[A few supplementary technical comments] \label{remark: moduli_of_elliptic_curves_disambiguations}
                        \noindent
                        \begin{itemize}
                            \item \textbf{(Categories of elliptic curves):}
                                \begin{enumerate}
                                    \item \textbf{(Morphisms of elliptic curves):} Fix a base scheme $B$. If one is to use definition \ref{def: moduli_of_elliptic_curves} as one's starting notion of what it means for a scheme to be an elliptic curve, then the morphisms in the category $(\calM_{1, 1})_{/S}(B)$ of elliptic $B$-curves are going to be \textit{group $B$-scheme homomorphisms}:
                                        $$
                                            \begin{tikzcd}
                                            	{E_1} && {E_2} \\
                                            	& B
                                            	\arrow[from=1-3, to=2-2]
                                            	\arrow["\phi", from=1-1, to=1-3]
                                            	\arrow[from=1-1, to=2-2]
                                            \end{tikzcd}
                                        $$
                                    \item \textbf{(Pullbacks):} It is well-known that thanks to the fact that limits commute, any limit of a group $B$-scheme taken in the ambient category $\Sch_{/B}$ would remain a group $B$-scheme. Is the same true for elliptic $B$-curves ? 
                                    
                                    To answer this question, one needs to verify that smoothness, properness, the dimension of elliptic curves being $1$, as well as fibres being geometrically connected are all properties stable under (finite) pullbacks. This is rather trivial for smoothness (as well as for relative dimensions) and to a lesser extent, for properness as well (which we recall to be the same as being separated, of finite type, and universally closed), so let us focus on geometric connectedness.
                                    
                                    Recall first of all that a scheme over a field is geometrically connected if and only if the underlying topological space of any of its geometric fibres is connected.
                                \end{enumerate}
                            \item \textbf{(Group structure on elliptic curves):} 
                                \begin{itemize}
                                    \item \textbf{(Group law):} Because limits of group schemes (taken in the ambient category of schemes) are still group schemes, the distinguished section $o_E: B \to E$ of a relative elliptic curve degenerates to a rational point $o_{E_x}: \Spec \kappa_x \to E_x$ (more generally, pullbacks do not interfere with the group structure of an existing elliptic curve). This is the usual \say{point at infinity} or \say{unit point} of an elliptic curve over a field (the residue field $\kappa_x$ in this case).
                                    \item \textbf{(Connectedness is important):}
                                \end{itemize}
                        \end{itemize}
                    \end{remark}
                        
                \paragraph{Good and bad reductions}
                    \begin{definition}[Models] \label{def: local_models_of_varieties} \index{Model! of an algebraic scheme}
                        Let $R$ be a discrete valutation ring with fraction field $K$ and let $X$ be an algebraic scheme $\Spec K$ (i.e. a scheme of finite type over $\Spec K$); often, $X$ will be a variety instead of a mere algebraic scheme. A \textbf{model for $X$ over a $\Spec R$} is a \textit{flat} and of \textit{finite type} $\Spec R$-scheme $\bbX$ whose pullback along the canonical arrow $\Spec K \to \Spec R$ is (isomorphic to) $X$, i.e. one fitting into the following pullback square:
                            $$
                                \begin{tikzcd}
                                	X & \bbX \\
                                	{\Spec K} & {\Spec R}
                                	\arrow[from=1-1, to=2-1]
                                	\arrow[from=2-1, to=2-2]
                                	\arrow[from=1-2, to=2-2]
                                	\arrow[from=1-1, to=1-2]
                                	\arrow["\lrcorner"{anchor=center, pos=0.125}, draw=none, from=1-1, to=2-2]
                                \end{tikzcd}
                            $$
                        In other words, a model for an algebraic scheme $X$ over the field of fractions $K$ of some discrete valuation ring $R$ is just a scheme $\bbX$ over $\Spec R$ whose generic fibre is precisely $X$. 
                        
                        One can also impose adjectives such as \say{smooth}, \say{\'etale}, or \say{proper} onto models of algebraic schemes, as long as morphisms characterised by these adjectives are flat and of finite type. If $P$ is a property described by one such adjective, then we say that a model $\bbX \to \Spec R$ is $P$ if and only if the morphism $\bbX \to \Spec R$ is $P$.
                    \end{definition}
                
                    \begin{definition}[Good and bad reductions] \label{def: reductions}
                        Let $X$ be an algebraic scheme over the field of fractions $K$ of a discrete valuation ring $R$, and let $\varpi \in R$ be a uniformiser (i.e. a generator of the unique maximal ideal of $R$). 
                        \begin{enumerate}
                            \item \textbf{(Good/bad reductions):} A reduction modulo $\varpi$ (or in other terminologies, at place $\varpi$) of a model $\bbX \to \Spec R$ of $X$ is just the pullback along the canonical morphism $\Spec R/\varpi \to \Spec R$ of $\bbX \to \Spec R$. It is called \textbf{good} if it is smooth or \say{better} (for instance, \'etale fibres can be considered good reductions). Otherwise, it is called \textbf{bad}.
                            
                            One thing to note is that smooth models and better ones \textit{a priori} have good reductions, since smoothness (and say, \'etale-ness) is preserved under arbitrary base changes (cf. proposition \ref{prop: compositions_and_base_changes_of_smooth_morphisms} and corollary \ref{coro: compositions_and_base_changes_of_etale_morphisms}).
                            \item \textbf{(Places of reduction):} Let $\varpi$ be a generator of the unique and (necessarily) principal maximal ideal of the discrete valuation ring $R$ (or equivalently, denote the unique and principal maximal ideal of $R$ by $(\varpi)$). An algebraic $K$-scheme $X$ is said to have a \textbf{place $\varpi$ of good reduction} if and only if there exist an $R$-model $\bbX \to \Spec R$ of $X$ whose reduction modulo $\varpi$ is good. One might also phrase things geometrically and say that $X$ has good reduction at the closed point $\Spec R/\varpi$.
                            
                            $X$ is said to have a \textbf{place $\varpi$ of bad reduction} if and only if there does not exist an $R$-model $\bbX \to \Spec R$ of $X$ whose reduction modulo $\varpi$ is good (or equivalently, if and only if the reduction modulo $\varpi$ of all $R$-models of $X$ are bad).
                            
                            For more information on so-called \say{places} and prime ideals, see convention \ref{conv: places_and_primes}.
                        \end{enumerate}
                    \end{definition}
                    \begin{example}[Reduction modulo $p$ of curves over $\Q$]
                        Let $p$ be a prime, and note firstly that $\Z_{(p)}$ is a discrete valuation ring by virtue of being a local PID that is not a field; its unique maximal ideal is $(p)$ (and hence one can take $p$ to be a uniformiser), as every element that is not equivalent to $0$ modulo $p$ has already been made to be invertible in $\Z_{(p)}$. Let $X$ be a smooth projective curve over $\Spec \Q$, and let $\bbX_{(p)} \to \Spec \Z_{(p)}$ be a model of $X$ (due to the Fundamental Theorem of Arithmetic, $\Z_{(p)}$ is has $\Q$ as its field of fractions). Then, a reduction modulo $p$ of $X$ is nothing but a fibre product $\bbX_{(p)} \x_{\Spec \Z_{(p)}} \Spec \F_p$. 
                    \end{example}
                    \begin{remark}[Locality of reduction]
                        It is worth noting that a curve $X \to \Spec K$ having good reduction at a place $\varpi$ is a local phenomenon. That is to say, a curve may have good reductions at all places $\varpi$ (for instance, there might be curves over $\Q$ such that all of their $\Z_{(p)}$ possess good modulo $p$ reductions), but the equation defining said curve may be singular in some manner after reduction modulo $\varpi$. 
                    \end{remark}
                    
                    \begin{lemma}[Existence of smooth models]
                        Let $R$ be a discrete valuation ring with field of fraction $K$. Then, there exist a flat, of finite presentation, and proper $R$-model $\bbX$ for any smooth scheme $X$ over $\Spec K$.
                    \end{lemma}
                        \begin{proof}
                            
                        \end{proof}
                    
                    \begin{theorem}[Shimura '55]
                        Let $X$ be a smooth projective variety over $\Spec \Q$. Then $X$ has finitely many places of bad reduction. That is to say, there are only finitely many places $p$ at which the reduction modulo $p$ of a $\Z_{(p)}$-model of $X$ is bad.
                    \end{theorem}
                        \begin{proof}
                            
                        \end{proof}
                    \begin{example}
                        \noindent
                        \begin{enumerate}
                            \item \textbf{(A cubic curve with places of bad reduction):} Consider the cubic curve given by the polynomial equation $x^3 - y^2  - 2 = 0$ (or in other words, consider the variety $X := \Spec \frac{\Q[x, y]}{(x^3 - y^2)}$), which we note to be smooth over $\Q$. Let $\bbX_{(p)}$ be a model over $\Spec \Z_{(p)}$ of $X$. Then, reductions modulo $p$ of $\bbX_{(p)}$ will remain smooth (i.e. good) at all places except at $p = 2$, because:
                                $$x^3 - y^2 - 2 \equiv x^3 - y^2 \pmod{2}$$
                            meaning that the reduction modulo $2$ of $\bbX_{(2)}$ has a cusp at the origin, and at $p = 3$, since:
                                $$x^3 - y^2 - 2 \equiv (x - 2)^3 - y^2 \pmod{3}$$
                            meaning that the reduction modulo $3$ of $\bbX_{(3)}$ has a cusp at $(x, y) = (2, 0)$.
                            \item \textbf{(Reductions of $p$-adic elliptic curves \cite[Example 5.2]{silverman_elliptic_curves}):} Let $p \geq 5$ be a prime and let $E_1, E_2$ be elliptic curves over $\Q_p$ that are given by the equations:
                                $$x^3 + px^2 - y^2 + 1 = 0$$
                            and:
                                $$x^3 - y^2 + p = 0$$
                            We know from the previous example that the second curve is cuspidal at place $p$, so let us focus our attention on the first curve. Reducing modulo $p$ gives a curve determined by the following equation:
                                $$x^3 - y^2 + 1 = 0$$
                            which we note to be smooth, and thus the place $p$ is of good reduction. The assumption that our prime $p$ is larger than or equal to $5$ is crucial for $p$ to be a place of good reduction for $E_1$, because when $p = 2$ or $p = 3$, the equation of the curve obtained after reduction modulo $p$ will be:
                                $$\text{$x^3 - (y + 1)^2 = 0$ or $(x - 1)^3 - y^2 = 0$}$$
                            and both are cuspidal.
                            \\
                            Over $\Q_p(\sqrt[6]{p})$, however, the curve $E_2$ does have good reduction at $p$: by setting $x = \sqrt[3]{p} s$ and $y = \sqrt[2]{p} t$, one gets the following equivalent equation defining the curve $E_2$:
                                $$ps^3 - pt^2 + p = 0$$
                            which becomes the trivial equation $0 = 0$ after reduction modulo $p$. 
                            \item \textbf{(Reduction of smooth projective $p$-adic surfaces):} Let $p$ be an arbitrary prime, let $K$ be a $p$-adic field (i.e. a finite extension of $\Q_p$), and let $X$ be a conic bundle over $\P^1_K$ with four degenrate fibres. First of all, note that $X$ is a smooth projective variety over $\Spec K$ whose $\ell$-adic cohomologies (with $\ell$ any prime distinct from $p$) are all unramified (cf. definition \ref{def: ramification_indices}). Second of all, by being a conic bundle, $X$ is $\overline{K}$-birational to $\P^1_K$. 
                            \item \textbf{(Reduction of abelian varieties):} Let $A$ be an abelian variety over $\Spec \Q$ and let $p, \ell$ be distinct primes. Then, by \cite[Theorem 1.1.1]{conradbrinon}, there exists a $\Z_{(p)}$-model of $A$ with good reduction at $p$ if and only if the $\ell$-adic \'etale cohomologies of $A$ are all unramified at $p$. A special case of this phenomenon is theorem 7.1 of \cite{silverman_elliptic_curves}, which gives a criterion by N\'eron-Ogg-Shafarevich for when elliptic curves admits good reduction at $p$;  
                        \end{enumerate}
                    \end{example}
                    
                    \begin{remark}[Global models ?] \label{remark: global_models}
                        Let $p$ be a prime, let $\bbX_p \to \Spec \Z_p$ model of some $p$-adic algebraic scheme $X_p \to \Spec \Q_p$, and suppose that for whatever reason, we want to be able lift this scheme to $\Q$; $X$ could be some sort of smooth projective geometrically connected curve and we might want to understand the compatibility of local and global class field theory via its \'etale fundamental group for example (for such a purpose, we would also want to be able to \say{localise} to $\Q_p$ from $\Q$), or perhaps we would want to check multiple primes and see where a given smooth curve has bad reductions. Then, even though $\Q$ is the same as the fraction field of say, $\Z_{(p)}$, we can use the fact that it is also the field of fractions of $\Z$ to construct some sort \say{global} model $\bbX_0 \to \Spec \Z$. The algebraic scheme $X_0 \to \Spec \Q$ be an object over the generic fibre $(0) \in |\Spec \Z|$, whereas $X_p \to \Spec \Q_p$ would be a \say{local} object over the special fibre $(p) \in |\Spec \Z|$.  
                            \begin{figure}[H]
                                \centering
                                \includegraphics[width=\linewidth,height=\textheight,keepaspectratio]{Figures/places of Spec Z.png}
                                \caption{Local and global points of $\Spec \Z$}
                                \label{fig: local_and_global_points_of_Spec_Z}
                            \end{figure}
                    \end{remark}
                    \begin{definition}[Global models] \label{def: global_models}
                        We want our parametrising scheme, like $\Spec \Z$, to be one where the infinitestimal neighbourhoods (i.e. formal completions) around correspond to spectra of complete discrete valuation rings (for instance, the infinitestimal neighbourhoods $\Spf \Z_p$ around the points $(p) \in |\Spec \Z|$ correspond uniquely to the affine schemes $\Spec \Z_p$), which essentially means we want . We also want $S$ to be connected so that any residue field at a generic point would automatically be the function field of $S$.
                        
                        Let $S$ be base scheme satisfying the above conditions and let $K_0$ be its function field. Then, a \textbf{global model} for an algebraic scheme $X_0$ over $K_0$ shall be a flat and of finite type $S$-scheme $\bbX_0 \to S$. 
                    \end{definition}
                    \begin{remark}[Local-global compatibility] \label{remark: global_to_local_for_models}
                        
                    \end{remark}
                    
                    \begin{definition}[Pointed curves] \label{def: pointed_curves}
                        A \textbf{pointed elliptic curve} is a \textit{finitely presented} and \textit{proper} global model:
                            $$\bbE_0 \to S$$
                        with a distinguised so-called unit section $o_{\bbE_0}: S \to \bbE_0$ of one of the following:
                            \begin{itemize}
                                \item an elliptic curve over a place of $S$,
                                \item a projective nodal cubic curve over a place of $S$ (cf. \cite[\href{https://stacks.math.columbia.edu/tag/0C46}{Tag 0C46}]{stacks}), or
                                \item a projective cuspidal cubic curve over a place of $S$ (a point is a cusp if it corresponds to a non-splitting prime with inertial degree $2$; cf. definition \ref{def: ramification_indices}). 
                            \end{itemize}
                        A fibre of the first kind is said to be over a place of \textbf{good reduction}, whereas fibres over places of the second and third kind are said to be of \textbf{bad reduction}, as one ends up with singularities at those places.
                    \end{definition}
                    
                \paragraph{The geometry of \texorpdfstring{$\calM_{1, 1}$}{}}
                    \begin{proposition}[Base-changing $\calM_{1, 1}$] \label{prop: base_changes_of_moduli_spaces_of_elliptic_curves}
                        Let $\phi: S_2 \to S_1$ be a morphism between two base schemes and let $(\calM_{1, 1})_{/S_1}$ be the  moduli spaces of elliptic curves over $S_1$. Then, there exists a moduli space of elliptic curves $(\calM_{1, 1})_{/S_2}$ fitting into the following pullback square of geometric stacks:
                            $$
                                \begin{tikzcd}
                                	{(\calM_{1, 1})_{/S_1}} & {(\calM_{1, 1})_{/S_1}} \\
                                	{S_2} & {S_1}
                                	\arrow[from=1-2, to=2-2]
                                	\arrow[from=1-1, to=2-1]
                                	\arrow["{\phi}", from=2-1, to=2-2]
                                	\arrow["{}", from=1-1, to=1-2]
                                	\arrow["\lrcorner"{anchor=center, pos=0.125}, draw=none, from=1-1, to=2-2]
                                \end{tikzcd}
                            $$
                    \end{proposition}
                        \begin{proof}
                            First of all, the pullback $(\calM_{1, 1})_{/S_2}$ is \textit{a priori} a geometric stack on $\Sch_{/S_2}$. Next, consider a pullback square of the following form in $\Sch_{/S_1}$ (i.e. a pullback between fibres of $(\calM_{1, 1})_{/S_1}$):
                                $$
                                    \begin{tikzcd}
                                    	{f_1^* E_1} & E \\
                                    	{B_1'} & {B_1}
                                    	\arrow["{f_1}", from=2-1, to=2-2]
                                    	\arrow[from=1-2, to=2-2]
                                    	\arrow[from=1-1, to=1-2]
                                    	\arrow[from=1-1, to=2-1]
                                    	\arrow["\lrcorner"{anchor=center, pos=0.125}, draw=none, from=1-1, to=2-2]
                                    \end{tikzcd}
                                $$
                            along with a base change diagram of the following form:
                                $$
                                    \begin{tikzcd}
                                    	&& {B_2} & {B_1} \\
                                    	{B_2'} & {B'_1} & {S_2} & {S_1} \\
                                    	{S_2} & {S_1}
                                    	\arrow["\phi", from=3-1, to=3-2]
                                    	\arrow[from=2-2, to=3-2]
                                    	\arrow[from=2-1, to=2-2]
                                    	\arrow[from=2-1, to=3-1]
                                    	\arrow["\lrcorner"{anchor=center, pos=0.125}, draw=none, from=2-1, to=3-2]
                                    	\arrow["\phi", from=2-3, to=2-4]
                                    	\arrow[from=3-2, to=2-4]
                                    	\arrow[from=3-1, to=2-3]
                                    	\arrow["{f_2}", from=2-1, to=1-3]
                                    	\arrow["{f_1}", from=2-2, to=1-4]
                                    	\arrow[from=1-3, to=1-4]
                                    	\arrow[from=1-4, to=2-4]
                                    	\arrow[from=1-3, to=2-3]
                                    	\arrow["\lrcorner"{anchor=center, pos=0.125}, draw=none, from=1-3, to=2-4]
                                    \end{tikzcd}
                                $$
                            Pulling the first diagram back along $\phi: S_2 \to S_1$ (using the second diagram as a guide in the process) thus yields:
                                $$
                                    \begin{tikzcd}
                                    	{\phi^* f_1^* E_1} & {\phi^*E_1} \\
                                    	{B_2'} & {B_2} & {f_1^* E_1} & {E_1} \\
                                    	&& {B'_1} & {B_1}
                                    	\arrow[from=2-1, to=3-3]
                                    	\arrow["{f_2}", from=2-1, to=2-2]
                                    	\arrow["{f_1}", from=3-3, to=3-4]
                                    	\arrow[from=2-2, to=3-4]
                                    	\arrow[from=2-4, to=3-4]
                                    	\arrow[from=1-2, to=2-2]
                                    	\arrow[from=1-2, to=2-4]
                                    	\arrow[from=2-3, to=2-4]
                                    	\arrow[from=2-3, to=3-3]
                                    	\arrow[from=1-1, to=2-1]
                                    	\arrow[from=1-1, to=1-2]
                                    	\arrow[from=1-1, to=2-3]
                                    	\arrow["\lrcorner"{anchor=center, pos=0.125}, draw=none, from=2-3, to=3-4]
                                    	\arrow["\lrcorner"{anchor=center, pos=0.125}, draw=none, from=1-1, to=2-2]
                                    \end{tikzcd}
                                $$
                            We know \textit{a prioi} that $f_1* E_1$ and $\phi^* f_1* E_1$ must be elliptic curves, and with this, the proof is done.
                        \end{proof}
                    
                    \begin{proposition}[$\calM_{1, 1}$ is a geometric stack] \label{prop: moduli_stacks_of_elliptic_curves}
                        Let $S$ be a base scheme. Then, the moduli space of elliptic curves $(\calM_{1, 1})_{/S}$ is a geometric stack on $\Sch_{/S, \tau}$, where:
                            $$\tau \in \{\text{Zariski, \'etale, fppf, fpqc}\}$$
                        is a Grothendieck topology on $\Sch_{/S}$.
                    \end{proposition}
                        \begin{proof}
                            We will take definition \ref{def: schemes} as our working definition of schemes. It will become clear in a moment why this is necessary.
                            \begin{enumerate}
                                \item \textbf{(Descent satisfaction):} The sheaf condition (up to invertible $2$-cells, of course) is an easy consequence of the fact that pullbacks of elliptic curves along arrows $f: B \to B'$ in $\Sch_{/S}$ are given by fibred products (cf. definition \ref{def: moduli_of_elliptic_curves}).
                                \item \textbf{(A smooth atlas):} 
                                \item \textbf{(Schematicity of the diagonal):} 
                            \end{enumerate}
                        \end{proof}
            
            \subsubsection{Moduli of higher-dimensional abelian varieties}
    \end{appendices}
	
	\printbibliography
	
	\printindex

\end{document}