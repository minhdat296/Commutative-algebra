\chapter{Lie theory and formal geometry} \label{chapter: formal_geometry}
    \begin{abstract}
        So, what is \say{formal geometry} within the derived context ? Well, within the context of this book, the term shall mean the study of objects of the category $[\Spec k]^{\laft, \defm}$ of prestacks on ${}^{k/}\Comm\Alg^{\op}$ that are locally almost of finite type and admit deformations - which we note to be a full subcategory of $[\Spec k]^{\laft}$ of prestacks that are locally of finite type on ${}^{k/}\Comm\Alg^{\op}$ - along with nil-isomorphisms between them. As we shall see, this category is a rather natural seting in which one may consider the following geometrically-motivated phenomena:
            \begin{enumerate}
                \item Quotients by equivalence relations induced by actions of internal groupoids.
                \item An algebro-geometric analogue of the Lie Group-Lie Algebra Correspondence, which particularly, is a correspondence between groups over a give base prestack $\calX$ and Lie algebras internal to the symmetric monoidal category $\Ind\Coh(\calX)$ of ind-coherent sheaves on $\calX$. 
                \item Universal enveloping algebras of Lie algebras and Lie algebroids, and other differential-geometric entities such as Hodge filtrations/de Rham resolutions.
                \item D-modules and the Riemann-Hilbert Correspondence. 
            \end{enumerate}
        We realise that this chapter leans more towards geometric representation theory rather than commutative algebra, but a glimpse of the true strength of commutative algebraic geometry should not be to anyone's detriment.
    \end{abstract}
    
    \minitoc
    
    \begin{convention}[Everything is derived!] \label{conv: formal_geometry_everything_is_derived}
        \noindent
        \begin{itemize}
            \item From now on until the end of the chapter, everything will be assumed to be derived. 
            \item By $1\-\Cat_1$, or simply $1\-\Cat$, we shall actually mean $(\infty, 1)\-\Cat_1$, i.e. the $(\infty, 1)$-category of $(\infty, 1)$-categories and functors between them, and by $1\-\Cat_2$ we will be referring to the $(\infty, 2)$-category of $(\infty, 1)$-categories, functors between them, and natural transformations between these functors. 
            
            Similarly, by $\Grpd^1$, or simply $\Grpd$, we will actually mean the $(\infty, 1)$-category of $\infty$-groupoids and functors between them, and by $\Grpd^2$, we shall mean the $(\infty, 2)$-category of $\infty$-groupoids, functors between them, and natural transformations between these functors.
            \item A subcategory of $1\-\Cat$ this is of particular interest is $\dg\Cat^{\cont}_2$ (or simply $\dg\Cat^{\cont}$), the $(\infty, 2)$-category of stable linear (i.e. differential-graded) $(\infty, 1)$-categories (see section \ref{section: homological_algebra} for the notion of stable $(\infty, 1)$-categories). Of course, we can also view $\dg\Cat^{\cont}$ as a mere $(\infty, 1)$-category; when necessary, we shall write $\dg\Cat^{\cont}_1$ to put emphasis on the disregard of $2$-morphisms.
        \end{itemize} 
    \end{convention}
    
    \section{Cohomology and representations of Lie algebras in characteristic zero}
        \begin{convention}
            From now until the end of the section let us fix a base field $k$ of characteristic $0$.
        \end{convention}
                
        \subsection{Lie operads and Koszul duality}
            \subsubsection{A bit about algebras over operads}
                \begin{definition}[Symmetric sequences] \label{def: symmetric_sequences}
                    Let $(\O, \tensor, \1)$ be a \textit{symmetric} monoidal category (typically taken to be $\Grp$) and let $\Sigma$ be an $\N$-graded monoid internal to $\O$ (i.e. a functor $\N \to \Mon(\O)$ or equivalently, a monoid internal to the functor category $[\N, \O]$, which inherits the symmetric monoidal structure of $\O$); $\Sigma$ is typically taken to be the graded monoid of symmetric groups, i.e. $\Sigma = \{S_n\}_{n \in \N}$. The naturally symmetric monoidal category of \textbf{$\Sigma$-symmetric $\O$-sequence} is thus simply the functor category $\O^{\Sigma} := [\bfB \Sigma, \O]$.
                \end{definition}
            
                \begin{convention}[Operads ?] \label{conv: symmetric_sequences_as_operads}
                    \noindent
                    \begin{enumerate}
                        \item \textbf{(Operads):} For the purposes of this subsection and section \ref{section: lie_algebras_and_formal_groups}, we shall think of \textbf{$\Sigma$-symmetric operads} in a given symmetric monoidal category $\O$ (for some $\N$-graded monoid $\Sigma$ internal to $\O$) as monoids in the symmetric monoidal category $\O^{\Sigma}$ of $\Sigma$-symmetric $\O$-sequences. For a more careful treatment, we refer the reader to section \ref{section: algebras_and_modules_over_operads} and particularly, subsection \ref{subsection: operads} therein. The category of $\Sigma$-symmetric operads shall be denoted by $\Opd(\O^{\Sigma})$ (it is actually just $\Mon(\O^{\Sigma})$, but we thought the notation could be a bit suggestive).
                        \item \textbf{(Regarding unitality):} Technically speaking, operads need not be unital, which is to say, for each given symmetric monoidal category $\O$ and a fixed $\N$-graded monoid $\Sigma$ therein, one can define $\Opd(\O^{\Sigma})$ to simply be $\Assoc\Alg(\O^{\Sigma})$. We will, however, only work with unital associative operads.
                    \end{enumerate}
                \end{convention}
                \begin{convention}
                    From now on, let us always take $\Sigma$ as the graded monoid of symmetric groups. Due to this specification of $\Sigma$, we shall only write $\Opd(\O)$ for the category of $\Sigma$-symmetric $\O$-operads (where, of course, $\O$ is a symmetric monoidal category).
                \end{convention}
                
                \begin{proposition}[Universal property of symmetric operads] \label{prop: symmetirc_operads_universal_property}
                    Let $(\O, \tensor, \1)$ be a symmetric monoidal category. Then, 
                \end{proposition}
                    \begin{proof}
                        
                    \end{proof}
                    
            \subsubsection{Co-operads and Koszul Duality}
            
        \subsection{Lie algebrs and cocommutative coalgebras}
        
    \section{Lie algebroids}
        
    \section{Formal groups and their Lie algebras} \label{section: lie_algebras_and_formal_groups}