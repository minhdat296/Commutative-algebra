\chapter{D-modules in geometric representation theory}
    \begin{abstract}
        
    \end{abstract}
    
    \minitoc
    
    \section{D-modules over characteristic zero}
        In this section, we will be giving a demonstration of how the theory of inf-schemes (cf. section \ref{section: formal_schemes_and_inf_schemes}) helps us define and describe D-modules as ind-coherent sheaves over a certain kind of prestacks, thereby naturally establishing the six-functor formalism \textit{\`a la} Grothendieck in a natural manner for D-modules. In particular, should a prestack $\calY$ be sufficiently nice, one would even be able to figure out the behaviours of D-modules on $\calY$ only from what one would expect from ind-coherent sheaves on $\calY$. 
        
        Before submerging ourselves in the tecnnical details, let us first recall some notable features of the classical theory of D-modules (in the sense of say, Kashiwara), both to equip ourselves with an intuitive sense of purpose for our endeavour, as well as to justify this highly abstract alternative approach to D-modules. 
        
        \paragraph*{The classical theory}
            Let $k$ be a field of characteristic $0$, and let $X$ be a scheme over $\Spec k$, which we will assume to be flat, proper, and of finite presentation for simplicity. If $X$ is smooth, then the relative cotangent complex $\bfL^*_{X/k}$ of $X/k$ shall be quasi-isomorphic to the zero complex (cf. definition \ref{def: cohomological_smoothness}), which means that the de Rham complex $\Omega^*_{X/k}$ attached to $X/k$ can only have non-zero terms in non-negative degrees; this, in turn, implies that the de Rham complex:
                $$
                    \begin{tikzcd}
                    	0 & \E & {\Omega^1_{X/k} \tensor_{\calO_{X/k}} \E} & {\Omega^2_{X/k} \tensor_{\calO_{X/k}} \E} & {...}
                    	\arrow["\nabla", from=1-4, to=1-5]
                    	\arrow["\nabla", from=1-3, to=1-4]
                    	\arrow["\nabla", from=1-2, to=1-3]
                    	\arrow[from=1-1, to=1-2]
                    \end{tikzcd}
                $$
            (henceforth abbreviated by $\E \tensor_{\calO_{X/k}} \Omega^*_{X/k}$) corresponding to any given \textit{flat} $\calO_{X/k}$-linear connection $\nabla: \E \to \Omega^1_{X/k} \tensor_{\calO_{X/k}} \E$ will also be concentrated in non-negative degrees. Recall also, that to specify a \textit{flat} $\calO_{X/k}$-linear connection $\nabla: \E \to \Omega^1_{X/k} \tensor_{\calO_{X/k}} \E$ on a $\calO_{X/k}$-vector bundle $\E$ over a smooth scheme $X/k$ is the same as specifying the structure of a left-$\D_{X/k}$-module on $\E$. Together, these two facts imply that the dg-category ${}^l\DMod(X/k)$ of chain complexes of left-$\D_{X/k}$-modules is equivalent to the category whose objects are the de Rham complexes corresponding to flat connections on vector bundles over $X/k$, which we will denote by ${}^{\geq 0}\Vect(X/k)^{\nabla}_{\dR}$. As a consequence, given any smooth morphism:
                $$f: X \to Y$$
            of schemes over $\Spec k$, one obtains a six-functor formalism between ${}^l\DMod(X/k)$ and ${}^l\DMod(Y/k)$ simply as that between ${}^{\geq 0}\Vect(X/k)^{\nabla}_{\dR}$ and ${}^{\geq 0}\Vect(Y/k)^{\nabla}_{\dR}$. While this is all well and good, categories of vector bundles equipped with flat connections (over smooth schemes) are rather unnatural as far as functorial constructions are concerned. 
            
            But what about cases wherein $X$ is not smooth ? In such a situation, one would embed $X$ into some smooth ambient scheme, say $Y$ (which can always be done, as one can take $X$ to be the affine singular locus and then embed that locus into a smooth deformation of its ambient scheme). Then, we can make use of what Kashiwara taught us, which is that regardless of our choice of embedding $X \hookrightarrow Y$, the restriction ${}^l\DMod(Y/k)|_X$ of the category of left-$\D_{Y/k}$-modules onto objects (set-theoretically) supported on $X$ will be the same. Choosing such an embedding into a smooth scheme, however, would still involve appealing to resolutions, which is a rather unnatural procedure (at least from a $\infty$-categorical standpoint).
            
            Base-changing D-modules is especially an issue, due to rings of differential operators being noncommutative, and consequently, so is establishing a pull-push yoga that is easy to work with. Consider a product in $\Sch_{/\Spec k}$, wherein both $X$ and $Y$ are smooth and proper over $\Spec k$:
                $$
                    \begin{tikzcd}
                    	{X \x_{\Spec k} Y} & Y \\
                    	X & {\Spec k}
                    	\arrow[from=1-2, to=2-2]
                    	\arrow[from=2-1, to=2-2]
                    	\arrow[from=1-1, to=1-2]
                    	\arrow[from=1-1, to=2-1]
                    	\arrow["\lrcorner"{anchor=center, pos=0.125}, draw=none, from=1-1, to=2-2]
                    \end{tikzcd}
                $$
            Because neither $\D_{X/k}$ nor $\D_{Y/k}$ is commutative, it is not entirely clear whether $\D_{X \x_{\Spec k} Y/k}$ should be $\D_{X/k} \tensor_{\calO_{X \x_{\Spec k} Y/k}} \D_{Y/k}$ or $\D_{Y/k} \tensor_{\calO_{X \x_{\Spec k} Y/k}} \D_{X/k}$. 
        
        \paragraph*{The new perspective}
            Again, suppose that $k$ is a field (perhaps of characteristic $0$) and that $X$ is a scheme that is flat, proper, and of finite presentation over $\Spec k$.
        
            Let us firstly that a crystal in quasi-coherent modules over $X$ is one which is isomorphic to its restriction to the maximal reduced closed subscheme, which is to say that:
                $$\E \cong \E|_{X_{\dR}}$$
            (here $X_{\dR}$ denotes the so-called \textbf{de Rham space} attached to $X$, which as a presheaf is given by $X_{\dR}(R) \cong X({}^{\red}R)$ for all commutative $k$-algebras $R$). Additionally, recall that to any crystal in quasi-coherent modules $\E \in \QCoh(X_{\dR}/k)$, one can canonically associate an $\calO_{X/k}$-linear flat connection:
                $$\nabla: \E \to \Omega^1_{X/k} \tensor_{\calO_{X/k}} \E$$
            As stated above, when $X$ is smooth over $\Spec k$, this is the same as giving a left-$\D_{X/k}$-module, so for smooth schemes, one might as well define the dg-category of left-$\D_{X/k}$-modules to literally be the dg-category $\QCoh(X_{\dR}/k)$ of crystals in quasi-coherent modules over $X$; in fact, as schemes smooth over fields are \textit{a priori} reduced, ${}^l\DMod(X/k)$ is nothing but $\QCoh(X/k)$. The nice thing about this approach is that we understand quasi-coherent sheaves very well, and better yet, the theory of quasi-coherent sheaves over schemes is essentially that of modules over commutative rings (cf. definition \ref{def: qcoh_def}). Moreover, it now makes sense to speak of D-modules over smooth algebraic stacks and so on (perhaps with some sort of properness or separatedness assumption imposed), as there is nothing preventing us from considering de Rham spaces attached to \textit{any presheaf} on ${}^{k/}\Comm\Alg^{\op}$, as the definition is completely functorial. This is very useful, as there are many geometric objects which appear naturally in geometric representation theory yet are not schemes, such as the stack $\Bun_G(X)$ of principal $G$-bundles over a smooth curve $X$, for $G$ a reductive group; in fact, the so-called \say{Automorphic Side} of the Geometric Global Langlands Correspondence is the category of D-modules on this stack.
        
        \begin{convention}
            Until the end of this section, we shall be working over a field of characteristic $0$ ($k = \Q$ as well as $k$ a number field is a case of particular interest).
        \end{convention}
    
        \subsection{Crystals as sheaves}
            \subsubsection{The de Rham prestack}
                \begin{remark}[Reduced affine schemes] \label{remark: reduced_affine_schemes}
                    \noindent
                    \begin{itemize}
                        \item Reduced objects of ${}^{k/}\Comm\Alg$ are nothing but commutative $k$-algebras whose connected components have vanishing nilradicals, i.e. they have no non-zero nilpotent elements, and because ring homomorphisms preserve $0$, a homomorphism:
                            $$\varphi: R \to S$$
                        between two reduced commutative $k$-algebras $R$ and $S$ will be nothing more than an ordinary algebra homomorphism. As a consequence, reduced $k$-algebras form a full subcategory of ${}^{k/}\Comm\Alg$; we shall denote it by ${}^{k/}\Comm\Alg^{\red}$; note that $k$ itself, by virtue of being a field of characteristic $0$, is trivially reduced as an algebra over itself, and hence is still the initial object of ${}^{k/}\Comm\Alg^{\red}$. 
                        \item Now, because there is a uniquely determined nilradical inside each commutative ring, and because the canonical fully faithful embedding:
                            $${}^{k/}\Comm\Alg^{\red} \hookrightarrow {}^{k/}\Comm\Alg$$
                        trivially preserves all (small) limits that exist in ${}^{k/}\Comm\Alg^{\red}$, there shall exist a left-adjoint:
                            $$(-)_{\dR}|_{{}^{k/}\Comm\Alg}: {}^{k/}\Comm\Alg \to {}^{k/}\Comm\Alg^{\red}$$
                        that associates to each commutative $k$-algebra $R$ its quotient by its nilradical ${}^{\red}R := R/\Nil(R)$ (the notation will make more sense once we reach remark \ref{remark: de_rham_prestacks_functoriality}): note that this is the correct left-adjoint because it preserves small colimits (as quotients by nilradicals are colimits themselves). 
                        \item The category of prestacks over ${}^{k/}\Comm\Alg^{\red}$ shall be denoted by $[\Spec k]^{\red}$. 
                    \end{itemize}
                \end{remark}
            
                \begin{definition}[The de Rham prestack] \label{def: de_rham_prestack}
                    We can associate to any prestack $\calY$ on ${}^{k/}\Comm\Alg^{\op}$ a prestack $\calY_{\dR}$ (also on ${}^{k/}\Comm\Alg^{\op}$), called \textbf{the de Rham prestack attached to $\calY$}, that is defined object-wise by the following formula:
                        $$\calY_{\dR}(R) \cong \calY({}^{\red}R)$$
                    where ${}^{\red}R$ is the classical commutative ring isomorphic to the quotient of the underlying classical commutative ring of $R$ by its nilradical. Note that this formula gives us a canonical arrow:
                        $$\calY \to \calY_{\dR}$$
                    coming from the canonical quotient map $R \to {}^{\red}R$. Sometimes de Rham prestacks might also go by the name \say{\textbf{de Rham spaces}}.
                \end{definition}
                \begin{remark}[Functoriality of de Rham prestacks] \label{remark: de_rham_prestacks_functoriality}
                    Consider a morphism:
                        $$\calX \to \calY$$
                    between prestacks on ${}^{k/}\Comm\Alg$. It is not hard to show, via evaluation at quotients by nilradicals, that such a morphism induced a morphism of de Rham spaces:
                        $$\calX_{\dR} \to \calY_{\dR}$$
                    which tells us that there is a so-called \textbf{de Rham space functor}:
                        $$(-)_{\dR}: [\Spec k]^{\red} \to [\Spec k]$$
                    that attaches de Rham spaces to prestacks. Furthermore, the canonical embedding ${}^{k/}\Comm\Alg^{\red} \hookrightarrow {}^{k/}\Comm\Alg$ admits a left-adjoint (cf. remark \ref{remark: reduced_affine_schemes}), the induced embedding of affine scheme categories:
                        $$\Sch^{\aff, \red}_{/\Spec k} \hookrightarrow \Sch^{\aff}_{/\Spec k}$$
                    should admit a right-adjoint:
                        $$(-)_{\dR}|_{\Sch^{\aff}_{/\Spec k}}: \Sch^{\aff}_{/\Spec k} \hookrightarrow \Sch^{\aff, \red}_{/\Spec k}$$
                    (still given by quotients by nilradicals) and hence the canonically induced pullback between prestack categories $(-)_{\dR}|_{\Sch^{\aff}_{/\Spec k}}^*: [\Spec k] \to [\Spec k]^{\red}$ must admit a \textit{right}-adjoint:
                        $$[\Spec k]^{\red} \to [\Spec k]$$
                    i.e. the right-Kan extension along the canonical embedding $\Sch^{\aff, \red}_{/\Spec k} \hookrightarrow \Sch^{\aff}_{/\Spec k}$. By the universal property of right-Kan extensions, and because there is a canonical morphism $\calY \to \calY_{\dR}$ for every prestack $\calY$, we thus have that the right-Kan extension $[\Spec k]^{\red} \to [\Spec k]$ along $(-)_{\dR}|_{\Sch^{\aff}_{/\Spec k}}: \Sch^{\aff}_{/\Spec k} \hookrightarrow \Sch^{\aff, \red}_{/\Spec k}$ is nothing but the de Rham space functor:
                        $$(-)_{\dR}: [\Spec k]^{\red} \to [\Spec k]$$
                \end{remark}
                
                \begin{proposition}[Gluing affine de Rham spaces] \label{prop: gluing_affine_de_rham_spaces}
                    The de Rham space functor:
                        $$(-)_{\dR}: [\Spec k]^{\red} \to [\Spec k]$$
                    can be viewed as the left-Kan extension of the affine de Rham space functor:
                        $$(-)_{\dR}|_{\Sch^{\aff}_{/\Spec k}}: \Sch^{\aff}_{/\Spec k} \to [\Spec k]$$
                    along the Yoneda embedding $\Sch^{\aff, \red}_{/\Spec k} \hookrightarrow [\Spec k]^{\red}$. 
                \end{proposition}
                    \begin{proof}
                        
                    \end{proof}
                    
                \begin{definition}[Infinitesimal closeness] \label{def: infinitesimally_closed_points}
                    Let $\calY$ be a prestack on ${}^{k/}\Comm\Alg^{\op}$. The \href{https://ncatlab.org/nlab/show/kernel+pair}{\underline{kernel pair}} of the canonical arrow:
                        $$\calY \to \calY_{\dR}$$
                    (cf. definition \ref{def: de_rham_prestack}) induces the following span:
                        $$
                            \begin{tikzcd}
                            	{} & {\calY \x_{\calY_{\dR}} \calY} & \calY \\
                            	& \calY
                            	\arrow["{\pr_2}"', from=1-2, to=2-2]
                            	\arrow["{\pr_1}", from=1-2, to=1-3]
                            \end{tikzcd}
                        $$
                    which can be easily shown to be a binary relation on $\calY$ (using the fact that the Yoneda embedding is left-exact). This relation shall be known as \textbf{the relation of infinitesimal closeness} on $\calY$, and shall be abbreviated by ${}^{\inf}\calY$, i.e.:
                        $${}^{\inf}\calY \cong \calY \x_{\calY_{\dR}} \calY$$
                    For all commutative rings $R$, we shall say that two $R$-points $x, y \in {}^{\inf}\calY(R)$ are \textbf{infinitesimally close}. 
                \end{definition}
                \begin{remark}[The congruence of infinitesimal closeness] \label{remark: infinitesimal_closeness}
                    In fact, the binary relation of being infinitesimally close is actually an equivalence relation, meaning that one can form quotients of prestacks by it. To see why the relation ${}^{\inf}\calY$ on some give prestack $\calY \in [\Spec k]$ is an equivalence relation, simply note that the internal category:
                        $$
                            \begin{tikzcd}
                            	{} & {{}^{\inf}\calY} & \calY \\
                            	& \calY
                            	\arrow["{\pr_2}"', from=1-2, to=2-2]
                            	\arrow["{\pr_1}", from=1-2, to=1-3]
                            \end{tikzcd}
                        $$
                    can be endowed with a groupoid structure via the following map on the object of arrows ${}^{\inf}\calY$:
                        $$i_{{}^{\inf}\calY}: {}^{\inf}\calY \to {}^{\inf}\calY$$
                    
                \end{remark}
                \begin{remark}[\v{C}ech nerves associated to de Rham spaces] \label{remark: cech_nerves_of_de_rham_spaces}
                    As $[\Spec k]$ is a category with all pullbacks, and because there is a canonical morphism:
                        $$\calY \to \calY_{\dR}$$
                    for every prestack $\calY \in [\Spec k]$, we can form the \v{C}ech nerve (cf. definition \ref{def: hypercovers} and remark \ref{remark: hypercovers_alt_def}) $\calY^{\bullet}_{/\calY_{\dR}}$ of $\calY_{\dR}$. 
                \end{remark}
                
                \begin{proposition}[The locally almost of finite type case] \label{prop: laft_de_rham_spaces}
                    If $\calY$ is a prestack that is locally almost of finite type then so is its associated de Rham prestack.  
                \end{proposition}
                    \begin{proof}
                        
                    \end{proof}
                    
            \subsubsection{Left and right-crystals}
                \begin{definition}[Left-crystals] \label{def: left_crystals}
                    The dg-category of \textbf{left-crystals} on a given prestack $\calY \in [\Spec k]$, denoted by ${}^l\Crys(\calY)$ is precisely the symmetric monoidal $k$-linear stable $\infty$-category $\QCoh(\calY_{\dR})$ of quasi-coherent modules on the de Rham prestack $\calY_{\dR}$. In notations, that is:
                        $${}^l\Crys(\calY) \cong \QCoh(\calY_{\dR})$$
                \end{definition}
                
                \begin{remark}[Universal property of left-crystals] \label{remark: universal_property_left_crystals}
                    Let $\calY$ be a prestack on ${}^{k/}\Comm\Alg^{\op}$. Then, via an application of proposition \ref{prop: qcoh_universal_property}, one can characterise the category of left-crystals on $\calY$ in the following universal manner:
                        $${}^l\Crys(\calY) \cong \underset{\Spec R \in \left(\Sch^{\aff}_{/\calY_{\dR}}\right)^{\op}}{\lim} \QCoh(\Spec R)$$
                    This allows us to view left-crystals on a given prestack $\calY$ as coming from the following composition of functors:
                        $$
                            \begin{tikzcd}
                            	& {\Cat^{\stab, \tensor}} \\
                            	{[\Spec k]^{\red} } & {[\Spec k]}
                            	\arrow["{(-)_{\dR}}", from=2-1, to=2-2]
                            	\arrow["{\QCoh^*}"', from=2-2, to=1-2]
                            	\arrow["{{}^{l}\Crys}", from=2-1, to=1-2]
                            \end{tikzcd}
                        $$
                \end{remark}
                
                \begin{definition}[Right-crystals] \label{def: right_crystals}
                    Over any prestack $\calY$ that is locally almost of finite type, one can define a dg-category of right-crystal: this is the category of ind-coherent sheaves on the de Rham space $\calY_{\dR}$ (which is well-defined because the de Rham space attached to a prestack locally almost of finite type is also locally almost of finite type; cf proposition \ref{prop: laft_de_rham_spaces}). 
                \end{definition}
                \begin{remark}[Universal property of right-crystals] \label{remark: universal_property_right_crystals}
                    Thanks to proposition \ref{prop: *_pushforwards_of_ind_coherent_sheaves}, we know that there is actually 
                \end{remark}
        
        \subsection{Crystals as functors on the category of correspondences}
        
        \subsection{D-modules}
        
    \section{The Riemann-Hilbert Correspondence}
    
    \section{Beilinson-Bernstein Localisation}