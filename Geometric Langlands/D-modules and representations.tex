\chapter{D-modules in geometric representation theory}
    \begin{abstract}
        
    \end{abstract}
    
    \minitoc
    
    \section{D-modules over characteristic zero} \label{section: D_modules_over_characteristic_0}
        \begin{convention}[Everything is derived!] \label{conv: D_modules_everything_is_derived}
            \noindent
            \begin{itemize}
                \item From now on until the end of the section, everything will be assumed to be derived. 
                \item By $1\-\Cat_1$, or simply $1\-\Cat$, we shall actually mean $(\infty, 1)\-\Cat_1$, i.e. the $(\infty, 1)$-category of $(\infty, 1)$-categories and functors between them, and by $1\-\Cat_2$ we will be referring to the $(\infty, 2)$-category of $(\infty, 1)$-categories, functors between them, and natural transformations between these functors. 
                
                Similarly, by $\Grpd^1$, or simply $\Grpd$, we will actually mean the $(\infty, 1)$-category of $\infty$-groupoids and functors between them, and by $\Grpd^2$, we shall mean the $(\infty, 2)$-category of $\infty$-groupoids, functors between them, and natural transformations between these functors.
                \item A subcategory of $1\-\Cat$ this is of particular interest is $\dg\Cat^{\cont}_2$ (or simply $\dg\Cat^{\cont}$), the $(\infty, 2)$-category of stable linear (i.e. differential-graded) $(\infty, 1)$-categories (see section \ref{section: homological_algebra} for the notion of stable $(\infty, 1)$-categories). Of course, we can also view $\dg\Cat^{\cont}$ as a mere $(\infty, 1)$-category; when necessary, we shall write $\dg\Cat^{\cont}_1$ to put emphasis on the disregard of $2$-morphisms.
            \end{itemize} 
        \end{convention}
    
        In this section, we will be giving a demonstration of how the theory of ind-coherent sheaves on inf-schemes (cf. section \ref{section: formal_schemes_and_inf_schemes}) helps us define and describe D-modules as ind-coherent sheaves over a certain kind of prestacks, thereby naturally establishing the six-functor formalism \textit{\`a la} Grothendieck in a natural manner for D-modules, in the sense that, should a prestack $\calZ$ be sufficiently nice, one would even be able to figure out the behaviours of D-modules on $\calZ$ only from what one would expect from ind-coherent sheaves on $\calZ$. 
        
        Before submerging ourselves in the tecnnical details, let us first recall some notable features of the classical theory of D-modules (in the sense of say, Kashiwara), both to equip ourselves with an intuitive sense of purpose for our endeavour, as well as to justify this highly abstract alternative approach to D-modules. 
            \begin{enumerate}
                \item \textbf{(The classical theory):} Let $k$ be a field of characteristic $0$, and let $X$ be a scheme over $\Spec k$, which we will assume to be flat, proper, and of finite presentation for simplicity. If $X$ is smooth, then the relative cotangent complex $\bfL^*_{X/k}$ of $X/k$ shall be quasi-isomorphic to the zero complex (cf. definition \ref{def: cohomological_smoothness}), which means that the de Rham complex $\Omega^*_{X/k}$ attached to $X/k$ can only have non-zero terms in non-negative degrees; this, in turn, implies that the de Rham complex:
                    $$
                        \begin{tikzcd}
                        	0 & \E & {\Omega^1_{X/k} \tensor_{\calO_{X/k}} \E} & {\Omega^2_{X/k} \tensor_{\calO_{X/k}} \E} & {...}
                        	\arrow["\nabla", from=1-4, to=1-5]
                        	\arrow["\nabla", from=1-3, to=1-4]
                        	\arrow["\nabla", from=1-2, to=1-3]
                        	\arrow[from=1-1, to=1-2]
                        \end{tikzcd}
                    $$
                (henceforth abbreviated by $\E \tensor_{\calO_{X/k}} \Omega^*_{X/k}$) corresponding to any given \textit{flat} $\calO_{X/k}$-linear connection $\nabla: \E \to \Omega^1_{X/k} \tensor_{\calO_{X/k}} \E$ will also be concentrated in non-negative degrees. Recall also, that to specify a \textit{flat} $\calO_{X/k}$-linear connection $\nabla: \E \to \Omega^1_{X/k} \tensor_{\calO_{X/k}} \E$ on a $\calO_{X/k}$-vector bundle $\E$ over a smooth scheme $X/k$ is the same as specifying the structure of a left-$\D_{X/k}$-module on $\E$. Together, these two facts imply that the dg-category ${}^l\Dmod(X/k)$ of chain complexes of left-$\D_{X/k}$-modules is equivalent to the category whose objects are the de Rham complexes corresponding to flat connections on vector bundles over $X/k$, which we will denote by ${}^{\geq 0}\Vect(X/k)^{\nabla}_{\dR}$. As a consequence, given any smooth morphism:
                    $$f: X \to Y$$
                of schemes over $\Spec k$, one obtains a six-functor formalism between ${}^l\Dmod(X/k)$ and ${}^l\Dmod(Y/k)$ simply as that between ${}^{\geq 0}\Vect(X/k)^{\nabla}_{\dR}$ and ${}^{\geq 0}\Vect(Y/k)^{\nabla}_{\dR}$. While this is all well and good, categories of vector bundles equipped with flat connections (over smooth schemes) are rather unnatural as far as functorial constructions are concerned. 
                
                But what about cases wherein $X$ is not smooth ? In such a situation, one would embed $X$ into some smooth ambient scheme, say $Y$ (which can always be done, as one can take $X$ to be the affine singular locus and then embed that locus into a smooth deformation of its ambient scheme). Then, we can make use of what Kashiwara taught us, which is that regardless of our choice of embedding $X \hookrightarrow Y$, the restriction ${}^l\Dmod(Y/k)|_X$ of the category of left-$\D_{Y/k}$-modules onto objects (set-theoretically) supported on $X$ will be the same. Choosing such an embedding into a smooth scheme, however, would still involve appealing to resolutions, which is a rather unnatural procedure (at least from a $\infty$-categorical standpoint).
                
                Additionally, the relationship between left-D-module and right-D-modules is rather unclear within this classical framework (it turns out that via the dualising complex, left and right-D-modules are equivalent, but this does not excuse the fact that the correspondence seems rather unnatural).
                \item \textbf{(The new perspective):} Again, suppose that $k$ is a field (perhaps of characteristic $0$) and that $X$ is a scheme that is flat, proper, and of finite presentation over $\Spec k$.
                
                Let us firstly that a crystal in quasi-coherent modules over $X$ is one which is isomorphic to its restriction to the maximal reduced closed subscheme (note that $|X| \cong |{}^{\red}X|$; cf. example \ref{example: reducedeness_and_nilpotency}), which is to say that:
                    $$\E \cong \E|_{X_{\dR}}$$
                (here $X_{\dR}$ denotes the so-called \textbf{de Rham space} attached to $X$, which as a presheaf is given by $X_{\dR}(R) \cong X({}^{\red}R)$ for all commutative $k$-algebras $R$). Additionally, recall that to any crystal in quasi-coherent modules $\E \in \QCoh(X_{\dR}/k)$, one can canonically associate an $\calO_{X/k}$-linear flat connection:
                    $$\nabla: \E \to \Omega^1_{X/k} \tensor_{\calO_{X/k}} \E$$
                As stated above, when $X$ is smooth over $\Spec k$, this is the same as giving a left-$\D_{X/k}$-module, so for smooth schemes, one might as well define the dg-category of left-$\D_{X/k}$-modules to literally be the dg-category $\QCoh(X_{\dR}/k)$ of crystals in quasi-coherent modules over $X$; in fact, as schemes smooth over fields are \textit{a priori} reduced, ${}^l\Dmod(X/k)$ is nothing but $\QCoh(X/k)$.
                
                The nice thing about this approach is that we understand quasi-coherent sheaves very well, and better yet, the theory of quasi-coherent sheaves over schemes is essentially that of modules over commutative rings (cf. definition \ref{def: qcoh_def}). Moreover, it now makes sense to speak of D-modules over smooth algebraic stacks and so on (perhaps with some sort of properness or separatedness assumption imposed), as there is nothing preventing us from considering de Rham spaces attached to \textit{any presheaf} on $\Comm\Alg^{\op}$, as the definition is completely functorial. This is very useful, as there are many geometric objects which appear naturally in geometric representation theory yet are not schemes, such as the stack $\Bun_G(X)$ of principal $G$-bundles over a smooth curve $X$, for $G$ a reductive group; in fact, the so-called \say{Automorphic Side} of the Geometric Global Langlands Correspondence is the category of D-modules on this stack.
                \item \textbf{(What happens in positive characteristics ?):} The new perspective that we have just described is not without flaws, however. One of its most prominent shortcoming is that it fails completely when $k$ is instead of some positive characteristic $p$. In that situation, one can still attach to each crystal in quasi-coherent sheaves $\E$ - in a canonical manner - a flat connection $\nabla: \E \to \Omega^1_{X/k} \tensor_{\calO_{X/k}} \E$ that would generate a de Rham complex:
                    $$
                        \begin{tikzcd}
                        	0 & \E & {\Omega^1_{X/k} \tensor_{\calO_{X/k}} \E} & {\Omega^2_{X/k} \tensor_{\calO_{X/k}} \E} & {...}
                        	\arrow["\nabla", from=1-4, to=1-5]
                        	\arrow["\nabla", from=1-3, to=1-4]
                        	\arrow["\nabla", from=1-2, to=1-3]
                        	\arrow[from=1-1, to=1-2]
                        \end{tikzcd}
                    $$
                (see \cite[\href{https://stacks.math.columbia.edu/tag/07J5}{Tag 07J5}]{stacks} for details). However, a crystal in quasi-coherent sheaves is now no longer simply an object of $\QCoh(X_{\dR}/k)$, even when $X$ is smooth. \todo{Continue this}
            \end{enumerate}
                    
            \begin{convention}
                Until the end of this section, we shall be working over a field of characteristic $0$ ($k = \Q$ and $k = \bbC$ are cases of particular interest). Additionally, by \say{prestacks}, we shall always mean \say{prestacks fibred in $\infty$-groupoids}. Due to these reasons, the $\infty$-presheaf $\infty$-topos $\Spec k$ shall be denoted somewhat ambiguously by $\Pre\Stk$; likewise, by $\Comm\Alg$ we shall mean ${}^{k/}\Comm\Alg$ and by $\Sch$ (respectively $\Sch^{\aff}$), we shall mean $\Sch_{/\Spec k}$ (respectively $\Sch^{\aff}_{/\Spec k}$).
            \end{convention}
    
        \subsection{Crystals as sheaves}
            \subsubsection{The de Rham prestack}
                As eluded to in the preceding introduction, crystals (and hence D-modules) are sheaves of modules over so-called \textbf{de Rham spaces}. Therefore, before actually diving in, we shall need to study the geometry of these de Rham spaces.
            
                Before we introduce the notion of de Rham spaces attached to prestacks, however, let us make some remark regarding reduced commutative $k$-algebras. 
                \begin{remark}[Reduced affine schemes] \label{remark: reduced_affine_schemes}
                    \noindent
                    \begin{itemize}
                        \item \textbf{(Reduced rings):} Reduced objects of $\Comm\Alg$ are nothing but $0$-connective commutative $k$-algebras with vanishing nilradicals, i.e. they have no non-zero nilpotent elements, and because ring homomorphisms preserve $0$, a homomorphism:
                            $$\varphi: R \to S$$
                        between two reduced commutative $k$-algebras $R$ and $S$ will be nothing more than an ordinary algebra homomorphism. As a consequence, reduced $k$-algebras form a full subcategory of $\Comm\Alg$ (or for that matter, of ${}^{\leq 0}\Comm\Alg$); we shall denote it by ${}^{\leq 0}\Comm\Alg^{\red}$; note that $k$ itself, by virtue of being a field of characteristic $0$, is trivially reduced as an algebra over itself, and hence is still the initial object of ${}^{\leq 0}\Comm\Alg^{\red}$. 
                        \item \textbf{(Prestacks over reduced affine schemes):} The category of prestacks over ${}^{\leq 0}\Comm\Alg^{\red}$ shall be denoted by $\Pre\Stk|_{{}^{\leq 0}\Sch^{\aff, \red}}$. This is nothing but the domain restriction of the presheaf $\infty$-topos $\Pre\Stk$ (if we would view the assignment of $\infty$-presheaves $\infty$-topoi to $\infty$-categories as fibration $\Pre\Stk \to 1\-\Cat_1$) down from $\Comm\Alg^{\op}$ onto $({}^{\leq 0}\Comm\Alg^{\red})^{\op}$. 
                    \end{itemize}
                \end{remark}
                
                And now, the definition of de Rham spaces:
                \begin{definition}[de Rham prestacks] \label{def: de_rham_prestacks}
                    We can associate to any prestack $\calZ \in \Pre\Stk$ a prestack $\calZ_{\dR} \in \Pre\Stk$, called \textbf{the de Rham prestack attached to $\calZ$}, that is defined object-wise by the following formula:
                        $$\calZ_{\dR}(R) \cong \calZ({}^{\red}R)$$
                    where ${}^{\red}R$ is the classical commutative ring isomorphic to the quotient of the underlying classical commutative ring of $R$ by its nilradical. Note that this formula gives us a canonical arrow:
                        $$\calZ \to \calZ_{\dR}$$
                    coming from the canonical quotient map $R \to {}^{\red}R$. Sometimes de Rham prestacks might also go by the name \say{\textbf{de Rham spaces}}.
                \end{definition}
                \begin{remark}[Functoriality of de Rham prestacks] \label{remark: de_rham_prestacks_functoriality}
                    \noindent
                    \begin{enumerate}
                        \item Consider a morphism:
                            $$\calX \to \calY$$
                        between prestacks on $\Comm\Alg^{\op}$. It is not hard to show, via evaluation at quotients by nilradicals, that such a morphism induced a morphism of de Rham spaces:
                            $$\calX_{\dR} \to \calY_{\dR}$$
                        which tells us that there exists a so-called \textbf{de Rham space functor}:
                            $$\dR: \Pre\Stk|_{{}^{\leq 0}\Sch^{\aff, \red}} \to \Pre\Stk$$
                        that assigns de Rham spaces $\calZ_{\dR}$ to prestacks $\calZ \in \Pre\Stk|_{{}^{\leq 0}\Sch^{\aff, \red}}$. 
                        \item Due to the fact that limits and colimits of prestacks are computed object-wise, the functor $\dR$ commutes with all limits and colimits in $\Pre\Stk|_{{}^{\leq 0}\Sch^{\aff, \red}}$.
                    \end{enumerate}
                \end{remark}
                
                \begin{proposition}[The locally almost of finite type case] \label{prop: laft_de_rham_spaces}
                    If $\calZ$ is a prestack that is locally almost of finite type then so is its associated de Rham prestack $\calZ_{\dR}$. This is to say, the essential image of the functor:
                        $$\dR|_{\Pre\Stk^{\laft}}: \Pre\Stk^{\laft} \to \Pre\Stk$$
                    is a subcategory of $\Pre\Stk^{\laft}$.
                \end{proposition}
                    \begin{proof}
                        We will need to show that for all $\calZ \in \Pre\Stk^{\laft}$, the corresponding de Rham space $\calZ_{\dR}$ is convergent, and that for all $n \in \N$, the de Rham space $({}^{\leq n}\calZ)_{\dR}$ attached to the $n$-coconnective prestack ${}^{\leq n}\calZ \in {}^{\leq n}\Pre\Stk^{\laft}$ is also $n$-coconnective.
                            \begin{enumerate}
                                \item 
                                \item 
                            \end{enumerate}
                    \end{proof}
                
                \begin{remark}[Universal property of the de Rham space functor] \label{remark: universal_property_of_de_rham_spaces}
                    
                \end{remark}
                    
            \subsubsection{Left and right-crystals}
                \begin{definition}[Right-crystals] \label{def: right-crystals}
                    By composing $\dR: \Pre\Stk^{\laft} \to \Pre\Stk^{\laft}$ with the functor:
                        $$\Ind\Coh^!: (\Pre\Stk^{\laft})^{\op} \to \dg\Cat^{\cont}_1$$
                    one obtains a so-called functor of \textbf{right-crystals}:
                        $$\Crys^! \cong \Ind\Coh^! \circ \dR$$
                \end{definition}
        
        \subsection{Crystals as functors on the category of correspondences}
        
        \subsection{D-modules}
        
        \subsection{Twistings}
            \subsubsection{The notion of twists}
            
            \subsubsection{Twisted crystals}
    
    \section{D-modules and representations of Lie algebras}
        \subsection{Some supplementary Lie theory}
            \subsubsection{The notion of Lie algebras}
                \paragraph{Lie algebras in symmetric monoidal categories and their enveloping algebras}
                    \begin{definition}[Algebras, coalgebras, and bialgebras] \label{def: algebras_and_coalgebras}
                        Let $(\O, \tensor, 1)$ be a monoidal category. 
                            \begin{enumerate}
                                \item A(n) (associative and unital) algebra $A$ internal to $\O$ is a monoid object of $\O$, i.e. one equipped with a so-called multiplication:
                                    $$\nabla: A \tensor A \to A$$
                                and unit:
                                    $$\eta: 1 \to A$$
                                satisfying the following commutative diagrams:
                                    $$
                                        \begin{tikzcd}
                                        	{A \tensor A \tensor A} & {A \tensor A} \\
                                        	{A \tensor A} & {A}
                                        	\arrow["{\id_A \tensor \nabla}"', from=1-1, to=2-1]
                                        	\arrow["{\nabla}", from=2-1, to=2-2]
                                        	\arrow["{\nabla \tensor \id_A}", from=1-1, to=1-2]
                                        	\arrow["{\nabla}", from=1-2, to=2-2]
                                        \end{tikzcd}
                                    $$
                                    $$
                                        \begin{tikzcd}
                                        	{1 \tensor A} & {A \tensor A} \\
                                        	{A \tensor A} & {A}
                                        	\arrow["{\nabla}", from=1-2, to=2-2]
                                        	\arrow["{\nabla}"', from=2-1, to=2-2]
                                        	\arrow["{\id_A \tensor \eta}"', from=1-1, to=2-1]
                                        	\arrow["{\eta \tensor \id_A}", from=1-1, to=1-2]
                                        \end{tikzcd}
                                    $$
                                \item A (coassociative and counital) coalgebra internal to $\O$ is then a comonoid object of $\O$, or in other words, an monoid object in $\O^{\op}$. Typically, the so-called multiplication and counit maps on a coalgebra will be denoted by $\Delta$ and $\e$.
                            \end{enumerate}
                        Algebras, and coalgebras internal to a monoidal category $\O$ form full subcategories, which we will denote, respectively, by $\Mon(\O)$ and $\co\Mon(\O)$.
                    \end{definition}
                    \begin{example}
                        \noindent
                        \begin{enumerate}
                            \item \textbf{(Rings and algebras over them)} All rings are associative and unital algebra in the (symmetric monoidal) category of abelian groups. More generally, for any given base ring $R$ (not necessarily commutative), associative and unital left/right/two-sided $R$-algebras are monoids in the (monoidal) category of left/right/two-sided $R$-modules.
                            \item \textbf{(Lie algebras and their universal enveloping algebras)} Lie algebras, despite their names, are not algebras in the sense of definition \ref{def: algebras_and_coalgebras}, as their Lie brackets are not associative (see definition \ref{def: lie_algebras} for more details). Their univeral enveloping algebras (cf. definition \ref{def: enveloping_algebras} and theorem \ref{theorem: universal_enveloping_algebras_universal_property}), on the other hand, are associative and unital algebras. In fact, they are coassociative and counital coalgebras too (cf. remark \ref{remark: universal_enveloping_algebras_are_bialgebras}). 
                        \end{enumerate}
                    \end{example}
                
                    Let us now take a closer look at how universal enveloping algebras are bialgebras. To do so, however, we will need to have a good understanding of these algebras behave. 
                    \begin{definition}[Lie algebras] \label{def: lie_algebras}
                        Let $k$ be a ring (not necessarily commutative) and let $(\O, \tensor, 1, \tau)$ be a symmetric monoidal $k$-linear category with braiding isomorphisms:
                            $$\tau_{x, y}: x \tensor y \cong y \tensor x$$
                        A Lie algebra internal to is then an object $\g \in \O$ equipped with a so-called Lie bracket:
                            $$[-,-]: \g \tensor \g \to \g$$
                        subject to two requirements:
                            \begin{enumerate}
                                \item \textbf{(Skew-symmetry)}
                                    $$[-,-] + [-,-] \circ \tau_{\g, \g} = 0$$
                                \item \textbf{(The Jacobi identity)}
                                    $$
                                        \begin{aligned}
                                            & \left[-, [-,-]\right]
                                            \\
                                            + & \left[-, [-,-]\right] \circ \left(\id_{\g} \tensor \tau_{\g, \g}\right) \circ \left(\tau_{\g, \g} \tensor \id_{\g}\right)
                                            \\
                                            + & \left[-, [-,-]\right] \circ \left(\tau_{\g, \g} \tensor \id_{\g}\right) \circ \left(\id_{\g} \tensor \tau_{\g, \g}\right)
                                            \\
                                            = & \: 0
                                        \end{aligned}
                                    $$
                            \end{enumerate}
                        Lie algebras internal to a symmetric monoidal $k$-linear category $\O$ form a full subcategory which we shall denote by $\Lie\Alg(\O)$. Its objects are the Lie algebra objects of $\g$, and its morphisms are arrows in $\O$ that intertwine with Lie brackets, i.e. they are arrows $\phi: \g \to \h$ such that:
                            $$[-,-]_{\h} \circ (\phi \tensor \phi) = \phi \circ [-,-]_{\g}$$
                        or equivalently, such that diagrams of the following form commute in $\O$:
                            $$
                                \begin{tikzcd}
                                	{\g \tensor \g} & {\g} \\
                                	{\h \tensor \h} & {\h}
                                	\arrow["{\phi}", from=1-2, to=2-2]
                                	\arrow["{\phi \tensor \phi}"', from=1-1, to=2-1]
                                	\arrow["{[-,-]_{\h}}", from=2-1, to=2-2]
                                	\arrow["{[-,-]_{\g}}", from=1-1, to=1-2]
                                \end{tikzcd}
                            $$
                    \end{definition}
                    
                    \begin{definition}[Enveloping algebras] \label{def: enveloping_algebras}
                        Let $k$ be a ring (not necessarily commutative) and let $(\O, \tensor, 1, \tau)$ be a symmetric monoidal $k$-linear category.
                            \begin{enumerate}
                                \item \textbf{(The Lie functor)} The Lie functor is the one that assigns to each associative and unital algebra $A$ internal to $\O$ a Lie algebra $\frakLie(A)$ whose underlying object is just $A$, and whose Lie bracket is given by:
                                    $$[-,-]_{\frakLie(A)} := \nabla_A - \nabla_A \circ \tau_{A,A}$$
                                \item \textbf{(Enveloping algebras)} Fix a Lie algebra object $\g$ of $\O$. Then, an enveloping algebra of a Lie algebra $\g$ internal to $\O$ is just a Lie algebra homomorphism:
                                    $$e: \g \to \frakLie(A)$$
                                for some $A \in \Mon(\O)$. These enveloping algebras form a category, whose objects are Lie algebra homomorphisms as described above, and whose morphisms are commutative triangles in $\Lie\Alg(\O)$ as follows:
                                    $$
                                        \begin{tikzcd}
                                        	& {\g} \\
                                        	{\frakLie(A)} && {\frakLie(A')}
                                        	\arrow[from=2-1, to=2-3]
                                        	\arrow["{e}"', from=1-2, to=2-1]
                                        	\arrow["{e'}", from=1-2, to=2-3]
                                        \end{tikzcd}
                                    $$
                                which we note to be induced by algebra homomorphisms $A \to A'$. We will denote the category of enveloping algebras of $\g$ by $\Env(\g)$. 
                                \\
                                The universal enveloping algebra of $\g$ (denoted by $\U(\g)$), if it exists, then its Lie algebra will be the initial object of $\Env(\g)$. 
                            \end{enumerate}
                    \end{definition}
                    
                    \begin{theorem}[Existence and uniqueness of universal enveloping algebras] \label{theorem: universal_enveloping_algebras_universal_property}
                         Let $k$ be a ring (not necessarily commutative) and let $(\O, \tensor, 1, \tau)$ be a symmetric monoidal $k$-linear category. Then:
                            \begin{enumerate}
                                \item \textbf{(Existence and uniqueness)} There is the following ($\O$-enriched) adjunction if $\O$ has all countable coproducts:
                                    $$
                                        \begin{tikzcd}
                                        	{(\U \ladjoint \frakLie): \Mon(\O)} & {\Lie\Alg(\O)}
                                        	\arrow["{\frakLie}"{name=0, swap}, from=1-1, to=1-2, shift right=2]
                                        	\arrow["{\U}"{name=1, swap}, from=1-2, to=1-1, shift right=2]
                                        	\arrow["\dashv"{rotate=-90}, from=1, to=0, phantom]
                                        \end{tikzcd}
                                    $$
                                wherein $\U$ is the functor sending each Lie algebra in $\O$ to its universal enveloping algebra.
                                \item \textbf{(Explicit construction)} If $\O$ also has all cokernels then we can explicitly characterise the universal enveloping algebra of a Lie algebra $\left(\g, [-,-]_{\g}\right)$ as the following quotient of the (Lie algebra canonically associated to) the tensor algebra $T(\g)$:
                                    $$\frakLie\left(\U(\g)\right) \cong \coker \bigg(\left([-,-]_{\frakLie T(\g)} - \left(\nabla_{T(\g)} - \nabla_{T(\g)} \circ \tau_{T(\g), T(\g)}\right)\right): T(\g) \tensor T(\g) \to T(\g)\bigg)$$
                            \end{enumerate}
                    \end{theorem}
                        \begin{proof}
                            \noindent
                            \begin{enumerate}
                                \item We will be using Freyd's Adjoint Functor Theorem \cite[Theorem V.6.2]{maclane} to prove this assertion, and to that end, let us first note that the category $\Mon(\O)$ is complete and locally small (this can be proven in the exact same way that one might prove that algebraic categories such as $\Ab$ or $\Ring$ are complete and locally small). Next, we will try to show that the functor $\frakLie$ satisfies the \href{https://ncatlab.org/nlab/show/solution+set+condition}{\underline{solution set condition}}, which we can do by finding a small indexing set $I$ such that for all enveloping algebras $\g \to \frakLie(A)$ of $\g$, there exists a family of algebras $A_i$ indexed by $i \in I$ and factorisations:
                                    $$
                                        \begin{tikzcd}
                                        	& {\g} \\
                                        	{\frakLie(A_i)} && {\frakLie(A)}
                                        	\arrow[from=2-1, to=2-3, dashed]
                                        	\arrow["{}"', from=1-2, to=2-1]
                                        	\arrow["{}", from=1-2, to=2-3]
                                        \end{tikzcd}
                                    $$
                                (we actually do not need to worry about the size of $I$, since we have already fixed a sufficiently large Grothendieck universe). To that end, note that because $\O$ has all countable coproducts, one can always construct tensor algebras, whose universal property implies that there is the following adjunction:
                                    $$
                                        \begin{tikzcd}
                                        	{(T \ladjoint \oblv): \Mon(\O)} & {\O}
                                        	\arrow["{\oblv}"{name=0, swap}, from=1-1, to=1-2, shift right=2]
                                        	\arrow["{T}"{name=1, swap}, from=1-2, to=1-1, shift right=2]
                                        	\arrow["\dashv"{rotate=-90}, from=1, to=0, phantom]
                                        \end{tikzcd}
                                    $$
                                In particular, this means that for each $A \in \Mon(\O)$ and each Lie algebra $\g$, there is a canonical morphism from $T(\g)$ into $A$. At the same time, note that because tensor algebras are constructed as coproducts of tensor powers, one has canonical inclusions of the tensor powers $\g^{\tensor n}$ into $T(\g)$. Thus, there are commutative diagrams in $\O$ as follows for each $A \in \Mon(\O)$ and each $n \in \N$:
                                    $$
                                        \begin{tikzcd}
                                        	{\g^{\tensor n}} \\
                                        	& {T(\g)} & {A}
                                        	\arrow[from=1-1, to=2-2]
                                        	\arrow[from=2-2, to=2-3]
                                        	\arrow[from=1-1, to=2-3]
                                        \end{tikzcd}
                                    $$
                                Thus, for each associative and unital algebra $A$, there is a universal morphism in $\Env(\g)$ as follows:
                                    $$
                                        \begin{tikzcd}
                                        	& {\g} \\
                                        	{\frakLie\left(T(\g)\right)} && {\frakLie(A)}
                                        	\arrow[from=2-1, to=2-3]
                                        	\arrow[from=1-2, to=2-1]
                                        	\arrow[from=1-2, to=2-3]
                                        \end{tikzcd}
                                    $$
                                proving the existence of a small index set $I$ (the singleton in this instance) and a family of objects $\{A_i\}_{i \in I}$ of $\Mon(\O)$ such that there are factorisations as below for all $i \in I$:
                                    $$
                                        \begin{tikzcd}
                                        	& {\g} \\
                                        	{\frakLie(A_i)} && {\frakLie(A)}
                                        	\arrow[from=2-1, to=2-3, dashed]
                                        	\arrow["{}"', from=1-2, to=2-1]
                                        	\arrow["{}", from=1-2, to=2-3]
                                        \end{tikzcd}
                                    $$
                                (for the sake of clarity, let us note that here, we take $A_i = T(\g)$ for all $i \in I$). Lastly, note that the category $\Mon(\O)$ is complete and locally small. Thus, all the conditions in Freyd's Adjoint Functor Theorem are satisfied, and therefore, $\frakLie$ is a right-adjoint. This of course means that we can construct a functor:
                                    $$\U: \Lie\Alg(\O) \to \Mon(\O)$$
                                to be left-adjoint to $\frakLie$. Then, as a consequence of the universal property of adjoint pairs, the category of enveloping algebras of $\g$ must have $\frakLie\left(\U(\g)\right)$ as an initial object, which by definition is the so-called universal enveloping algebra of $\g$.
                                \item This is trivial if we note that taking the quotient of $T(\g)$ by the equivalence relation generated by the image of $[-,-]_{\frakLie T(\g)} - \left(\nabla_{T(\g)} - \nabla_{T(\g)} \circ \tau_{T(\g), T(\g)}\right)$ (which incidentally, also canonically endows $T(\g)$ with a Lie bracket) is just to ensure that the canonical map:
                                    $$\g \to (\frakLie \circ \U \circ \frakLie \circ T)(\g)$$
                                is a Lie algebra homomorphism for every $\g$.
                            \end{enumerate}
                        \end{proof}
                    \begin{corollary}
                        $\frakLie$ is left-exact and $\U$ is right-exact. In particular, we have:
                            $$\U\left(\g \oplus \g'\right) \cong \U(\g) \tensor \U(\g')$$
                        for any pair $\g, \g'$ of Lie algebras. 
                    \end{corollary}
                        \begin{proof}
                            These are general properties of adjoint functors.
                        \end{proof}
                    \begin{remark}
                        The adjunction as presented in the preceding theorem can be understood as fitting into the following (non-commutative) diagram:
                            $$
                                \begin{tikzcd}
                                	{\Mon(\O)} & {} & {\O} \\
                                	& {\Lie\Alg(\O)}
                                	\arrow["{\oblv}"{name=0, swap}, from=1-1, to=1-3, shift right=2]
                                	\arrow["{T}"{name=1, swap}, from=1-3, to=1-1, shift right=2]
                                	\arrow["{L}"{name=2, swap}, from=2-2, to=1-3, shift right=5]
                                	\arrow[""{name=3, inner sep=0}, from=1-3, to=2-2, shift left=1]
                                	\arrow["{\U}"{name=4, swap}, from=2-2, to=1-1, shift left=1]
                                	\arrow["{\frakLie}"{name=5, swap}, from=1-1, to=2-2, shift right=5]
                                	\arrow["\dashv"{rotate=-90}, from=1, to=0, phantom]
                                	\arrow["\dashv"{rotate=61}, from=5, to=4, phantom]
                                	\arrow["\dashv"{rotate=129}, from=2, to=3, phantom]
                                \end{tikzcd}
                            $$
                    \end{remark}
                    \begin{remark}[Universal enveloping algebras are bialgebras] \label{remark: universal_enveloping_algebras_are_bialgebras}
                        Let $k$ be a ring and let $\g$ be a Lie algebra internal to some $k$-linear symmetric monoidal category $(\V, \tensor, 1, \tau)$. Then, its universal enveloping algebra $\U(\g)$ is a cocommutative bialgebra internal to $\V$, which is commutative if and only if $\g$ is abelian. 
                        
                        This becomes trivial if we let the comultiplication be given by:
                            $$\Delta_{\U(\g)} := \id_{\U(\g)} \tensor \e + \e \tensor \id_{\U(\g)}$$
                        and the counit be:
                            $$\e_{\U(\g)} := 0$$
                    \end{remark}
                        
                    \begin{lemma}[Embedding of Lie algebras into their universal enveloping algebras]
                        Let $k$ be a ring (not necessarily commutative) and let $(\O, \tensor, 1, \tau)$ be a symmetric monoidal $k$-linear \textbf{abelian} category. Also, suppose that $\g$ is a (faithfully ?) flat Lie algebra object internal to $\O$, i.e. that the functor $\g \tensor -$ is (faithfully ?) flat. Then, there is a canonical monomorphism embedding $\g$ itno $\frakLie \U(\g)$
                    \end{lemma}
                        \begin{proof}
                            \todo{Currently I'm not certain that this statement is entirely correct.}
                        \end{proof}
                    
                    \begin{example}[The abelian case]
                        If $\g$ is an abelian Lie algebra (i.e. one whose Lie bracket is just the zero morphism), then as a direct consequence of the definition of symmetric algebras and of theorem \ref{theorem: universal_enveloping_algebras_universal_property}, we have the following isomorphism of associative and unital algebras:
                            $$\U(\g) \cong \Sym(\g)$$
                    \end{example}
                    \begin{example}[The Poincar\'e-Birkhoff-Witt Theorem]
                        Let $k$ be a commutative and unital $\Q$-algebra, let $(\O, \tensor, 1, \tau)$ be a symmetric monoidal $k$-linear category with all countable coproducts and cokernels, and let $\g$ be a Lie algebra object of $\O$ that is (faithfully ?) flat (i.e. one such that the functor $\g \tensor -$ is left-exact). Then, there is the following isomorphism of cocommutative $\N$-graded $k$-bialgebras:
                            $$\U(\g) \cong \Sym(\g)$$
                        In other words, one can understand representations of such a Lie algebra $\g$ via representations of the symmetric algebra $\Sym(\g)$, which in turn are just representations of symmetric groups.
                    \end{example}
                    \begin{example}[A counter-example]
                        Let $k$ be a field of characteristic $0$ and consider the symmetric monoidal category of finite-dimensional $k$-vector spaces. Then clearly, $\O$ does not have all countable coproducts (for example, the vector space $k^{\oplus \aleph_0}$, which is a coproduct indexed by the countable infinite cardinal $\aleph_0$, is not an object as it is infinite-dimensional), and thus not all Lie $k$-algebras have a universal enveloping algebra.
                    \end{example}
                    \begin{example}[Open problem]
                        It is not known if the functor:
                            $$\U: \frakLie(\O) \to \Lie\Alg(\O)$$
                        is faithful, and even if that is not the case in general, we also do not have a good understanding of which extra assumptions to impose on our ambient symmetric monoidal linear category $\O$ so that in such a setting, $\U$ might be faithful. 
                    \end{example}
                    
                    \begin{lemma}[Tannaka duality] \label{lemma: tannaka_duality}
                        Let $\O$ be a closed symmetric monoidal category that is locally small (such as the symmetric monoidal category of vector spaces over a field), so that it may be enriched over itself via its internal homs (which exist thanks to the monoidal closure assumption), and let $A$ be a monoid object of $\O$. If we denote the category of $\O$-representations on $A$ by:
                            $$\Rep_{\O}(A) := \O\-\Cat(\bfB A^{\op}, \O)$$
                        then the algebra $\End_{\O\-\Cat(\Rep_{\O}(A), \O)}(F)$ of $\O$-natural endomorphisms on the canonical forgetful functor $F: \Rep_{\O}(A) \to \O$ is isomorphic to $A$.
                    \end{lemma}
                        \begin{proof}
                            Apply the $\O$-enriched Yoneda's lemma:
                                $$
                                    \begin{aligned}
                                        \End_{\O\-\Cat(\Rep_{\O}(A), \O)}(F) & \cong \O\-\Cat(\Rep_{\O}(A), \O)(F, F)
                                        \\
                                        & \cong \Psh_{\O}\left(\Psh_{\O}(\bfB A)\right)\bigg(\Psh_{\O}(\bfB A)(*, -), \Psh_{\O}(\bfB A)(*, -)\bigg)
                                        \\
                                        & \cong \Psh_{\O}(\bfB A)(*, *)
                                        \\
                                        & \cong \Rep_{\O}(A)(*, *)
                                        \\
                                        & \cong \End_{\O}(A)
                                        \\
                                        & \cong A
                                    \end{aligned}
                                $$
                        \end{proof}
                    \begin{theorem}
                        If $k$ is a ring, $\O$ is a \textit{locally small} \textit{closed} symmetric monoidal $k$-linear category with all coproducts and cokernels, and $\g$ is a (faithfully ?) flat Lie algebra object of $\O$, then there is an equivalence of abelian monoidal $k$-linear categories as follows:
                            $$\Rep_{\O}(\g) \cong {\U(\g)}\mod$$
                        wherein $\Rep_{\O}(\g)$, the category of representations of $\g$ on $\O$, is the category whose objects are Lie algebra homomorphisms from $\g$ to $\frakgl(V)$ (with $\frakgl(V)$ the canonical Lie algebra associated to the associative and unital endomorphism algebra $\End_{\O}(V)$), and morphisms are commutative triangles in $\Lie\Alg(\O)$ of the form:
                            $$
                                \begin{tikzcd}
                                	& {\g} \\
                                	{\frakgl(V)} && {\frakgl(V')}
                                	\arrow[from=2-1, to=2-3]
                                	\arrow[from=1-2, to=2-1]
                                	\arrow[from=1-2, to=2-3]
                                \end{tikzcd}
                            $$
                        Also, note that ${\U(\g)}\mod$ is \href{https://ncatlab.org/nlab/show/module+over+a+monoid}{\underline{well-defined}} as the category of $\U(\g)$-left-equivariant objects of $\O$, and it exhibits all properties that one might expect of a module category.
                    \end{theorem}
                        \begin{proof}
                            Before we prove this claim, let us first note that because $\O$ has all countable coproducts and cokernels, all Lie algebras possess universal enveloping algebras. With that out of the way, let us note that by Tannaka duality (cf. lemma \ref{lemma: tannaka_duality}), each left-$\U(\g)$-module is actually just a $\U(\g)$-representation. Thus, to show that:
                                $$\Rep_{\O}(\g) \cong {\U(\g)}\mod$$
                            it will suffice to show that:
                                $$\Rep_{\O}(\g) \cong \Rep_{\O}\left(\U(\g)\right)$$
                            (this might seem like a round-about method, but without Tannaka duality, guaranteeing that the equivalence is actually between abelian monoidal $k$-linear categories instead of just between ordinary categories will be difficult). In turn, one can do this via showing that there is the following equivalence of categories of Lie algebra representations:
                                $$\Rep_{\O}(\g) \cong \Rep_{\O}\left(\frakLie \U(\g)\right)$$
                            thanks to the uniqueness of the universal enveloping algebra. Then, we can simply apply lemma 1.1.1, which states that $\g$ is a subobject of $\frakLie \U(\g)$, and consider commutative diagrams as follows in $\Lie\Alg(\O)$:
                                $$
                                    \begin{tikzcd}
                                    	{\g} \\
                                    	& {\frakgl(V)} \\
                                    	{\frakLie \U(\g)}
                                    	\arrow[from=1-1, to=2-2]
                                    	\arrow[from=3-1, to=2-2]
                                    	\arrow[from=1-1, to=3-1, tail]
                                    \end{tikzcd}
                                $$
                            to show that for each representation of $\g$ on $V \in \O$, there is a representation of $\frakLie \U(\g)$ on $V$ as well, and vice versa.
                        \end{proof}
                    
                \paragraph{Factoring Lie algebra homomorphisms}
                    \begin{convention}
                        From now on, we work within a fixed $k$-linear tensor category $\O$ (i.e. a $k$-linear abelian dualisable symmetric monoidal category). For all intents and purposes, one might think of $\O$ as being $k\mod$.
                    \end{convention}
                    
                    The upshot for this segment is that categories of Lie algebras are not quite abelian. In fact, Lie algebras behave more like groups than modules (unsurprising, given the Lie group-Lie algebra Correspondence), in the sense that monomorphisms of Lie algebras are kernels if and only if they are \textit{normal}. It should be noted, however, the non-unitality of Lie algebras makes categories of Lie algebras a little better than categories of groups. 
                    
                    We begin with the notions of subalgebras and ideals.
                    \begin{definition}[Subalgebras and ideals] \label{def: lie_subalgebras_and_lie_ideals}
                        Let $\g$ be a Lie algebra internal to $\O$. 
                            \begin{enumerate}
                                \item \textbf{(Subalgebras):} A \textbf{Lie subalgebra} of $\g$ is thus 
                                \item \textbf{(Ideals):}
                            \end{enumerate}
                    \end{definition}
                    
            \subsubsection{Homological algebra of Lie algebras}
                \paragraph{Cohomologies and homologies of Lie algebra}
                    \begin{definition}[Lie algebra cohomologies] \label{def: lie_algebra_cohomologies}
                        Let $\g$ be a Lie algebra over a commutative ring $k$ and let $A$ be a $\g$-module. Then, we define the $n^{th}$ cohomology group of $\g$ with coefficient in $A$ as:
                            $$H^n(\g, A) \cong \Ext^n_{\U(\g)}(k, A)$$
                        where we view $k$ as the $1$-dimensional $\g$-representation on the right-hand side. 
                    \end{definition}
                    \begin{remark}
                        It is not hard to see via induction on the cohomological dimension $n \in \N$ that $H^n(\g, A)$ (as in definition \ref{def: lie_algebra_cohomologies}) actually has a $k$-module structure.
                    \end{remark}
                    
                    Let us now attempt to compute cohomologies of Lie algebras in low dimensions (namely, $n = 0, 1, 2$) as well as give meaning to these spaces (cf. proposition \ref{prop: low_dimensional_cohomologies_of_lie_algebras}). We shall require, first of all, the following lemma:
                    \begin{lemma}[de Rham resolutions of Lie algebras] \label{lemma: de_rham_resolutions_of_lie_algebras}
                        For $\g$ a Lie algebra over a commutative ring $k$ and $A$ any $\g$-module, one has the following isomorphism of cohomology groups:
                            $$H^n(\g, A) \cong \Ext^n_k(\Lambda^{\bullet}(\g), A)$$
                    \end{lemma}
                        \begin{proof}
                            By definition, we have:
                                $$H^n(\g, A) \cong \Ext^n_{\U(\g)}(k, A)$$
                            Through noting that $\Lambda^{\bullet}(\g) \cong \{k \to \g \to \g \wedge \g \to \cdots\}$ is an injective resolution of $k \in \U(\g)\mod$, one sees that these isomorphisms of cohomologies \say{lift} to the following quasi-isomorphism of cochain complexes:
                                $$H^{\bullet}(\g, A) \cong_{\qis} \R\Hom_{\U(\g)}\left(\U(\g) \tensor_k^{\L} \Lambda^{\bullet}(\g), A\right)$$
                            An application of the tensor-hom adjunction then gives:
                                $$H^{\bullet}(\g, A) \cong_{\qis} \R\Hom_k(\Lambda^{\bullet}(\g), A)$$
                            which implies:
                                $$H^n(\g, A) \cong \Ext^n_k(\Lambda^{\bullet}(\g), A)$$
                        \end{proof}
                    
                    Now, because the low-dimensional cohomology groups of Lie algebras are interpreted in terms of various extra structures on these Lie algebras, let us first write down some auxiliary definitions.
                    \begin{definition}[Derivations on Lie algebras] \label{def: lie_algebra_derivations}
                        A \textbf{derivation} on a Lie algebra $\g$ internal to any suitable tensor category (cf. definition \ref{def: lie_algebras}) is nothing but a derivation in the sense of definition \ref{def: derivations}, i.e. an endomorphism:
                            $$D: \g \to \g$$
                        making the following diagram (expressing the Leibniz Rule) commute:
                            $$
                                \begin{tikzcd}
                                	{\g \tensor \g} & \g \\
                                	{\g \tensor \g} & {\g}
                                	\arrow["{D \tensor \id_{\g} + \id_{\g} \tensor D}"', from=1-1, to=2-1]
                                	\arrow["{[-,-]}", from=2-1, to=2-2]
                                	\arrow["{[-,-]}", from=1-1, to=1-2]
                                	\arrow["D", from=1-2, to=2-2]
                                \end{tikzcd}
                            $$
                        Here, 
                    \end{definition}
                    \begin{remark}
                        The space of derivations on a given Lie algebra $\g$ over a commutative ring $k$ is also a Lie algebra over $k$, whose Lie bracket is inherited from the endomorphism algebra $\frakgl(\g)$. We denote this Lie algebra by $\der(\g)$.
                    \end{remark}
                    \begin{proposition}[Lie algebra derivation universal property] \label{prop: lie_algebra_derivations_universal_property}
                        Let $\g$ be a Lie algebra over a commutative ring $k$\footnote{One can generalise this result rather easily to a setting wherein $\g$ is interal to a tensor category: $k$ can simply be replace by the monoidal unit.}. Then, $\der(\g, -): \g\mod \to k\mod$ is not only a functor, but also left-adjoint to the forgetful functor $\oblv: k\mod \to \g\mod$. Furthermore, it is corepresented by the augmentation ideal:
                            $$\rmI(\g) := \ker(\Delta_{\U(\g)}: \U(\g) \tensor \U(\g) \to \U(\g): f \tensor g \mapsto fg)$$
                        i.e.:
                            $$\der(\g, -) \cong \Hom_{(\g)}(\rmI(\g), -)$$
                    \end{proposition}
                        \begin{proof}
                            
                        \end{proof}
                    \begin{definition}[Inner derivations on Lie algebras] \label{def: lie_algebra_inner_derivations}
                        Any derivation $D: \g \to \g$ on a Lie algebra $\g$ such that:
                            $$\forall y \in \g: \exists x \in \g: D(y) = \ad_{\g}(x)(y) = [x, y]$$
                        is known as an \textbf{inner derivation}\footnote{This is in analogy with inner automorphisms of groups}.
                    \end{definition}
                    \begin{remark}[Adjoint maps are derivations]
                        It is an easy exericse (one can simply manipulate the Jacobi identity defining Lie brackets) to show that any adjoint map:
                            $$[x, -]: \g \to \g$$
                        is actually a derivation. We thus leave it up to the reader to fill in the details. 
                    \end{remark}
                    \begin{convention}
                        The set of inner derivations on a given Lie algebra $\g$ over a commutative ring $k$ is a Lie algebra over $k$ whose structure is inherited from the endomorphism algebra $\frakgl(\g)$ (namely, its Lie bracket is the commutator with respect to composition of endomorphisms). We denote this Lie algebra by $\inn(\g)$.
                    \end{convention}
                    \begin{remark}
                        $\inn(\g)$ is a Lie subalgebra of $\der(\g)$ (cf. definition \ref{def: lie_subalgebras_and_lie_ideals}), and hence of $\frakgl(\g)$. 
                    \end{remark}
                        
                    We can now compute the low-dimensional cohomology groups of Lie algebras.
                    \begin{proposition}[Low-dimensional cohomologies of Lie algebras] \label{prop: low_dimensional_cohomologies_of_lie_algebras}
                        For $\g$ a Lie algebra over a commutative ring $k$ and $A$ any $\g$-module, one has the following interpretations of the low-dimensional cohomology spaces of $\g$ with coeffcients in $A$:
                        \begin{enumerate}
                            \item \textbf{($n = 0$: Invariants):} There exists a canonical $k$-module isomorphism:
                                $$H^0(\g, A) \cong \{a \in A \mid \forall x \in \g: x \cdot a = 0\}$$
                            When $A \cong \g$, this is furthermore an isomorphism of Lie algebras over $k$, with the Lie bracket on the left-hand side being inherited from $\g$.
                            \item \textbf{($n = 1$: Derivations):} There exists a canonical Lie algebra isomorphism:
                                $$H^1(\g, \g) \cong \out(\g)$$
                            \item \textbf{($n = 2$: Extensions):} There exists a natural $k$-module isomorphism between the space of isomorphism classes of extensions of $\g$ by $A$ and $H^2(\g, A)$.
                        \end{enumerate}
                    \end{proposition}
                        \begin{proof}
                            \noindent
                            \begin{enumerate}
                                \item \textbf{($n = 0$: Invariants):}
                                \item \textbf{($n = 1$: Derivations):}
                                \item \textbf{($n = 2$: Extensions):}
                            \end{enumerate}
                        \end{proof}
                    
                    Lemma \ref{lemma: de_rham_resolutions_of_lie_algebras} also help us makes sense of homologies of Lie algebras, defined as follows:
                    \begin{definition}[Lie algebra homologies] \label{def: lie_algebra_homologies}
                        Let $\g$ be a Lie algebra over a commutative ring $k$ and let $A$ be a $\g$-module. Then, we define the $n^{th}$ cohomology group of $\g$ with coefficient in $A$ as:
                            $$H_n(\g, A) \cong \Tor_n^{\U(\g)}(k, A)$$
                        where we view $k$ as the $1$-dimensional $\g$-representation on the right-hand side. 
                    \end{definition}
                    \begin{remark}
                        It is not hard to show, using the associativity of the derived tensor product, that one has the following quasi-isomorphism:
                            $$\Tor_{\bullet}^{\U(\g)}(k, A) \cong_{\qis} \Lambda^{\bullet}(\g) \tensor_k^{\L} A$$
                    \end{remark}
                
                \paragraph{Cohomology of semi-simple Lie algebras over characteristic \texorpdfstring{$0$}{} and Whitehead's two lemmas}
                    We begin with some algebraic prelimiaries. Specifically, we want to understand the notion of (semi)simplicity within the context of Lie theory, due to simple Lie algebras not quite being simple objects in categories of Lie algebras.
                    \begin{definition}[Simple Lie algebras] \label{def: simple_lie_algebras}
                        A Lie algebra internal to an appropriate tensor category $\O$ (cf. definition \ref{def: lie_algebras}) is \textbf{simple} if and only if it is a \textit{non-abelian} (i.e. the Lie bracket in question is not the zero morphism) simple object of $\Lie\Alg(\O)$.
                        
                        More algebraically, one might say that a simple Lie algebra is a non-abelian Lie algebra with no non-trivial proper ideal.
                    \end{definition}
                
                    Let us now move on to two important results concerning semi-simple Lie algebras, namely:
                        \begin{itemize}
                            \item the fact that finite-dimensional representations of semi-simple Lie algebras are completely reducible, and
                            \item that every Lie algebra which is finite-dimensional splits into the direct sum of its so-called \say{radical} and some semi-simple Lie algebra. 
                        \end{itemize}
                    During the way, we shall prove two technical lemmas, commonly known as Whitehead's First and Second Lemmas.
                    
                    \begin{convention} \label{conv: cohomology_of_semi_simple_lie_algebras_conventions}
                        Throughout this paragraph, $\g$ shall denote a finite-dimensional semi-simple Lie algebra over a field $k$ of characteristic $0$, and $A$ shall denote a $\g$-module that is of finite dimension as a $k$-vector space (i.e. a finite-dimensional $\g$-representation) with structural homomorphism:
                            $$\rho: \g \to \frakgl(A)$$
                    \end{convention}
                    
                    \begin{proposition}[Associated invariant bilinear forms] \label{prop: associated_invariant_bilinear_forms_of_lie_algebras}
                        To $\g$, $A$, and $\rho$ as in convention \ref{conv: cohomology_of_semi_simple_lie_algebras_conventions}, there exists a symmetric bilinear form:
                            $$\beta: \g \tensor \g \to k$$
                        given by\footnote{Note that $\beta(x, y)$ should technically be written as $\beta(x \tensor y)$.}:
                            $$x \tensor y \mapsto \beta(x, y) := \trace(\rho(x) \rho(y))$$
                        Furthermore, such a bilinear form is $\ad_{\g}$-invariant, i.e.:
                            $$\beta([x, y], z) = \beta(x, [y, z])$$
                        for all $x, y, z \in \g$.
                    \end{proposition}
                        \begin{proof}
                            Bilinearity is a straightforward consequence of the linearity of $\rho$ (in the two factors $x$ and $y$) and the linearity of the trace map $\trace: \frakgl(A) \to k$. As for symmetry, it is a consequence of the well-known fact that traces of endomorphisms on finite-dimensional vector spaces (whihc $\frakgl(A)$ is, since $A$ is a finite-dimensional $k$-vector space) are invariant under cyclic permutations.
                            
                            Now, to prove that $\beta$ is $\ad_{\g}$-invariant, consider the following:
                                $$
                                    \begin{aligned}
                                        \beta([x, y], z) & = \trace(\rho([x, y]) \rho(z))
                                        \\
                                        & = \trace([\rho(x), \rho(y)] \rho(z))
                                        \\
                                        & = \trace((\rho(x)\rho(y) - \rho(y)\rho(x)) \rho(z))
                                        \\
                                        & = \trace(\rho(x) \rho(y)\rho(z)) - \trace(\rho(y)\rho(x)\rho(z))
                                        \\
                                        & = \trace(\rho(x) \rho(y)\rho(z)) - \trace(\rho(x)\rho(z)\rho(y))
                                        \\
                                        & = \trace(\rho(x)[\rho(y), \rho(z)])
                                        \\
                                        & = \beta(x, [y, z])
                                    \end{aligned}
                                $$
                        \end{proof}
                    \begin{example}[The Killing Form] \label{example: the_killing_form}
                        Since $\g$ is assumed to be finite-dimensional over $k$, one can meaningfully construct ymmetric bilinear forms in the fashion of proposition \ref{prop: associated_invariant_bilinear_forms_of_lie_algebras} for the case $A \cong \g$. In particular, when $\rho$ is the adjoint representation:
                            $$\ad_{\g}: \g \to \End_k(\g): x \mapsto \left(x \mapsto [x, -]: \g \to \g\right)$$
                        the associated symmetric bilinear form is the \textbf{Killing Form} $\kappa: \g \tensor \g \to k$, which is defined via:
                            $$\kappa(x, y) := \trace(\ad_{\g}(x) \ad_{\g}(y))$$
                    \end{example}
                    
                    \begin{theorem}[Non-degeneracy of associated invariant bilinear forms] \label{theorem: nondegeneracy_of_associated_invariant_bilinear_forms_of_lie_algebras}
                        The associated invariant bilinear form is non-degenerate when the representation $\rho: \g \to \frakgl(A)$ is faithful.
                    \end{theorem}
                        \begin{proof}
                            
                        \end{proof}
                    \begin{corollary}
                        The adjoint representation is faithful, so the Killing Form is non-degenerate. 
                    \end{corollary}
                    
            \subsubsection{Structure and classification of compact Lie algebras}
    
        \subsection{Localisation of \texorpdfstring{$\g$}{}-modules; the Beilinson-Bernstein Equivalence}
            \begin{convention} \label{conv: beilinson_bernstein_localisation_conventions}
                \noindent
                \begin{itemize}
                    \item We work with a complex algebraic group $G$ of adjoint type with \textit{a priori} simple Lie algebra $\g$, along with universal enveloping algebra $\U(\g)$ and centre $\rmZ(\g)$ thereof. $B$ shall be a Borel subgroup thereof. 
                    \item $\Gr_G$ shall denote the loop affine Grassmannian $G(\!(t)\!)/G[\![t]\!]$ attached to $G$.
                    \item If $T$ is a maximal torus inside $G$ and $\lambda \in \bbX(T)$ is a weight then we shall denote by $\Dmod(G/B)^{(\lambda)}$ the category of $\lambda$-twisted D-modules on $G/B$ (i.e. D-modules on $G/B$ which act on the line bundle $G \x_B \lambda$), where $B \leq G$ is a choice of Borel subgroup that contains the fixed torus $T$.
                    \item The Lie algebras of $G, B$, and $T$ shall be denoted - respectively - by $\g, \b$, and $\t$.
                \end{itemize}
            \end{convention}
            
            In this subsection, we shall be presenting a proof by Gaitsgory and Frenkel of the Beilinson Bernstein Localisation Theorem, which shall then (in subsection \ref{subsection: localisation_of_affine_lie_algebras}) be generalised to obtain an analogous statement regarding localisations of modules over affine Lie algebras. Let us first jot down some initial observations and then give an outline of the idea behind the proof.
            
            Let $\pi: G \to G/B$ denote the canonical projection and observe that there exists a canonically determined sheaf pullback functor:
                $$\pi^*: \Dmod(G/B) \to \Dmod(G)$$
            which in turn induces a \say{twisted} pullback functor, compatible with the aforementioned pullback $\pi^*$ in some appropriate and natural sense:
                $$(\pi^{(\lambda)})^*: \Dmod(G/B)^{(\lambda)} \to \Dmod(G)$$
            Also, note the fact that for all $\calM \in \Dmod(G)$, the global section $\Gamma(G, \calM)$ has a natural $\U(\g)$-bimodule structure thanks to the self-actions of $G$ via left and right-translations. Among other things, this implies that there exists the following natural composition of functors:
                $$
                    \begin{tikzcd}
                    	{\Dmod(G/B}) & {\Dmod(G)} & {\U(\g)\bimod}
                    	\arrow["{\pi^*}", from=1-1, to=1-2]
                    	\arrow["{\Gamma(G, -)}", from=1-2, to=1-3]
                    \end{tikzcd}
                $$
            which induces the following composition wherein the first functor is now \say{twisted} by the character $\lambda$:
                $$
                    \begin{tikzcd}
                    	{\Dmod(G/B)^{(\lambda)}} & {\Dmod(G)} & {\U(\g)\bimod}
                    	\arrow["{(\pi^{(\lambda)})^*}", from=1-1, to=1-2]
                    	\arrow["{\Gamma(G, -)}", from=1-2, to=1-3]
                    \end{tikzcd}
                $$
            Set:
                $$\Gamma^{(\lambda)} := \Gamma(G, -) \circ (\pi^{(\lambda)})^*$$
            
            Now, let $\calV_{\lambda}$ denote the Verma module attached to a given weight $\lambda$ and set:
                $$\bfGamma^{(\lambda)} := \Hom_{\U(\g)}(\calV_{\lambda}, -) \circ \Gamma^{(\lambda)}$$
            Eventually, we will be establishing an adjunction:
                $$
                    \begin{tikzcd}
                    	{\Dmod(G/B)^{(\lambda)}} & \O
                    	\arrow[""{name=0, anchor=center, inner sep=0}, "{\bfGamma^{(\lambda)}}"', shift right=2, from=1-1, to=1-2]
                    	\arrow[""{name=1, anchor=center, inner sep=0}, "{\Delta^{(\lambda)}}"', shift right=2, from=1-2, to=1-1]
                    	\arrow["\dashv"{anchor=center, rotate=-90}, draw=none, from=1, to=0]
                    \end{tikzcd}
                $$
            wherein $\O$ is the so-called \say{Category $\O$} whose construction we shall get to\footnote{For now, think of $\O$ as a suitable subcategory of the category of right-$\U(\g)$-modules.}; here $\Delta^{(\lambda)}$ is given by:
                $$M \mapsto \D_{G/B} \tensor^{\L}_{\U(\g)} M$$
                
            All of this is simply to say that, the global section of the pullback along $\pi: G \to G/B$ of any D-modules on $G/B$ twisted by some character $\lambda$, remains unchanged under \say{twisting} by the Verma module $\calV_{\lambda}$. The adjunction $\Delta^{(\lambda)} \ladjoint \bfGamma^{(\lambda)}$ - as complicated as it may seem - is actually just an enhanced version of the usual tensor-hom adjunction.
            
            Now, this is not all there is to the theorem; it in fact says a lot more. Interesting phenomena occur when the weight $\lambda + \rho$ (where $\rho$ is the half-sum of the positive roots of $\g$) is \textit{dominant}: specifically, should this be the case, the Verma module $\calV_{\lambda}$ shall be a \textit{projective} object of the category $\O$, and hence the \say{twisted} global section functor:
                $$\bfGamma^{(\lambda)}: \Dmod(G/B)^{(\lambda)} \to \O$$
            shall be the composition of two exact functors, and thus \textit{exact} itself. One can interpret this fact as $G/B$ being \say{D-affine} up to a twisting by a weight, in the same sense that higher cohomologies of quasi-coherent sheaves over Noetherian affine schemes vanish. 
                    
            \subsubsection{Kazhdan-Lusztig modules, the category \texorpdfstring{$\O$}{}, and Verma modules}
                    
            \subsubsection{Proving the theorem}
                The following geometric result shall help us make better sense of the Beilinson-Bernstein Localisation Theorem (cf. theorem \ref{theorem: beilinson_bernstein_localisation}). In fact, it is the true \say{localisation theorem}: the Beilinson-Bernstein Theorem simply refines the adjunction established by this results down to an adjoint equivalence (which, of course, is of great representation-theoretic significance)\footnote{Additionally, note how the notion of weight does not appear at all in theorem \ref{theorem: localisation_adjunction_for_D_modules}}.
                
                \begin{lemma}[The localisation adjunction for quasi-coherent associative algebras] \label{lemma: localisation_adjunction_for_quasi_coherent_associative_algebras}
                    Let $X$ be a scheme and consider $\calB \in \Assoc\Alg(\QCoh(X))$ with global section $B \cong \Gamma(X, \calB)$. Then, one has the following derived adjunction:
                        $$
                            \begin{tikzcd}
                            	{{}^l\calB\mod} & {B\bimod}
                            	\arrow[""{name=0, anchor=center, inner sep=0}, "{\R\Gamma(X, -)}"', shift right=2, from=1-1, to=1-2]
                            	\arrow[""{name=1, anchor=center, inner sep=0}, "{\calB \tensor_B^{\L} -}"', shift right=2, from=1-2, to=1-1]
                            	\arrow["\dashv"{anchor=center, rotate=-90}, draw=none, from=1, to=0]
                            \end{tikzcd}
                        $$
                \end{lemma}
                    \begin{proof}
                        By definition, $\R\Gamma(X, -) \cong \R\Hom_{{}^l\calB\mod}(\calB, -)$, so this is just the hom-tensor adjunction.
                    \end{proof}
                \begin{corollary} \label{coro: localisation_adjunction_for_quasi_coherent_associative_algebras}
                    If $B$ (as in lemma \ref{lemma: localisation_adjunction_for_quasi_coherent_associative_algebras}) is a bialgebra over some \say{deeper} base associative ring $B_0$ then the adjunction of lemma \ref{lemma: localisation_adjunction_for_quasi_coherent_associative_algebras} will extend down to:
                        $$
                            \begin{tikzcd}
                            	{{}^l\calB\mod} & {B_0\bimod}
                            	\arrow[""{name=0, anchor=center, inner sep=0}, "{\R\Gamma(X, -)}"', shift right=2, from=1-1, to=1-2]
                            	\arrow[""{name=1, anchor=center, inner sep=0}, "{\calB \tensor_{B_0}^{\L} -}"', shift right=2, from=1-2, to=1-1]
                            	\arrow["\dashv"{anchor=center, rotate=-90}, draw=none, from=1, to=0]
                            \end{tikzcd}
                        $$
                \end{corollary}
                \begin{theorem}[The localisation adjunction for D-modules] \label{theorem: localisation_adjunction_for_D_modules}
                    Let $X$ be a smooth algebraic variety with a $G$-action\footnote{$G$ still as in convention \ref{conv: beilinson_bernstein_localisation_conventions}} (which means that $G$ \textit{a priori} has the structure of an $X$-scheme through a structural morphism $\pi: G \to X$). Then, there exists the following derived adjunction:
                        $$
                            \begin{tikzcd}
                            	{\Dmod(X)} & {\U(\g)\bimod}
                            	\arrow[""{name=0, anchor=center, inner sep=0}, "{\bfGamma}"', shift right=2, from=1-1, to=1-2]
                            	\arrow[""{name=1, anchor=center, inner sep=0}, "{\D_X \tensor^{\L}_{\U(\g)} -}"', shift right=2, from=1-2, to=1-1]
                            	\arrow["\dashv"{anchor=center, rotate=-90}, draw=none, from=1, to=0]
                            \end{tikzcd}
                        $$
                    wherein the (derived) functor $\bfGamma: \Dmod(X) \to \U(\g)\bimod$ returns global sections of pullbacks of D-modules on $X$ to $G$, i.e. for all $\calM \in \Dmod(X)$, one has:
                        $$\bfGamma(\calM) \cong \Gamma(G, \pi^*\calM)$$
                \end{theorem}
                    \begin{proof}
                        Corollary \ref{coro: localisation_adjunction_for_quasi_coherent_associative_algebras} tells us that there exists the following adjunction:
                            $$
                                \begin{tikzcd}
                                	{\Dmod(G)} & {\U(\g)\bimod}
                                	\arrow[""{name=0, anchor=center, inner sep=0}, "{\R\Gamma(G, -)}"', shift right=2, from=1-1, to=1-2]
                                	\arrow[""{name=1, anchor=center, inner sep=0}, "{\D_G \tensor^{\L}_{\U(\g)} -}"', shift right=2, from=1-2, to=1-1]
                                	\arrow["\dashv"{anchor=center, rotate=-90}, draw=none, from=1, to=0]
                                \end{tikzcd}
                            $$
                        Next, consider the following pair of derived adjunctions:
                            $$
                                \begin{tikzcd}
                                	{\Dmod(X)} & {\Dmod(G)} & {\U(\g)\bimod}
                                	\arrow[""{name=0, anchor=center, inner sep=0}, "{\R\pi^*}"', shift right=2, from=1-1, to=1-2]
                                	\arrow[""{name=1, anchor=center, inner sep=0}, "{\D_X \tensor^{\L}_{\D_G} -}"', shift right=2, from=1-2, to=1-1]
                                	\arrow[""{name=2, anchor=center, inner sep=0}, "{\R\Gamma(G, -)}"', shift right=2, from=1-2, to=1-3]
                                	\arrow[""{name=3, anchor=center, inner sep=0}, "{\D_G \tensor^{\L}_{\U(\g)} -}"', shift right=2, from=1-3, to=1-2]
                                	\arrow["\dashv"{anchor=center, rotate=-90}, draw=none, from=1, to=0]
                                	\arrow["\dashv"{anchor=center, rotate=-90}, draw=none, from=3, to=2]
                                \end{tikzcd}
                            $$
                        They are trivially composable, and the resulting pair of functors is \textit{a priori} the sought-for derived adjunction:
                            $$
                                \begin{tikzcd}
                                	{\Dmod(X)} & {\U(\g)\bimod}
                                	\arrow[""{name=0, anchor=center, inner sep=0}, "{\bfGamma}"', shift right=2, from=1-1, to=1-2]
                                	\arrow[""{name=1, anchor=center, inner sep=0}, "{\D_X \tensor^{\L}_{\U(\g)} -}"', shift right=2, from=1-2, to=1-1]
                                	\arrow["\dashv"{anchor=center, rotate=-90}, draw=none, from=1, to=0]
                                \end{tikzcd}
                            $$   
                        thanks to the right/left-exactness (with respect to the canonical t-structures) of the composite functors (the composition of the two tensoring-up functors is trivially right-exact, and the composition $\R\Gamma(G, -) \circ \R\pi^*$ is left-exact since both factors are \textit{a priori} left-exact functors).  
                    \end{proof}
                \begin{corollary}[An application to flag varieties]
                    Since $G/B$ is a smooth variety, theorem \ref{theorem: localisation_adjunction_for_D_modules} specialises to the case of $X \cong G/B$.
                \end{corollary}
                
                It now remains to establish a functor from the category of $\U(\g)$-bimodules to the category $\O$, which shall be $\Hom_{\U(\g)}(\calV_{\lambda}, -)$ where $\calV_{\lambda}$ is the Verma module attached to a weight $\lambda$. 
                \begin{theorem}[The Beilinson-Bernstein Localisation Theorem] \label{theorem: beilinson_bernstein_localisation}
                    
                \end{theorem}
                    \begin{proof}
                        
                    \end{proof}
                \begin{corollary}[The Borel-Weil-Bott Theorem] \label{coro: borel_weil_bott_theorem}
                    
                \end{corollary}
        
        \subsection{Chiral D-modules on the affine Grassmannians and representations of Kac-Moody algebras} \label{subsection: localisation_of_affine_lie_algebras}
            \begin{convention} \label{conv: chiral_beilinson_bernstein_localisation_conventions}
                \noindent
                \begin{itemize}
                    \item We work with a complex algebraic group $G$ of adjoint type with \textit{a priori} simple Lie algebra $\g$, along with universal enveloping algebra $\U(\g)$ and centre $\rmZ(\g)$ thereof. $B$ shall be a Borel subgroup thereof. 
                    \item $\Gr_G$ shall denote the loop affine Grassmannian $G(\!(t)\!)/G[\![t]\!]$ attached to $G$.
                    \item If $T$ is a maximal torus inside $G$ and $\lambda \in \bbX(T)$ is a weight then we shall denote by $\Dmod(G/B)^{(\lambda)}$ the category of $\lambda$-twisted D-modules on $G/B$ (i.e. D-modules on $G/B$ which act on the line bundle $G \x_B \lambda$), where $B \leq G$ is a choice of Borel subgroup that contains the fixed torus $T$.
                    \item The Lie algebras of $G, B$, and $T$ shall be denoted - respectively - by $\g, \b$, and $\t$.
                \end{itemize}
            \end{convention}
            
            \begin{convention} \label{conv: loop_algebras_1}
                If $\g$ is a simple complex Lie algebra then we shall denote by $\g(\!(t)\!)$ the corresponding loop algebra, which is defined to be isomorphic to $\g \tensor_{\bbC} \bbC(\!(t)\!)$.
            \end{convention}
            
            \subsubsection{\textit{Pr\'elude}: Affine Kac-Moody algebras}
                We begin by recalling the following result, which classifies $\ad_{\g}$-invariant inner products on our given simple Lie algebra $\g$:
                \begin{lemma}[Invariant inner products on simple complex Lie algebras] \label{lemma: invariant_inner_products_on_simple_complex_lie_algebras}
                    On any given simple complex Lie algebra $\g$, there exists (up to non-zero scalar multiples) only one $\ad_{\g}$-invariant $\kappa$, namely the Killing Form, which is defined to be a linear map:
                        $$\kappa: \g \tensor \g \to \bbC$$
                    given by:
                        $$x \tensor y \mapsto \trace(\ad_{\g}(x) \ad_{\g}(y))$$
                    Furthermore, this space of invariant inner products is isomorphic (as a $\bbC$-vector space) to $H^2(\g(\!(t)\!), \bbC)$.
                \end{lemma}
                    \begin{proof}
                        
                    \end{proof}
                We shall also need to recall the general fact that extensions of a given Lie algebra $\g$ (for now, not necessarily simple), i.e. short exact sequences of $R$-modules of the form $0 \to M \to \hat{\g}_M \to \g \to 0$ by some $\g$-module $M$\footnote{Note that $M$ is necessarily a Lie algebra ideal of $\hat{g}_M$ and hence a Lie subalgebra of $\hat{g}$, due to Lie algebras being non-unital.}, are parametrised by $2$-cocyles $\kappa \in H^2(\g, M)$:
                \begin{lemma}[Extensions of Lie algebras are $2$-cocycles] \label{lemma: lie_algebra_extensions_are_2_cocycles}
                    Let $\g$ be an arbitrary Lie algebra over any commutative ring (cf. definition \ref{def: lie_algebras}) and let $M$ be a $\g$-module. Then, extensions $\hat{\g}_M$ of $\g$ by $M$ are in bijective correspondence with elements $\kappa \in H^2(\g, M)$.
                \end{lemma}
                    \begin{proof}
                        
                    \end{proof}
                \begin{corollary}
                    Central extensions of any Lie algebra $\g$ over a commutative ring $k$ are in bijection with elements of $H^2(\g, R)$.
                \end{corollary}
                    
                Now, by putting lemma \ref{lemma: invariant_inner_products_on_simple_complex_lie_algebras} and lemma \ref{lemma: lie_algebra_extensions_are_2_cocycles} together, one can meaningfully speak of central extensions of \say{central extensions of the loop algebra $\g(\!(t)\!)$ with respect to a given invariant inner product $\kappa$ on $\g$} in the following manner:
                \begin{definition}[Central extension with respect to invariant inner products] \label{def: central_extensions_with_respect_to_invariant_inner_products}
                    Let $\g$ be a simple complex Lie algebra and let $\kappa$ be an $\ad_{\g}$-invariant inner product thereon. Then, an \textbf{extension of $\g(\!(t)\!)$ with respect to $\kappa$} is nothing but a (central) extension $\hat{\g}_{\kappa}$ of $\g$ with respect to $\bbC$.
                    
                    Due to $H^2(\g(\!(t)\!), \bbC)$ being of dimension $1$ over $\bbC$, the extension $\hat{\g}_{\kappa}$ is actually unique up to scalar multiples, and thus is usually known as \textbf{\textit{the} affine Kac-Moody algebra associated to $\g$ of level $\kappa$}.
                \end{definition}
                
                Let us also state the following accompanying definition:
                \begin{definition}[Representations of the Kac-Moody algebra] \label{def: kac_moody_algebra_representations}
                    \noindent
                    \begin{enumerate}
                        \item 
                        \item 
                    \end{enumerate}
                \end{definition}
                
            \subsubsection{Towards a chiral version of the Beilinson-Bernstein Localisation Theorem}
            
            \subsubsection{The case of the Kac-Moody algebra at negative and at irrational levels}
            
            \subsubsection{The case of the Kac-Moody algebra at the critical level}
            
            \subsubsection{Global sections of chiral D-modules on the affine Grassmannian}
            
        \subsection{Localisation of \texorpdfstring{$\hat{\g}$}{}-modules}
            \begin{convention} \label{conv: quantum_beilinson_bernstein_localisation_conventions}
                \noindent
                \begin{itemize}
                    \item We work with a complex algebraic group $G$ of adjoint type with \textit{a priori} simple Lie algebra $\g$, along with universal enveloping algebra $\U(\g)$ and centre $\rmZ(\g)$ thereof. $B$ shall be a Borel subgroup thereof. 
                    \item $\Gr_G$ shall denote the loop affine Grassmannian $G(\!(t)\!)/G[\![t]\!]$ attached to $G$.
                    \item If $T$ is a maximal torus inside $G$ and $\lambda \in \bbX(T)$ is a weight then we shall denote by $\Dmod(G/B)^{(\lambda)}$ the category of $\lambda$-twisted D-modules on $G/B$ (i.e. D-modules on $G/B$ which act on the line bundle $G \x_B \lambda$), where $B \leq G$ is a choice of Borel subgroup that contains the fixed torus $T$.
                    \item The Lie algebras of $G, B$, and $T$ shall be denoted - respectively - by $\g, \b$, and $\t$.
                \end{itemize}
            \end{convention}
            
            \subsubsection{The Hecke Category}
            
            \subsubsection{The Main Theorem}
            
    \section{The Riemann-Hilbert Correspondence}