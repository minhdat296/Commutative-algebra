\chapter{Chiral algebras; Opers}
    \begin{abstract}
        
    \end{abstract}
    
    \section{D-schemes}
        \subsection{Pseudo-tensor categories; compound tensor categories} \label{subsection: pseudo_and_compound_tensor_categories}
            We introduce the following terminology in keeping with \cite[Chapter 1]{beilinson2004chiral}
            \begin{definition}[Pseudo-tensor categories] \label{def: }
                A \textbf{pseudo-tensor category} is a coloured operad enriched in some (symmetric monoidal) category of vector spaces. 
            \end{definition}
    
        \subsection{The compound tensor structure on D-modules}
            In this subsection we establish how the category of D-modules on a smooth scheme possesses the structure of a compound tensor category. For details on pseudo-tensor categories and compound tensor structures, see subsection \ref{subsection: pseudo_and_compound_tensor_categories}.
    
        \subsection{The geometry of D-schemes}
    
    \section{Chiral algebras}
    
    \section{Opers and Miura Opers}
        \subsection{Opers}
            \subsubsection{The definition of opers}
                \begin{definition}[Opers] \label{def: opers}
                    
                \end{definition}
        
        \subsection{Miura Opers}
    
    \section{Wakimoto Modules}
    
    \section{\texorpdfstring{$q$}{}-opers}