\chapter{Algebraic topology and higher category theory}
    \begin{abstract}
        
    \end{abstract}
    
    \minitoc
    
    \section{Higher operads and higher categories}
        \subsection{Monoidal categories; enrichments}
            \subsubsection{Classical monoidal categories}
                \begin{definition}[Monoidal categories] \label{def: monoidal_categories}
                    A monoidal category is a quintuple $(\calV, \tensor, 1, \alpha, (\lambda, \rho))$ of:
                        \begin{enumerate}
                            \item a category $\calV$;
                            \item a bifunctor $\tensor: \calV \x \calV \to \calV$;
                            \item a distinguished object $1 \in \calV$ - called the \textbf{unit} - which we shall view as a functor $\eta: \pt \to \calV$ from the terminal category;
                            \item a natural isomorphism of functors - called the \textbf{associator} - as follows:
                                $$
                                    \begin{tikzcd}
                                    	{(\calV \x \calV) \x \calV} && {\calV \x (\calV \x \calV)} \\
                                    	\\
                                    	{\calV \x \calV} && {\calV \x \calV} \\
                                    	& \calV
                                    	\arrow["{\tensor \x \id_{\calV}}"', from=1-1, to=3-1]
                                    	\arrow["\tensor"', from=3-1, to=4-2]
                                    	\arrow["\tensor", from=3-3, to=4-2]
                                    	\arrow["{\id_{\calV} \x \tensor }", from=1-3, to=3-3]
                                    	\arrow["\cong", from=1-1, to=1-3]
                                    	\arrow[""{name=0, anchor=center, inner sep=0}, "{(- \tensor -) \tensor -}"{description}, from=1-1, to=4-2]
                                    	\arrow[""{name=1, anchor=center, inner sep=0}, "{- \tensor (- \tensor -)}"{description}, from=1-3, to=4-2]
                                    	\arrow["\alpha", shorten <=26pt, shorten >=26pt, Rightarrow, from=0, to=1]
                                    \end{tikzcd}
                                $$
                            \item natural isomorphisms $\lambda$ and $\rho$ - known, respectively, as the \textbf{left and right-unitors} - as follows:
                                $$
                                    \begin{tikzcd}
                                    	{\calV \x \pt} & {\calV \x \calV} & {\pt \x \calV} \\
                                    	\calV & \calV & \calV
                                    	\arrow["{\tensor}"{description}, from=1-2, to=2-2]
                                    	\arrow["{\eta \x \id_{\calV}}"', from=1-3, to=1-2]
                                    	\arrow["{\id_{\calV} \x \eta}", from=1-1, to=1-2]
                                    	\arrow["{\id_{\calV}}"', from=2-1, to=2-2]
                                    	\arrow["{\pr_1}"', from=1-1, to=2-1]
                                    	\arrow["{\pr_2}", from=1-3, to=2-3]
                                    	\arrow["{\id_{\calV}}", from=2-3, to=2-2]
                                    	\arrow["\lambda"', shorten <=13pt, shorten >=13pt, Rightarrow, from=1-2, to=2-1]
                                    	\arrow["\rho", shorten <=13pt, shorten >=13pt, Rightarrow, from=1-2, to=2-3]
                                    \end{tikzcd}
                                $$
                            (note that the functors $\pr_1: \calV \x \pt \to \calV$ and $\pr_2: \pt \x \calV \to \calV$ are equivalences \textit{a priori}, thanks to the universal property of the terminal objects and that of products).
                        \end{enumerate}
                    
                \end{definition}
                \begin{definition}[Lax-monoidal categories] \label{def: lax_monoidal_categories}
                    Let $(\calV, \tensor, 1, \alpha, (\lambda, \rho))$ a quintuple as in definition \ref{def: monoidal_categories}, but now, suppose that $\alpha: (- \tensor -) \tensor - \to - \tensor (- \tensor -)$ is a non-invertible $2$-cell. Then, this quintuple will define a so-called \textbf{lax-monoidal category} if and only if 
                \end{definition}
                \begin{definition}[Non-unital monoidal categories]
                    
                \end{definition}
                
                \begin{definition}[Braidings and symmetries] \label{def: braided_and_symmetric_monoidal_categories}
                    
                \end{definition}
                
            \subsubsection{Categories enriched over monoidal categories; 2-categories}
            
            \subsubsection{Unbiased monoidal categories}
        
        \subsection{Operads and multicategories}
            \begin{definition}[Cartesian categories] \label{def: cartesian_categories}
                \noindent
                \begin{enumerate}
                    \item \textbf{(Cartesian categories):} A category is said to be \textbf{Cartesian} if it has all pullbacks. A \textbf{Cartesian} functor (not necessarily between Caterisan categories) is one that preserves pullbacks. 
                    \item \textbf{(Cartesian natural transformations):} A natural transformation:
                        $$
                            \begin{tikzcd}
                            	\C & \D
                            	\arrow[""{name=0, anchor=center, inner sep=0}, "G"', shift right=3, from=1-1, to=1-2]
                            	\arrow[""{name=1, anchor=center, inner sep=0}, "F", shift left=3, from=1-1, to=1-2]
                            	\arrow["\alpha", shorten <=2pt, shorten >=2pt, Rightarrow, from=1, to=0]
                            \end{tikzcd}
                        $$
                    between two functors $F, G: \C \to \D$ (where $\C, \D$ need not be Cartesian categories) if the naturality square induced by any arrow $f: x \to y$ in $\C$ is a pullback square in $\D$:
                        $$
                            \begin{tikzcd}
                            	Fx & Fy \\
                            	Gx & Gy
                            	\arrow["Ff", from=1-1, to=1-2]
                            	\arrow["Gf", from=2-1, to=2-2]
                            	\arrow["{\alpha_x}"', from=1-1, to=2-1]
                            	\arrow["{\alpha_y}", from=1-2, to=2-2]
                            	\arrow["\lrcorner"{anchor=center, pos=0.125}, draw=none, from=1-1, to=2-2]
                            \end{tikzcd}
                        $$
                    \item \textbf{(Cartesian monads):} Let $\E$ be a Cartesian category .A so-called \textbf{Cartesian monad} $(T: \E \to \E, \mu: T^2 \to T, \eta: \id \to T)$ internal to $\E$ is thus one wherein the defining endomofunctor $T: \E \to \E$ is Carterian, and so are the natural transformations $\mu: T^2 \to T$ and $\eta: \id \to T$.
                \end{enumerate}
            \end{definition}
            \begin{remark}[$2$-category of Catersian categories] \label{remark: 2_category_of_cartesian_categories}
                There exists a natural 
            \end{remark}
            \begin{example}[Cartesian categories] \label{example: cartesian_categories}
                    
            \end{example}
            \begin{example}[Cartesian monads] \label{example: cartesian_monads}
                    
            \end{example}
        
        \subsection{Weak \texorpdfstring{$n$}{}-categories}
            \subsubsection{Globular operads}
            
            \subsubsection{The many definitions of weak \texorpdfstring{$n$}{}-categories}
        
    \section{Algebraic K-theory}
    
    \section{\texorpdfstring{$(\infty, 1)$}{}-categories and \texorpdfstring{$\infty$}{}-topoi}
    
    \section{\texorpdfstring{$(\infty, 2)$}{}-categories}