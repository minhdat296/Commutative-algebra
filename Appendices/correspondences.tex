\chapter{The \texorpdfstring{$(\infty, 2)$}{}-category of correspondences}
    \begin{abstract}
        
    \end{abstract}
    
    \minitoc

    \section{The 2-category of correspondences}
        \subsection{The paradigm of spans}
            \subsubsection{What are correspondences ?}
                \begin{remark}[Why correspondences ?] \label{remark: why_correspondences}
                    Let $\C$ be an arbitrary base category and let $\S$ be a target category of \say{spaces} ($\S$ can be ${}^{\infty}\Grpd$ or $\Cat^{\dg, \cont}$ for example), and suppose that there exists the following pullback square in $\C$:
                        $$
                            \begin{tikzcd}
                            	{c_1'} & {c_1} \\
                            	{c_2'} & {c_2}
                            	\arrow["f", from=1-2, to=2-2]
                            	\arrow["g", from=2-1, to=2-2]
                            	\arrow["{f'}"', from=1-1, to=2-1]
                            	\arrow["{g'}", from=1-1, to=1-2]
                            	\arrow["\lrcorner"{anchor=center, pos=0.125}, draw=none, from=1-1, to=2-2]
                            \end{tikzcd}
                        $$
                    Consider now a product functor:
                        $$\Phi_* \x \Phi^!: \C \x \C^{\op} \to \S \x \S$$
                    
                \end{remark}
            
                \begin{definition}[Spans] \label{def: spans}
                    \noindent
                    \begin{enumerate}
                        \item \textbf{(Spans):} A \textbf{span} (also known as a \textbf{correspondence}) from an object $x$ to another object $y$ via an object $s$ inside a given category $\C$ is a diagram therein that is of the following form:
                            $$
                                \begin{tikzcd}
                                	&&& s & y \\
                                	{} &&& x
                                	\arrow["f"', from=1-4, to=2-4]
                                	\arrow["g", from=1-4, to=1-5]
                                \end{tikzcd}
                            $$
                        wherein the tip $s$ is some \textit{choice} of object of $\C$. As any span with either or both arrow therein being an identity is just a normal morphism, the notion of spans can be thought of as a generalisation of morphisms.  
                        \item \textbf{(Categories of spans):} Within a category $\C$ with pullbacks, one can compose, say, a span from $x$ to $y$ via $s$ with another from $y$ to $z$ via $s'$ via taking the pullback of the \say{inner} arrows in the manner depicted by the following diagram:
                            $$
                                \begin{tikzcd}
                                	{s \x_{g, y, f'} s'} & {s'} & z \\
                                	s & y \\
                                	x
                                	\arrow["f"', from=2-1, to=3-1]
                                	\arrow["g"', from=2-1, to=2-2]
                                	\arrow["{f'}", from=1-2, to=2-2]
                                	\arrow["{g'}", from=1-2, to=1-3]
                                	\arrow[from=1-1, to=2-1]
                                	\arrow[from=1-1, to=1-2]
                                	\arrow["\lrcorner"{anchor=center, pos=0.125}, draw=none, from=1-1, to=2-2]
                                \end{tikzcd}
                            $$
                        to obtain a so-called \say{composite} span from $x$ to $z$ via $s \x_{g, y, f'} s'$ (which visually, can be thought of either as the upper \say{roof} or the big \say{roof} covering the two smaller ones - that being the previously specified spans from $x$ to $y$ via $s$ and the span from $y$ to $z$ via $s'$ - in the above diagram); alternatively, one might visualise this composition via the following commutative diagram of spans:
                            $$
                                \begin{tikzcd}
                                	x & y & z
                                	\arrow["{(f,g)}", "\shortmid"{marking}, from=1-1, to=1-2]
                                	\arrow["{(f' g')}", "\shortmid"{marking}, from=1-2, to=1-3]
                                \end{tikzcd}
                            $$
                        With the use of this style of composition, one obtains, for every category $\C$ with pullbacks in tandem with a choice of natural number $n \geq 1$, a (lax) \textbf{$n$-category of spans} $\Span^{\leq n}(\C)$ that is defined via:
                            \begin{enumerate}
                                \item objects being those of $\C$ itself,
                                \item $1$-morphisms being spans between objects, which shall henceforth be known as $1$-spans,
                                \item and for all $2 \leq k \leq n$, $k$-morphisms, which shall henceforth be called $k$-spans, being $k$-cells between $(k-1)$-morphisms; for instance, a $2$-span between two $1$-spans is a commutative diagram of the following form:
                                    $$
                                        \begin{tikzcd}
                                        	\bullet \\
                                        	& \bullet & \bullet \\
                                        	& \bullet
                                        	\arrow[from=1-1, to=2-2]
                                        	\arrow[from=1-1, to=3-2]
                                        	\arrow[from=2-2, to=3-2]
                                        	\arrow[from=2-2, to=2-3]
                                        	\arrow[from=1-1, to=2-3]
                                        \end{tikzcd}
                                    $$
                            \end{enumerate}
                    \end{enumerate}
                \end{definition}
                \begin{convention}[Regarding notations] \label{conv: span_notations}
                    Obviously, writing out spans explicitly takes up a lot of effort and frankly, these diagrams can get confusing rather quickly. However, a quick observation tells us that because composites of spans are given by pullbacks, they actually satisfy the universal property of products taken in the arrow category of some given span category. That is to say, given an ambient category $\C$ with all pullbacks and two composable spans:
                        $$
                            \begin{tikzcd}
                            	& \bullet & \bullet \\
                            	\bullet & \bullet \\
                            	\bullet
                            	\arrow["f"', from=2-1, to=3-1]
                            	\arrow["g", from=2-1, to=2-2]
                            	\arrow["{f'}", from=1-2, to=2-2]
                            	\arrow["{g'}", from=1-2, to=1-3]
                            \end{tikzcd}
                        $$
                    therein, their composite:
                        $$
                            \begin{tikzcd}
                            	\bullet & \bullet & \bullet \\
                            	\bullet & \bullet \\
                            	\bullet
                            	\arrow["f"', from=2-1, to=3-1]
                            	\arrow["g"', from=2-1, to=2-2]
                            	\arrow["{f'}", from=1-2, to=2-2]
                            	\arrow["{g'}", from=1-2, to=1-3]
                            	\arrow[from=1-1, to=2-1]
                            	\arrow[from=1-1, to=1-2]
                            	\arrow["\lrcorner"{anchor=center, pos=0.125}, draw=none, from=1-1, to=2-2]
                            \end{tikzcd}
                        $$
                    is nothing but the product:
                        $$
                            \begin{tikzcd}
                            	{(f, g) \x (f', g')} & {(f', g')} \\
                            	{(f, g)}
                            	\arrow[dashed, from=1-1, to=2-1]
                            	\arrow[dashed, from=1-1, to=1-2]
                            \end{tikzcd}
                        $$
                    in $\Mor\left(\Span^{\leq 1}(\C)\right)$ (which will probably be commonly written as $\Span^{\leq 1}(\C)_1$ from now on, so as to make the notion of spans fit snuggly into the language of internal categories, and also to cut back on parentheses) Therefore, our proposition of an alternative notation is as follows: the composition of two spans $(f, g)$ and $(f', g')$ shall instead be denoted by $(f, g) \x (f', g')$.
                \end{convention}
                \begin{convention}[Associators]
                    Fix a \textit{weak} $n$-category $\C$, with $n \geq 1$. Now, for all $1 \leq k \leq n$ and all triples of \textit{composable} $(k - 1)$-morphisms $f, g, h$ of $\C$, let us call the $k$-cell:
                        $$(fg)h \to f(gh)$$
                    (which we note to be invertible thanks to the weakness assumption on $\C$) a \textbf{$k$-associator}. An associator is called \textbf{trivial} if and only if it is an identity. See \href{https://ncatlab.org/nlab/show/associator}{\underline{here}} for more details. 
                \end{convention}
                \begin{remark}[Associators in span categories]
                    Since pullbacks are merely unique up to unique isomorphisms, $k$-associators in the $n$-category $\Span^{\leq n}(\C)$ of spans of a given category $\C$ with pullbacks are generally non-trivial, but they are invertible. This implies that $n$-categories of spans are \textit{weak} $n$-categories, as opposed to simply being lax $n$-categories, but generally they are not strict $n$-categories. 
                \end{remark}
                
                \begin{proposition}[Limits and colimits of spans]
                    
                \end{proposition}
            
            \subsubsection{Correspondences via grids}
        
        \subsection{The universal property of the category of correspondences}
        
        \subsection{Enlarging classes of 2-morphisms}
        
        \subsection{Factorisation}
        
    \section{Cohomological base-changing} \label{section: cohomological_base_change}
        
    \section{The symmetric monoidal structure on the category of correspondences}