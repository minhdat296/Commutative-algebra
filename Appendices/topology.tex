\chapter{Algebraic topology}
    \begin{abstract}
        
    \end{abstract}
    
    \minitoc
    
    \section{Condensed mathematics} \label{section: condensed_mathematics}
        \begin{remark}[Regular cardinals]
            We will be using the notion of regular cardinals often. For details on the notion, see definition \ref{def: limit_cardinal}.
        \end{remark}
    
        \subsection{Basics of condensed mathematics}
            \subsubsection{Condensed sets}
                \begin{definition}[Condensation] \label{def: condensation}
                    Let $\kappa$ be a fixed regular cardinal and let $\C$ be a hypercomplete $\infty$-category with enough $\kappa$-small limits and enough $\kappa$-small filtered colimits. We then define so-called \textbf{condensed objects} of $\C$ to be $\C$-valued sheaves over the $\kappa$-small pro-\'etale site of a point (i.e. the pro-\'etale site of the spectrum of a field). 
                    
                    Clearly condensed objects of a given $\infty$-category $\C$ satisfying the above conditions form a category. We shall denote it by $\C^{\cond}$.
                \end{definition}
                \begin{remark}
                    We refer the reader to proposition \ref{prop: perfect_pro_etale_sites}, and remarks \ref{remark: non_affine_perfect_pro_etale_sites} and \ref{remark: non_perfect_pro_etale_sites} for the definition of pro-\'etale coverages. In particular, recall that the $\kappa$-small pro-\'etale site of a point is equivalent to the site $\Pro_{\kappa}(\Sets^{\fin})$ of $\kappa$-small profinite sets (\textit{viewed as totally disconnected $\kappa$-small compact Hausdorff spaces}), whose coverage is generated by jointly surjective families. 
                \end{remark}
                \begin{example}
                    \noindent
                    \begin{itemize}
                        \item \textbf{(\textit{Small} condensed sets):} The pro-\'etale topos $\Sh(*_{\proet}^{< \kappa})$ over a point is, by definition, the category of sheaves of sets on the pro-\'etale site of a point. Therefore, this topos is the category of \textbf{$\kappa$-small condensed sets}. Whenenver we wish to put emphasis on the fact that $\Sh(*_{\proet}^{< \kappa})$ is actually the category of $\kappa$-small condensed sets, we will write $\Sets^{\cond, < \kappa}$ instead.
                        \item \textbf{(Condensed abelian groups and modules):} If $R$ is a condensed commutative ring, then the category ${}_R\Mod^{\cond}$ of condensed $R$-modules is a \href{https://ncatlab.org/nlab/show/Grothendieck+category}{\underline{Grothendieck category}}; on the other hand, topological abelian groups even fails to form an abelian category, and as we well know: no abelian categories means no homological algebra. This is a biggy, so we shall bestow upon it the dignity of theorem-hood (see theorem \ref{theorem: abelian_categories_of_condensed_modules}).
                    \end{itemize}
                \end{example}
                \begin{remark}[Condensation and profiniteness] \label{remark: condensation_and_profiniteness}
                    \noindent
                    \begin{enumerate}
                        \item Fix a regular cardinal $\kappa$. The $\kappa$-pro-\'etale site of a point is equivalent to the category $\Pro_{\kappa}(\Sets^{\fin})$ of $\kappa$-small profinite sets equipped with the coverage generated by jointly surjective finite families of surjective functions. One can show this using the fact that the pro-\'etale site of a point is the same as the pro-\'etale site of the spectrum of a field, and subsequently, the Fundamental Theorem of Galois Theory, namely the fact that the Galois group of any Galois extension $L/K$ is the filtered limit over the galois groups $\Gal(E/K)$ over all \textit{finite} Galois subextensions $E/K$.
                        \item One important fact to keep in mind is that profinite sets are compact and Hausdorff. This will be used in lemma \ref{lemma: sheaves_over_compact_hausdorff_spaces} to show that the sheaf tops over the category of small compact Hausdorff spaces is the same as the sheaf topos of $\kappa$-small condensed sets.
                    \end{enumerate}
                \end{remark}
                
                \begin{lemma}[Sheaves over compact Hausdorff spaces] \label{lemma: sheaves_over_compact_hausdorff_spaces}
                    Fix a regular cardinal $\kappa$ and denote by $\Sh(\Comp^{\kappa})$ the sheaf topos over the site of $\kappa$-small compact Hausdorff topological spaces (with coverage given by jointly surjective families). Then, one has the following equivalence of topoi:
                        $$\Sh(\Comp^{< \kappa}) \cong \Sh(*_{\proet}^{< \kappa})$$
                    between the aforementioned sheaf topos and the topos of $\kappa$-small condensed sets.
                \end{lemma}
                    \begin{proof}
                        \noindent
                        \begin{enumerate}
                            \item \textbf{(Sheaves over Stone spaces):} A $\kappa$-small Stone spaces is a totally disconnected $\kappa$-small compact Hausdorff topological space; it is not hard to see that such a space would admit a (finite) covering by profinite sets. 
                            \item \textbf{(Sheaves over compact Hausdorff spaces):}
                        \end{enumerate}
                    \end{proof}
                
                \begin{definition}[Extremally disconnected sets] \label{def: extrememly_disconnected_sets}
                    A compact Hausdorff space $S$ is extremally disconnected if and only if any epimorphism $e: S' \to S$ splits.
                \end{definition}
                \begin{example}
                    \noindent
                    \begin{enumerate}
                        \item \textbf{(Stone-\v{C}ech compactifications):}
                        \item \textbf{(A counter-example: the $p$-adics):} For a fixed prime $p$, the $p$-adic rationals $\Q_p$ is only totally disconnected, not extremally disconnected. 
                    \end{enumerate}
                \end{example}
                
                \begin{lemma}[Small condensed sets and extremally disconnected sets] \label{lemma: small_condensed_sets_and_extremally_disconnected_sets}
                    Fix a regular cardinal $\kappa$ and denote the topos of sheaves of sets on the site of $\kappa$-small extremally disconnected sets with coverage given by \textit{finite} jointly surjective families by $\Sh(\sfExt^{< \kappa})$. Then, one has the following equivalence of topoi:
                        $$\Sh(\sfExt^{< \kappa}) \cong \Sh(*_{\proet}^{< \kappa})$$
                    between the aforementioned sheaf topos and the topos of $\kappa$-small condensed sets.
                \end{lemma}
                    \begin{proof}
                        
                    \end{proof}
                    
                \begin{lemma}[Enlargements of condensed sets] \label{lemma: large_condensed_sets}
                    Let $\kappa$ be a regular cardinal and let $\lambda > \kappa$ be a possibly uncountable, possibly strong limit cardinal (cf. definition \ref{def: limit_cardinal}). The natural embedding of $\Sets^{\cond, < \kappa}$ into $\Sets^{\cond, < \lambda}$ is compatible with pro-\'etale sheafification. This is to say, the following diagram of topoi commutes:
                        $$
                            \begin{tikzcd}
                            	{\Sh(*_{\proet}^{< \kappa})} & {\Psh(*_{\proet}^{< \kappa})} \\
                            	{\Sh(*_{\proet}^{< \lambda})} & {\Psh(*_{\proet}^{< \lambda})}
                            	\arrow[hook, from=1-1, to=2-1]
                            	\arrow[hook, from=1-2, to=2-2]
                            	\arrow["{{}^{\sh}(-)}"', from=1-2, to=1-1]
                            	\arrow["{{}^{\sh}(-)}"', from=2-2, to=2-1]
                            \end{tikzcd}
                        $$
                \end{lemma}
                    \begin{proof}
                        
                    \end{proof}
                
            \subsubsection{Abelian categories of condensed objects}
                \begin{theorem}[Abelian categories of condensed modules] \label{theorem: abelian_categories_of_condensed_modules}
                    Fix a regular cardinal $\kappa$. Categories of modules ${}_R\Mod^{\cond}$ over $\kappa$-condensed commutative rings $R$ satisfy the following Grothendieck homological axioms:
                        \begin{enumerate}
                            \item They are abelian categories.
                            \item \textbf{($AB3$ \& $AB3^*$):} They are $\kappa$-small complete and $\kappa$-small cocomplete.
                            \item \textbf{($AB3$ \& $AB4^*$):} Coproducts of monics remain monic, and dually, products of epics remain epic.
                            \item \textbf{($AB5$):} Filtered colimits of exact sequences remain exact. Furthermore, categories of condensed modules are compactly generated: this is to say, its generator, call it $\Lambda$ is a compact object (i.e. the copresheaf ${}_R\Mod^{\cond}(\Lambda, -)$ preserves filtered colimits).
                            \item \textbf{($AB6$):} $\kappa$-small products commute with $\kappa$-small filtered colimits.
                        \end{enumerate}
                \end{theorem}
                    \begin{proof}
                        \noindent
                        \begin{enumerate}
                            \item Categories of condensed modules are categories of internal modules - specifically to the pro-\'etale topos over a point - and so are trivially abelian. 
                            \item \textbf{($AB3$ \& $AB3^*$):} Again, condensed modules are internal modules, and thus the categories they form are \textit{a priori} $AB5$, and hence $AB3$.
                            \item \textbf{($AB4$ \& $AB4^*$):} 
                            \item \textbf{($AB5$):} We have already shown that categories of condensed modules are $AB5$, so it remains to show that the generator of ${}_R\Mod^{\cond}$ is compact.
                            \item \textbf{($AB6$):} 
                        \end{enumerate}
                    \end{proof}
                \begin{corollary}[Properties of categories of condensed modules] \label{coro: condensed_modules_properties}
                    By virtue of being a Grothendieck category (i.e. an $AB5$-category with a generator), any condensed module category ${}_R\Mod^{\cond}$ enjoy the following properties:
                        \begin{enumerate}
                            \item 
                                \begin{enumerate}
                                    \item If a presheaf $F: ({}_R\Mod^{\cond})^{\op} \to \Sets$ preserves $\kappa$-small limits, then it is representable (i.e. the Yoneda embedding on ${}_R\Mod^{\cond}$ preserves all $\kappa$-small limits, not just the finite ones).
                                    \item If a presheaf $({}_R\Mod^{\cond})^{\op} \to \Sets$ commutes with all $\kappa$-small colimits, then it possesses a right-adjoint. 
                                \end{enumerate}
                            \item By the \href{https://ncatlab.org/nlab/show/Gabriel-Popescu+theorem}{\underline{Gabriel-Popescu Theorem}}, we can realise ${}_R\Mod^{\cond}$ as a reflective localisation of some category of modules over a commutative ring. 
                            \item ${}_R\Mod^{\cond}$ is presentable. 
                        \end{enumerate}
                \end{corollary}
                
            \subsubsection{Cohomology of condensed modules}
            
        \subsection{Symmetric monoidal structures; solidity}
            \begin{convention}[Condensed local systems]
                From now on, if $L \in \Sets$ is a set then the corresponding condensed local system (i.e. pro-\'etale local system over a point) shall be suggestively denoted by $L^{\cond}$.
            \end{convention}
        
            First of all, let us clarify that it is not that tensor products of condensed modules do not exist. However, such tensor products will usually end up being topologically pathological or just outright nonsensical. Take for instance, the local system $\Z_p^{\cond} \in \Sets^{\cond}$. Its tensor product with other non-archimedean local systems can be easily topologised via formal completion, but if we were to consider say, $\Z_p^{\cond} \tensor \R^{\cond}$ or $\Z_p^{\cond} \tensor \Z_{\ell}^{\cond}$ (where $\ell \not = p$ is another prime), then it is not very clear what the corresponding topological completion should be.  
            
            
        
        \subsection{Analyticity}
        
    \section{Algebraic K-theory}