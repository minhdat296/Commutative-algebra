\section{The Fargues-Fontaine Curve}
        In this section, we will motivate and introduce the construction of the Fargues-Fontaine Curve. We will also touch on its applications, not just to $p$-adic Hodge theory, but also to the larger $p$-adic Local Langlands Correspondence (in the sense of Fargues-Scholze; cf. \cite{fargues_scholze_geometrization_of_local_langlands}).
    
        \subsection{Shtukas}
            \subsubsection{Shtukas over equal-characteristic field}
                \begin{definition}[The schematic affine Grassmannian] \label{def: schematic_grassmannian}
                    Suppose that $G$ is an affine algebraic group over some arbitrary ground field $k$. Then, the \textbf{affine Grassmannian} attached to $G$, for which we shall write $\Gr_{G/k}$, will be the \'etale quotient stack $\left[G(\!(t)\!)/G[\![t]\!]\right]$: as a sheaf on $\Sch^{\aff}_{/\Spec k}$, it is given by:
                        $$\Gr_{G/k}(\Spec R) \cong G(\bbD^{\circ}_R)/G(\bbD_R)$$
                    where $\bbD_R$ is the \say{formal disc} $\Spec R[\![t]\!]$ and $\bbD^{\circ}_R$ is the \say{punctured disc} $\Spec R(\!(t)\!)$.
                \end{definition}
                \begin{remark}
                    Via some cumbersome and formal mapping stack arguments, one can show that the schematic affine Grassmannian attached to a given affine algebraic group $G$ is nothing but the moduli stack of principal $G$-bundles over the punctured formal disc $\bbD_k^{\circ}$. 
                \end{remark}
                    
                \begin{definition}[Schematic shtukas] \label{def: schematic_shtukas}
                    Let $k$ be an arbitrary ground field and let $G$ be an affine algebraic group over $k$.
                        \begin{enumerate}
                            \item \textbf{(Shtukas with $1$ leg):} The \textbf{moduli stack of shtukas with $1$ leg} attached to $G$ is the prestack:
                                $$\Sht^1_{G/k}: \left(\Sch^{\aff}_{/\Spec k} \x \Sch_{/\Spec k}\right)^{\op} \to \Grpd$$
                            given by:
                                $$\Sht^1_{G/k}(\Spec R, Y) \cong \Bun_G\left(\Spec R \x_{\Spec k} (Y \setminus \{y\}_{\Spec R})\right)$$
                            where $\{y\}_{\Spec R}$ is the scheme-theoretic image of some $y \in Y(\Spec R)$ (incidentally, called the \textbf{leg} of the shtuka $\Sht^1_{G/k}(\Spec R, Y)$), and the corresponding complement is also scheme-theoretic as opposed to set-theoretic. 
                            \item \textbf{(Shtukas with $n$ legs):} Now, by noticing that an $n$-tuple of maps $y_1, ..., y_n \in Y(\Spec R)$ is the same as a map $\vec{y} \in Y^n(\Spec R)$ (where $Y^n$ is the $n$-fold product $Y \x_{\Spec k} ... \x_{\Spec k} Y$), we can define \textbf{moduli stacks of shtukas with $n$ legs} for every fixed scheme $Y \in \Sch_{/\Spec k}$ to be given pointwise by:
                                $$\Sht^n_{G/k}(\Spec R, Y) \cong \Bun_G\left(\Spec R \x_{\Spec k} (Y^n \setminus \{\vec{y}\}_{\Spec R})\right)$$
                        \end{enumerate} 
                    Note that should the base scheme $Y$ be fixed, our definition of moduli stacks of shtukas reduces down to the usual one.
                \end{definition} 
                
                \begin{example}[Grassmannians are moduli stacks of shtukas on the punctured unit disc]
                    Let $X$ be a curve that is smooth, proper, and geometrically connected algebraic curve.
                    \begin{enumerate}
                        \item Notice that we have:
                            $$\Gr_{G/k}\cong \Sht^1_{G/k}(-, \bbD_k)$$
                        thanks to the fact that:
                            $$\bbD_R \setminus \{t\}_{\Spec R} \cong \bbD_R^{\circ}$$
                        This is to say, over the unit disc $\bbD_k$, the moduli stack of shtukas with $1$ legs attached to a given affine algebraic group $G$ is nothing but the affine Grassmannian associated to the same group $G$. We can also denote the formal discs centered at $x$ by $\bbD_x \cong \Spec \calO_{X, x}$ and $\bbD_x^{\circ} \cong \Spec K_{X, x}$ (with $\calO_{X, x}$ the stalk of the structure sheaf of $X$ at a closed point $x \in |X|$, and $K_{X, x}$ its fraction field) and then realise that in some sense, there exists a \say{local} affine Grassmannian around $x \in |X|$, given by:
                            $$\Gr_{G/k, (X, x)}^{\loc} \cong \Sht^1_{G/k}(-, \bbD_x)$$
                        \item Interestingly, when $\chara k = p$ for some prime $p$, one can view the affine Grassmannian as the moduli stack of principal $G$-bundles over the unit disc. This is because then, $K_{X, x}$ \textit{a priori} would admit $k$ as a subfield and hence the \say{local} unit discs $\bbD_x$ and $\bbD_x^{\circ}$ will actually be schemes over $\Spec k$; one should keep the following commutative diagram in mind:
                            $$
                                \begin{tikzcd}
                                	& {\bbD_x^{\circ}} \\
                                	{\Spec k} & {\bbD_x}
                                	\arrow[from=1-2, to=2-2]
                                	\arrow[from=2-1, to=2-2]
                                	\arrow[dashed, from=2-1, to=1-2]
                                \end{tikzcd}
                            $$
                        The following isomorphism, coming from the very definition of shtukas, then makes sense:
                            $$\Sht^1_{G/k}(-, \bbD_x) \cong \Bun_G\left(- \x_{\Spec k} \bbD_x^{\circ}\right)$$
                        From this, one gets the following characterisation of the local affine Grassmannian at the closed point $x \in |X|$:
                            $$\Gr_{G/k, (X, x)}^{\loc} \cong \Bun_G\left(- \x_{\Spec k} \bbD_x^{\circ}\right)$$
                    \end{enumerate}
                \end{example}
                
                \begin{convention}
                    From now on, when $k$ is understood, we shall write $\Sht^n_{G, X}$ instead of $\Sht^n_{G/k, X}$.
                \end{convention}
                
                \begin{remark}[There is a Frobenius action] \label{remark: frobenii_on_shtukas}
                    Let $X$ be a curve that is smooth, proper, and geometrically connected algebraic curve, and let $G$ be an affine algebraic group, both over some field $k$ of characteristic $p$.
                    
                    Each scheme $S$ over a field $k$ of positive characteristic comes equipped with a $p^{th}$ power Frobenius endomorphism $\Frob_{S/k}$. By functoriality, such an action would induce a Frobenius endomorphism on the moduli space of shtukas with $n$ legs attached $G$ over the pair $(S, X)$: specifically, one has:
                        $$\Frob_{\Sht^n_{G, X}/k} = \Bun_G\left(\Frob_{S/k} \x_{\Spec k} (X \setminus \{\vec{x}\})\right)$$
                    where $\vec{x} \in X^n(S)$ be some $n$-tuple of $S$-points of $X$. 
                    
                    Now what should we do with this information ? One can, for instance, tilt the argument, say $S$, of $\Sht^n_{G, X}$, in the sense of definition \ref{def: tilting_schemes}. Then evaluate to get $\Sht^1_{G, X}(S^{\flat})$, and since the essential image of the tilting functor on $\Sch_{/\Spec k}$ is the category of perfect scheme $\Sch^{\perf}_{/\Spec k^{\flat}}$. Thus, whenever $k$ is a perfect field (say, $\F_q$), one restrict the domain of $\Sht^n_{G, X}$ down onto $\Sch^{\perf}_{/\Spec k}$ instead of consider the functor on the entirety of $\Sch_{/\Spec k}$. Let us denote this restricted moduli stack of shtukas by $(\Sht^n_{G, X})^{\flat}$. This, however, begs the following question: what is the point ? The point, dear reader, is that for $K$ a perfectoid field, because $\Sch^{\perf}_{/\Spec K}$ embeds fully faithfully into the category of (small) v-sheaves on $\Perfd_{/\Spa K}$, which means that there ought to be some sort of extension of $(\Sht^n_{G, X})^{\flat}$ to the adic setting. 
                \end{remark}
            
            \subsubsection{Mixed-characteristic shtukas}
                Suppose that $X$ is a sufficiently nice scheme (for formal completion purposes, so it might be noetherian and so on). Additionally, fix a point $x \in |X|$ and denote the formal completion of $X$ at $x$ by $\hat{X}$. Then, via some mapping stack arguments (recall that $\Bun_G \cong [-, \bfB G]$) and through exchanging certain limits and colimits, one can show that:
                    $$(\Sht^1_{G, X})^{\wedge} \cong \Sht^1_{\hat{G}, \hat{X}}$$
                In fact, we can take this further, using the fact that pullbacks commute, to obtain:
                    $$(\Sht^1_{\hat{G}, \hat{X}})^{\ad} \cong \Sht^1_{\hat{G}, \hat{X}^{\ad}}$$
                (recall that adification is defined via pullbacks; cf. definition \ref{def: adification}). In doing this, however, we have unfortunately invoked a technical difficulty: now, the stack $\Bun_G$ is no longer a stack taking in schemes as test objects, but rather, adic spaces. 
        
        \subsection{The Fargues-Fontaine Curve}
            The following description of the Fargues-Fontaine Curve, aside from detailing its applications to Local Class Field Theory, will also serve as a continuation of section \ref{section: perfectoid_spaces}, for The Curve can be viewed as the moduli space of untilts a given algebraically closed perfectoid field of positive characteristic. Specifically, what we want to know are perfectoid fields $E$ of mixed characteristics $(0, p)$ such that $E^{\flat} \cong F$ for some prescribed perfectoid field $F$ of characteristic $p$. Also, let us note that this is in no way an ill-posed question, as for instance, we know that the tilts of both $\Q_p(p^{\frac{1}{p^{\infty}}})^{\wedge}$ and $\Q_p(\mu_{p^{\infty}})^{\wedge}$ are isomorphic to $\F_p(\!(t^{\frac{1}{p^{\infty}}})\!)$. The Fargues-Fontaine Curve, which we shall denote by $X_{\FF}$, should thus be some sheaf of sets on $\Perfd_{/\Spec \F_p}$ whose fibres over adic spectra of algebraic closures of adic residue fields (understood as \textit{geometric points}) shall be sets of untilts of that very algebraically closed field.
            
            We would like our to-be adic Fargues-Fontaine Curve to be a curve over a base field that is:
                \begin{itemize}
                    \item projective, so it would have a chance of admitting a GAGA-esque functor,
                    \item smooth, so that its topology would behave nicely (read: so that we would be able to apply \'etale cohomology) and so important cases such as elliptic curves would be covered, 
                    \item proper (or according to certain authors such as D. Gaitsgory, \say{complete}, although we will avoid this terminology since we would frequently be thinking of topologically complete objects), so the Proper Base Change Theorem from the theory of \'etale cohomology would apply 
                \end{itemize}
            and most importantly, such that it would have a very straight forward connection to the Weil group of a $p$-adic number field because then, certain local systems on The Curve would correspond with the Weil-Deligne representations of the aforementioned Weil group, which is something that we would want for the establishment of the Galois/spectral side of the Local Langlands Correspondence. Also, due to this last point, one should imagine the Fargues-Fontaine Curve as a sort of $p$-adic Riemann surface; in fact, we will see (eventually) that it does behave similarly to Riemann surfaces in many ways, particularly via its connection to $p$-adic Shimura varieties.
            
            \begin{convention}[Relative Witt vectors] \label{conv: relative_witt_vectors}
                Let $p$ be a prime number. If $E/\Q_p$ is a finite $p$-adic number field with residue field $\F_q$ (for $q$ some power of $p$) and if $F$ is a perfectoid field of characteristic $p$ (which is \textit{a priori} perfect and therefore can contain $\F_q$; consider fields such as $\F_q(\!(t^{\frac{1}{p^{\infty}}})\!)$ or $\widehat{\overline{\F_q(\!(t)\!)}}$ for example), then let us write:
                    $$\W_{E^{\circ}}(F) \cong \W(F) \tensor_{\W(\F_q)} E^{\circ}$$
                for the base change along $E^{\circ} \to $ of the ring $\W(F)$ of (unramified) $p$-typical Witt vectors with coefficients in $F$ along the canonically induced arrow $\W(\F_q) \to E^{\circ}$. For instance, when $q = p$ we have:
                    $$\W_{E^{\circ}}(\F_p(\!(t^{\frac{1}{p^{\infty}}})\!)) \cong \Z_p[p^{\frac{1}{p^{\infty}}}] \tensor_{\Z_p} \Z_p \cong \Z_p[p^{\frac{1}{p^{\infty}}}]$$
                Slightly more generally, one can speak of a relative $p$-typical Witt vector functor from the category of perfect commutative $\F_q$-algebras to the category ${}^{E^{\circ}/}\Comm\Alg^{\wedge}$ of $p$-adically complete $E^{\circ}$-algebras:
                    $$\W_{E^{\circ}}: {}^{\F_q/}\Comm\Alg^{\perf} \to {}^{E^{\circ}/}\Comm\Alg^{\wedge}$$
                which in particular, gives us the following canonical arrow:
                    $$\W_{E^{\circ}}(F^{\circ}) \to \W_{E^{\circ}}(F)$$
            \end{convention}
            
            \begin{remark}[Witt vectors over perfect rings] \label{remark: witt_vectors_over_perfect_rings}
                \noindent
                \begin{enumerate}
                    \item One somewhat non-trivial fact to keep in mind is that if $B$ is a \textit{perfect} $\F_q$-domain (for some power $q$ of a prime $p$) with field of fractions $K$, then we have the following natural characterisation of the ring of $p$-typical Witt vectors over $K$ (which we note to be trivially perfect as an $\F_q$-algebra):
                    $$(\W(B)_{(p)})^{\wedge} \cong \W(K)$$
                    In particular, $\W(B)_{(p)}$ is an unramified extension of $\W(\F_q)$. We refer the reader to \cite[Proposition 5.2]{shimomoto2014witt} for a proof.
                    \item This result extends trivially to the relative setting, as relative Witt vectors are defined via a pushout (cf. convention \ref{conv: relative_witt_vectors}), which is in particular a finite limit, and since adic completions are filtered limits, the two procedures can be exchanged, which gives:
                        $$(\W_{E^{\circ}}(B)_{(p)})^{\wedge} \cong \W_{E^{\circ}}(K)$$
                    where now, $E$ is a finite $p$-adic number field with residue field $\F_q$ (note that its pseudo-uniformiser is actually just $p$, like $\Q_p$).
                    \item Another notable property of rings (relative) Witt vectors is that if $B$ is a perfect $\F_q$-algebra (not necessarily a domain), then $\W_{E^{\circ}}(B)$ is $p$-torsion-free, and:
                        $$\W_{E^{\circ}}(B)/p \cong B$$
                \end{enumerate}
            \end{remark}
            \begin{example}
                One might think of the following example:
                    $$(\W_{E^{\circ}}(\F_q)_{(p)})^{\wedge} \cong \W_{E^{\circ}}(\F_q) \cong E^{\circ}$$
                (note that $E^{\circ}$ is \textit{a priori} complete), or the following slightly subtler one:
                    $$( \W_{E^{\circ}}( \F_q[\![t^{\frac{1}{p^{\infty}}}]\!] )_{(p)} )^{\wedge} \cong E^{\circ}[p^{\frac{1}{p^{\infty}}}]^{\wedge} \cong \W_{E^{\circ}}( \F_q(\!(t^{\frac{1}{p^{\infty}}})\!) ) \cong E(p^{\frac{1}{p^{\infty}}})^{\wedge, \circ}$$
                (indeed, $E^{\circ}[p^{\frac{1}{p^{\infty}}}]^{\wedge} \cong E^{\circ}[\![t]\!]/(t^{p^{\infty}} - p)$ and so $E^{\circ}[p^{\frac{1}{p^{\infty}}}]^{\wedge}/p \cong \F_q[\![t^{\frac{1}{p^{\infty}}}]\!]$). In both cases, note that $\F_q$ and $\F_q[\![t^{\frac{1}{p^{\infty}}}]\!]$ are both perfect domains (the latter being the $p$-tilt of $\F_q[\![t]\!]$).
            \end{example}
            \begin{convention}[Fontaine's infinitesimal period ring] \label{conv: A_inf}
                If $R$ is a perfectoid ring of characteristic $p$, if $\varphi \in R^{\circ \circ}$ is some pseudo-uniformiser of $R$, and if $R^{\circ}/\varpi$ is some $\F_q$-algebra for $q$ some power of $p$, then let us denote the aboslute Witt vector functor $\W \cong \W_{\Z_p}$ by $\A_{\inf}$. The ring $\A_{\inf}(R)$ is typically known as the \textbf{Fontaine infinitesitmal period ring} associated to $R$, although for reasons that we will not get into here; \textit{The} Fontaine infinitestimal period ring is usually understood to be that associated to $\bbC_p$, the completion of the algebraic closure of $\Q_p$.
            \end{convention}
            
            We refer the reader to convention \ref{conv: prestacks} for the notion of (pre)stacks.
            \subsubsection{The Curve as a moduli space of untilts}
                \begin{definition}[Prediamonds] \label{def: prediamonds}
                    Let $X$ be a perfectoid space of positive characteristic. The category $\Pre\Dia_{/X}$ of \textbf{prediamonds} over $X$ is thus nothing but the presheaf topos on $\Perfd_{/X}$.
                \end{definition}
                
                For now we shall refer to the to-be the Fargues-Fontaine Curve as the \say{Fargues-Fontaine Space}, as there is not yet a good reason to legitimately call it a \say{curve}.
                \begin{definition}[The Frobenius Space] \label{def: the_frobenius_space}
                    
                \end{definition}
                
                \begin{definition}[The Fargues-Fontaine Space] \label{def: the_fargues_fontaine_space}
                    
                \end{definition}
                \begin{remark}[Tilting the Fargues-Fontaine Space] \label{remark: tilting_the_fargues_fontaine_space}
                    
                \end{remark}
                
                \begin{theorem}[The Fargue-Fontaine Space as the moduli curve of untilts] \label{theorem: fargues_fontaine_space_as_moduli_space_of_untilts}
                    
                \end{theorem}
                    \begin{proof}
                        
                    \end{proof}
                    
                \begin{theorem}[The Fargue-Fontaine Space as a punctured disc] \label{theorem: fargues_fontaine_space_as_punctured_disc}
                    
                \end{theorem}
                
                \begin{lemma}[The Covering Space is a diamond] \label{the_covering_space_is_a_diamond}
                    
                \end{lemma}
                    \begin{proof}
                        
                    \end{proof}
                \begin{theorem}[The Fargues-Fontainte Space is perfectoid] \label{theorem: the_fargues_fontaine_space_is_perfectoid}
                    
                \end{theorem}
                    \begin{proof}
                        
                    \end{proof}
                \begin{corollary}[Discontinuity of Frobenius] \label{coro: discontinuity_of_frobenius}
                    
                \end{corollary}
                
            \subsubsection{The Curve as a \texorpdfstring{$p$}{}-adic Riemann surface}
            
            \subsubsection{A GAGA theorem}
            
            \subsubsection{Vector bundles on The Curve}