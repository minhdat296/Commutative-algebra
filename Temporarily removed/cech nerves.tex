\begin{definition}[\v{C}ech nerves and hypercovers] \label{def: hypercovers} \index{\v{C}ech nerves} \index{Hypercovers}
                    \noindent
                    \begin{enumerate}
                        \item \textbf{(Simplicial objects):} First of all, \textbf{the simplex category} (which we shall denote by $\simp$) shall to us be the full $1$-subcategory spanned by the \textit{non-empty} finite ordinals of the category $\Cat$ of \textit{small} categories:
                            $$\simp \cong
                                \left\{
                                    \begin{tikzcd}
                                    	{[0]} & {[1]} & {[2]} & \cdots
                                    	\arrow[shift right=2, from=1-1, to=1-2]
                                    	\arrow[shift right=1, from=1-2, to=1-1]
                                    	\arrow[from=1-2, to=1-1]
                                    	\arrow[shift right=2, from=1-2, to=1-3]
                                    	\arrow[shift right=1, from=1-2, to=1-3]
                                    	\arrow[shift right=1, from=1-3, to=1-2]
                                    	\arrow[shift right=2, from=1-3, to=1-2]
                                    	\arrow[shift right=3, from=1-3, to=1-2]
                                    	\arrow[shift right=1, from=1-4, to=1-3]
                                    	\arrow[shift right=2, from=1-4, to=1-3]
                                    	\arrow[shift right=3, from=1-4, to=1-3]
                                    	\arrow[shift right=2, from=1-3, to=1-4]
                                    	\arrow[shift right=1, from=1-3, to=1-4]
                                    	\arrow[shift right=4, from=1-4, to=1-3]
                                    	\arrow[shift right=3, from=1-3, to=1-4]
                                    \end{tikzcd}
                                \right\}
                            $$
                        The forward arrows are injective functions called \textbf{face maps}, and the backward arrows are surjective functions known as \textbf{degeneracy maps}; see \cite[Definition 2.2]{nlab:simplicial_set} for more details. Then, let us simply define a \textbf{simplicial object} of a category $\E$ with enough pullbacks to be a presheaf on the simplex category, i.e. a functor:
                            $$X: \simp^{\op} \to \E$$
                        or in other words, a diagram in $\E$ of shape $\simp^{\op}$. 
                        \item \textbf{(\v{C}ech nerves):} Let $\E$ be a category with enough pullbacks, and let:
                            $$u \to x$$
                        be some morphism of $\E$. Then, the \textbf{\v{C}ech nerve} of $u$ over $x$ - for which we shall write $u^{\bullet}_{/x}$ - is the following simplicial object of $\E$, whose face maps are canonical projections, and the degeneracy maps are equalities (note that the direction in which these arrows point have been flipped since simplicial objects are diagrams of shape $\simp^{\op}$):
                            $$u^{\bullet}_{/x} \cong
                                \left\{
                                    \begin{tikzcd}
                                    	u & {u \x_x u} & {u \x_x u \x_x u} & \cdots
                                    	\arrow[shift right=1, from=1-2, to=1-1]
                                    	\arrow[shift left=1, from=1-2, to=1-1]
                                    	\arrow[from=1-3, to=1-2]
                                    	\arrow[shift right=2, from=1-3, to=1-2]
                                    	\arrow[shift left=2, from=1-3, to=1-2]
                                    	\arrow[shift left=1, from=1-4, to=1-3]
                                    	\arrow[shift right=1, from=1-4, to=1-3]
                                    	\arrow[shift left=3, from=1-4, to=1-3]
                                    	\arrow[shift right=3, from=1-4, to=1-3]
                                    \end{tikzcd}
                                \right\}
                            $$
                        When $u$ is implicitly understood, one may say ambiguous things such as \say{the \v{C}ech nerve on $x$}.
                        
                        One very important thing to note is that the \v{C}ech nerve associated to some given morphism is the \href{https://ncatlab.org/nlab/show/internal+category#internal_nerve}{\underline{internal nerve}} of the internal groupoid in $\E$ that is the kernel pair:
                            $$u \toto x$$
                        corresponding to the given arrow $u \to x$ (cf. definition \ref{def: equivalence_relations}): particularly, if one was to write $u^i_{/x}$ for the $i$-fold pullback of $u \to x$ along itself (so for instance, $u^1_{/x} \cong u \x_x u$), then it would be clear that each span of the following form would be an internal groupoid (in fact, a kernel pair) inside $\E$:
                            $$
                                \begin{tikzcd}
                                	{u^{i + 1}_{/x}} & {u^i_{/x}} \\
                                	{u^i_{/x}}
                                	\arrow[from=1-1, to=2-1]
                                	\arrow[from=1-1, to=1-2]
                                \end{tikzcd}
                            $$
                        To pause and handwave a bit (so as to keep traditions alive), one might imagine the \v{C}ech nerve of a given morphism (which itself can be thought of as the inclusion of a space into a larger ambient space) as the system of $2$-fold, $3$-fold, ... intersections of subspaces, which helps us approximate the ambient space (via its cohomology groups!) better with each successive intersection.
                        \item \textbf{(Hypercovers):} 
                    \end{enumerate}
                \end{definition}
                \begin{remark}[\v{C}ech nerves and hypercovers via functors] \label{remark: hypercovers_alt_def}
                    \noindent
                    \begin{enumerate}
                        \item \textbf{(\v{C}ech nerves):} We apologise in advance for invoke the forbidden word, but (un)fortunately, the most natural settings for introducing hypercovers turns out to be that of $\infty$-categories. 
                        
                        Recognise the simplex category $\simp$ as a full $\infty$-subcategory the category $\Sets^{\fin}$ of finite sets with the same morphisms but fewer ($1$-truncated) morphisms (since there can be functions between finite sets which are neither injective nor surjective, for instance) via a fully faithful $\infty$-embedding:
                            $$\simp \hookrightarrow \Sets^{\fin}$$
                        and consider a commutative diagram of the following form, wherein $\E$ is an $\infty$-category with all $\infty$-pullbacks, $x$ is an object of $\E$:
                            $$
                                \begin{tikzcd}
                                	{\simp^{\op}} && {\Sets^{\fin, \op}} \\
                                	& \E_{/x}
                                	\arrow[hook, from=1-1, to=1-3]
                                	\arrow[from=1-1, to=2-2]
                                	\arrow[from=1-3, to=2-2]
                                \end{tikzcd}
                            $$
                        and with the functor $\Sets^{\fin, \op} \to \E_{/x}$ being the one that returns pointwise exponentials of objects of $\E_{/x}$ (i.e. it is given by $u \mapsto u^n$, with $u^n$ being the $n$-fold pullback of $u \to x$ along itself). Then, take the right $\infty$-Kan extension of the composite $\simp^{\op} \to \Sets^{\fin, \op} \to \E_{/x}$ along the composite $\simp^{\op} \to \E_{/x} \to \E$, with $\E_{/x} \to \E$ the evident forgetful functor, which fits into the following diagram whose outer portion (i.e. the part without the dashed arrow) commutes:
                            $$
                                \begin{tikzcd}
                                	& {\simp^{\op}} && {\Sets^{\fin, \op}} \\
                                	{\E_{/x}} && {\E_{/x}} \\
                                	&& \E
                                	\arrow[from=1-2, to=1-4]
                                	\arrow[from=1-2, to=2-3]
                                	\arrow[from=1-4, to=2-3]
                                	\arrow[from=1-4, to=3-3]
                                	\arrow[""{name=0, anchor=center, inner sep=0}, from=1-2, to=3-3]
                                	\arrow[dashed, from=2-3, to=3-3]
                                	\arrow[from=1-2, to=2-1]
                                	\arrow["\oblv"', from=2-1, to=3-3]
                                	\arrow[shorten >=2pt, Rightarrow, from=2-3, to=0]
                                \end{tikzcd}
                            $$
                        We shall denote this right $\infty$-Kan extension by $(-)^{\bullet}_{/x}$ and call it the \textbf{\v{C}ech nerve functor over $x$}. At this point, let us take note of the fact that by a property of general right $\infty$-Kan extensions, we can realise \v{C}ech nerve functors as certain $\infty$-limits of $\infty$-functors:
                            $$(-)^{\bullet}_{/x} \cong {}^{(\infty, 1)}\underset{n \in \N}{\lim} (-)^{[n]}_{/x}$$
                        wherein we view $\N$ as an ordered set, and by $u^{[n]}_{/x}$, we mean the following $n$-truncated object of $\E_{/x}$:
                            $$
                                u^{[n]}_{/x} \cong
                                \left\{
                                    \begin{tikzcd}
                                    	u & {u \x_x u} & {u \x_x u \x_x u} & \cdots & {u^n}
                                    	\arrow[shift left=1, from=1-4, to=1-3]
                                    	\arrow[shift right=1, from=1-4, to=1-3]
                                    	\arrow[shift left=3, from=1-4, to=1-3]
                                    	\arrow[shift right=3, from=1-4, to=1-3]
                                    	\arrow[from=1-3, to=1-2]
                                    	\arrow[shift right=2, from=1-3, to=1-2]
                                    	\arrow[shift left=2, from=1-3, to=1-2]
                                    	\arrow[shift right=1, from=1-2, to=1-1]
                                    	\arrow[shift left=1, from=1-2, to=1-1]
                                    	\arrow[""{name=0, anchor=center, inner sep=0}, shift left=4, from=1-5, to=1-4]
                                    	\arrow[""{name=1, anchor=center, inner sep=0}, shift right=4, from=1-5, to=1-4]
                                    	\arrow["\cdots"{marking}, Rightarrow, draw=none, from=1, to=0]
                                    \end{tikzcd}
                                \right\}
                            $$
                        (i.e. the objects $u^{[n]}_{/x}$ are \say{simplicial powers} of sort). This is evident if we would notice that each $u^{[n]}_{/x}$ is the image of the simplex $[n]$ under the composition:
                            $$\simp^{\op} \hookrightarrow \Sets^{\fin, \op} \to \E_{/x} \to \E$$
                        wherein $\E_{/x} \to \E$ is the forgetful functor. This tells us that objects in the essential image of the \v{C}ech nerve functor $(-)^{\bullet}_{/x}$ need to be of the form:
                            $$
                                u^{\bullet}_{/x} \cong
                                \left\{
                                    \begin{tikzcd}
                                    	u & {u \x_x u} & {u \x_x u \x_x u} & \cdots
                                    	\arrow[shift right=1, from=1-2, to=1-1]
                                    	\arrow[shift left=1, from=1-2, to=1-1]
                                    	\arrow[from=1-3, to=1-2]
                                    	\arrow[shift right=2, from=1-3, to=1-2]
                                    	\arrow[shift left=2, from=1-3, to=1-2]
                                    	\arrow[shift left=1, from=1-4, to=1-3]
                                    	\arrow[shift right=1, from=1-4, to=1-3]
                                    	\arrow[shift left=3, from=1-4, to=1-3]
                                    	\arrow[shift right=3, from=1-4, to=1-3]
                                    \end{tikzcd}
                                \right\}
                            $$
                        which means that they are precisely the \v{C}ech nerves associated to the structure morphisms $u \to x$, and also, that:
                            $$\tau_{\leq n} (-)^{\bullet}_{/x} \cong (-)^{[n]}_{/x}$$
                        for all natural numbers $n$, which means that \textit{\v{C}ech nerve functors are limits of their \href{https://ncatlab.org/nlab/show/n-truncated+object+of+an+\%28infinity\%2C1\%29-category}{\underline{truncations}}}. Our purpose for bothering with all this abstract nonsense is, that the new formulation helps simplify the computations of (co)limits indexed by \v{C}ech nerves somehwhat. For example, in proposition \ref{prop: affine_schemes_are_schemes} below, we wish to establish the fact that when given a scheme $X$ (which we shall view as an object of the category $[\Spec \Z]_{\et}$ - or the full subcategory $[\Spec \Z]_{\et}^{\geom}$ for that matter, as both have all pullbacks), we can realise it as the colimit of the \v{C}ech nerve associated any of its Zariski atlases $U \to X$, i.e.:
                            $$X \cong \colim U^{\bullet}_{/X}$$
                        and we are interpreting $U^{\bullet}_{/X}$ as the image of the object $(U \to X) \in [\Spec \Z]_{/X, \et}$ under the \v{C}ech nerve functor $(-)^{\bullet}_{/X}$. Another advantage of this perspective is that it allows us to study properties of \v{C}ech nerves via asking questions such as whether or not \v{C}ech nerve functors preserve (co)limits (and since \v{C}ech nerve functors are right Kan extensions, they should enjoy all sorts of nice properties). 
                        
                        Lastly, because the spans of the following form would are internal groupoids inside $\pi_n(\E)$:
                            $$
                                \begin{tikzcd}
                                	{u^{n + 1}_{/x}} & {u^n_{/x}} \\
                                	{u^n_{/x}}
                                	\arrow[from=1-1, to=2-1]
                                	\arrow[from=1-1, to=1-2]
                                \end{tikzcd}
                            $$
                        it is rather easy to see how \v{C}ech complexes are actually internal $\infty$-groupoids (in fact, $\infty$-kernel pairs). 
                        \item \textbf{(Hypercovers):}
                    \end{enumerate}
                \end{remark}