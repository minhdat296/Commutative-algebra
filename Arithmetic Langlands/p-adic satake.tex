\chapter{The \texorpdfstring{$p$}{}-adic Geometric Satake Equivalence}
    \begin{abstract}
        
    \end{abstract}
    
    \minitoc
    
    \section{The Geometric Satake Equivalence in mixed characteristics}
        \begin{convention} \label{conv: p_adic_geometric_satake_conventions}
            \noindent
            \begin{itemize}
                \item Let us fix once and for all a non-archimedean local field $E$ with residue field $\F_q$, along with some choice of pseudo-uniformiser $\varpi \in E^{\circ \circ}$. 
                \item Additionally, $X_S$ shall denote the Fargues-Fontaine Curve over any given perfectoid space $S \in \Perfd_{/\Spa \F_q}$. We refer the reader to section \ref{section: the_fargues_fontaine_curve} for details on the construction and properties of The Curve. 
                \item Lastly, $G$ shall be a connected reductive group over $\Spd E$.
            \end{itemize}
        \end{convention}
        
        In this section we attempt to understand a version of the Geometric Satake Correspondence in mixed characteristics as written down in \cite{fargues_scholze_geometrization_of_local_langlands}, whose statement asserts an equivalence between certain kinds of sheaves on the affine Grassmannian attached to $G$ and the category of representations of the Langlands dual of $G$, thereby establishing a rudimentary version of Langlands Duality. 
        
        \subsection{The Satake Metropolitan and its suburbs}
            The Geometric Satake Correspondence is a very deep and complicated result. It is, after all, the result where one observes the first instance of Langlands Duality. Due to this, it shall be beneficial to first give an expository presentation, not just of the Satake Equivalence itself, but also, surrounding results and ideas, so that we will have a better understanding of how this result fits into the larger picture of the Langlands Programme.
            
            \subsubsection{The Geometric Satake Equivalence for \texorpdfstring{$p$}{}-adic groups}
            
            \subsubsection{A motivic reformulation}
                \begin{convention}
                    For the sake of this subsection, let us assume that the reader has some familiarity with the theory or pure motives. In fact, on that note, let us denote by:
                        $$\Mot^{\pure} \to (\Var_{/\Spec \F_q}^{\smooth, \proj})^{\op}$$
                        $$\Mot^{\num} \to (\Var_{/\Spec \F_q}^{\smooth, \proj})^{\op}$$
                    (respectively) the Karoubian rigid symmetric monoidal category of pure motives and of numerical motives over the category of smooth projective algebraic varieties over $\Spec \F_q$. Both shall implicitly be taken as being $\Q$-linear.s
                \end{convention} 
            
            \subsubsection{Categorical traces and Shimura varieties}
    
        \subsection{The \texorpdfstring{$\B_{\dR}$}{}-affine Grassmannian}
            \subsubsection{The affine Grassmannian over a point}
                \begin{convention}[$\B_{\dR}$-discs]
                    For each $\Spa(R, R^+) \in \Perfd^{\affd}_{/\Spa E^{\flat}}$, let us write $\bbD_{\dR}^+(R)$ for the so-called \textbf{$\B_{\dR}$-disc} $\Spec \B^+_{\dR}(R)$, and $\bbD_{\dR}(R)$ for the \textbf{punctured $\B_{\dR}$-disc} $\Spec \B_{\dR}(R)$.
                \end{convention}
            
                \begin{definition}[Local $\B_{\dR}$-affine Grassmannians] \label{def: local_B_dR_affine_grassmannian}
                    There is a canonically defined moduli space, denoted by $\Gr_G^{\loc}$ and called the \textbf{\textit{local} $\B_{\dR}$-affine Grassmannian} attached to $G$. It is the prestack which assigns to each $\Spa R \in \Perfd_{/\Spd E}^{\affd}$ the groupoid that is the core of the category of \'etale $G$-torsors on $\bbD_{\dR}^+(R)$ that trivialise over $\bbD_{\dR}(R)$.  
                \end{definition}
                \begin{remark}
                    \noindent
                    \begin{itemize}
                        \item It is rather easy to see that $\Gr_G^{\loc}$ satisfies \'etale descent and hence tautologically a stack on $(\Perfd^{\affd}_{/\Spd E})_{\et}$.
                        \item Furthermore, by introducing the so-called $\B_{\dR}$-loop and $\B_{\dR}$-arc groups $G_{\dR}^+$ and $G_{\dR}$\footnote{Note that in this context $(-)_{\dR}$ is not the functor of de Rham spaces like in section \ref{section: D_modules_over_characteristic_0}.}, defined by $G_{\dR}^+(R) \cong G(\B_{\dR}^+(R))$ and $G_{\dR}(R) \cong G(\B_{\dR}(R))$ respectively, one can show that:
                            $$\Gr_G^{\loc} \cong G_{\dR}/G_{\dR}^+$$
                        using the fact that $G_{\dR}^+$ acts on $\Gr_G^{\loc}$ by changing the trivialisation. This is an important description of the affine Grassmannian, so let us state and prove it properly (cf. proposition \ref{prop: B_dR_affine_grassmannian_as_coset_spaces}).
                    \end{itemize}
                \end{remark}
                
                Let us now investigate the geometry of $\Gr_G^{\loc}$. In particular, our aim is to establish properties of $\Gr_G^{\loc}$ that would make the consideration of (equivariant) perverse sheaves thereon a meaningful process.
                \begin{lemma}[Affine Grassmannians are $v$-sheaves] \label{lemma: B_dR_affine_grassmannians_are_v_sheaves}
                    $\Gr_G^{\loc}$ satisfies $v$-descent, and hence pro-\'etale descent as well.
                \end{lemma}
                    \begin{proof}
                        
                    \end{proof}
                \begin{proposition}[Affine Grassmannians as coset spaces] \label{prop: B_dR_affine_grassmannian_as_coset_spaces}
                    Let $G$ be an algebraic group. Then, $\Gr_G^{\loc} \cong G_{\dR}/G_{\dR}^+$, with the right-hand side implicitly meaning the \'etale sheafification of quotient presheaf.
                \end{proposition}
                    \begin{proof}
                        
                    \end{proof}
                \begin{corollary}[Loop group action on Grassmannians] \label{coro: loop_group_action_on_B_dR_grassmannians}
                    There is a $G_{\dR}^+$-action on $\Gr_G^{\loc}$, which means that one can now meaningfully discuss equivariance of sheaves on the affine Grassmannian $\Gr_G^{\loc}$. In particular, we shall be interested in $G_{\dR}^+$-equivariant perverse sheaves on $\Gr_G^{\loc}$.
                \end{corollary}
                
                \begin{proposition}[Structure of affine Grassmannians] \label{prop: structure_of_B_dR_affine_grassmannian}
                    As a $v$-stack over $\Spd E$, the affine Grassmannian $\Gr_G^{\loc}$ is separated (hence quasi-separated) and proper.
                \end{proposition}
                    \begin{proof}
                        \noindent
                        \begin{enumerate}
                            \item \textbf{(Separatedness):}
                            \item \textbf{(Properness):}
                        \end{enumerate}
                    \end{proof}
                
                \begin{proposition}[Functoriality of the affine Grassmannian] \label{prop: B_dR_affine_grassmannian_functoriality}
                    Let $H$ be a closed reductive subgroup (also over $\Spd E$) of $G$. Then, the induced map $\Gr_H^{\loc} \to \Gr_G^{\loc}$ is a closed embedding as well.
                \end{proposition}
                    \begin{proof}
                        
                    \end{proof}
                \begin{remark}
                    Proposition \ref{prop: B_dR_affine_grassmannian_functoriality} is tremendously useful. This is because every reductive group is in particular a linear algebraic group, and the \say{maximal} reductive group - that being $\GL_n$ - is also the \say{maximal} linear algebraic group. One can thus prove assertions for $\Gr_{\GL_n}$ (which is relatively simple to understand) before extending these results to other reductive groups.  
                \end{remark}
                    
            \subsubsection{Bruhat Decomposition, Schubert cells, and ind-structures}
            
            \subsubsection{The Demazure Resolution for the affine Grassmannian}
            
            \subsubsection{Representability of the affine Grassmannian}
            
        \subsection{Beilinson-Drinfeld Grassmannians}
            \subsubsection{Definition}
            
            \subsubsection{Semi-infinite orbits}
            
            \subsubsection{Sheaves on Beilinson-Drinfeld Grassmannians}
            
        \subsection{The Satake Equivalence}
            \subsubsection{Convolution of equivariant perverse sheaves on the affine Grassmannian and properties thereof}
            
            \subsubsection{Hecke eigensheaves}
            
            \subsubsection{Fibre functors and weights}
            
            \subsubsection{Langlands Duality}
    
    \section{Relative perverse sheaves}