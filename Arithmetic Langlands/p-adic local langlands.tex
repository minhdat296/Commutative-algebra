\chapter{Geometrisation of the \texorpdfstring{$p$}{}-adic Local Langlands Programme}
    \begin{abstract}
        
    \end{abstract}
    
    \minitoc
    
    \section{The Fargues-Fontaine Curve} \label{section: the_fargues_fontaine_curve}
        In this section, we will motivate and introduce the construction of the Fargues-Fontaine Curve. We will also touch on its applications, not just to $p$-adic Hodge theory, but also to the larger $p$-adic Local Langlands Correspondence (in the sense of Fargues-Scholze; cf. \cite{fargues_scholze_geometrization_of_local_langlands}).
    
        \subsection{Construction of The Fargues-Fontaine Curve}
            The following description of the Fargues-Fontaine Curve, aside from detailing its applications to Local Class Field Theory, will also serve as a continuation of section \ref{section: perfectoid_spaces}, for The Curve can be viewed as the moduli space of untilts a given algebraically closed perfectoid field of positive characteristic. Specifically, what we want to know are perfectoid fields $E$ of mixed characteristics $(0, p)$ such that $E^{\flat} \cong F$ for some prescribed perfectoid field $F$ of characteristic $p$. Also, let us note that this is in no way an ill-posed question, as for instance, we know that the tilts of both $\Q_p(p^{\frac{1}{p^{\infty}}})^{\wedge}$ and $\Q_p(\mu_{p^{\infty}})^{\wedge}$ are isomorphic to $\F_p(\!(t^{\frac{1}{p^{\infty}}})\!)$. The Fargues-Fontaine Curve, which we shall denote by $X_{\FF}$, should thus be some sheaf of sets on $\Perfd_{/\Spec \F_p}$ whose fibres over adic spectra of algebraic closures of adic residue fields (understood as \textit{geometric points}) shall be sets of untilts of that very algebraically closed field.
            
            We would like our to-be adic Fargues-Fontaine Curve to be a curve over a base field that is:
                \begin{itemize}
                    \item projective, so it would have a chance of admitting a GAGA-esque functor,
                    \item smooth, so that its topology would behave nicely (read: so that we would be able to apply \'etale cohomology) and so important cases such as elliptic curves would be covered, 
                    \item proper, so the Proper Base Change Theorem from the theory of \'etale cohomology would apply 
                \end{itemize}
            and most importantly, such that it would have a very straight forward connection to the Weil group of a $p$-adic number field because then, certain local systems on The Curve would correspond with the Weil-Deligne representations of the aforementioned Weil group, which is something that we would want for the establishment of the Galois/spectral side of the Local Langlands Correspondence. Also, due to this last point, one should imagine the Fargues-Fontaine Curve as a sort of $p$-adic Riemann surface; in fact, we will see (eventually) that it does behave similarly to Riemann surfaces in many ways, particularly via its connection to $p$-adic Shimura varieties.
        
        \subsection{A GAGA theorem for vector bundles on The Curve}
    
    \section{Cohomology of the moduli space of shtukas}
    
    \section{The spectral action of the stack of L-parameters}